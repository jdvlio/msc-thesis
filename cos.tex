%! TEX root = ./cos.tex
\documentclass{report}[12pt, A4paper]

% Set pdf page size
\pdfpagewidth=\paperwidth\pdfpageheight=\paperheight

% syntax only ---------
% comment out to build:
% \usepackage{syntonly}

% load packages
\usepackage[utf8]{inputenc}
% \usepackage[
%    language=english,
%    title={A L\"auchli and Leonard style result for complete linear orders},
%    author={Jean-Pierre de Villiers},
%    supervisor={Dr.\ Ruaan Kellerman},
%    examiner={},
%    type={Magister Scientiae},
%    institute={University of Pretoria},
%    course={MSc Mathematics},
%    startdate={2019--02--01},
%    enddate={2020--08--30}
% ]{scientific-thesis-cover}
\usepackage{mathtools}
\usepackage{amssymb}
\usepackage{amsfonts}
\usepackage{amsthm}
\usepackage{bm}
\usepackage{faktor}
\usepackage{enumitem}
\usepackage{tikz-cd}
\usepackage[sectionbib]{chapterbib}
\usepackage[nopostdot,symbols,nomain,nonumberlist]{glossaries}
\usepackage{stmaryrd}

% double spacing for easy annotation
% \usepackage[doublespacing]{setspace}

% packages that generally MUST be loaded last
% \usepackage{hyperref}

% packages that must be loaded after hyperref
% \usepackage[symbols,nomain]{glossaries}

% add definition theorem environments
\theoremstyle{definition}
\newtheorem{dfn}{Definition}[chapter]
\newtheorem{exm}[dfn]{Example}
\newtheorem{conv}[dfn]{Convention}
\newtheorem{assn}[dfn]{Assumption}

% add plain theorem environments
\theoremstyle{plain}
\newtheorem{thm}[dfn]{Theorem}
\newtheorem{lem}[dfn]{Lemma}
\newtheorem{cor}[dfn]{Corollary}
\newtheorem{prp}[dfn]{Proposition}

\newcounter{nclaim}[dfn]
\newcounter{ncase}[dfn]
% add remark theorem environments
\theoremstyle{remark}
\newtheorem{rem}[dfn]{Remark}
\newtheorem*{notation}{Notation}
\newtheorem{claim}[nclaim]{Claim}
\newtheorem{case}[ncase]{Case}

% common sets of numbers
\newcommand{\nats}{\mathbb{N}}
\newcommand{\posnats}{\mathbb{N}^{+}}
\newcommand{\ints}{\mathbb{Z}}
\newcommand{\rats}{\mathbb{Q}}
\newcommand{\reals}{\mathbb{R}}
\newcommand{\irrats}{\mathbb{I}}

% finite ordinals
\newcommand{\zero}{\mathbf{0}}
\newcommand{\one}{\mathbf{1}}
\newcommand{\two}{\mathbf{2}}
\newcommand{\n}{\mathbf{n}}

% special classes of linear orders
\newcommand{\VD}{\mathcal{VD}}
\newcommand{\Mzero}{\mathcal{M}_0}
\newcommand{\M}{\mathcal{M}}
\newcommand{\linear}{\mathcal{L}}
\newcommand{\scattered}{\mathcal{S}}
\newcommand{\C}{\mathcal{C}}
\newcommand{\Cast}{\mathcal{C}^\ast}
\newcommand{\dense}{\mathcal{D}}
\newcommand{\Dense}{\mathcal{D}^\ast}
\newcommand{\continuous}{\mathcal{L}_\lambda}
\newcommand{\ordinals}{\mathcal{O}\mathrm{rd}}

% important sentences
\newcommand{\axmlin}{\sigma_{\mathrm{L}}}
\newcommand{\axmden}{\sigma_{\mathrm{D}}}

% aliases
\newcommand{\define}{\coloneqq}
\newcommand{\eqsymb}{\approx}
\newcommand{\setbuild}[2]{\left\lbrace #1\colon #2 \right\rbrace}
\newcommand{\set}[1]{\left\lbrace #1\right\rbrace}
\newcommand{\dual}[1]{{#1}^\ast}
\newcommand{\up}[1]{\left\uparrow#1\right.}
\newcommand{\down}[1]{\left\downarrow#1\right.}
\newcommand{\powerset}[1]{\mathcal{P}(#1)}
\newcommand{\family}[2]{(#1)_{#2}}
\newcommand{\domain}[1]{\partial^{#1}}
\newcommand{\del}{\partial}
\newcommand{\card}[1]{\left\lvert#1\right\rvert}
\newcommand{\subsets}[2]{\left[#2\right]^{#1}}
\newcommand{\nequiv}[1]{\equiv^{#1}}
\newcommand{\eq}[1]{{#1}^{\mathrm{eq}}}
\newcommand{\embed}{\preceq}
\newcommand{\elmsub}{\preccurlyeq}
\newcommand{\tto}{\twoheadrightarrow}
\newcommand{\dcl}[1]{{#1}^\mathrm{D}}
\newcommand{\forward}{\noindent$\Rightarrow$:\quad}
\newcommand{\backward}{\noindent$\Leftarrow$:\quad}
\newcommand{\cha}[2]{\llbracket#1\rrbracket^{#2}}
\newcommand{\godel}[1]{\lfloor #1\rfloor}
\newcommand{\pto}{\to}
\newcommand{\inv}[1]{{#1}^{-1}}
\newcommand{\gen}[2]{\langle #1\rangle_{#2}}
\newcommand{\Left}{\mathcal{L}}
\newcommand{\Right}{\mathcal{R}}
\newcommand{\pprime}{{\prime\prime}}
\newcommand{\A}[1]{\bm{\forall}_{#1}}
\newcommand{\E}[1]{\bm{\exists}_{#1}}
\newcommand{\length}{\ell}

% splittings
\newcommand{\fsplit}[1][]{\prescript{#1}{}{\pi}_\mathrm{F}}

% roman font operators
\DeclareMathOperator{\Int}{Int}
\DeclareMathOperator{\cf}{cf}
\DeclareMathOperator{\range}{ran}
\DeclareMathOperator{\con}{Con}
\DeclareMathOperator{\defcon}{Con_d}
\DeclareMathOperator{\trclos}{Tr}
\DeclareMathOperator{\Th}{Th}
\DeclareMathOperator{\Mod}{Mod}
\DeclareMathOperator{\admis}{Admis}
\DeclareMathOperator{\kernel}{ker}
\DeclareMathOperator{\EF}{EF}
\DeclareMathOperator{\diag}{diag}
\DeclareMathOperator{\diagp}{{diag}^{+}}

% italic font operators
\newcommand{\rank}{\mathit{rank}}
\newcommand{\vdrank}{\mathit{rank}_\mathcal{VD}}
\newcommand{\qrank}{\mathit{qr}}
\newcommand{\id}{\mathit{id}}
\newcommand{\height}{\mathit{h}}

% calligraphic operators
\newcommand{\comp}{\mathcal{C}}
\newcommand{\level}[1]{\mathcal{L}_{#1}}

% AMS-Latex customisation
\newcommand{\noqed}{\renewcommand{\qedsymbol}{}}

% define glossary entries
\loadglsentries{symbols}
\makeglossaries

% Bibliography style
\bibliographystyle{amsalpha}


% glossary entries
\glsadd{domain}
\glsadd{cofinality}
\glsadd{ksubsets}
\glsadd{positive integers}
\glsadd{reals}
\glsadd{rationals}
\glsadd{naturals}
\glsadd{initial ordinals}
\glsadd{range}


% build only the specified chapters
% \includeonly{prelim/pre, scattered/scatter, mono/mono}

\begin{document}
   	% \maketitle

   	% frontmatter
	\bibstyle{amsalpha}

\chapter{Introduction}


\section{Linear orders and coloured expansions}


\section{Basic operations on linear orders}


\section{Lattices}


\section{Model theory and logic}


\section{Decidability}

	\chapter{Preliminaries}


\section{Model theory and logic}

Our primitive (logical) symbols are $\bot$, $\rightarrow$, $\wedge$ and $\exists$ with their usual semantics.  We will also make use of the following symbols to abbreviate certain formulas in the way one would normally expect in (classical) first-order logic: $\top$, $\neg$, $\vee$ and $\forall$.  Since there is more than one way to do this

On occasion we might encounter infinite conjunctions and disjuctions by employing the (respective) symbols $\bigwedge$ and $\bigvee$, with or without subscripts.  These do not occur in the language of first-order logic as such but rather in its infinitary extensions $L_{\kappa\omega}$ with $\kappa$ an initial ordinal and $L$ ranging over all possible signatures.  For a fixed signature $L$ the notation $L_{\infty\omega}$ is also encountered in the literature to denote the (smallest) language containg $L_{\kappa\omega}$, for each initial ordinal $\kappa$.  An example of a minimalistic but infinitary extension of a first-order language would be the language $L_{\omega_1\omega}$ which also allows formulas to have denumerable conjunctions and disjunctions such as in $\bigvee_{i<\omega}\bigwedge_{j<\omega}\varphi_{ij}$.

\begin{conv}[Signatures and languages]
	We will make a habit of identifying the language $L_{\omega\omega}$ (the first-order language over the signature $L$) with its underlying signature as these uniquely determine each other.
\end{conv}

\begin{conv}
	We indentify any language $L_{\kappa\omega}$ with its corresponding set of formulas, allowing us to make assertions such as ``$\varphi\in L$'' in the case $L=L_{\omega\omega}$.
\end{conv}

\begin{dfn}[Finitary languages]
	A language $L$ is called \textbf{finitary} if $L$ is a language over a finite signature.
\end{dfn}

\begin{prp}\label{prp:finequiv}
	Suppose we fix any $n\in\nats$ and $L$ is a finitary first order language. If $v_0,\dotsc,v_{k-1}$ are distinct variables then, up to logical equivalence, there are only finitely many formulas $\varphi=\varphi(v_1,\dotsc,v_{k-1})$ such that $\qrank(\varphi)\leq n$.
\end{prp}

\begin{dfn}[Characteristic formulas]
	If $L$ is some finitary language, $\mathfrak{M}$ is some $L$-structure and $\bar{a}\in\domain{k}{\mathfrak{M}}$ then we define the \textbf{$\mathbf{n}$-characterstic formula} $\cha{\bar{a}}{n}$ of $\bar{a}$ relative to $\mathfrak{M}$, recursively, as follows:
	\begin{enumerate}
		\item $\cha{\bar{a}}{0}=\bigwedge\setbuild{\varphi(\bar{x})\in L_k}{\mathfrak{M}\models\varphi(\bar{a})}$;
		\item $\cha{\bar{a}}{n+1}=\bigwedge_{b\in\domain{}\mathfrak{M}}\exists v_n \cha{\bar{a}b}{n}\wedge\forall v_n\bigvee_{b\in\domain{}\mathfrak{M}}\cha{\bar{a}b}{n}$, for each $n\in\nats$.
	\end{enumerate}
	If $\bar{a}$ is the empty tuple, and $n\in\nats$, then we write $\cha{\mathfrak{M}}{n}$ for $\cha{\bar{a}}{n}$ and call $\cha{\mathfrak{M}}{n}$ the \textit{$\mathit{n}$-characteristic sentence} of the structure $\mathfrak{M}$.
\end{dfn}

\begin{prp}
	If $L_{\omega\omega}$ is a finitary language, $\mathfrak{M}$ is an $L$-structure and $n\in\nats$ then the $L_{\omega_1\omega}$-sentence $\cha{\mathfrak{M}}{n}$ is logically equivalent to an $L_{\omega\omega}$-sentence $\sigma$.
\end{prp}
\begin{proof}
	From proposition \ref{prp:finequiv} it follows that there exists sentences $\sigma_0,\dotsc,\sigma_{k-1}$, of quantifier rank at most $n$, such that for every sentence $\sigma\in\Th(\mathfrak{M})$ it holds that $\qrank(\sigma)\leq n$ implies that $\sigma$ is logically equivalent to exactly one of $\sigma_0,\dotsc,\sigma_{k-1}$.  Without loss of generality we may assume that $\mathfrak{M}\models\sigma_i$ for $i=0,\dotsc,k-1$ as we could otherwise simply discard sentences false in $\mathfrak{M}$. By definition of a characteristic sentence, it then follows that $\cha{\mathfrak{M}}{n}$ is logically equivalent to $\sigma=\bigwedge_{0\leq i<k_0}\sigma_i$, which is the required sentence.
\end{proof}

In light of the previous proposition, when $L$ is a finitary first-order language and $n\in\nats$, for a class $\mathcal{S}$ of $L$-structures we will identify the $L_{\omega_1\omega}$-sentences $\bigvee_{\alpha\in\mathcal{S}}\cha{\alpha}{n}$ and $\bigwedge_{\alpha\in\mathcal{S}}\cha{\alpha}{n}$ with any of their respective first-order equivalents.

\begin{lem}\label{lem:diagH}
	Suppose $\mathfrak{A}$ and $\mathfrak{B}$ are $L$-structures.  If $\bar{c}$ is a (possibly infinite) tuple of constant symbols and $\bar{a}\in A^{\ell(\bar{c})}$ and $\bar{b}\in B^{\ell(\bar{c})}$ then
	\begin{enumerate}
		\item 	for every atomic $\sigma\in L_0(\bar{c})$, if $(\mathfrak{A},\bar{a})\models\sigma$ then $(\mathfrak{B},\bar{b})\models\sigma$,
		\item  there exists a homomorphism $h\colon\gen{\bar{a}}{\mathfrak{A}}\to\mathfrak{B}$ such that $f[\bar{a}]=f[\bar{b}]$.
	\end{enumerate}
	Additionally, if $h$ exists then it is unique and, furthermore, $h$ is an embedding iff $(\mathfrak{A},\bar{a})\nequiv{0}(\mathfrak{B},\bar{b})$.
\end{lem}

\begin{lem}[Positive Diagram Lemma]\label{lem:diagp}
	Suppose $\mathfrak{A}$ and $\mathfrak{B}$ are $L$-structures.  If $f$ is a map $f\colon A\to B$ then the following are equivalent:
	\begin{enumerate}
		\item	$h$ is a homomorphism $h\colon\mathfrak{A}\to\mathfrak{B}$,
		\item	for each $n\in\nats$ and every atomic formula $\varphi\in L_n$ it holds that, whenever $\bar{a}\in A^n$,
			\begin{equation}
				\mathfrak{A}\models\varphi(\bar{a}) \quad\implies\quad \mathfrak{B}\models\varphi(h[\bar{a}])
			\end{equation}
		\item	it holds that $(\mathfrak{B},h[A])\models\diagp\mathfrak{A}$.
	\end{enumerate}
\end{lem}

\begin{lem}[Diagram Lemma]\label{lem:diag}
	Suppose $\mathfrak{A}$ and $\mathfrak{B}$ are $L$-structures.  If $f$ is a map $f\colon A\to B$ then the following are equivalent:
	\begin{enumerate}
		\item	$h$ is an embedding $h\colon\mathfrak{A}\hookrightarrow\mathfrak{B}$,
		\item	for each $n\in\nats$ and every quantifier-free formula $\varphi\in L_n$ it holds that, whenever $\bar{a}\in A^n$,
			\begin{equation}
				\mathfrak{A}\models\varphi(\bar{a}) \quad\iff\quad \mathfrak{B}\models\varphi(h[\bar{a}])
			\end{equation}
		\item	it holds that that $(\mathfrak{B},h[A])\models\diag\mathfrak{A}$.
	\end{enumerate}
\end{lem}

\begin{dfn}[Quantifier elimination]
	An $L$-theory $T$ is said to \textbf{admit quantifier elimination} (or \textit{has} quantifier elimination) if for every $k\in\nats$ and every $\varphi\in L_k$ there exists a quantifier-free formula $\psi\in L_k$ such that $\varphi$ is $T$-equivalent (or equivalent modulo $T$) to $\psi$ i.e.\ $T\models\forall\bar{x}(\varphi(\bar{x})\leftrightarrow\psi(\bar{x}))$.
\end{dfn}

\begin{dfn}[Substructure-completeness]
	A first-order theory $T$, in some language $L$, is said to be \textbf{substructure-complete} whenever for every model $\mathfrak{M}$ of $T$ and every substructure $\mathfrak{A}$ of $\mathfrak{M}$ it necessarily holds that $T\cup\diag(\mathfrak{A})$ is a complete $L(A)$-theory.
\end{dfn}

\begin{dfn}[Literal]
	A formula $\varphi\in L$ is called a \textbf{literal} whenever it is either an atomic $L$-formula or, otherwise, the negation of such an atomic formula.
\end{dfn}

\begin{dfn}[Simply primitive]
	A formula $\varphi(\bar{x})\in L_n$, for some $n\in\posnats$, is said to be \textbf{simply primitive} whenever there exists literals $\varphi_0(\bar{x},y),\dotsc,\varphi_{n-1}(\bar{x},y)\in L_n$ such that $\varphi(\bar{x})=\exists y\bigwedge_{0\leq i<n}\varphi_i(\bar{x},y)$.  We denote the class of simply primitive formulas as $\E{}^\ast$.
\end{dfn}

\begin{prp}\label{prp:sprm}
	An $L$-theory $T$ admits quantifier elimination iff every simply primitive $L$-formula is $T$-equivalent to a quantifier-free formula.
\end{prp}
\begin{proof}
	\forward Trivial.

	\backward We argue on the complexity of the formula $\varphi\in L$ that it is $T$-equivalent to a quantifier-free formula.  If $\varphi$ is atomic then it trivially follows that $\varphi$ is quantifier-free.  Note $\bot$ is quantifier-free as it is atomic and suppose now that $\varphi_0$ and $\varphi_1$ are, respectively, $T$-equivalent to quantifier-free formulas $\psi_0,\psi_1\in L$.  It follows  by definition that the formulas $\varphi_0\rightarrow\varphi_1$ and $\varphi_0\wedge\varphi_1$ are $T$-equivalent to the respective quantifier-free formulas $\psi_0\rightarrow\psi_1$ and $\psi_0\wedge\psi_1$.  Lastly, note that $\exists y\varphi_0$, by the induction hypothesis, is $T$-equivalent to the simply primitive formula $\exists y\psi_0$.  As simply primitive $L$-formulas (by assumption) are $T$-equivalent to quantifier-free formulas, we may conclude that $\exists y\psi_0$, and thus $\exists y\varphi_0$, is $T$-equivalent to a quantifier-free formula.
\end{proof}

\begin{dfn}[Boolean closure]
	If $\Phi\subseteq L$ is a set of $L$-formulas then its \textbf{boolean closure} (denoted as $\widetilde{\Phi}$) is the smallest set of $L$-formulas such that $\widetilde{\Phi}\supseteq\Phi\cup\set{\bot,\top}$ and is closed under negation, conjunction and disjunction i.e.\ if $\varphi,\psi\in\widetilde{\Phi}$ then it must follow that $\set{\neg\varphi, \varphi\wedge\psi,\varphi\vee\psi}\subseteq\widetilde{\Phi}$.
\end{dfn}

\begin{rem}
	Note in the definition of a boolean closure that the condition that $\widetilde{\Phi}$ be closed under disjunctions is redundant.
\end{rem}

\begin{lem}\label{lem:bcls}
	Let $T$ be an $L$-theory and fix some $\sigma\in L_0$.  If $\Sigma$ is an $L$-theory then $\sigma$ is $T$-equivalent to a sentence from $\widetilde{\Sigma}$ iff it holds for every $\mathfrak{M},\mathfrak{N}\models T$ that:
	\begin{equation}
		\mathfrak{M}\equiv_\Sigma\mathfrak{N} \quad\implies\quad \mathfrak{M}\equiv_\sigma\mathfrak{N}
	\end{equation}
\end{lem}
\begin{proof}
	\forward  Note that no generality is lost by assuming that $\sigma\in\widetilde{\Sigma}$.  As the case $\sigma\in\set{\bot,\top}$ is trivial, suppose not.  We argue by induction on the complexity of $\sigma$.  If $\sigma\in\Sigma$ then it follows by definition that $\mathfrak{M}\equiv_\Sigma\mathfrak{N}$.  Assume now that $\sigma_0,\sigma_1\in\widetilde{\Sigma}$ and either $\sigma=\neg\sigma_0$, $\sigma=\sigma_0\wedge\sigma_1$, where (for $i=0,1$) it holds, for $\mathfrak{M},\mathfrak{N}\models T$, that
	\begin{equation}
		\mathfrak{M}\equiv_\Sigma\mathfrak{N} \quad\implies\quad \mathfrak{M}\equiv_{\sigma_i}\mathfrak{N}.
	\end{equation}
	In both cases it necessarily then follows from the usual definition of Tarski semantics that, for $\mathfrak{M},\mathfrak{N}\models T$, we must have
	\begin{equation}
		\mathfrak{M}\equiv_\Sigma\mathfrak{N} \quad\implies\quad \mathfrak{M}\equiv_\sigma\mathfrak{N}.
	\end{equation}

	\backward  Consider the set $S_L$ of complete $L$-theories, under its usual Stone topology, and define the sets $S=\gen{\sigma}{}\cap\bigcap_{\sigma^\prime\in T}\gen{\sigma^\prime}{}$ and $S^\prime=\gen{\neg\sigma}{}\cap\bigcap_{\sigma^\prime\in T}\gen{\sigma^\prime}{}$.  Note, if $T\cup\set{\sigma}$ is inconsistent then it follows that $T\models\neg\sigma$ and thus $\sigma$ is $T$-equivalent to $\bot$ so that the result follows trivially.  Similarly,  if $T\cup\set{\neg\Sigma}$ is inconsistent then $\sigma$ is $T$-equivalent to $\top$ and the result again follows trivially.  Now assume both $T\cup\set{\sigma}$ and $T\cup\set{\neg\sigma}$ are consistent so that $S\neq\emptyset\neq S^\prime$.

	Note now that, for every $T_0\in S$ and $T_1\in S^\prime$, it follows our assumption (by contraposition) that if $\mathfrak{M}_0\models T_0$ and $\mathfrak{M}_1\models T_1$ then there exists a $\sigma_{T_0,T_1}\in\Sigma$ such that $\mathfrak{M}_0\models\sigma_{T_0,T_1}$ but $\mathfrak{M}_1\models\neg\sigma_{T_0,T_1}$.  Note that, for every $T_0\in S$, the set $\setbuild{\gen{\neg\sigma_{T_0,T^\prime}}{}}{{T^\prime\in S^\prime}}$ is an open cover of $S^\prime$.  Since $S^\prime$ is an intersection of closed sets, it now follows that $S^\prime$ itself is closed and therefore must be compact as it is a closed subset of the compact topological space $S_L$.  By definition of compactness, there must then exist $T^\prime_0,\dotsc,T^\prime_{k-1}\in S^\prime$ (for some $k\in\posnats$) such that the set $\set{\gen{\neg\sigma_{T_0,T^\prime_0}}{},\dotsc,\gen{\neg\sigma_{T_0,T^\prime_{k-1}}}{}}$ is an open cover of $S^\prime$.  We may now conclude that
	\begin{equation}
		T\cup\set{\neg\sigma}\models\bigvee_{i<k}\neg\sigma_{T_0,T^\prime_i},
	\end{equation}
	and therefore
	\begin{equation}
		T\models\neg\sigma\rightarrow\bigvee_{i<k}\neg\sigma_{T_0,T^\prime_i}.\label{eq:tmnsr}
	\end{equation}
	Now, for each $T_0\in S$, define $\sigma_{T_0}=\bigwedge_{i<k}\sigma_{T_0,T^\prime_i}$ and note that $\sigma_{T_0}\in\widetilde{\Sigma}$ as it is a conjunction of sentences from $\Sigma\subseteq\widetilde{\Sigma}$.  Note also, by contraposition, that
	\begin{equation}
		T\models\sigma_{T_0}\rightarrow\sigma.
	\end{equation}
	Note now for every $T_0\in S$, and each natural $i<k$, that we necessarily have $T_0\models\sigma_{T_0,T_i^\prime}$.  Consequently $T_0\models\sigma_{T_0}$, for each $T_0\in S$, and therefore the set $\setbuild{\gen{\sigma_{T_0}}{}}{T_0\in S}$ is an open cover of $S$.  Since $S$ is a closed, and therefore compact, subset of $S_L$ it follows that there exists a finite subcover $\set{\gen{\sigma_{T^\pprime_0}}{},\dotsc,\gen{\sigma_{T^\pprime_{m-1}}}{}}$ of the aforementioned open cover.  Consequently,
	\begin{equation}
		T\cup\set{\sigma}\models\bigvee_{i<m}\sigma_{T^\pprime_i}.
	\end{equation}
	We may now assume, without loss of generality, it holds that $T\cup\set{\sigma}\models\sigma_{T^\pprime_0}$ and therefore
	\begin{equation}
		T\models\sigma\rightarrow\sigma_{T^\pprime_0}.
	\end{equation}
	Since it has already been determined that $T\models\sigma_{T_0^\pprime}\rightarrow\sigma$ and it holds that $\sigma_{T^\pprime_0}\in\widetilde{\Sigma}$, the result then follows.
\end{proof}

\begin{prp}\label{prp:bcls}
	Let $T$ be an $L$-theory and, for any $n\in\nats$, fix some $\varphi\in L_n$.  If $\Delta\subseteq L_n$ then $\varphi$ is $T$-equivalent to a formula from $\widetilde{\Delta}$ iff the following holds for every $\mathfrak{M},\mathfrak{N}\models T$ and each $\bar{a}\in M^n$ and $\bar{b}\in N^n$:
	\begin{equation}
		 \mathfrak{M}\models\delta(\bar{a})\text{ iff }\mathfrak{N}\models\delta(\bar{b})\text{, for all $\delta\in\Delta$,} \implies \mathfrak{M}\models\varphi(\bar{a})\text{ iff }\mathfrak{N}\models\varphi(\bar{b}).
	\end{equation}
\end{prp}
\begin{proof}
	Let $n\in\nats$ and suppose that $\varphi=\varphi(\bar{x})\in L_n$ and $\Delta=\Delta(\bar{x})\subseteq L_n$.  Consider now a tuple $\bar{c}$ of constant symbols which is of length $n$.  After convincing oneself that $\widetilde{\Delta(\bar{c})}=\widetilde{\Delta}(\bar{c})$, that is to say that the operation of taking a boolean closure of a set of formulas and that of substition by constant symbols commute, it follows from lemma \ref{lem:bcls} that $\varphi(\bar{c})$ is equivalent to a sentence from $\widetilde{\Delta}(\bar{c})$ iff
	\begin{equation}
		\mathfrak{M}^\prime\equiv_{\Delta(\bar{c})}\mathfrak{N}^\prime \quad\implies\quad \mathfrak{M}^\prime\equiv_{\varphi(\bar{c})}\mathfrak{N}^\prime,
	\end{equation}
	for all $L(\bar{c})$-expansions $\mathfrak{M}^\prime$ and $\mathfrak{N}^\prime$ of the respective $L$-structures $\mathfrak{M},\mathfrak{N}\models T$.  Redressing this equivalence purely in terms of $L$-formulas and $L$-structures (by taking their $L$-reducts) then yields the desired result.
\end{proof}

\begin{dfn}
	Recursively define the following classes of $L_{\infty\omega}$ formulas, with $L$ ranging over all signatures:
	\begin{enumerate}
		\item	$\E{0}=\A{0}$ is the class of formulas $\varphi$ such that $\varphi$ is a quantifier-free formula in some first-order language $L$,
		\item	if $n\in\nats$, $\E{n+1}\supseteq\A{n}$ is defined to the smallest class of formulas, closed under $\bigwedge$ and $\bigvee$, such that $\exists\bar{x}\varphi\in\E{n+1}$ whenever $\bar{x}$ is finite tuple of variables and $\varphi\in\E{n+1}$,
		\item	if $n\in\nats$, $\A{n+1}\supseteq\E{n}$ is defined to be the smallest class of formulas, closed under $\bigwedge$ and $\bigvee$, such that $\forall\bar{x}\varphi\in\A{n+1}$ whenever $\bar{x}$ is finite tuple of variables and $\varphi\in\A{n+1}$.
	\end{enumerate}
	In particular, members of $\E{1}$ are known as the \textbf{existential formulas} while the members of $\A{1}$ are referred to as \textbf{universal formulas}.
\end{dfn}

\begin{lem}\label{lem:qelim}
	The following are equivalent for a consistent $L$-theory $T$:
	\begin{enumerate}
		\item	$T$ admits quantifier elimination,
		\item	$T$ is a substructure-complete theory,
		\item	if $\mathfrak{A}$ is finitely generated, $f\colon\mathfrak{A}\to\mathfrak{M}$ and $g\colon\mathfrak{A}\to\mathfrak{N}$ are embeddings and $\mathfrak{M},\mathfrak{N}\models T$ then, for every $n\in\nats$ and each simply primitive $\varphi\in L_n(A)$, it holds for every $\bar{a}\in A^n$ that
		\begin{equation}
			\mathfrak{M}\models\varphi(f[\bar{a}]) \quad\iff\quad \mathfrak{N}\models\varphi(g[\bar{a}]).\label{eq:qelim}
		\end{equation}
	\end{enumerate}
\end{lem}
\begin{proof}
	1$\Rightarrow$2:  Choose any $\mathfrak{M}\models T$ and any $\mathfrak{A}\subseteq\mathfrak{M}$.  Note, since $T$ has quantifier elimination, it follows from the diagram lemma (choosing $h$ as the inclusion map) that
	\begin{equation}
		\Th(\mathfrak{M},A)\subseteq T\cup\diag\mathfrak{A}.
	\end{equation}
	Since the former is complete it immediately follows that the latter is complete as well.

	2$\Rightarrow$3:  Suppose $\mathfrak{A}$ is a finitely generated structure and let $f\colon\mathfrak{A}\hookrightarrow\mathfrak{M}$ and $g\colon\mathfrak{A}\hookrightarrow\mathfrak{N}$ be embeddings into the respective models $\mathfrak{M}$ and $\mathfrak{N}$ of $T$.  Without loss of generality we may assume $f$ and $g$ are inclusion maps and therefore $\mathfrak{A}\subseteq\mathfrak{M},\mathfrak{N}$.  By assumption the $L(A)$-theory $T\cup\diag\mathfrak{A}$ is complete.  Therefore, since $(\mathfrak{M},A)\models T\cup\diag\mathfrak{A}$ and $(\mathfrak{N},A)\models T\cup\diag\mathfrak{A}$, it follows that $(\mathfrak{M},A)\equiv(\mathfrak{N},A)$.  By definition, this is equivalent to (\ref{eq:qelim}).

	3$\Rightarrow$1:  By Proposition \ref{prp:sprm}, it is enough to show that every simply primitive formula is $T$-equivalent to a quantifier-free formula.  In turn, Proposition \ref{prp:bcls} reduces this to showing, for each $n\in\nats$, that if $\bar{c}$ is an $n$-tuple of constant symbols and $\mathfrak{M},\mathfrak{N}\models T$ are $L$-structures then all of their (respective) $L(\bar{c})$-expansions $\mathfrak{M}^\prime$ and $\mathfrak{N}^\prime$ satisfy
	\begin{equation}
		\mathfrak{M}^\prime\equiv_{\A{0}}\mathfrak{N}^\prime \quad\implies\quad \mathfrak{M}^\prime\equiv_{\E{}^\ast}\mathfrak{N}^\prime.
	\end{equation}

	Let $\mathfrak{M},\mathfrak{N}\models T$ be $L$-structures and choose, for an arbitrary $n\in\nats$, any $\bar{a}\in\domain{n}\mathfrak{M}$ and any $\bar{b}\in\domain{n}\mathfrak{N}$ such that
	\begin{equation}
		(\mathfrak{M},\bar{a})\equiv_{\A{0}}(\mathfrak{N},\bar{b}).
	\end{equation}
	Define the structures $\mathfrak{A}=\gen{\bar{a}}{\mathfrak{M}}$ and $\mathfrak{B}=\gen{\bar{b}}{\mathfrak{N}}$ and note, by definition, that both of them are finitely generated.  From lemma \ref{lem:diagH}, it follows that there exists an embedding $h\colon\mathfrak{A}\to\mathfrak{N}$ such that $h[\bar{a}]=h[\bar{b}]$.  Since $\mathfrak{B}$ is generated by $\bar{b}=h[\bar{a}]$, we may consider $h$ to be an isomorphism $h\colon\mathfrak{A}\to\mathfrak{B}$.  Therefore, we may identify the structures $\mathfrak{A}$ and $\mathfrak{B}$ so that the desired conlusion then follows from (3) by choosing $f$ and $g$ to be the respective inclusion maps $f\colon\mathfrak{A}\hookrightarrow\mathfrak{M}$ and $g\colon\mathfrak{A}\hookrightarrow\mathfrak{N}$.
\end{proof}

\begin{dfn}[Model-completeness]
	An $L$-theory $T$ is said to be \textbf{model-complete} if every embedding $f\colon\mathfrak{M}\to\mathfrak{N}$, between models $\mathfrak{M}$ and $\mathfrak{N}$ of $T$, is elementary.
\end{dfn}

\begin{prp}
	The following are equivalent for an $L$-theory $T$:
	\begin{enumerate}
		\item	$T$ is model-complete,
		\item	for all $\mathfrak{N}\models T$, if $\mathfrak{M}\subseteq\mathfrak{N}$ and $\mathfrak{M}\models T$ then $T\cup\diag(\mathfrak{M})$ is a complete $L(M)$-theory.
	\end{enumerate}
\end{prp}
\begin{proof}
	1$\Rightarrow$2: Suppose $\mathfrak{M},\mathfrak{N}\models T$ and $\mathfrak{M}\subseteq\mathfrak{N}$.  Note now that for any $L(M)$-structure $\mathfrak{A}^\prime\models T\cup\diag(\mathfrak{M})$ if $\mathfrak{A}$ is the $L$-reduct of $\mathfrak{A}^\prime$ then there exists an embedding $h\colon\mathfrak{M}\hookrightarrow\mathfrak{A}$ such that $\mathfrak{A}^\prime=(\mathfrak{A},h[M])$.  From (1) it follows that $h$ is elementary and thus $(\mathfrak{M},M)\equiv\mathfrak{A}^\prime$.  Since $\mathfrak{A}^\prime$ was arbitrary it follows that any two models of $T\cup\diag(\mathfrak{M})$ will be elementarily equivalent to $(\mathfrak{M},M)$ and thus to each other.

	2$\Rightarrow$1:  Suppose $\mathfrak{M},\mathfrak{N}\models T$ and let $f\colon\mathfrak{M}\hookrightarrow\mathfrak{N}$ be an embedding.  We are now required to show that $f$ is elementary.  By the Elementary Diagram Lemma this amounts to showing that
	\begin{equation}\label{eq:mc2}
		(\mathfrak{N},f[M])\models\Th(\mathfrak{M},M).
	\end{equation}
	Noting that, since $f$ is an embedding, it follows from the Diagram Lemma that $(\mathfrak{N},f[M])\models T\cup\diag{\mathfrak{M}}$.  Also, since $(\mathfrak{M},M)\models T\cup\diag{\mathfrak{M}}$ and $T\cup\diag{\mathfrak{M}}$ is complete it follows that the deductive closure of $T\cup\diag{\mathfrak{M}}$ is the complete theory $\Th(\mathfrak{M},M)$.  As a consequence of these observations, (\ref{eq:mc2}) follows immediately.
\end{proof}

\begin{cor}\label{cor:qemc}
	If $T$ is a first-order theory that admits quantifier elmination then $T$ is model-complete.
\end{cor}
\begin{proof}
	Follows immediately from the definition of model-completeness and the implication (2)$\Rightarrow$(1) in the previous proposition.
\end{proof}

\begin{dfn}[$n$-Type]
	Let $\mathfrak{A}$ be an $L$-structure and $n\in\nats$ while supposing, for some $A_0\subseteq A$, that $\Phi=\Phi(\bar{x})\subseteq L_n(A_0)$ is deductively closed and closed under finite conjunctions.  We call $\Phi$ an $\mathbf{n}$\textbf{-type of }$\bm{\mathfrak{A}}$\textbf{ over }$\bm{A_0}$ iff there exists some $\mathfrak{B}\succcurlyeq\mathfrak{A}$ and a $\bar{b}\in B^n$ such that $\mathfrak{B}\models\Phi(\bar{b})$ i.e.\ $\mathfrak{B}\models\varphi(\bar{b})$, for every $\varphi\in\Phi(\bar{x})$.  Additionally, if $A_0=\emptyset$ then $\Phi$ is also referred to as an $\mathbf{n}$\textbf{-type of the complete theory }$T=\Th(\mathfrak{A})$.
\end{dfn}

\begin{dfn}[Realisation]
	A tuple $\bar{b}$ of length $n$ in a structure $\mathfrak{B}\succcurlyeq\mathfrak{A}$ is said to \textbf{realise} a type $\Phi(\bar{x})\subseteq L_n(A)$ of a structure $\mathfrak{A}$ iff $\mathfrak{B}\models\Phi(\bar{b})$.
\end{dfn}

\begin{dfn}[Isolated and principal types]
	An $n$-type $\Phi(\bar{x})\subseteq L_n(A)$ of an $L$-structure $\mathfrak{A}$ is called \textbf{isolated} whenever there exists a formula $\varphi_0\in L_n(A)$, satisfiable in $\mathfrak{A}$, such that $(\mathfrak{A},A)\models\forall\bar{x}(\varphi_0(\bar{x})\rightarrow\varphi(\bar{x}))$, for each $\varphi\in\Phi(\bar{x})$.  If $\varphi_0$ can be chosen such that $\varphi_0\in\Phi(\bar{x})$ then $\Phi$ is also said to be \textbf{principal}.
\end{dfn}

\begin{dfn}[Saturation]
	Given any infinite cardinal $\kappa$, an $L$-structure $\mathfrak{A}$ is called $\bm{\kappa}$\textbf{-saturated} whenever $\kappa\leq\card{\mathfrak{A}}$ and each $1$-type of $\mathfrak{A}$, over any $A_0\subseteq A$ such that $\card{A_0}<\kappa$, is realised by some element in $\mathfrak{A}$.  We call $\mathfrak{A}$ \textbf{saturated} iff it is $\card{\mathfrak{A}}$-saturated.
\end{dfn}

\begin{prp}\label{prp:sat0}
	If $\mathfrak{A}$ is an $\aleph_0$-saturated $L$-structure then, for each $n\in\posnats$, $\mathfrak{A}$ realises all $n$-types over subsets of finite cardinality.
\end{prp}
\begin{proof}
	We argue by induction on $n$.  The case $n=1$ follows immediately from the definition of $\aleph_0$-saturation so assume for every $k\leq n$ it holds that $\mathfrak{A}$ realises all $k$-types of $\mathfrak{A}$ over any of its finite subsets.  Let $\Phi(\bar{x})\subseteq L_{n+1}(A_0)$ be an $(n+1)$-type of $\mathfrak{A}$, for some finite $A_0\subseteq A$.  By definition, there must exist an elementary extension $\mathfrak{B}\succcurlyeq\mathfrak{A}$ and some $\bar{b}\in B^{n+1}$ such that $\mathfrak{B}\models\Phi(\bar{b})$.  Now define $\Phi^\prime(x_n)\subseteq L_1(A_0)$ to be the deductive closure of the set
	\begin{equation}
		\setbuild{\exists x_0\dotso\exists x_{n-1}\varphi(x_0,\dotsc,x_n)}{\varphi\in\Phi}.
	\end{equation}

	Supposing that $\bar{b}=(b_0,\dotsc,b_n)$, note that $\mathfrak{B}\models\Phi^\prime(b_n)$ and thus $\Phi^\prime$ is a $1$-type of $\mathfrak{A}$ over $A_0$ and consequently, since $\mathfrak{A}$ is $\aleph_0$-saturated, there exists an $a_n\in A$ such that $\mathfrak{A}\models\Phi^\prime(a_n)$.  Aiming to show that $\Phi(x_0,\dotsc,x_{n-1},a_n)$ is an $n$-type of $\mathfrak{A}$ over $A_1\coloneqq A_0\cup\set{a_n}$, let $\bar{c}$ be an $n$-tuple of constant symbols.  Note that $\Phi(x_0,\dotsc,x_{n-1},a_n)$ is an $n$-type of $\mathfrak{A}$ iff $\Phi(\bar{c},a_n)\cup\Th(\mathfrak{A})$ is a consistent $L(A_1)$-theory.  Also, since $\Phi$ is closed under conjunctions, if $\psi=\psi(x_0,\dotsc,x_n)$ is a finite conjunction of formulas from $\Phi$ then $\psi\in\Phi(\bar{x})$ and thus $\exists x_0\dotso\exists x_{n-1}\psi\in\Phi^\prime(x_n)$.  Therefore, since $\psi$ was arbitrary and $\mathfrak{A}\models\Phi^\prime(a_n)$, it follows that $T\coloneqq\Th(\mathfrak{A})\cup\Phi(\bar{c},a_n)$ is a finitely satisfiable $L(A_1)$-theory so that, by the compactness theorem, $T$ must have an $L(A_1\cup\bar{c})$-model $(\mathfrak{M},\bar{d}\string^a_n)$, for some $L(A_0)$-structure $\mathfrak{M}$ and some $\bar{d}\in M^n$.  By definition it follows that $\mathfrak{M}\succcurlyeq\mathfrak{A}$ is an $L(A_0)$-structure such that $(\mathfrak{M},a_n)\models\Phi(\bar{d}\string^a_n)$.  Therefore, $\Phi(x_0,\dotsc,x_{n-1},a_n)$ is an $n$-type over $A_1$ which, by the induction hypothesis, is realised by some $(a_0,\dotsc,a_{n-1})\in A^n$ and thus if $\bar{a}=(a_0,\dotsc,a_{n-1},a_n)$ then $\mathfrak{A}\models\Phi(\bar{a})$.
\end{proof}

\begin{prp}
	If $\alpha\models\Th(\dense)$
\end{prp}

\begin{prp}
	For any ordinal $\alpha$, if $\mathfrak{A}$ is an $\aleph_\alpha$-saturated $L$-structure then, for each $n\in\posnats$ and any ordinal $\beta<\alpha$, $\mathfrak{A}$ realises all $n$-types over subsets of cardinality $\aleph_\beta$.
\end{prp}
\begin{proof}
	The case $\alpha=0$ is given by Proposition \ref{prp:sat0}.  Consider now the case $\alpha>0$ and note that if $A_0\subseteq A$ and $\card{A_0}=\aleph_\beta$, for some ordinal $\beta<\alpha$, then it follows that $\card{A_0\cup\set{a}}=\card{A_0}=\aleph_\beta$.  By leveraging this property of infinite cardinals, an induction argument similar to that in the proof of Proposition \ref{prp:sat0} yields the desired result.
\end{proof}

\begin{prp}
	If $\Phi(\bar{x})\subseteq L_n(A)$ is an isolated type of the nonempty $L$-structure $\mathfrak{A}$ then there exists an $\bar{a}\in A^n$ which realises $\Phi$.
\end{prp}
\begin{proof}
	Suppose $\Phi(\bar{x})$ is an $n$-type of $\mathfrak{A}$ over $A$, for some $n\in\posnats$.  Since $\Phi$ is isolated, there exists a $\varphi_0\in L_n(A)$ such that $\varphi_0(\bar{x})$ is satisfiable in $\mathfrak{A}$ and $\mathfrak{A}\models\forall\bar{x}(\varphi_0(\bar{x})\rightarrow\varphi(\bar{x}))$ for each $\varphi\in\Phi(\bar{x})$.  By definition, there must exist some $\bar{a}\in A^n$ such that $\mathfrak{A}\models\varphi_0(\bar{a})$.  Consequently, it follows from the definition of $\varphi_0$ that $\mathfrak{A}\models\Phi(\bar{a})$, as required.
\end{proof}

\begin{dfn}[Atomicity]
	An $L$-structure $\mathfrak{A}$ is called \textbf{atomic} whenever the only types over $\emptyset$ which are realised in $\mathfrak{A}$ are isolated.
\end{dfn}


\section{Games on structures}

In what follows we will describe, for some arbitrary ordinal $\gamma$, the \textbf{Ehrenfeucht-Fra\"iss\'e game} $\EF_\gamma(\mathfrak{A},\mathfrak{B})$ of length $\gamma$ on $L$-structures $\mathfrak{A}$ and $\mathfrak{B}$.  In this game there are two players $\Left$ and $\Right$ (respectively pronounced ``Left'' and ``Right'').  This first move of the game is always made by $\Left$ and consists of some choice of element from either one of the structures $\mathfrak{A}$ or $\mathfrak{B}$.  The players alternate turns and on each of  $\Right$'s turns he is obliged to choose an element from the opposing structure i.e.\ from $\mathfrak{B}$ if $\Left$'s last move was an element in $\mathfrak{A}$ or vice-versa.

\begin{dfn}[Position]
	A pair $(\bar{a},\bar{b})$, consisting of transfinite sequences $\bar{a}=\family{a_i}{i<\gamma}$ and $\bar{b}=\family{b_i}{i<\gamma}$, is called a \textbf{position} of the game $\EF_\gamma(\mathfrak{A},\mathfrak{B})$ whenever, for each $i<\gamma$, $a_i\in\domain{}\mathfrak{A}$ and $b_i\in\domain{}\mathfrak{B}$ are the elements played during the $i$-th round of the game $\EF_\gamma(\mathfrak{A},\mathfrak{B})$.  We refer to $\gamma$ as the \textit{length} of the position.
\end{dfn}

\begin{dfn}[Play]
	A \textbf{play} in the game $\EF_\gamma(\mathfrak{A},\mathfrak{B})$ is any maximal position in the game.
\end{dfn}

\begin{dfn}[Winning play]
	A position $(\bar{a},\bar{b})$ of the game $\EF_\gamma(\mathfrak{A},\mathfrak{B})$ is said to be \textbf{a winning play for} $\bm{\mathcal{R}}$ iff there exists some isomorphism $f\colon\gen{\bar{a}}{\mathfrak{A}}\to\gen{\bar{b}}{\mathfrak{B}}$ such that $f[\bar{a}]=f[\bar{b}]$.
\end{dfn}

\begin{dfn}[Back-and-forth equivalence]
	We define $\mathfrak{A}\sim_\gamma\mathfrak{B}$ for an ordinal $\gamma$ if there exists a winning strategy for $\Right$ in the game $\EF_\gamma(\mathfrak{A},\mathfrak{B})$.  For the special case $\gamma=\omega$: if $\mathfrak{A}\sim_\omega\mathfrak{B}$ then it is said that $\mathfrak{A}$ and $\mathfrak{B}$ are \textbf{back-and-forth equivalent}.
\end{dfn}

\begin{dfn}[Winning position]
	In the back-and-forth game $\EF_\omega(\mathfrak{A},\mathfrak{B})$ a position $(\bar{c},\bar{d})$ of length $n\in\nats$ is called a $\textbf{winning position}$ for a particular player if whenever that player can win the $\EF_\gamma(\mathfrak{A},\mathfrak{B})$ when starting from position $(\bar{c},\bar{d})$.
\end{dfn}

\begin{prp}\label{prp:bfiso}
	Suppose $\mathfrak{A}$ and $\mathfrak{B}$ are countable $L$-structures, for some language $L$, and let $(\bar{c},\bar{d})$ be a winning position for $\Right$ in $\EF_\omega(\mathfrak{A},\mathfrak{B})$.  If $\mathfrak{A}\sim_\omega\mathfrak{B}$ then there exists an isomorphism $f\colon\mathfrak{A}\to\mathfrak{B}$ such that $f[\bar{c}]=f[\bar{d}]$ i.e.\ an isomorphism $f\colon(\mathfrak{A},\bar{c})\to(\mathfrak{B},\bar{d})$.
\end{prp}
\begin{proof}
	Let $\bar{a}=\family{a_i}{i<\omega}$ and $\bar{a}=\family{a_i}{i<\omega}$, respectively, be enumerations of the structures $\mathfrak{A}$ and $\mathfrak{B}$.  In what follows we describe the remainder of the game $\EF_\omega(\mathfrak{A},\mathfrak{B}$.  On every odd round $\Left$ plays the first element occuring in $\bar{a}$ which he has not played at any previous position of the game and does not occur in $\bar{c}$.  Similarly, on every even round $\Left$ plays the first element occuring in $\bar{b}$ which he has not played at a previous position of the game and does not occur in $\bar{d}$.  Naturally, $\Right$ simply responds using his winning strategy on every round.

	Note, given any element from either one of the two structures, that $\Left$ will play that element after finitely many rounds.  Hence all the elements from both $\mathfrak{A}$ and $\mathfrak{B}$ make an appearance at some point in the game.  By definition of a winning play, there then exists an isomorphism $f\colon\gen{\bar{c}\string^\bar{a}}{\mathfrak{A}}\to\gen{\bar{d}\string^\bar{b}}{\mathfrak{B}}$ such that $f[\bar{c}\string^\bar{a}]=f[\bar{d}\string^\bar{b}]$ and hence $f[\bar{c}]=f[\bar{d}]$.  Since $\gen{\bar{a}}{\mathfrak{A}}=\mathfrak{A}$ and $\gen{\bar{b}}{\mathfrak{B}}=\mathfrak{B}$, the result follows as intended.
\end{proof}


\section{Interpretations}

\begin{dfn}[(Un)nested term]
	If $L$ is any language and $t$ is an $L$-term then $t$ is said to be \textbf{unnested} whenever $t$ one of the following holds:
	\begin{enumerate}
		\item	$t$ is a variable $x$,
		\item	$t$ is a constant symbol $c$,
		\item	or, for some $n\in\posnats$, there exists an $n$-ary function symbol $f$ in $L$ and an $n$-tuple $\bar{x}$ of variables such that $t=t(\bar{x})$ is of the form $f(\bar{x})$.
	\end{enumerate}
	A term which is not unnested is (unsurprisingly) referred to as being \textbf{nested}.
\end{dfn}

\begin{dfn}[Unnested atomic formula]
	An \textit{atomic} $L$-formula $\varphi$ is said to be \textbf{unnested} provided that $\varphi$ is either of the form $s=t$, for some unnested $L$-terms $s$ and $t$, or there exists a $n\in\posnats$, an $n$-ary relation symbol $R$ and an $n$-tuple $\bar{x}$ of variables such that $\varphi=\varphi(\bar{x})$ is $R(\bar{x})$.

\end{dfn}

\begin{dfn}[(Un)nested formula]
	An arbitrary formula $\varphi$ is called \textbf{unnested} whenever each of its atomic subformulas is unnested.  A \textbf{nested formula} is then simply a formula which is not unnested.
\end{dfn}

\begin{dfn}[Substitutability of terms]
	Suppose $\varphi=\varphi(\bar{x},y)$ is an $L$-formula and $t=t(\bar{v})$ is any $L$-term such that $\bar{v}=(v_0,\dotsc,v_k)$.  We say that $t$ is \textbf{substitutable} for $y$ in $\varphi$ iff no instance of a variable $v_i$ in $t$ (for $i<k$) is bound by a quantifier in $\varphi(\bar{x},t/y)$.  In other words, no free instance of $y$ in $\varphi(\bar{x},y)$ lies in the scope of some $\exists v_i$ (hence also $\forall v_i$) such that $i<k$.
\end{dfn}

\begin{dfn}[Substitutability]
	Suppose $\varphi=\varphi(\bar{x},\bar{y})$ is an $L$-formula and, for some $n\in\nats$, $\bar{y}$ has length $n$ and $\bar{t}$ is an $n$-tuple of terms.  We say that $\bar{t}$ is \textbf{substitutable} for $\bar{y}$ in $\varphi$ whenever, for each $i<n$, $t_i$ is substitutable for $y_i$ in $\varphi$.
\end{dfn}

\begin{dfn}[Interpretation]
	Suppose $L$ and $L^\prime$ are first-order languages, fix an enumeration $\family{x_i}{i<\omega}$ of the set of variables $V$ and, for each $i<\omega$, let $\bar{x}_i$ denote the $n$-tuple $\left(x_i,\dotsc,x_{i+(n-1)}\right)$.  If $\Gamma\colon L\to L^\prime$ is a map and $\del_\Gamma\coloneqq\Gamma(\exists x_1(x_0=x_1))$ then $\Gamma$ is called an $\mathbf{n}$\textbf{-dimensional interpretation} of $L$ in $L^\prime$ whenever the following holds:
	\begin{enumerate}
		\item	for every $k\in\nats$, if $\varphi=\varphi(x_0,\dotsc,x_{k-1})$ then
			\begin{equation}
				\Gamma(\varphi)=\Gamma(\varphi)(\bar{x}_0,\dotsc,\bar{x}_{k-1}),
			\end{equation}
			i.e\ the free variables of $\Gamma(\varphi)$ are among the components of $\bar{x}_0,\dotsc,\bar{x}_{k-1}$,
		\item	$\Gamma(\bot)=\bot$,
		\item	$\Gamma(\varphi\wedge\psi)=\Gamma(\varphi)\wedge\Gamma(\psi)$,
		\item	$\Gamma(\varphi\rightarrow\psi)=\Gamma(\varphi)\rightarrow\Gamma(\psi)$,
		\item	if $k\in\nats$ and $\varphi=\varphi(x_0,\dotsc,x_{k+1})$ then it necessarily follows that $\Gamma(\exists x_{k+1}\varphi)$ is the formula
			\begin{equation}
				\exists\bar{x}_{k+1}\left(\del_\Gamma(\bar{x}_{k+1}/\bar{x}_0)\wedge\Gamma(\varphi)\right),
			\end{equation}
		\item	for each $n\in\nats$ and every $\bar{x},\bar{x}^\prime\in V^n$: if $\varphi=\varphi(\bar{x})$, $\bar{x}^\prime$ is substitutable for $\bar{x}$ in $\varphi$ and $\psi=\varphi(\bar{x}^\prime/\bar{x})$ then it follows that $\Gamma(\psi)$ is is obtainable from $\Gamma(\varphi)$ by a change of variables.
	\end{enumerate}
\end{dfn}

It should be empasised that the first clause in the preceding definition merely exists for convenience: simplifying the task of keeping track of free variables accross interpretations.  The final clause on the other hand is to guarantee that formulas which differ only in their choice of free variables remain as such when taking their respecitve images under $\Gamma$.  The remaining clauses server to preserve the syntactic structure of formulas under $\Gamma$.  Those familiar with algebraic logic will recognise that $\Gamma$, in an obvious manner, gives rise to homomorphism of cylindrical algebras.  For the rest of us, $\Gamma$ merely acts as a ``translator'' of formulas in $L$ to formulas in $L^\prime$.

A model theorist, however, is far more concerned with theories and models than formal languages for their own sake.  Fortunately, interpretations can serve a purpose that the model theorist would find pleasing.  We refer to the ability of reducing truth of sentences in the presence of a theory to truth of their interpretation in a more appropriate or better understood theory.

\begin{dfn}[Interpration of theories]
	Suppose $L$ and $L^\prime$ are first-order languages.  If $T$ is an $L$-theory and $T^\prime$ is an $L^\prime$-theory then an \textbf{interpretation of }$\bm{T}$\textbf{ in }$\bm{T^\prime}$ is an interpretation $\Gamma$ of $L$ in $L^\prime$ such that, for every sentence $\sigma$ in $L$,
	\begin{equation}
		T\models\sigma\quad\implies\quad T^\prime\models\Gamma(\sigma).
	\end{equation}
\end{dfn}

\begin{prp}
	Suppose $T$ is an $L$-theory and $T^\prime$ is an $L^\prime$-theory and let $\Gamma$ be an interpretation of $T$ in $T^\prime$.  If $T$ is complete then, for every $L$-sentence $\sigma$,
	\begin{equation}
		T^\prime\models\Gamma(\sigma)\quad\implies\quad T\models\sigma
	\end{equation}
\end{prp}
\begin{proof}
	Assume to the contrary that $T^\prime\not\models\Gamma(\sigma)$ but $T\models\sigma$.  Since $T$ is complete it immediately follows that $T\models\neg\sigma$ and thus, by definition of $\Gamma$, we may conclude that $T^\prime\models\Gamma(\sigma)$ --- which is the desired contradiction.
\end{proof}

\begin{dfn}[Interpretation of structures]
	Let $L$ and $L^\prime$ be first-order languages and suppose that $\mathfrak{A}$ is an $L$-structure and $\mathfrak{B}$ is an $L^\prime$-structure.  If $\Gamma$ is an $n$-dimensional ($n\in\nats$) interpretation of $L^\prime$ in $L$ and $f\colon\del_\Gamma(A^n)\to B$ is a surjection then we call the pair $(\Gamma,f)$ an \textbf{interpration of }$\bm{\mathfrak{B}}$\textbf{ in }$\bm{\mathfrak{A}}$ whenever it holds for every unnested atomic formula $\varphi\in L^\prime_k$ and each $\bar{a}_i\in A^n$ ($i<k$) that
	\begin{equation}
		\mathfrak{B}\models\varphi(f(\bar{a}_0),\dotsc,f(\bar{a}_{k-1}))\quad\iff\quad\mathfrak{A}\models\Gamma(\varphi)(\bar{a}_0,\dotsc,\bar{a}_{k-1}).
	\end{equation}
\end{dfn}

Having introduced a third concept sharing the common name of ``interpration'' we would like them all be related in some manner.  As the connection between the first two should be clear we now turn our attention towards interpretation of structures.

\begin{prp}\label{prp:unnest}
	If $L$ is a first-order language then, for each atomic $\varphi\in L$, there exists an unnested $\E{1}$-formula $\varphi^\exists$ and an unnested $\A{1}$-formula $\varphi^\forall$ such that $\varphi$ is logically equivalent to $\varphi^\exists$ as well as $\varphi^\forall$.
\end{prp}
\begin{proof}
	Suppose $\varphi$ is atomic and argue by induction on the maximum number of nested function symbols in $\varphi$.  The base case (i.e.\ when no function symbols occur in $\varphi$) follows from the fact that atomic formulas are simultaneously also $\A{1}$-formulas and $\E{1}$-formulas.

	Suppose, for some $n\in\nats$, the result holds whenever the maximum number of nested function symbols in $\varphi$ is $k<n$.  Suppose now that $\varphi$ has at most $n$ nested occurences of any function symbol in $L$.
	\begin{case}
		$\varphi$ is $s=t$ for some $L$-terms $s$ and $t$.
	\end{case}
	\begin{proof}
		Consider the case where $s$ is of the form $f(s_0,\dotsc,s_{m-1})$, for $L$-terms $s_0,\dotsc,s_{m-1}$, and $t$ is the variable $x$.  Let $\bar{y}$ be an $m$-tuple of variables then it follows that $\varphi$ is logically equivalent to the $\E{1}$-formula
		\begin{equation}
			\exists\bar{y}\left(\left(\bigwedge_{i<m}y_i=s_i\right)\wedge f(\bar{y})=x\right),
		\end{equation}
		which has at most $n$ nested occurences of any function symbol belonging to $L$.  In a similar fashion $\varphi$ is also logically equivalent to the $\A{1}$-formula
		\begin{equation}
			\forall\bar{y}\left(\left(\bigwedge_{i<m}y_i=s_i\right)\rightarrow f(\bar{y})=x\right),
		\end{equation}
		which also has at most $n$ nested occurences of any function symbol.  One may now apply the induction hypothesis and the remaining cases will follow similarly.
	\end{proof}

	\begin{case}
		There exists a relation symbol $r$ in $L$, having arity $m\in\nats$, and $L$-terms $t_0,\dotsc,t_{m-1}$ such that $\varphi$ is $r(t_0,\dotsc,t_{m-1})$.
	\end{case}
	\begin{proof}
		Let $\bar{y}$ be an $m$-tuple of variables.  The formula $\varphi$ must necessarily be logically equivalent to the $\E{1}$-formula
		\begin{equation}
			\exists\bar{y}\left(\left(\bigwedge_{i<m}y_i=t_i\right)\wedge r(y_0,\dotsc,y_{m-1})\right),
		\end{equation}
		as well as the $\A{1}$-formula
		\begin{equation}
			\forall\bar{y}\left(\left(\bigwedge_{i<m}y_i=t_i\right)\rightarrow r(y_0,\dotsc,y_{m-1})\right),
		\end{equation}
		both of which can have no more than $n$ nested occurences of any function symbol in $L$.  Now applying the induction hypothesis yields the desired result.
	\end{proof}
	The result has thus been proven for all atomic $\varphi$ belonging to $L$, concluding the proof.
\end{proof}

\begin{cor}\label{cor:unnest}
	If $\varphi$ is any $L$-formula then there exists an unnested $L$-formula $\psi$ which is logically equivalent to $\varphi$.
\end{cor}
\begin{proof}
	Leveraging Proposition \ref{prp:unnest}, we may replace all unnested atomic subformulas of $\varphi$ with unnested eqvuivalents in order to obtain a new formula $\varphi^\prime$.  As no atomic subformula of $\varphi^\prime$ is nested, and $\varphi^\prime$ is certainly logically equivalent to $\varphi$, the result follows.
\end{proof}

\begin{prp}\label{prp:strint}
	If $(\Gamma,f)$ is an $n$-dimensional interpretation of the $L^\prime$-structure $\mathfrak{B}$ in the $L$-structure $\mathfrak{A}$ and $k\in\nats$ then, for every $\varphi\in L^\prime_k$ and each $\bar{a}_0,\dotsc,\bar{a}_{k-1}\in A^n$, it follows that
	\begin{equation}
		\mathfrak{B}\models\varphi(f(\bar{a}_0),\dotsc,f(\bar{a}_{k-1}))\quad\iff\quad\mathfrak{A}\models\Gamma(\varphi)(\bar{a}_0,\dotsc,\bar{a}_{k-1})\label{eq:interp}
	\end{equation}
\end{prp}
\begin{proof}
	Observe that a formula is unnested iff each of its subformulas are unnested.  Additionally, \ref{cor:unnest} tells us we need only concern ourselves with unnested formulas.  Employing these facts, we now argue by induction on the complexity of formulas that \ref{eq:interp} holds whenever $\varphi$ is unnested.

	The atomic case is trivial as per the definition of an interpretation of structures.  Suppose now that $\varphi,\psi\in L$ satisfy \ref{eq:interp}.  Note that we may assume $\varphi,\psi\in L_k$, for some $k\in\nats$:  if $y$ is any variable which has an occurence in $\psi$ but not in $\varphi=\varphi(\bar{x})$ then $\varphi$ is logically equivalent to $\varphi(\bar{x})\wedge y=y$.  If $\varphi\wedge\psi$ is unnested then it follows that both $\varphi$ and $\psi$ are (respectively) unnested and, thus, from the inductive hypothesis and the Tarski definition of $\models$ it must hold, for every $\bar{a}_0,\dotsc,\bar{a}_k\in A^n$, that
	\begin{equation}
		\mathfrak{B}\models(\varphi\wedge\psi)(f(\bar{a}_0),\dotsc,f(\bar{a}_k))\quad\iff\quad\mathfrak{A}\models\Gamma(\varphi\wedge\psi)(\bar{a}_0,\dotsc,\bar{a}_k).
	\end{equation}
	Similarly, it can be shown that if $\varphi\rightarrow\psi$ is unnested then
	\begin{equation}
		\mathfrak{B}\models(\varphi\rightarrow\psi)(f(\bar{a}_0),\dotsc,f(\bar{a}_k))\quad\iff\quad\mathfrak{A}\models\Gamma(\varphi\rightarrow\psi)(\bar{a}_0,\dotsc,\bar{a}_k).
	\end{equation}
	 Suppose now that $\varphi=\varphi(\bar{x},y)$, for some $(k-1)$-tuple of variables $\bar{x}$, and that $\exists y\varphi(\bar{x},y)$ is unnested then it follows that $\varphi$ is unnested.  If $\bar{b}\in B^{k-1}$ and $\mathfrak{B}\models\exists y\varphi(\bar{x},y)$ then there exists a $d\in B$ such that
	 \begin{equation}
		 \mathfrak{B}\models\varphi(\bar{b},d).
	 \end{equation}
	 Since $f$ is a surjection there exists $\bar{a}_0,\dotsc,\bar{a}_{k-2},\bar{c}\in A^n$ such that $f(\bar{c})=d$ and $f(\bar{a}_i)=b_i$, for $i<n$.  From the definition of $\varphi$ it then follows that
	 \begin{equation}
		 \mathfrak{A}\models\Gamma(\varphi)(\bar{a}_0,\dotsc,\bar{a}_{k-2},\bar{c}),
	 \end{equation}
	 and thus we may conclude
	 \begin{equation}
		 \mathfrak{A}\models\exists\bar{y}\Gamma(\varphi)(\bar{a}_0,\dotsc,\bar{a}_{k-2},\bar{y}).\label{eq:exint}
	 \end{equation}
	 The converse is even simpler:  start from (\ref{eq:exint}) and take the image under $f$ of the initial tuples in $A^n$ as well as the tuple acting in the role of $\bar{c}$.
\end{proof}

\begin{cor}
	If $(\Gamma,f)$ is an interpretation of the structure $\mathfrak{B}$ in the structure $\mathfrak{A}$ then it follows that $\Gamma$ is an interpretation of $\Th(\mathfrak{B})$ in $\Th(\mathfrak{A})$.
\end{cor}
\begin{proof}
	Obtained as a special case from Proposition \ref{prp:strint} by considering sentences only.
\end{proof}

\begin{dfn}[Definitional expansion]
	Suppose $\mathfrak{A}$ is an $L$-structure and let $L^\prime\supseteq L$.  We call an $L^\prime$-structure $\mathfrak{A}^\prime$ a \textbf{definitional expansion} of $\mathfrak{A}$ whenever the following holds:
	\begin{enumerate}
		\item	$A=A^\prime$,
		\item	for each constant symbold $c$ belonging to $L^\prime\setminus L$, there exists an $L$-formula $\varphi\in L_1$ such that, for every $a\in A$,
			\begin{equation}
				\mathfrak{A}\models\varphi(a)\quad\iff\quad a=c^{\mathfrak{A}^\prime},
			\end{equation}
		\item	for each $n\in\nats$ and every $(n+1)$-ary function symbol $f$ belonging to $L^\prime\setminus L$ there exists an $L$-formula $\varphi=\varphi(\bar{x},y)$ such that $\varphi\in L_{n+1}$ and, for every $\bar{a}\in A^n$ and $b\in A$,
			\begin{equation}
				\mathfrak{A}\models\varphi(\bar{a},b)\quad\iff\quad f^{\mathfrak{A}^\prime}(\bar{a})=b,
			\end{equation}
		\item	for each $n\in\posnats$ and every $n$-ary relation symbol $r$ belonging to $L^\prime\setminus L$ there exists an $L$-formula $\varphi=\varphi(\bar{x})$ such that $\varphi\in L_n$ and, for each $\bar{a}\in A^n$,
			\begin{equation}
				\mathfrak{A}\models r(\bar{a})\quad\iff\quad r^{\mathfrak{A}^\prime}(\bar{a}).
			\end{equation}
	\end{enumerate}
\end{dfn}

\begin{dfn}[Recursively enumerable theory]
	If $L$ is a countable first-order language and $T$ is an $L$-theory then we refer to $T$ as being \textbf{recursively enumerable} whenever $\dcl{T}$ is a recursively enumerable set.
\end{dfn}

\begin{dfn}[Regular expansion]
	Suppose $\mathfrak{A}$ is an $L$-structure and let $L^\prime\supseteq L$.  If the signature $L^\prime$ is obtained from $L$ by adding finitely many additional relation symbols and the $L^\prime$-structure $\mathfrak{A}^\prime$ satisfies $A=A^\prime$ then $\mathfrak{A}^\prime$ is said to be a \textbf{regular expansion} of $\mathfrak{A}$.
\end{dfn}

\begin{dfn}[Proof sequence]
	Suppose $L$ is a finitary first-order language and let $T$ be a recursively enumerable $L$-theory.  If $\mathcal{A}$ is a recursively enumerable class of regular expansions of models of $T$ then a \textbf{proof sequence} over $\mathcal{A}$ is a finite sequence of pairs $(\mathfrak{M}_0,\sigma_0),\dotsc,(\mathfrak{M}_{n-1},\sigma_{n-1})$ such that, for each $i<n$, $\mathfrak{M}_i\in\mathcal{A}$ and one of the following holds:
	\begin{enumerate}
		\item	$\sigma_i$ is a logical axiom,
		\item	$\sigma_i\in T$,
		\item	$\sigma_i$ follows via an inference rule from $\setbuild{\sigma_j}{j<i}$,
		\item	$i>0$ and, for some $j<i$, there exists an interpretation $\Gamma$ of $\mathfrak{M}_j$ in $\mathfrak{M}_i$ such that $\sigma_i=\Gamma(\sigma_j)$.
	\end{enumerate}
\end{dfn}


\section{Linear orders and coloured expansions}

Note that, from this section onwards, $L$ will exclusively denote the \textit{language of partial orders} i.e.\ the language whose signature consists of only the binary relation symbol $<$.


\begin{dfn}[Dense linear order]
	A linear order $\alpha$ is called \textbf{dense} whenever, for every $a,b\in\alpha$, there exists $a\in\alpha$ such that $a<c<b$.  Additionally, $\alpha$ is said to be \textit{trivially dense} if the order type of alpha is either $\zero$ or $\one$.
\end{dfn}

\begin{dfn}[Coloured linear order]
	If $\alpha$ is a linear order and $\bar{r}$ is a (finite) tuple $(r_0,\dotsc,r_{k-1})$ of unary relations on $\alpha$ then the expansion $(\alpha,\bar{r})$ is said to be a $\mathbf{k}$\textbf{-coloured linear order}.
\end{dfn}

\begin{dfn}[Spectrum]
	Suppose $\mathcal{A}$ is a class of $k$-coloured linear orders.  We call $F$ an $\mathbf{n}$\textbf{-spectrum} for the class $\mathcal{A}$ when $F$ is a minimal set of linear orders with the property that for every $\alpha\in\mathcal{A}$ there exists a $\chi\in F$ such that $\alpha\nequiv{n}\chi$.
\end{dfn}


\section{Basic operations on linear orders}

\begin{lem}\label{lem:fvsum}
	Suppose $\family{\alpha_i}{i\in I}$ and $\family{\beta}{i\in I}$ are families of (possibly coloured) linear orders, indexed by a linearly ordered set $I\neq\emptyset$.  Now fix some $n\in\nats$ and assume $\alpha_i\nequiv{n}\beta_i$, for each $i\in J$, then it follows that
	\begin{equation}
		\sum_{i\in I}\alpha_i\nequiv{n}\sum_{i\in I}\beta_i.
	\end{equation}
\end{lem}

\begin{prp}
	If $\beta\models\Th(\omega)$ then there exists a linear order $\alpha$ such that $\beta=\omega+\zeta\cdot\alpha$.
\end{prp}

\begin{prp}\label{prp:omega}
	The first-order theory $\Th(\omega)$ is finitely axiomatisable.
\end{prp}
\begin{proof}
	Let $T$ denote the first-order theory consisting of only the sentences:
	\begin{enumerate}
		\item	$\axmlin$,
		\item	$\exists x\forall y(x\neq y\rightarrow x<y)$,
		\item	$\forall x\exists y(x<y\wedge\neg\exists z(x<z<y))$,
		\item	$\forall x\big(\exists y(y<x)\rightarrow\exists y(y<x\wedge\neg\exists z(y<z<x))\big)$.
	\end{enumerate}
	Together these express (in order) that if $\alpha$ is a model of $T$ then:
	\begin{enumerate}
		\item	$\alpha$ is a linear order,
		\item	$\alpha$ has a least element,
		\item	every element of $\alpha$ has an immediate successor,
		\item	every non-minimal element of $\alpha$ has an immediate predecessor.
	\end{enumerate}
\end{proof}


\section{Lattices}

\begin{thm}[Knaster-Tarski Theorem]
	Suppose $\Lambda$ is a complete lattices.  If $h$ is some endomorphism on $\Lambda$ then there must exist a (least) $x_0\in \Lambda$ such that $x_0$ is a fix0ed point of $h$, i.e.\ $h(x_0)=x_0$.
\end{thm}
\begin{proof}
	We start by exhibiting a fixed point of $h$.  Since $\Lambda$ is nonempty there exists a top $1\in\Lambda$, since by completeness, taking the join of $\Lambda$ we get an element $1=\bigvee\Lambda$.  Since $h$ is isotone we may conclude that $h(1)=1$.

	We are now obligated to find a least fixed point of $h$ so define
	\begin{equation}
		S=\setbuild{x\in\Lambda}{h(x)=x}
	\end{equation}  then clearly $S\neq\emptyset$ since $1\in S$.  Define $x_0=\bigwedge S$ and note that, since $x_0\leq x$ for each $x\in S$, by definition of isotonicity it follows that $h(x_0)\leq x$, for $x\in S$. Therefore $h(x_0)\leq\bigwedge S=x_0$.  Consequently, since $h(x_0)$ is a lower bound of $S$.  Since $x_0$ is the infimum of $S$ we cannot have $h(x_0)<x_0$.  Thus the only remaining possibility is $h(x_0)=x_0$, implying that $x\in S$.
\end{proof}


\section{Categories and functors}


\section{Galois connections}



\section{Decidability}

	\begin{prp}
		If $\Sigma$ is a set of $L$-sentences then $\Sigma$ is decidable set of sentences iff $\Sigma$ and $L_0\setminus\Sigma$, each, are recursively enumerable.
	\end{prp}

	\begin{prp}\label{prp:sdth}
		Suppose $\mathcal{A}$ is a class of structures.  If $\Sigma$ is a recursively enumerable set of sentences and its deductive closure $\dcl{\Sigma}$ satisfies
		\begin{equation}
			\dcl{\Sigma}=\Th(\mathcal{A}),
		\end{equation}
		then $\Th(\mathcal{A})$ is recursively enumerable.
	\end{prp}

	\begin{dfn}[Encoding]
		An \textbf{encoding} $p$ of a countable set $X$ is an injective recursive partial function $p\colon X\pto\nats$.  If $X$ is a first order language then $p$ is in stead referred to as a \textit{G\"odel numbering}.
	\end{dfn}

	\begin{dfn}[Frameworks]
		Suppose for some $A\subseteq X$, where $X$ is of (coloured) linear orders, there exists a family $\family{\Gamma_{alpha,\beta}}{(\alpha,\beta)\in A\times X}$
		\begin{enumerate}
			\item	$p\colon X\to\nats$ is an encoding of the (countable) set of coloured linear orders,
			\item 	the binary relation $\preceq$ is a well-founded preorder defined on $X$,
			\item	the map $\rho\colon X\to\nats$ is the rank function associated with $(X,\preceq)$.  If, for each $x,y\in X$, it holds that
		\end{enumerate}
	\end{dfn}

	\begin{dfn}[Proof sequences]
		Suppose $X$ is a countable set of countable coloured linear orders and let $p\colon X\pto\nats$ be an encoding of $X$.  A \textbf{proof sequence} is then a finite sequence
		\begin{equation}
			(\alpha_0,\sigma_0),\dotsc,(\alpha_{n-1},\sigma_{n-1})\in X\times L_0
		\end{equation}
		and finite expansions $\alpha^\prime_0,\dotsc,\alpha^\prime_{n-1}$ of the (respective) linear orders $\alpha_0,\dotsc,\alpha_{n-1}$ so that, for each $i=1,\dotsc,n-1$, there exists an interpretation $\Gamma_i$ of $\alpha_{i-1}$ in $\alpha^\prime_i$.  Additionally, when $0\leq i<n$, we require that one of the following holds:
		\begin{enumerate}
			\item	$\sigma_i$ is a logical axiom,
			\item	$\sigma_i$ follows via an inference rule from $\sigma_0,\dotsc,\sigma_{i-1}$,
			\item	$\sigma_i\in\admis(\Gamma_i)$ or
			\item	$\sigma_i=\Gamma_i\Gamma_{i-1}\dotsb\Gamma_j\sigma_j$, for some natural $j<i$.
		\end{enumerate}
	\end{dfn}
\vfill
	\begin{verbatim}
		-------------------------------------------------------------------------
	\end{verbatim}
\noindent[\textbf{Alles in hierdie hoofstuk is tentatief.  Byvoegings en (veral) verwyderings sal gemaak word soos ek deur die tesis vorder}]

	\bibliographystyle{amsalpha}

\chapter{Ramsey theory}

In the area of mathematics known as Ramsey Theory, one studies the relationship
between the cardinality of a specified mathematical structure and the occurence
(or total absence) of certain ``uniformities''.  The archetypical example
involves colourings of the edges of the (complete) graph $K_n$ on $n$ vertices:
how large should $n$ be to guarantee the existence of a particular monochromatic
cycle?

Problems of this flavour can also be formulated for infinite structures and form
the foundation of a body of set theory commonly referred to as combinatorial
set theory.  Many of the problems we will encounter will be of this nature.

We lay out the terminology for and briefly discuss Ramsey's theorem as it
appears in the context of $k$-colourings of $n$ element subsets of $\nats$.
This is followed by an analagous result, formulated by Shelah, which pertains to
$k$-colourings of subintervals of a given (infinite) linear order, irrespective
of the countability of the underlying set.

\section{Classical Ramsey Theorem}

A formalisation is required of the aforementioned notion of ``uniformity''.
This would be the essence of the following definition, formulated in terms of
partitions.  These partitions, of which there will be finitely many, are
intended to represent the ``colours'' of the elements: two elements belong to
the same member of the partition iff they share a colour.

\begin{dfn}[Homogeneous set for a partition]
   If $A$ is a nonempty set and, for some $k\in\posnats$, $\mathcal{C}$ is a
   partition of $\subsets{k}{A}$ then we will call $X\subseteq A$
   \textbf{homogeneous for the partition} $\mathcal{C}$ if there exists a $C\in
   \mathcal{C}$ such that $\subsets{k}{X}\subseteq C$.
\end{dfn}

Rephrased in terms of colours, a set $X$ is homogeneous for a partition whenever
$\subsets{k}{X}$ is a monochromatic subset of $\subsets{k}{\nats}$.  That is to
say all its elements share the same colour.

\begin{thm}[Ramsey's Theorem]
	Suppose $\mathcal{C}$ is a finite partition of $\subsets{k}{X}$ and
	$\card{X}=\aleph_0$, for some $k\in\posnats$, then there exists an infinite
	set $H\subseteq\nats$ such that $H$ is homogeneous for $\mathcal{C}$.
\end{thm}
\begin{proof}
	Note that it is enough to consider the case $X=\nats$ and proceed by
	induction on $k$. If $k=1$ then there must exist an infinite
	$C\in\mathcal{C}$ and thus $H=C$ is homogeneous for $\mathcal{C}$.

	Suppose the result holds for $k\in\posnats$ and let
	$\mathcal{C}=\set{C_0,\dotsc,C_{p-1}}$ be a partition of
	$\subsets{k+1}{\nats}$.  We proceed by recursively defining a sequence
	$a=\family{a_i}{i<\omega}$ of natural numbers and a sequence
	$\family{H_i}{i<\omega}$ of subsets of $\nats$ as follows.  For $i<p$,
	$b\in\nats$ and $S\subseteq\nats$, the set $C_i(S,b)$ is defined by
	\begin{equation}
		C_i(b,S)=\setbuild{A\in\subsets{k}{S\setminus\set{b}}}{A\cup\set{b}\in
		C_i}.
	\end{equation}

	Choose $a_0=0$ and $H_0=\nats\setminus\set{0}$.  Now assume $a_j$ and $H_j$
	have been defined, for some $j<\omega$, and that $H_j$ is infinite.  Define
	$a_{j+1}$ to be the least element of $H_j$ then
	\begin{equation}
		\mathcal{B}_{j+1}=\setbuild{C_i\big(a_{j+1},H_j\big)}{
		i<p}\setminus\set{\emptyset}
	\end{equation}
	is a partition of $\subsets{k}{H_j\setminus\set{a_{j+1}}}$.

	From the inductive hypothesis, there exists an infinite $H_{j+1}\subseteq
	H_j\setminus\set{a_{j+1}}$ such that $H_{j+1}$ is homogeneous for
	$\mathcal{B}_{j+1}$.  This completes the recursive definition so that all that
	remains is to show that the set
	\begin{equation}
		H=\setbuild{a_i}{0<i<\omega}
	\end{equation}
	is homogeneous for $\mathcal{C}$.

	Without loss of generality, we may assume it holds that
	\begin{equation}
		\subsets{k}{H_1}\subseteq C_0(a_1,H_0)
	\end{equation}
	and thus $\subsets{k}{H_1}\subseteq C_0$.  Consequently, it follows by
	definition that $\subsets{k}{H_j}\subseteq C_0$, for all $j>0$.

	By construction, for each $j>0$, all sets which are of the form
	$X\cup\set{a_j}$, for some $X\in\subsets{k+1}{H_{j+1}}$, must necessarily be
	members of $C_0$.  Now, suppose $X\in\subsets{k}{H}$ and let $b$ be the
	least element of $X$.  Define $X_0=X\setminus\set{b}$ and note that there
	exists some $j_0>0$ such that $b=a_{j_0}$.

	Since $a$ is clearly (strictly) increasing, the members of $X_0$ all occur
	in the sequence $a$ at some point after the index $j_0$. Therefore,
	$X_0\in\subsets{k}{H_{j+1}}$ and thus we may conclude that $X\in C_0$.  This
	then proves that $\subsets{k+1}{H}\subseteq C_0$, as required.
\end{proof}

In an appeal to the imagination, a colouring of a countably infinite set always
gives rise to an infinite monochromatic subset.  Thus, it is impossible to
arrange the colours in such a manner that all the monochromatic subsets are
finite.

It would now be prudent to formally define the term \textit{colouring}.  This is
done in terms of surjective maps and will serve in giving us a better (or at
least different) view of certain concepts already introduced.

\begin{dfn}[Colouring]\label{def:Col}
	If $S$ is a nonempty set and $\card{C}=k\in\posnats$ then a
	$\mathbf{k}$\textbf{-colouring} of $S$ is a surjective map $f\colon  S\to
	C$.  We call a surjection $f$ simply a \textbf{colouring} if it is a
	$k$-colouring for some $k$.
\end{dfn}

Note that if $\mathcal{C}$ is some partition of a nonempty set $S$ then there
exists a unique surjection $f\colon S\to \mathcal{C}$ such that $x\in f(x)$, for
each $x\in S$. Since every surjection $f\colon A\to B$ induces a partition of
its domain $A$, by taking $\mathcal{C}=\setbuild{f^{-1}[x]}{x\in B}$, this
justifies the following definition:

\begin{dfn}[Homogeneous set for a colouring]
	Suppose $A$ is a nonempty set and $f\colon A\to C$ is a colouring.  We call
	$X\subseteq A$ \textbf{homogeneous for} $\bm{f}$ whenever there exists a $c\in
	C$ such that $X\subseteq f^{-1}[c]$ or, equivalently, $f(x)=f(y)$ for every
	$x,y\in X$.
\end{dfn}

If $A$ is some linearly ordered set and $k\in\posnats$ we say that a tuple
$\bar{a}\in A^k$ is \textit{increasing} whenever its components form an
increasing sequence i.e.\ it holds that $a_0<\dotsb<a_{k-1}$.

Denote the set of increasing $k$-tuples of $A$ as $A^+(k)$ and let $f\colon
\subsets{k}{A}\to C$ be some arbitrary colouring of $A$.  Defining the map
$\tau_f\colon A^+(k)\to C$ such that
$\tau_f(a_0,\dotsc,a_{k-1})=f(\set{a_0,\dotsc,a_{k-1}})$, for
$(a_0,\dotsc,a_{k-1})\in A^+(k)$, there exists a unique bijection $h\colon
\subsets{k}{A}\to A^+(k)$ such that the diagram
\begin{center}
	\begin{tikzcd}
		\subsets{k}{A}\arrow[r,"h"]\arrow[rd,"f"']	& A^+(k)\arrow[d,"\tau_f"] \\
		& C
	\end{tikzcd}
\end{center}
commutes.  The unconvinced reader need only consider the function $h$ inverse
to:
\begin{equation}
	(a_0,\dotsc,a_{k-1})\mapsto\set{a_0,\dotsc,a_{k-1}}.
\end{equation}
Noting the surjectivity of $\tau_f$ guarantees the uniqueness of $h$,
relative to the commutativity of the diagram above.

\begin{conv}[Colourings]\label{rem:Col}
	In light of these observations, we will henceforth identify the maps $f$ and
	$\tau_f$.  This allows us to write $f(a_0,\dotsc,a_{k-1})$ as
	shorthand for $f(\set{a_0,\dotsc,a_{k-1}})$.

	When $k=2$ we take this a step further by also identifying $f$ with the map
	that sends subintervals $[a,b]$ of $A$, under the condition $a<b$, to the
	colour $\tau_f(a,b)$.
\end{conv}

\begin{rem}[Homogeneity]\label{rem:hom}
	Note that if $f$ is a colouring of $\subsets{k}{S}$, for some nonempty set
	$S$ and some $k\in\posnats$, then it follows by definition that $H\subseteq
	S$ is homogeneous for the partition $\setbuild{f^{-1}[c]}{c\in\range f}$ if
	and only if $\subsets{k}{H}$ is homogeneous for $f$.  This establishes the
	relationship between the two notions of homogeneity.
\end{rem}


\begin{dfn}[Cofinality]
	If $\alpha$ is a linear order then $\cf(\alpha)$, called the
	\textbf{cofinality} of $\alpha$, is the least ordinal $\beta$ with the
	following property: there exists an embedding $f\colon\beta\to\alpha$ such
	that $f[\beta]$ is not bounded above by a non-maximal element of $\alpha$.
\end{dfn}

In clarification of the following definition, we use the phrase
\textit{``transfinite sequence''} to refer to any function whose domain is some
ordinal.  Hence, finite sequences (tuples) may also be considered in this
regard.

\begin{dfn}[Cofinal sequence]
	A \textbf{cofinal sequence} in a linear order $\alpha$ is any (strictly)
	increasing transfinite sequence $(a_\gamma)_{\gamma<\beta}$, where
	$\beta\geq\cf(\alpha)$, such that $\setbuild{a_\gamma}{\gamma<\beta}$ is
	not bounded above by any non-maximal element of $\alpha$.
\end{dfn}

\begin{dfn}[Homogeneous sequence]
	If $\alpha$ is a linear order and $f$ is a colouring of
	$\subsets{2}{\domain\alpha}$ then a \textbf{homogeneous sequence for
	$\bm{f}$} is a cofinal sequence $\family{a_\gamma}{\gamma<\beta}$ in
	$\alpha$ such that
	\begin{equation}
		H=\setbuild{a_\gamma}{\gamma<\beta}
	\end{equation}
	is homogeneous for $f$.
\end{dfn}

\begin{cor}[Existence of homogeneous sequences]\label{cor:Cofinal}
	Suppose $A$ is an infinite linearly ordered set without a greatest element.
	If $A$ has cofinality $\omega$ and $f$ is a colouring of $\subsets{2}{A}$
	then there exists a homogeneous sequence $x=\family{a_i}{i<\omega}$ for $f$
	in $A$.
\end{cor}
\begin{proof}
	Choose $y=(b_i)_{i<\omega}$ to be cofinal in $A$.  Define
	\begin{equation}
		B=\setbuild{b_i}{i<\omega}
	\end{equation}
	and let $g=f\restriction B$.

	As superfluous colours may be discarded, thereby restricting the codomain of
	$g$, we may safely assume that $g$ is surjective and thus a colouring of
	$B$.  It follows from Remark \ref{rem:hom} and Ramsey's Theorem there exists
	an $H\subseteq B$ which is homogeneous for $g$ and, clearly then, also for
	$f$.

	Since $y$ is increasing, there exists a unique subsequence
	$x=(b_{k_i})_{i<\omega}$ of $y$ with image $H$.  By definition $x$ is the
	desired homogeneous sequence for the colouring $f$.
\end{proof}

We conclude this exposition by providing an illustrative example.  The reader is
encouraged to thoroughly internalise its contents.  Similar arguments will be
employed throughout the text and it is, consequently, more significant than
first appearances may suggest.

\begin{exm}[Cofinal sequences in $\lambda$]
	We aim to show that every colouring $f$ of $\reals$ is accompanied by a
	corresponding homogeneous sequence.  Note that $\cf(\lambda)=\omega$ since
	$\lambda$ has no greatest element and $\family{k}{k<\omega}$ is a cofinal
	sequence in $\lambda$.

	Now suppose $f$ is a colouring of $\subsets{2}{\reals}$ and let $x=(x_i)_{i<\omega}$
	be a cofinal sequence in $\lambda$.  Let $X=\setbuild{x_i}{i<\omega}$ and
	define $g=f\restriction \subsets{2}{X}$.

	Similar to the proof of Corollary \ref{cor:Cofinal}, we may assume $g$ is
	surjective and, hence, a colouring of the set $\subsets{2}{X}$.  Since $x$
	is clearly cofinal in $X$, it follows from Corollary \ref{cor:Cofinal} that
	there exists a subsequence $y$ of $x$ which is homogeneous for $g$.  As $y$
	is homogeneous for $g$ and $X$ is a cofinal subset of $\lambda$, it follows
	by definition of $g$ that $y$ is a homogeneous sequence for $f$.
\end{exm}


\section{Additive Ramsey Theorem}

A natural question that arises is whether or not Corollary \ref{cor:Cofinal}
could be extended to linear orders of \textit{uncountable} cofinality.
Unfortunately this is not the case, as illustrated by the following example.

\begin{exm}[Sierpi\'nski colouring]
	Suppose $\alpha=(\reals,\prec)$ is a well-order and assume
	$\cf(\alpha)\geq\omega_1$.  Note that the uncountability of $\reals$
	guarantees the existence of such an $\alpha$ e.g.\ by choosing $\alpha$
	isomorphic to the initial ordinal of cardinality $2^{\aleph_0}$.

	The \textit{Sierpi\'nski colouring} $s\colon\subsets{2}{\reals}\to\set{0,1}$
	is defined, for every $x,y\in\reals$, such that
	\begin{equation}
		s(x,y)=
		\begin{cases}
			1,  &\text{if }x\prec y\text{ and }x<y,\\
			1,  &\text{if }x\nprec y\text{ and }x\not< y,\\
			0,   &\text{otherwise.}
		\end{cases}
	\end{equation}
	That is to say that $s(x,y)=1$ iff $<$ and $\prec$ agree on the order of
	$x$ and $y$.

	Since $\lambda$ is clearly not a well-order it follows that $s$ is
	necessarily a surjection and, thus, a $2$-colouring of
	$\subsets{2}{\reals}$.  By way of contradiction, we assume there exists a
	homogeneous sequence $x=(x_{\gamma})_{\gamma<\beta}$ for $s$ in $\alpha$.

	By definition, $x$ must be either an increasing or (otherwise) decreasing
	sequence in $\lambda$.  Since $\lambda\cong\dual{\lambda}$, we may assume
	without loss of generality that $x$ is increasing.

	It now follows, by assumption, that $\beta\geq\cf(\alpha)\geq\omega_1$.
	However, this implies that there exists a (necessarily uncountable) set
	\begin{equation}
		\setbuild{(x_\gamma,x_{\gamma+1})}{\gamma<\omega_1}
	\end{equation}
	of pairwise disjoint open intervals of $\lambda$, thereby yielding the
	desired contradiction.
\end{exm}

Although our hopes may have been dashed by the late Sierpi\'nski, all hope is
not lost.  Our saving grace turns out to be the following definition:
\begin{dfn}[Additive colouring \cite{ShelahOrder}]
	An \textbf{additive colouring} of a linear order $\alpha$ is a $2$-colouring
	$f$ of $\domain{\alpha}$ such that the equations $f(x_0,y_0)=f(x_1,y_1)$ and
	$f(y_0,z_0)=f(y_1,z_1)$ imply $f(x_0,z_0)=f(x_1,z_1)$.
\end{dfn}

In the context of a $2$-colouring $f$ of the domain of a linear order $\alpha$,
one may interpret the ``joining together'' of subintervals $(a,b)$ and $(b,c)$
to form the interval $(a,c)$ as the process of mixing the colours $f(a,b)$ and
$f(b,c)$ to obtain the colour $f(a,c)$.

In this intuitive framework, one could think of $f$ as being additive precisely
when colours of intervals, under $f$, mix in a consistent fashion.  For example,
if a blue interval and a red interval can be mixed to create a purple interval
then this should be the case for \textit{any} choice of the component intervals,
given that the first one is blue and the second red.

\begin{rem}[Addition of colours]
	As the name might suggest, an \textit{additive} colouring $f$ of a linear
	order $\alpha$ gives rise to a (partial) binary operation $\oplus$ on the
	set of colours (i.e.\ the range of the function $f$).  That is to say one
	defines, in the obvious manner,
	\begin{equation}
		f(a,b)\oplus f(b,c)=f(a,c).
	\end{equation}
\end{rem}

\begin{thm}[Additive Ramsey Theorem \cite{ShelahOrder}]
	If $\delta$ is a limit ordinal, $\beta=\cf(\delta)$ and $f$ is an additive
	colouring of $\delta$ then there exists a homogeneous sequence
	$x=(\alpha_\gamma)_{\gamma<\beta}$ for $f$.
\end{thm}

\begin{proof}
	For every $\alpha,\alpha^\prime<\delta$, define $\alpha\sim\alpha^\prime$ whenever there exists a $\gamma_0<\delta$ such that $\alpha_0,\alpha_1<\gamma_0$ and $f(\alpha,\gamma_0)=f(\alpha^\prime,\gamma_0)$.  We now prove the following claim:
	\begin{claim}
		The binary relation $\sim$ is an equivalence relation
	\end{claim}
	\begin{proof}[Proof of claim.]
		Fix any $\alpha_0,\alpha_1,\alpha_2<\delta$.  Since $f(\alpha_0,\alpha_0+1)=f(\alpha_0,\alpha_0+1)$ it follows that $\sim$ is reflexive.

		If there exists a $\gamma_0>\alpha_0,\alpha_1$ such that $\gamma_0<\delta$ and $f(\alpha_0,\gamma_0)=f(\alpha_1,\gamma_0)$ then $f(\alpha_1,\gamma_0)=f(\alpha_0,\gamma_0)$ so that $\sim$ is symmetric.

		All that remains is to esablish transitivity.
		Suppse that $\gamma_0>\alpha_1,\alpha_0$ and $\gamma_1>\alpha_1,\alpha_2$
		such that $\gamma_0,\gamma_1<\delta$.  Furthermore, assume both the
		equations:
		\begin{align}
			f(\alpha_0,\gamma_0)&=f(\alpha_1,\gamma_0),\\
			f(\alpha_1,\gamma_1)&=f(\alpha_2,\gamma_1).
		\end{align}

		If $\gamma_0=\gamma_1$ then it follows immediately that
		$f(\alpha_0,\gamma_0)=f(\alpha_2,\gamma_0)$ and thus
		$\alpha_0\sim\alpha_2$, so suppose instead that $\gamma_0<\gamma_1$.

		Since, trivially, $f(\gamma_0,\gamma_1)=f(\gamma_0,\gamma_1)$, it
		follows by additivity of $f$ that
		$f(\alpha_0,\gamma_1)=f(\alpha_1,\gamma_1)$ and, therefore,
		$f(\alpha_0,\gamma_1)=f(\alpha_2,\gamma_1)$. so that
		$\alpha_0\sim\alpha_2$, as required.\noqed

		Since the case $\gamma_1<\gamma_0$ is similar, the relation $\sim$ is
		transitive, implying $\sim$ is an equivalence relation.
	\end{proof}

	Note that $\sim$ has at most $\card{\range f}<\aleph_0$ equivalence classes
	and there must exist an equivalence class $C$ under $\sim$ which is
	unbounded in $\delta$.  We will now define a cofinal sequence
	$x=(\alpha_{\gamma})_{\gamma<\beta}$, where $\beta=\cf(\delta)$, by means
	of transfinite recursion.

	Let $\alpha_0$ be the least element of $C$ and define, for each $c\in\range
	f$, the set $I_c=\setbuild{\alpha\in C}{\alpha_0<\alpha\text{ and
	}f(\alpha_0,\alpha)=c}$.  It then follows, by definition, that
	\begin{equation}
		C\setminus\set{\alpha_0}=\bigcup_{c\in\range f}I_c.
	\end{equation}

	Since $C$ is unbounded in $\delta$ and $\card{\range f}<\card{\cf(\delta)}$
	it follows that there exists a $d\in\range f$ such that $I_d$ is unbounded
	in $\delta$.  Note that there exists a cofinal sequence
	$(\delta_\gamma)_{\gamma<\beta}$ in $I_d$.

	Now, continuing the recursion, choose $\alpha_1$ to be the least member of
	$I_d$ and assume $\alpha_\gamma\in I_d$ has been defined for each
	$\gamma<\epsilon$ and some ordinal $\epsilon<\beta=\cf(\delta)$.  We
	suppose, whenever $\gamma<\xi<\epsilon$ and
	$\gamma^\prime<\xi^\prime<\epsilon$, that
	\begin{equation}
		f(\alpha_\gamma,\alpha_\xi)=f(\alpha_{\gamma^\prime},\alpha_{\xi^\prime}).
	\end{equation}
	For each $\gamma<\xi<\epsilon$ define $\sigma_{\gamma,\xi}$ to be the least
	member of $C$ such that
	\begin{equation}\label{eq:addram}
		f(\alpha_\gamma,\sigma_{\gamma,\xi})=f(\alpha_\xi,\sigma_{\gamma,\xi}).
	\end{equation}

	Choose $\alpha_\epsilon$ to be the first $\delta_\gamma$ such that
	$\gamma\geq\epsilon$ and
	$\delta_\gamma>\sup\setbuild{\sigma_{\nu,\xi}}{\nu<\xi<\epsilon}$.
	Consequently, for each $\gamma<\xi<\epsilon$, it follows from
	(\ref{eq:addram}) and the additivity of $f$ that
	\begin{equation}
		f(\alpha_\gamma,\alpha_\epsilon)=f(\alpha_\xi,\alpha_\epsilon).
	\end{equation}
	It now follows from the construction that the transfinite sequence
	$\family{\alpha_\gamma}{\gamma<\beta}$ is homogeneous for $f$.
\end{proof}

\begin{rem}
	If in the above proof we substitute for the colouring $f$ any surjection
	with the property that $\card{\range f}<\card{\cf(\delta)}$ then the proof
	remains valid.  Hence, it is not necessary for the proof that $\card{\range
	f}<\aleph_0$ when $\cf(\delta)\geq\omega_1$.
\end{rem}

One could also view additive colourings, through a categorical lense, as a
functor between appropriately chosen categories.  Leting $\cat{\alpha}$ denote
the category whose \textit{objects} are the elements of $\alpha$ and whose
\textit{arrows} are (closed) intervals of the form $[a,b]\colon a\to b$.  It is
readily verified that $\cat{\alpha}$ is in fact a legitimate category.

Given this perspective, an additive colouring of $\alpha$ is then nothing more
than a functor $\cat{\alpha}$ to a monoid $\cat{M}$, informally a "group without
inverses", considered as a category (specifically, a category with a single
object).

The arrows of $\cat{M}$, i.e.\ the monoid elements, then represent what we have
referred to as ``colours'' while $\circ$ corresponds simultaneously to the
aforementioned $\oplus$ operation and multiplication within the monoid itself.

\begin{cor}[Existence of homogeneous sequences]
	Suppose $\alpha$ is a linear order, without a greatest element, and $f$ is
	an additive colouring of $\alpha$. There must then exist a limit ordinal
	$\delta\geq\cf(\alpha)$ and a homogeneous sequence
	$(a_\gamma)_{\gamma<\delta}$ for $f$.
\end{cor}

\begin{proof}
	Let $(b_\gamma)_{\gamma<\delta^\prime}$ be some cofinal sequence in $\alpha$
	such that $\delta^\prime=\cf(\alpha)$.  Note that, since $\alpha$ has no
	greatest element, $\delta^\prime=\cf(\alpha)$ is a limit ordinal.

	Define $B=\setbuild{b_\gamma}{\gamma<\delta^\prime}$ and let
	$g=f\restriction{\subsets{2}{B}}$ then we may assume, without loss of
	generality, that $g$ is surjective and hence a colouring of $B$.

	Since $B$ has order type $\delta^\prime$, it follows from the additive
	Ramsey Theorem that there exists, for some $\delta\geq\cf(\alpha)$, a
	transfinite sequence $x=(a_\gamma)_{\gamma<\delta}$ which is homogeneous for
	$g$.  Since $B$ is not bounded above, $\delta$ must be a limit ordinal and,
	by definition of $g$, $x$ is a homogeneous sequence for $f$, as required.
\end{proof}


\bibliography{references}

	\bibliographystyle{amsalpha}

\chapter{Scattered linear orders}

The scattered linear orders are antithetical to the \textit{dense linear
orders}, the latter of which includes specimens such as $\eta$ and $\lambda$.
They are precisely the linear orders which do not embed any dense linear order
or, equivalently, do not embed a copy of the order type $\eta$ of the
rationals.

As we will see, the scattered linear orders can constructed in a systematic
fashion from the ground up by starting with the order types $\zero$ and $\one$
and iterating relatively ``simple'' operations.

From the perspective of first-order logic, the class $\scattered$ is also
well-behaved in that its theory $\Th(\scattered)$ is decidable.  A recursively
enumerable class $\Mzero\subseteq\scattered$ is presented in order to establish
this fact.  The argument proceeds as in \cite{RosLin}, owing to the techniques
of L\"auchli and Leonard.

\section{Splittings and congruence lattices}

The following definition is reminiscent of the analogous concept from algebra
and, in particular, lattice theory.
\begin{dfn}[Congruence]
	Suppose $\sim$ is an equivalence relation on the domain of the linear order
	$\alpha$.  We call $\sim$ a \textbf{congruence} on $\alpha$ whenever, for
	every $a,b\in\alpha$ such that $a<b$ and $a\nsim b$, the following
	holds:
	\begin{equation}
		a^\prime\sim a\text{ and }b^\prime\sim b\quad\implies\quad
		a^\prime<b^\prime,
	\end{equation}
	for all $a^\prime,b^\prime\in\alpha$.
\end{dfn}

Essentially, congruences are the equivalence relations that are compatible with
the order structure of $\alpha$.

Upon careful inspection, the above definition can be seen to be equivalent to
the definition of a lattice-theoretic congruence.  One need only view $\alpha$
as a lattice.  The lattice operations are derived in the usual fashion from the
order relation.

We will, ofcourse, now need a means of constructing various condensations along
our travels.  The following Lemma accomplished this feat by starting with an
arbitrary transitive relation.

\begin{lem}[Congruence construction]
	\label{lem:IndCong}
	Suppose $\alpha$ is a linear order and $R$ is a transitive binary relation
	on $\alpha$.  Now define a another binary relation $\sim$ on $\alpha$ such
	that, for $a,b\in\alpha$, we have $a\sim b$ whenever one of the following
	is satisfied:
	\begin{enumerate}
		\item   $a=b$,
		\item   $a<b$ and $aRb$,
		\item   $b<a$ and $bRa$.
	\end{enumerate}
	Under these assumptions, $\sim$ is a congruence on $\alpha$.
\end{lem}
\begin{proof}
	By definition, $\sim$ must be both reflexive and transitive.  To establish
	transitivity, fix any $a,b,c\in\domain{}\alpha$.

	Without loss of generality we may assume that $a\leq b\leq c$.  However,
	sine the cases $b=c$ and $a=b$ are trivial, it suffices to consider only the
	case $a<b<c$.  The result then follows by trasitivity of the linear order
	relation.
\end{proof}

\begin{dfn}[Induced congruence]
	The congruence $\sim$ in Lemma (\ref{lem:IndCong}) is referred to as the
	\textbf{congruence induced by $R$}.
\end{dfn}

\begin{dfn}[Splitting]
	If $\alpha,\beta$ are linear orders then a \textbf{splitting} is a
	surjective homomorphism $\pi\colon\alpha\rightarrow\beta$.
\end{dfn}

The definition of a splitting should be somewhat reminiscent of the
\textit{factor maps}, also called quotient maps, from algebra and topology.  Our
analog of a quotient structure is the following:

\begin{dfn}[Condensations]
	If $\sim$ is a congruence of the linear order $\alpha$ and
	$\pi_\sim\colon\alpha\to\faktor{\alpha}{\sim}$ is the (unique) splitting such
	that, for every $a,b\in\alpha$, it holds that:
	\begin{equation}
		a\sim b\iff\pi_\sim(a)=\pi_\sim(b),
	\end{equation}
	then we call $\pi_\sim$ the \textit{splitting of $\alpha$ induced
	by} $\sim$, refer to the quotient $\faktor{\alpha}{\sim}$ as a
	\textbf{condensation}.
\end{dfn}

We single out the following splitting in particular.  It will play a crucial
role in some of the results to come.

\begin{dfn}[Finite splitting]
	Suppose $\alpha$ is a linear order and $\sim$ is the congruence on $\alpha$
	induced by the relation $R$ defined by:
	\begin{equation} aRb\quad\iff\quad
		a<b\text{ and }[a,b]\text{ is finite}.
	\end{equation} For any linear order
	$\alpha$ we let $\fsplit[\alpha]$ denote the map $\fsplit[\alpha]\colon
	a\mapsto\faktor{a}{\sim}$ for $a\in\alpha$.  We refer to $\fsplit[\alpha]$
	as the $\textbf{finite splitting}$ on $\alpha$ and we omit the prescript
	whenever $\alpha$ is clear from the context.
\end{dfn}

In the result that follows, for any binary relation $R$, we use the notation
$\trclos(R)$ to denote the \textit{transitive closure} of $R$.  In other words,
$\trclos(R)$ is the smallest (w.r.t.\ set inclusion) transitive binary relation
containing $R$ as a subset.

\begin{prp}[Congruence lattices]
	\label{prp:conlat}
	Let $\con \alpha\subseteq\powerset{\domain{2}\alpha}$ be the set of all
	congruences on $\alpha$ and, for each $X\subseteq \con\alpha$, define:
	\begin{align}
		\bigwedge X &\coloneqq\bigcap X,\\ \bigvee X
					&\coloneqq\bigwedge\setbuild{a\in\con\alpha}{x\subseteq
					a,\forall x\in X}.
	\end{align}
	It then follows that $(\con\alpha,\vee,\wedge)$ is a complete lattice.
	Furthermore, for any $X\subseteq\con\alpha$, it holds that $\bigvee
	X=\trclos(\bigcup X)$.
\end{prp}
\begin{proof}
	Let $X$ be an arbitrary (non-empty) subset of $\con\alpha$.  We now proceed
	to argue that $(\con\alpha,\vee,\wedge)$ is a complete lattice under the
	defined operations.  To achieve this, we are first required to show that (as
	defined above) the sets $\bigvee X$ and $\bigwedge X$ are in fact
	congruences.

	We first make the case for $\bigwedge X$.  Let $I$ denote the identity
	relation on $\domain{}\alpha$ then, since the members of $X$ are all
	reflexive, we must have $I\subseteq x$ for each $x\in X$ and thus
	$I\subseteq\bigcap X=\bigwedge X$.

	Therefore $\bigwedge X$ is in fact a reflexive (binary) relation.  Also,
	since the members of $X$ are all symmetric it immediately follows that
	$\bigwedge X$ is also, since $(a,b)\bigcap X$ implies $(a,b)\in x$ for every
	$x\in X$.  The symmetry of the members of $x$ now yield the
	corresponding $(b,a)\in\bigcap X$.

	In a somewhat similar fashion, an arbitrary intersection of transitive
	(binary) relations will again yield a transitive relation, all from first
	principles.

	Next we consider the case for $\bigvee X$.  Note that $I\subseteq \bigvee X$
	since $X$ is non-empty and $I\subseteq x$ for each $x\in X$.  Therefore
	$\bigvee X$ is reflexive.  Note now that $\bigvee X$ is an intersection of
	symmetric (binary) relations and thus must itself also be symmetric.  In a
	similar fashion, since $\bigvee X$ is an intersection of transtive relations
	it must itself also be transitive.

	All that remains is to show that $\bigvee X=\trclos(\bigcup X)$.  Clearly,
	since $\bigvee X$ is transitive, we already have $\trclos(\bigcup
	X)\subseteq\bigvee X$.  Noting then that $\trclos(\bigcup X)\in X$, we may
	conclude $\bigvee X\leq\trclos(\bigcup X)$, as required.
\end{proof}

The sublattice of $\con\alpha$ generated by the set of $0$\textit{-definable}
congruences on $\alpha$ will be denoted as $\defcon{\alpha}$.  It should be clear
that, in general, $\defcon{\alpha}$ will not be complete as it might lack
certain \textit{infinite} joins or meets.

Ideally, one would want every condensation of a member of $\con\alpha$ to again
belong to $\con\alpha$.  Up to isomorphism this is in fact the case, as
suggested by the following ``homomorphism'' theorem.  Take note that we use
$\pi_\sim$ to denote the splitting
$\pi_\sim\colon\alpha\to\faktor{\alpha}{\sim}$ induced by the congruence $\sim$
on $\alpha$.

\begin{prp}[Homomorphisms]
	 Suppose $f\colon\alpha\to\beta$ is a splitting. There exists a unique
	 congruence relation $\sim$ on $\alpha$ and a unique order isomorphism
	 $\iota\colon\faktor{\alpha}{\sim}\to\beta$ that makes the following diagram
	 commute:
	\begin{equation}
		\begin{tikzcd}
			\alpha \arrow[r, "\pi_\sim",rightarrow]&   \faktor{\alpha}{\sim}\\
			 &   \beta\arrow[from=u,"\iota", dashrightarrow]
			 \arrow[from=ul,"f"']
		\end{tikzcd}
	\end{equation}
\end{prp}
\begin{proof}
	The desired congruence relations is obtained in a manner familiar from
	algebra:  identify elements in $\alpha$ whenever they have the same image
	under $f$.  Borrowing a definition from universal algebra and lattice
	theory, we let $\sim$ be a binary relation such that:
	\begin{equation}
		\sim{}=\kernel f=\setbuild{(a,b)\in\domain{2}\alpha}{f(a)=f(b)}.
	\end{equation}

	It is readily verified that the required isomorphism is given by
	$\iota([a])=f(a)$, for each $a\in\alpha$.  That $\iota$ is well-defined
	follows simply from the definition of ${\sim}=\kernel f$.

	We are now required to establish the uniqueness of $\sim$ and $\iota$ in
	their respective roles.  Suppose $\sim_0$ is a congruence on $\alpha$ and
	there exists a unique isomorphism $\iota_0$ that makes
	\begin{equation}
		\begin{tikzcd}
			\alpha\arrow[r, "\pi_{\sim_0}",rightarrow]&
			\faktor{\alpha}{\sim_0}\\
			 &   \beta\arrow[from=u,"\iota", dashrightarrow]
			 \arrow[from=ul,"f"']
		\end{tikzcd}
	\end{equation}
	commute.  This then clearly implies that ${\sim_0}=\ker f={\sim}$ and,
	consequently, also $\iota_0=\iota$.
\end{proof}


\section{Hausdorff's characterisation of the countable scattered linear orders}

\begin{prp}[Operations on $\scattered$]\label{prp:OpScattered}
        The following properties hold:
        \begin{enumerate}
			\item   If $I\in\scattered$ and $\alpha_i\in\scattered$ for each
				$i\in I$ then $\sum_{i\in I}\alpha_i\in\scattered$;
			\item   if $\alpha,\beta\in\scattered$ then
				$\alpha+\beta,\alpha\cdot\beta\in\scattered$.
        \end{enumerate}
\end{prp}
\begin{proof}
		(1):  Suppose, by way of contradiction, that $\delta\subseteq\sum_{i\in
		I}\alpha_i$ is countable and dense.  Define
		$\delta_i=\delta\cap\alpha_i$, for each $i\in I$.

		Note that, since $\delta$ is dense, we cannot have
		$1<\card{\delta_i}<\aleph_0$ for any $i\in I$.  Therefore, each
		$\delta_i$ is either infinite or has at most one element.

		If $\delta_i$ is finite for each $i\in I$ then $\delta\preceq I$,
		contradicting the scatteredness of $I$.  Consequently, the must exist
		some $j\in I$ such that $\delta_j$ is infinite.  However, since
		$\delta_j\subseteq\alpha_j$, this implies that $\alpha_j$ has a
		countable dense subset --- the desired contradiction.

		(2):  Choosing $I=\two$, $\alpha_0=\alpha$ and $\alpha_1=\beta$ in (1)
		yields $\alpha+\beta\in\scattered$.  Instead, choosing $I=\beta$ and
		$\alpha_i=\alpha$ for each $i\in\beta$ we get
		$\alpha\cdot\beta\in\scattered$, as required.
\end{proof}

The result above gives us the mantra: \textit{scattered sums} of scattered
linear orders are themselves scattered.  In particular, products and (finite)
sums of scattered linear orders are also scattered.

This observation suggests the possibility that one could generate the class of
all scattered linear orders recursively as sums of previously constructed
(scattered) linear orders.  We do this for the countable case but the approach
readily extends to the class of \text{all} scattered linear orders.

\begin{dfn}[The class $\VD$]
		By way of transfinite recursion define, for each ordinal
		$\gamma<\omega_1$, the class of linear orders
		$\VD_{\gamma}\subseteq\linear$ to be the smallest class which is
		\textit{closed under isomorphisms} and satisfies:
        \begin{enumerate}
            \item   $\zero,\one\in\VD_0$;
			\item   if $\alpha_i\in\bigcup_{\beta<\gamma}\VD_{\beta}$, for each
				$i\in \zeta$, then $\sum_{i\in\zeta}\alpha_i\in\VD_\gamma$.
        \end{enumerate}
		The class $\VD=\bigcup_{\gamma<\omega_1}\VD_\gamma$ is called the class
		of $\textbf{(countable) very discrete}$ linear orders.
\end{dfn}

\begin{dfn}[$\VD$-rank]
		If $\alpha$ is a very discrete linear order then its
		$\bm{\mathcal{VD}}$\textbf{-rank} $\vdrank(\alpha)$ is the least ordinal
		$\beta$ such that $\alpha\in\VD_{\beta}$.
\end{dfn}

\begin{lem}\label{prp:vdsct}
	Every very discrete linear order is scattered.  That is to say
	$\VD\subseteq\scattered$.
\end{lem}
\begin{proof}
	We argue by transfinite induction on $\gamma$ that
	$\VD_\gamma\subseteq\scattered$.  Finite linear orders are (trivially)
	scattered and thus $\VD_0\subseteq\scattered$.

	Fix $\delta<\omega_1$ and assume $\VD_\gamma\subseteq\scattered$, for each
	$\gamma<\delta$.  By definition, if $\vdrank(\alpha)=\delta$ then there
	exists, for each $i\in\zeta$, an $\alpha_i\in\VD_{\gamma_i}$, for some
	ordinal
	$\gamma_i<\delta$, such that
	\begin{equation}
		\alpha=\sum_{i\in\zeta}\alpha_i.
	\end{equation}

	It follows from Proposition \ref{prp:OpScattered}, as well as the
	inductive hypothesis, that $\alpha$ is a scattered sum of scattered linear
	orders.  Therefore we must have $\alpha\in\scattered$, as requred.
\end{proof}

Suppose $\pi\colon\alpha\to\faktor{\alpha}{\sim}$ is a splitting. Define $\pi^0$
to be the splitting $a\mapsto\set{a}$ and let $\pi^1=\pi$.  If $\pi^\gamma$ has
been defined and $\pi^\prime$ is the finite splitting of
$\faktor{\alpha}{\sim_\gamma}$ then let
$\pi^{\gamma+1}=\pi^\prime\circ\pi^\gamma$.  Now choose $\delta$ to be some
limit ordinal and assume $\pi^\gamma$ has been defined for all $\gamma<\delta$.

Recall from Proposition \ref{prp:conlat} that, given any linear order $\alpha$,
the lattice $\con\alpha$ is complete.  Hence, $\sim_\gamma$ is the congruence of
$\alpha$ induced by $\pi_\gamma$, for each $\gamma<\delta$, we may define
$\sim_\delta=\bigvee_{\gamma<\delta}\sim_\gamma$ and choose $\pi_\delta$ to be
the (corresponding) induced splitting of $\alpha$.

Note that there necessarily exists a least ordinal $\beta$ with the property:
for every ordinal $\delta\geq\beta$, there exists an isomorphism $\iota$ such
that the diagram
\begin{equation}
	\begin{tikzcd}
		\alpha\arrow[dd,"{\pi^\beta}"']\arrow[rdd,"{\pi^\delta}"]&\\
									 &\\
		\faktor{\alpha}{\sim_\beta}\arrow[r,"{\iota}"]&\faktor{\alpha}{\sim_\delta}
	\end{tikzcd}
\end{equation}
commutes.  In the special case $\pi=\fsplit$ we refer to $\beta$ as the
\textit{Hausdorff rank} of $\alpha$ and denote it by $\hrank(\alpha)$.

\begin{prp}
	If $\alpha$ is a linear order, $\beta<\hrank(\alpha)$ is a limit ordinal
	and $\pi=\fsplit$ then it follows, for every $\gamma<\beta$, that there
	exists a unique splitting
	$h_\gamma\colon\faktor{\alpha}{\sim_\gamma}\to\faktor{\alpha}{\sim_\beta}$ so that
	\begin{equation}
		\begin{tikzcd}
			\alpha\arrow[rd,"{\pi^\gamma}"]\arrow[dd,"{\pi^\beta}"']&\\
			&\faktor{\alpha}{\sim_{\gamma}}\arrow[ld,dashrightarrow,"{h_\gamma}"]\\
			\faktor{\alpha}{\sim_\beta}&
		\end{tikzcd}
	\end{equation}
	commutes.
\end{prp}
\begin{proof}
	Fix any $\gamma<\beta$ then note that, for any $a,b\in\alpha$,
	\begin{equation}
		a\sim_\gamma b\implies a\sim_\beta b.
	\end{equation}
	Denoting by $a_\gamma$ the equivalence class $[a]_{\sim_\gamma}$, for
	$a\in\alpha$, it follows that the map $a_\gamma\mapsto\pi_\beta(a)$ is
	well-defined.

	Let $h_\gamma\colon\faktor{\alpha}{\sim_\gamma}\to\faktor{\alpha}{\sim_\beta}$ be
	the map with the above action then the diagram in question commutes.  Since
	$\pi_\gamma$ is surjective, it also follows that $h_\gamma$ is unique.

	All that remains is to establish the surjectivity of $h_\gamma$.  To this
	effect, it is sufficient to note: for every $a\in\alpha$ it holds that
	$h_\gamma(a_{\gamma})=a_\beta$.

\end{proof}

Note, for a scattered linear order $\alpha$ and any ordinal
$\beta\geq\hrank(\alpha)$, we necessarily have
$\faktor{\alpha}{\sim_\beta}\cong\one$.  It is enough to argue that this is the
case when $\beta=\hrank(\alpha)$ so suppose the contrary.

By assumption there must exist some $a,b\in\alpha$ such that $a_\beta<b_\beta$.
Since $\pi_\beta$ is surjective, it follows from the definition of $\beta$ that
there exists a unique isomorphism
$\iota\colon\faktor{\alpha}{\sim_\beta}\to\faktor{\alpha}{\sim_{\beta+1}}$ such
that
\begin{equation}
	\begin{tikzcd}
		\alpha\arrow[rd,"{\pi^\beta}"]\arrow[dd,"{\pi^{\beta+1}}"']&\\
		&\faktor{\alpha}{\sim_{\beta}}\arrow[ld,dashrightarrow,"\iota"]\\
		\faktor{\alpha}{\sim_{\beta+1}}&
	\end{tikzcd}
\end{equation}
commutes.

Since $\iota$ is injective it follows that $a_{\beta+1}<b_{\beta+1}$.  By
definition of $\pi_{\beta+1}$, we may now conclude that the interval
$[a_\beta,b_\beta]$, and thus also $I=(a_\beta,b_\beta)$, is infinite.

Working towards a contradiction, we now argue that the interval $I$ is a dense
linear order.  If $c_\beta<d_\beta$ both belong to $I$, for some $c,d\in\alpha$,
then it follows from the injectivity of $\iota$ that $c_{\beta+1}<d_{\beta+1}$
and, thus, $[c_\beta,d_\beta]$ is infinite.  Consequently, the open interval
$(c_\beta,d_\beta)$ is nonempty, as required.

This is the desired contradiction.  Therefore, we may conclude that
$\faktor{\alpha}{\sim_\beta}\cong\one$ whenever $\alpha$ is scattered.

\begin{lem}
	\label{lem:consplit}
	Suppose $\alpha$ is a countable linear order then the following holds:
	\begin{enumerate}
		\item if $\alpha$ is very discrete and $\delta$ is a convex subset of
			$\alpha$ then $\delta$ is a very discrete linear order.
		\item if $\alpha$ is scattered, $\hrank(\alpha)$ is a limit
			ordinal $\rho>0$ and
			\begin{equation}
				\label{eq:limrank}
				\lim_{i<\omega}\rho_i=\rho
			\end{equation} then,
			for every $a<b$ in $\alpha$, there exists an $n<\omega$ such that
			$\fsplit^{\rho_i}(a)=\fsplit^{\rho_i}(b)$ whenever $i\geq n$.
	\end{enumerate}
\end{lem}
\begin{proof}\leavevmode
	\begin{enumerate}[topsep=0pt]
		\item If $\vdrank(\alpha)=0$ then $\alpha$, and thus $\delta$, must be
			finite so that the result holds trivially.

			Suppose now that $\vdrank(\alpha)>0$ and assume the result holds for
			all very discrete linear orders of $\VD$-rank less than
			$\vdrank(\alpha)$.  It follows by definition that there exists
			a family $\family{\alpha_i}{i\in\zeta}$ of very discrete linear
			orders such that
			\begin{equation}
				\alpha=\sum_{i\in\zeta}\alpha_i.
			\end{equation}

			For each $i\in\zeta$, define $\delta_i=\delta\cap\alpha_i$.  We may
			then conclude that
			\begin{equation}
				\delta=\sum_{i\in\zeta}\delta_i
			\end{equation}
			and, for each $i\in\zeta$, it holds that
			$\delta_i\subseteq\alpha_i$.  From the inductive hypothesis the
			result then readily follows.
			\smallskip

		\item Choose $\pi=\fsplit$ and let $\sim$ denote the induced congruence.
			Define $\family{\rho_i}{i<\omega}$ to be a strictly increasing
			sequence of ordinals such that (\ref{eq:limrank}) is
			satisfied.

			Choose the least $n<\omega$ such that
			\begin{equation}
				\hrank\left([a,b]\right)\leq\rho_n,
			\end{equation}
			and fix any $\delta=\rho_m$ such that $m\geq n$.  Note, if
			$\sim^\prime$ denotes the congruence
			$\sim_\delta\restriction[a,b]$ then it follows that
			\begin{equation}
				\faktor{[a,b]}{\sim^\prime}\cong
				\left[\faktor{a}{\sim_\delta}\mathpunct{,}\faktor{b}{\sim_\delta}\right].
			\end{equation}
			However, since $\alpha$ is scattered, it follows from part 1 that
			the subinterval $[a,b]$ is also scattered.  Hence, we may conclude
			that
			\begin{equation}
				\left[\faktor{a}{\sim_\delta},\faktor{b}{\sim_\delta}\right]\cong\one.
			\end{equation}
			Therefore, it follows that $\pi^\delta(a)=\pi^\delta(b)$, as
			required.
	\end{enumerate}
\end{proof}

\begin{lem}\label{prp:sctvd}
	If $\alpha\in\scattered$ and $\card{\alpha}\leq\aleph_0$ then $\alpha$ is
	very discrete.
\end{lem}
\begin{proof}
	Suppose $\alpha\in\scattered$ is at most countable and define a relation $R$
	on $\domain{}\alpha$ such that, for every $a,b\in\alpha$, we have $aRb$
	whenever $a\leq b$ and $[a,b]$ is finite.  Note that $R$ is transitive and
	thus, by Lemma \ref{lem:IndCong}, it induces a congruence relation $\sim$ on
	$\domain{}\alpha$.

	Define the condensation $\beta=\faktor{\alpha}{\sim}$ and argue by
	transfinite induction on $\hrank(\alpha)$.  If $\hrank(\alpha)=0$ then it
	follows immediately that $\alpha\embed\zeta$ and thus $\alpha\in\VD$.

	Assume the result holds for each $\gamma<\rho=\hrank(\alpha)$ and consider
	the case where $\rho=\rho_0+1$.  It follows that
	$\pi_{\rho_0}[\alpha]\embed\zeta$ and thus $\alpha$ can be written as a
	$\zeta$-sum of members of $\beta$.  This implies that $\alpha$ is very
	discrete, since $\hrank(\xi)<\rho$ whenever $\xi\in\beta$.

	Suppose now that $\rho$ is a limit ordinal then there exists a (strictly)
	increasing sequence
	$\family{\rho_i}{i<\omega}$ of ordinals such that
	$\lim_{i<\omega}\rho_i=\rho$.  Fix some $a_0\in\alpha$ and construct a
	sequence $\family{a_i}{i<\omega}$ as follows.  If $a_i$ has been defined for
	some $i<\omega$ then choose $a_{i+1}$ such that $a_{i+1}=a_i$ whenever
	\begin{equation}
		\up{a_0}\cap\pi^{\rho_{i+1}}(a_i)=\up{a_0}\cap\pi^{\rho_i}(a_i),
	\end{equation}
	otherwise, let
	$a_{i+1}\in\up{a_0}\cap(\pi^{\rho_{i+1}}(a_i)\setminus\pi^{\rho_i}(a_i))$.

	In a similar fashion, construct a sequence $\family{a_{-i}}{i<\omega}$ in
	$\down{a_0}$.  Define the indexed family $\family{I_j}{j\in\zeta}$
	such that
	\begin{equation}
		I_j=\begin{cases}
			\set{a_0}&\text{if }j=0,\\
			(a_{j-1},a_j]&\text{if }j>0,\\
			[a_j,a_{j+1})&\text{if }j<0.
		\end{cases}
	\end{equation}
	It follows from Lemma \ref{lem:consplit} that each $I_j$, for $i\in\ints$,
	is very discrete and we must have $\bigcup_{j\in\ints} I_j=\domain{}\alpha$.
	We may therefore write
	\begin{equation}
		\alpha=\sum_{j\in\zeta}I_j
	\end{equation}
	and thus conclude, by definition, that $\alpha$ is indeed a very discrete
	linear order.
\end{proof}

\begin{thm}[Hausdorff's Theorem]
	A linear order $\alpha$ is countable and scattered iff it is very discrete.
\end{thm}
\begin{proof}

	\forward\	Lemma \ref{prp:sctvd}.

	\backward\	Proposition \ref{prp:vdsct}.
\end{proof}


\section{The first-order theory of scattered linear orders}

A technical, but useful, feature of splittings is that they necessarily always
have \textit{right} inverses: if $s\colon\alpha\to\beta$ is a splitting then
there exists a homomorphism $h\colon\beta\to\alpha$ such that $sh=\id_\alpha$.
Additionally, $h$ must be an embedding of linear orders.  This latter remark
follows from the fact that if $b<b^\prime$ are members of $\beta$ then we must
have $sh(b)<sh(b^\prime)$ and thus $h(b)\neq h(b^\prime)$.

Splittings turn out to be an instance of what a category theorist would refer to
as a \textit{split epimorphism}.  This serves to justify our choice of
terminology as splittings are therefore, in fact, split!  Henceforth, we will
use this property without explicit mention of it.

From an ontological standpoint, the definition of scatteredness is
\textit{negative} in nature.  That is to say, scattered linear orders are
defined by stating what they are \textit{not}.  We now seek to remedy this in
the result that follows.

\begin{prp}[Condensations and scatteredness]
	The following are equivalent for any linear order $\alpha$:
	\begin{enumerate}
		\item  $\alpha$ is scattered,
		\item  no (nontrivial) condensation of $\alpha$ is dense,
		\item  the lattice $\con\alpha$ is atomic.
	\end{enumerate}
\end{prp}
\begin{proof}
	1$\Rightarrow$2:  If $\beta$ is a nontrivial condensation of $\alpha$ and
	$s\colon\alpha\to\beta$ is the induced splitting then there exists a right
	inverse $h$ of $s$ which is an embedding of $\beta$ in $\alpha$.  Hence,
	$\beta$ cannot be dense as this would contradict the scatteredness of
	$\alpha$.

	\smallskip\noindent 2$\Rightarrow$3:  Suppose ${\sim}\in\con\alpha$ is not the identity
	congruence.  By assumption, $\faktor{\alpha}{=}$ cannot be dense.  Hence,
	there exists $a<a^\prime$ in $\alpha$ such that $a\sim a^\prime$ and
	$a^\prime$ is an immediate successor of $a$.

	Now define the congruence $\sim_0$ on $\alpha$ such that $x\sim_0 y$ iff
	either $x=y$ or $x,y\in\set{a,a^\prime}$.  It follows immediately, by
	definition, that ${\sim_0}\subseteq{\sim}$ and no congruence $\sim_1$ on
	$\alpha$ satisfies ${\sim_1}\subsetneq{\sim_0}$ other than the identity
	congruence.  Thus, $\con\alpha$ is indeed atomic.

	\smallskip\noindent 3$\Rightarrow$1:  Suppose, with an aim to a
	contradiction, that there exists an embedding $h\colon\eta\to\alpha$.  We
	may then assume, without loss of generality, that $f[\beta]=\rats$.  Now, if
	$\sim$ is an atomic element of $\con\alpha$ then it follows that
	${\sim}\cap\rats\times\rats$ must be an atomic element of $\con\eta$,
	contradicting the density of $\eta$.
\end{proof}

The implication 1$\Rightarrow$2, in the previous proposition, will serve to aid
in the choice of a first-order approximation of the property of scatteredness.
This is done by requiring that 2 holds for \textit{definable} congruences only.

\begin{dfn}[Congruence formula]
	If $\alpha$ is any linear order and, for some $\bar{b}\in\domain[n]{\alpha}$ the
	formula $\varphi(x,y,\bar{b})$ defines a congruence $\sim$ on $\alpha$ then
	we call the formula $\varphi=\varphi(x,y,\bar{z})$, with $\bar{z}$ an
	$n$-tuple of variables, a \textbf{congruence formula} and refer to $\bar{b}$
	as \textbf{a parameter of} $\bm{\varphi}$.
\end{dfn}

Furthermore, we will use the notation $\faktor{\alpha}{\varphi}$ to refer
to condensation $\faktor{\alpha}{\sim}$ and, for any $a\in\alpha$,
$a_\varphi=\faktor{a}{\varphi}$ will mean the equivalence class
$\faktor{a}{\sim}$.  In this given scenario, we will refer to
$\faktor{\alpha}{\varphi}$ as being a \textit{definable condensation} of
$\alpha$.

\begin{dfn}[Definably scattered]
	A linear order $\alpha$ is \textbf{definably scattered} whenever none of its
	(infinite) definable condensations are dense.
\end{dfn}

Na\"ively, one would expect that scatteredness and its ``definable
approximation'' do not coincide.  This is suggessted by the fact that ordinary
scatteredness is itself a \textit{dyadic} second order property: if $X$ is a
binary relation variable, $\varphi(X)$ expresses that $X$ is a (nontrivial)
congruence and $\psi(X)$ expresses that the induced condensation is not dense
then it follows that $\alpha$ is scattered iff
\begin{equation}
	\alpha\models\forall X\big(\varphi(X)\rightarrow\psi(X)\big).
\end{equation}

It turns out that this expectation is in fact correct!  The demonstration of
this fact is a routine application of the Diagram Lemma and Compactness Theorem.

\begin{exm}[A definably scattered linear order which is not scattered]
	\label{exm:dscat}
	Consider the language $L$ which is obtained, from the language of linear
	orders, by adding a new constant symbol $c_q$, for each $q\in\rats$.  Also,
	define $T$ to be the first-order theory of definably scattered linear
	orders.

	Since $\zeta$ is scattered, it follows that $\zeta\models T$.  Furthermore,
	if $\Sigma\subseteq\diag\eta$ is finite then, for each $q\in\rats$,
	$z_q={c_q}^\zeta\in\ints$ may be chosen in a manner such that
	$(\zeta,\family{z_q}{q\in\rats})\models T\cup\Sigma$.  This is possible due
	to the fact that only a finite number of the new constant symbols will have
	an occurence in some member of $\Sigma$.

	From the Compactness Theorem, it follows that $T\cup\diag\eta$ must have
	some model $\mathfrak{M}$.  Therefore, from the Diagram Lemma, there exists
	a linear order $\alpha$ and an embedding $h\colon\eta\to\alpha$ such that
	$(\alpha,h[\rats])=\mathfrak{M}$.

	By construction, $\alpha$ cannot be scattered but, nonetheless, must
	be a model of the first-order theory $T$.  Consequently, $\alpha$ is
	definably scattered but not scattered.
\end{exm}

\begin{prp}
	If $\alpha$ is some linear order then the following are equivalent:
	\begin{enumerate}
		\item   $\alpha$ is definably scattered,
		\item   the lattice $\defcon\alpha$ is atomic,
		\item   there exists no dense linear order $\delta$ and no
			$k,n\in\posnats$ such that $k\leq 2$ and, for some
			$\bar{b}\in\domain[n]{\delta}$ and $\bar{a}\in\domain[kn]{\alpha}$,
			$(\delta,\bar{b})$ is interpretable in $(\alpha,\bar{a})$,
	\end{enumerate}
\end{prp}
\begin{proof}
	1$\Rightarrow$2:  By way of contradiction, assume the contrary so that
	$\delta$ is a dense linear order and $(\Gamma,f)$ is an interpretaion of
	$(\delta,\bar{b})$ in $(\alpha,\bar{a}_0,\dotsc,\bar{a}_{n-1})$.
\end{proof}

\begin{prp}\label{prp:dcform}
	For each formula $\varphi=\varphi(x,y,\bar{z})$, with $\bar{z}$ possibly empty, define the formula $\epsilon_\varphi(\bar{z})$ to be the conjunction of the following formulas:
	\begin{enumerate}
		\item	$\forall x\varphi(x,x,\bar{z})$,\hfill(reflexivity)
		\item	$\forall x\forall y(\varphi(x,y,\bar{z})\rightarrow\varphi(y,x,\bar{z}))$,\hfill (symmetry)
		\item 	$\forall x\forall y\forall w(\varphi(x,y,\bar{z})\wedge\varphi(y,w,\bar{z})\rightarrow\varphi(x,w,\bar{z}))$.\hfill (transitivity)
		\item 	$\forall x_0\forall x_1\forall y_0\forall y_1(x_0<x_1\wedge\neg\varphi(x_0,x_1,\bar{z})\wedge\varphi(x_0,y_0,\bar{z})\wedge\varphi(x_1,y_1,\bar{z})\rightarrow y_0<y_1)$.\phantom{}\hfill(compatibility)
	\end{enumerate}
	For every finite tuple $\bar{a}$ of $\alpha$ (of the same length as $\bar{z}$), it holds that $\alpha\models\epsilon_\varphi(\bar{a})$ iff $\varphi(x,y,\bar{a})$ defines a congruence of $\alpha$.
\end{prp}

\begin{thm}
	For every countable, definably scattered linear order $\alpha$ and every  $n\in\nats$ there exists a $\beta_n\in\scattered$ such that $\alpha\nequiv{n}\beta_n$.
\end{thm}
\begin{proof}
	Suppose $\alpha\models\Th(\scattered)$ is countable and fix any $n\in\nats$.  Now define a binary relation $R$ on $\domain{}\alpha$ such that $aRb$ iff there exists a $\beta\in\scattered$ such that $(a,b)\nequiv{n}\beta$ in the language of $k$-coloured linear orders (for some $k\in\nats$).  Since $R$ is clearly transitive, it induces a cogruence $\sim$ on $\alpha$.  Moreover, $\sim$ is defined by the $L_{\omega_1\omega}$ formula $\varphi(x,y)$ given by:
	\begin{equation}
		x=;y\vee(x<y\wedge\bigvee_{\beta\in\scattered}(\cha{\beta}{n})^{(x,y)})\vee(y<x\wedge\bigvee_{\beta\in\scattered}(\cha{\beta}{n})^{(y,x)}).
	\end{equation}
	Note that, since our language is finitary, we need only choose finitely many characterstics $\alpha_i\in\scattered$ such that $\alpha_0,\dotsc,\alpha_{m-1}$ is a traversal of the $n$-equivalence classes of $\scattered$. Thus we may identify $\varphi$ with its first-order equivalent which, has finite length.

	\begin{claim}\label{clm:LL1}
		Each $\gamma\in\faktor{\alpha}{\sim}$ has a scattered $n$-equivalent.
	\end{claim}
	\begin{proof}
		Choose any $\gamma\in\faktor{\alpha}{\sim}$.  If there exists a least element $a_0\in\gamma$ then, since $\alpha$ and thus $\gamma$ is countable, there exists a cofinal sequence $\family{a_i}{i<\omega}$.  By Ramsey's Theorem, there then exists a homogeneous subsequence $\family{a^\prime_i}{i<\omega}\subseteq\family{a^\prime_i}{i<\omega}$ for the colouring $h$ that sends intervals $(a_i,a_j)$, with $i<j$, to a characteristic among $\alpha_0,\dotsc,\alpha_{m-1}$.  Consequently, $\gamma$ is an $\omega$-sum
		\begin{equation}
			\gamma=\sum_{i<\omega}[a^\prime_i,a^\prime_{i+1})
		\end{equation}
		where all of the summands are $n$-equivlent to the same $\one+\alpha_{i_0}$ for some $i_0\in\set{0,\dotsc,m-1}$ so that, since summation preserves $n$-equivalence, it follows that $\gamma$ is $n$-equivalent to the scattered linear order:
		\begin{equation}
			\gamma^\prime=(\one+\alpha_{i_0})\cdot\omega.
		\end{equation}
		Since $\gamma^{<a^\prime_0}=[a_0,a^\prime_0))$, by defintion of $\sim$, has an $n$-equivalent (($\gamma^\prime_0$ say) it follows that $\beta_n=\gamma_0+\gamma^\prime$ is the desired scatterred $n$-equivalent of $\gamma$.

		A similar argument can be obtained for when $\gamma$ has a greatest element by dualising the argument above (i.e. respectively swap around $<$ and $>$ as well as $\leq$ and $\geq$.  In te case where $\gamma$ has neither a largest nor a smallest element choose some fixed $a\in\gamma$ and apply a similar argument in the respective suborderings $\gamma^{\geq a}$ and $\gamma^{<a}$.  The sum of their $n$-equivalents will then yield a scattered $n$-equivalent of the equivalence class $\gamma$, as required.\noqed
	\end{proof}

	\begin{claim}
		The equivalence classes are dense under the induced order.
	\end{claim}
	\begin{proof}
		Suppose $\epsilon_0,\epsilon_1\in\faktor{\alpha}{\sim}$ satisfy the property $\epsilon_0<\epsilon_1$ and and then assume, by way of contradiction, that for no $c\in\alpha$ does it hold that:
		\begin{equation}
			\epsilon_0<c<\epsilon_1.
		\end{equation}
		However, since $n$-equivalence is preserved under sums it follows that $\epsilon_0+\epsilon_1$ is $n$-equivalent to some scattered linear order which implies tha by choosing an $a_0\in\epsilon_1$ and an $a_1\in\epsilon_1$ that $(a_0,a_1)$ posess a scattered $n$-equivalent.  This is the desired contradiction as then, by defition, it would have to hold that $a_0\sim a_1$: a contradiction.\noqed
	\end{proof}
	Since $\faktor{\alpha}{\sim}$ is dense and $\alpha$ is definably scattered, the only possible conclusion is that $\faktor{\alpha}{\sim}\cong\one$.  From claim \ref{clm:LL1} the result then follows.
\end{proof}

\begin{prp}\label{prp:rescat}
	For each formula $\varphi(x,y,\bar{z})$ define $\theta_\varphi(\bar{z})$ to be the formula
	\begin{equation}
		\forall x\forall y\big((x<y\wedge\neg\varphi(x,y,\bar{z}))\rightarrow\exists w(x<w<y\wedge\neg\varphi(x,w,\bar{z})\wedge\neg\varphi(w,y,\bar{z}))\big)
	\end{equation}
	which formalises the statement: ``the condensation which is induced by the congruence, defined by the formula $\varphi$, is dense'' and let each $\epsilon_\varphi(\bar{z})$ be defined as in Proposition \ref{prp:dcform}.  It then holds that:
	\begin{enumerate}
		\item the theory $\Sigma=\set{\axmlin}\cup\setbuild{\forall\bar{z}(\epsilon_\varphi(\bar{z})\rightarrow\neg\theta_\varphi(\bar{z}))}{\varphi=\varphi(x,y,\bar{z})}$ axiomatises $\Th(\scattered)$, i.e.\ $\dcl{\Sigma}=\Th(\scattered)$,

		\item $\Sigma$ is recursively enumerable.
	\end{enumerate}
	\begin{proof}
		\begin{enumerate}[nosep]
			\item	$\alpha\models\Sigma$ iff $\alpha$ is definably scattered iff $\alpha\models\Th(\scattered)$.
			\item	Since our signature is finite it follows that the set of formulas in our language can be enumerated (via a G\"odel numbering $\godel{-}\colon L\to\nats$).  We can then enumerate $\Sigma$ by declaring a map acting on $\Sigma$ such that $\axmlin\mapsto 0$ and, for every tuple $\bar{z}$ of variables,
			\begin{equation}
				\forall \bar{z}(\epsilon_\varphi(\bar{z})\rightarrow\theta_\varphi(\bar{z})))\mapsto 2^{\godel{\varphi(x,y,\bar{z})}}.
			\end{equation}
		\end{enumerate}
	\end{proof}
\end{prp}


\section{Theorems of L\"auchli and Leonard}

\begin{dfn}[The class $\Mzero$]
	The class $\Mzero$ is the smallest class of linear orders which satisfies the following:
	\begin{enumerate}
		\item	$\zero,\one\in\Mzero$,
		\item	if $\alpha,\beta\in\Mzero$ then $\alpha+\beta\in\Mzero$,
		\item	if $\alpha\in\Mzero$ then $\alpha\cdot\omega,\alpha\cdot\dual{\omega}\in\Mzero$,
	\end{enumerate}
\end{dfn}

\begin{prp}\label{prp:M0sum}
	Let $n\in\nats$ and suppose for each $i<\omega$ it holds that $\alpha_i\in\Mzero$ then there exists a $\beta\in\Mzero$ such that
	\begin{equation}
		\sum_{i<\omega}\alpha_i\nequiv{n}\beta.
	\end{equation}
\end{prp}
\begin{proof}
	For each $i,j<\omega$, if $i<j$ then define $\alpha_{ij}=\alpha_{i+1}+\alpha_{i+2}+\dotsb+\alpha_j$ and note that $\alpha_{ij}\in\Mzero$ since $\Mzero$ is closed under finite sums.  Choose $S=\set{\beta_0,\dotsc,\beta_{k-1}}\subseteq\Mzero$ to be an $n$-spectrum for $\Mzero$.  Consequently, there exists a colouring $h$ of $\omega$ whose codomain is $\set{0,\dotsc,k-1}$ such that $h(i,j)=\ell$ iff $\alpha_{ij}\nequiv{n}\beta_\ell$.

	It now follows from Ramsey's theorem that there exists a homogeneous sequence $\family{i_j}{j<\omega}$ for $h$.  By definition, there exists a natural $m$ such that $0\leq m<k$ and $\alpha_{i_ji_{j+1}}\nequiv{n}\beta_m$, for every $j<\omega$.  Note, from Lemma \ref{lem:fvsum}, it follows that
	\begin{align}
		\sum_{i<\omega}\alpha_i&\cong\alpha_0+\alpha_1+\dotsb+\alpha_{i_0}+\sum_{j<\omega}\alpha_{i_ji_{j+1}}\\
		&\nequiv{n}\alpha_0+\dotsb+\alpha_{i_0}+\beta_m\cdot\omega,
	\end{align}
	Since $\Mzero$ is closed under finite sums, we may thus conclude that if $\beta=\alpha_0+\dotsb+\alpha_{i_0}+\beta_m\cdot\omega$ then $\sum_{i<\omega}\alpha_i\nequiv{n}\beta$, as required.
\end{proof}

\begin{prp}\label{prp:M0sumop}
	Let $n\in\nats$ and suppose for each $i\in\dual{\omega}$ it holds that $\alpha_i\in\Mzero$ then there exists a $\beta\in\Mzero$ such that
	\begin{equation}
		\sum_{i\in\dual{\omega}}\alpha_i\nequiv{n}\beta.
	\end{equation}
\end{prp}
\begin{proof}
	Dualise the argument in the proof of Proposition \ref{prp:M0sum}.
\end{proof}

\begin{prp}\label{prp:M0sumint}
	Let $n\in\nats$ and suppose for each $i\in\ints$ it holds that $\alpha_i\in\Mzero$ then there exists a $\beta\in\Mzero$ such that
	\begin{equation}
		\sum_{i\in\zeta}\alpha_i\nequiv{n}\beta.
	\end{equation}
\end{prp}
\begin{proof}
	Choose any fixed $a_0\in\alpha_0$ and define $\alpha=\sum_{i\in\zeta}\alpha_i$ apply Propositions \ref{prp:M0sum} and \ref{prp:M0sum} respectively to $\alpha^{>a_0}$ and $\alpha^{<a_0}$ to obtain their (respective $n$-equivelants $\beta^+\in\Mzero$ and $\beta^-\in\Mzero$.  It then follows, since $\Mzero$ is closed under finite sums, that if $\beta=\beta^-+\one+\beta^+$ then $\alpha\nequiv{n}\beta$, as required.
\end{proof}

\begin{thm}[L\"auchli and Leonard]
	For every countable $\alpha\in\scattered$, and each $n\in\nats$, there exists a $\beta_n\in\Mzero$ such that $\alpha\nequiv{n}\beta_n$.
\end{thm}
\begin{proof}
	Suppose $\alpha\in\scattered$ is countable.  From Hausdorff's theorem it follows that $\alpha$ is very discrete so we may argue by induction on the $\VD$-rank of $\alpha$. Since $\zero,\one\in\Mzero$, the result holds when $\vdrank(\alpha)=0$.

	Assume now there exists an ordinal $\gamma<\omega_1$ such that if $\vdrank(\alpha)<\gamma$ then, for every $n\in\nats$, there exists a $\beta_n\in\Mzero$ such that $\alpha\nequiv{n}\beta_n$.  Now fix any $n\in\nats$ and  suppose $\vdrank(\alpha)=\gamma$ then, by definition of $\VD$, there exists an $\alpha_i\in\VD$ such that $\vdrank(\alpha_i)<\gamma$, for each $i\in\ints$, such that
	\begin{equation}
		\alpha=\sum_{i\in\zeta}\alpha_i.
	\end{equation}
	By the induction hypothesis there exists, for each $i\in\ints$, some $\alpha^\prime_i\in\Mzero$ such that $\alpha_i\nequiv{n}\alpha_i^\prime$.  It now follows from Lemma \ref{lem:fvsum}, if we choose $\alpha^\prime=\sum_{i\in\zeta}\alpha_i^\prime$, that $\alpha\nequiv{n}\alpha^\prime$.  According to Proposition \ref{prp:M0sumint} there exists a $\beta\in\Mzero$ such that $\alpha^\prime\nequiv{n}\beta$ and, therefore, $\alpha\nequiv{n}\beta$, as required.
\end{proof}

\begin{prp}
	Suppose $\alpha$ and $\beta$ are linear orders and consider the families $\family{\mathfrak{A}_i}{i\in\alpha}$ and $\family{\mathfrak{B}_j}{j\in\beta}$ such that, for each $i\in\alpha$ and $j\in\beta$, $\mathfrak{A}_i,\mathfrak{B}_j$ are regular $L^\prime$-expansions of linear orders.  If $n\in\nats$, $\alpha\nequiv{n}\beta$ and, for each $i\in\alpha$ and $j\in\beta$,
	\begin{equation}
		\mathfrak{A}_i\nequiv{n}\mathfrak{B}_j
	\end{equation}
	then it necessarily follows that
	\begin{equation}
		\sum_{i\in\alpha}\mathfrak{A}_i\nequiv{n}\sum_{j\in\beta}\mathfrak{B}_j.
	\end{equation}
\end{prp}
\begin{proof}
	We describe a winning strategy for $\Right$ in the $n$-round game
	\begin{equation}
		\EF_n\left(\sum_{i\in\alpha}\mathfrak{A}_i,\sum_{j\in\beta}\mathfrak{B}_j\right).
	\end{equation}
	A move $(a,i)$ in $\sum_{i\in\alpha}\mathfrak{A}_i$ by $\Left$ is countered by $\Right$ by first employing his winning strategy in the game $\EF_n(\alpha,\beta)$ to obtain some $j\in\beta$ followed by an application of his winning strategy in the game $\EF_n(\mathfrak{A}_i,\mathfrak{B}_j)$ in order to acquire a $b\in B_j$.  The countermove by $\Right$ is then $(b,j)$.  A similar choice is made in the case that $\Left$ instead plays some $(b^\prime,j^\prime)$ in the opposing sum.
\end{proof}

\begin{lem}\label{lem:rescat}
	The following set is recursively enumerable:
	\begin{equation}
		R\coloneqq\setbuild{(\alpha,\sigma)}{\alpha\in\Mzero\text{ and }\alpha\models\sigma}.
	\end{equation}
\end{lem}
\begin{proof}
	Recursively on $\Mzero$, for each $\alpha\in\Mzero$, by adding finitely many (new) relation symbols to $L$ we shall define a language $L_\alpha\supseteq L$ and first-order theories $S_\alpha$, $T_\alpha$ such that $T_\alpha\supseteq S_\alpha$ is a recursively enumerable complete theory.  The role of each $S_\alpha$ will be that of a finitely axiomatisable theory whose models interpret a finitely axiomatisable linear order.

	If $\alpha=\zero$ then we choose $L_\alpha$ to be the language of linear orders and we define $T_\alpha=S_\alpha$ to consist only of the sentence $\neg\exists x(x=x)$.  While, when $\alpha=\one$, we choose the same $L_\alpha$ as before and define $T_\alpha=S_\alpha$ to consist only of the sentence $\axmlin\wedge\exists x\forall y(x=y)$.

	In both cases above, $T_\alpha$ is recursively enumerable (since it is finite) as well as complete:  $T_\alpha$ determines its models up to isomorphism.

	Assume now that $L_\alpha$, $T_\alpha$ and $S_\alpha$ have been defined for $\alpha\in\set{\alpha_0,\alpha_1}$ and that $T_{\alpha_0},T_{\alpha_1}$ are recursively enumerable (complete) theories.  Suppose that $\alpha=\alpha_0+\alpha_1$ and expand $L$ by adding the unary relation symbols $r_0$ and $r_1$ in order to obtain $L_\alpha$.  These new symbols will serve to ``identify'' the linear orders $\alpha_0$ and $\alpha_1$, respectively, within $\alpha$.  Let $S_\alpha$ consist of exactly the $L_\alpha$-sentences $\axmlin$, $\forall x(r_0(x)\vee r_1(x))$ and $\forall x\forall y(r_0(x)\wedge r_1(y)\rightarrow x<y)$ while also choosing
	\begin{equation}
		T_\alpha=S_\alpha\cup\setbuild{\sigma^{r_0(v)}}{\alpha_0\models\sigma}\cup\setbuild{\sigma^{r_1(v)}}{\alpha_1\models\sigma}.
	\end{equation}

	We now argue that $T_\alpha$ is complete.  Note that if $\mathfrak{M},\mathfrak{N}\models T_\alpha$ then, by definition, their respective order types, $\beta$ and $\delta$, can be decomposed as $\beta=\beta_0+\beta_1$ and $\delta=\delta_0+\delta_1$ such that,
	\begin{equation}
		\beta_i,\delta_i\models\Th(\alpha_i), \quad\text{for }i=0,1.
	\end{equation}
	Therefore $\beta_0\equiv\delta_0$ and $\beta_1\equiv\delta_1$ so that, for each $n\in\nats$, $\Right$ has the (respective) winning strategies $s_n$ and $t_n$ for the games $\EF_n(\beta_0,\delta_0)$ and $\EF_n(\beta_1,\delta_1)$.

	The winning strategy for $\Right$ in the game $\EF_n(\mathfrak{M},\mathfrak{N})$ is laid out as follows.  A move in the $\beta_0$-part of $\mathfrak{M}$ is countered using the strategy $s_n$, as if playing the game $\EF_n(\beta_0,\delta_0)$ while a move in the $\beta_1$-part is similarly countered but using the strategy $t_n$.  If instead $\Left$ plays a member of $\mathfrak{N}$ then a similar methodology may be employed.

	If $\alpha=\alpha_0\cdot\omega$ then we obtain $L_\alpha$ from $L$ by adding a single binary relation symbol $\theta$.  Choose $\Gamma$ to be the (unique) interpretation of $L$ in $L_\alpha$ such that $\Gamma(x<y)$ is the formula $x<y\wedge\neg\theta(x,y)$.

	Let $\Sigma$ consist of the (finitely many) $L_\alpha$-sentences which, together, express the assertion ``$\theta$ is an equivalence relation'' as well as the sentence
	\begin{equation}
		\forall x_0\forall x_1\forall y_0\forall y_1( x_0<y_0\wedge\neg\theta(x_0,y_0)\wedge\theta(x_1,x_0)\wedge\theta(y_1,y_0)\rightarrow x_1<y_1).
	\end{equation}
	so that, in fact, $\theta$ is a congruence and not simply a mere equivalence relation.

	Define $S_\alpha$ to be a finite axiomatisation of the $L_\alpha$-theory
	\begin{equation}
		\Gamma\left[\Th(\omega)\right]\cup\Sigma\cup\set{\axmlin}
	\end{equation}
	and let $T_\alpha=S_\alpha\cup\setbuild{\forall w\sigma^{\theta(w,v)}}{\alpha_0\models\sigma}$.  The latter set of sentences expresses the sentiment that each $\theta$-equivalence class is elementarily equivalent to $\alpha_0$.

	Note, it follows from the induction hypothesis that $T_\alpha$ is recursively enumerable.  The burden is now on us to show that the $L_\alpha$-theory $T_\alpha$ is complete.  Consider any $\mathfrak{M},\mathfrak{N}\models T_\alpha$ then the respective order types, $\beta$ and $\delta$, of $\mathfrak{M}$ and $\mathfrak{N}$ can be decomposed as
	\begin{equation}
		\beta=\beta_0\cdot\gamma_0\qquad\text{and}\qquad\delta=\delta_0\cdot\gamma_1,
	\end{equation}
	such that $\gamma_0\equiv\omega\equiv\gamma_1$ and $\beta_0,\delta_0\models\Th(\alpha_0)$.

	Fix any $n\in\nats$ then we describe a winning strategy for $\Right$ in the game $\EF_n(\mathfrak{M},\mathfrak{N})$.  If $\Left$ plays the move $(a,i)\in\domain{}\beta\times\domain{}\gamma_0$ then $\Right$ chooses a $j\in\gamma_1$ using his winning strategy in the game $\EF_n(\gamma_0,\gamma_1)$.  This is then followed by a choice of $b\in\delta_0$ using his winning strategy in the game $\EF(\beta_0,\delta_0)$.  This determines the countermove $(b,j)$.  If $\Left$ instead plays a move in the opposite structure then a countermove be obtained by $\Right$ in a similar fashion.

	The case when $\alpha=\alpha_0\cdot\dual{\omega}$ is approached in a similar manner: let $\theta$, $L_\alpha$, $\Gamma$ and $\Sigma$ be defined as before.  Choose $S_\alpha$ to be a finite axiomatisation of the theory
	\begin{equation}
		\Gamma[\Th(\dual{\omega})]\cup\Sigma\cup\set{\axmlin}
	\end{equation}
	and define $T_\alpha$ as in the previous case.  It again follows that $T_\alpha$ is recursively enumerable and a similar argument as before will suffice to show that $T_\alpha$ is complete.

	All that remains is that a mechanical listing of the set
	\begin{equation}
		R^\prime\coloneqq\setbuild{(\alpha,\sigma)}{\alpha\in\Mzero\text{ and }\sigma\text{ is an }L_\alpha\text{-sentence such that }T_\alpha\models\sigma}
	\end{equation}
	need be given.  Once this procedure has been established it will follow that $R$ is also recursively enumerable since, for any $\alpha\in\Mzero$ and $\sigma\in L_\alpha$, the problem of whether or not $\sigma$ is a member of $L$ is decidable.

	The list starts with all pairs of the form $(\alpha,\sigma)$ such that $\alpha\in\set{\zero,\one}$ and $\sigma\in S_\alpha$, noting that the aforementioned pairs are finite in number.  Now assume the entries $(\alpha_i,\sigma_i)$, for $i=0,\dotsc,n_0-1$, have been listed.  Continue the list sequentially with entries $(\alpha_i,\sigma_i)$, for $i\geq n_0$, as follows:
	\begin{enumerate}
		\item 	list all pairs of the form $(\alpha_i,\sigma)$ such that $\sigma\in S_{\alpha_i}$ and $i<n_0$ and let $n_1$ denote the resulting number of entries in the list;
		\item 	list all pairs of the form $(\alpha_i, \sigma)$, for $i<n_1$, where $\sigma$ is the direct consequence of an inference rule from $\sigma_{i_0},\dotsc,\sigma_{i_{k-1}}$ and, for each $j<k$, $(\alpha_i,\sigma_{i_j})$ has already appeared in the list and let $n_2$ denote the resulting number of entries;
		\item	list all pairs of the form $(\alpha,\sigma)$ where, for $i,j<n_2$, $\alpha=\alpha_i+\alpha_j$ and $\sigma=\sigma_i^{r_0(v)}\wedge\sigma_j^{r_1(v)}$ and let $n_3$ denote the resulting number of entries in the list;
		\item	list all pairs of the form $(\alpha_i\cdot\omega,\sigma)$, for $i<n_3$, such that $\sigma=\forall w\sigma_i^{\theta(w,v)}$ and and let $n_4$ denote the resulting number of entries in the list;
		\item	list all pairs of the form $(\alpha_i\cdot\dual{\omega},\sigma)$, for $i<n_4$, such that $\sigma=\forall w\sigma_i^{\theta(w,v)}$ and and redefine $n_0$ to be the resulting number of entries in the list;
		\item	repeat steps 1 to 5.
	\end{enumerate}

	By way of induction (on $\alpha\in\Mzero$) it can be shown that each member of the set $R^\prime$ does indeed appear as some entry in the above list.  Consequently, we may conclude that the set $R$ is in fact recursively enumerable, as required.
\end{proof}

\begin{thm}
	The theory $\Th(\scattered)$ is decidable.
\end{thm}
\begin{proof}
	From Proposition \ref{prp:rescat} and Lemma \ref{lem:rescat}, there exists machines $M_0$ and $M_1$ which, respectively, list the members of $\Th(\scattered)$ and the set of pairs
	\begin{equation}
		R=\setbuild{(\alpha,\sigma)}{\alpha\in\Mzero)\text{ and }\alpha\models\sigma}.
	\end{equation}

	Now consider a machine $M$ which, when given a sentence $\sigma$, alternately produces entries from each list until either $\sigma$ or, for some $\alpha\in\Mzero$, $(\alpha,\neg\sigma)$ appears in the list, at which point the machine halts and returns the value "True" if $\sigma$ appeared on the list as member of $\Th(\scattered)$ and returns "False" otherwise.

	We now need to argue that $M$ always halts irrespective of the choice of input sentence.  The case where $\sigma\in\Th(\scattered)$ is trivial so suppose that $\sigma\notin\Th(\scattered)$.  By definition, there must exist an $\alpha\in\scattered$ such that $\alpha\models\neg\sigma$.

	Choose $n=\qrank(\sigma)$ and recall it follows from the theorem of L\"auchli and Leonard that, for some $\beta_n\in\Mzero$, it holds that $\beta_n\nequiv{n}\alpha$ and thus $\beta_n\models\neg\sigma$.  Consequently, $(\beta_n,\neg\sigma)\in R$ and thus $(\beta_n,\neg\sigma)$ will appear in the list and $M$ will return "False".  Therefore, the machcine $M$ always halts --- as required.
\end{proof}

\bibliography{references}

	\chapter{Dense linear orders}


\section{The shuffle operation}

\begin{prp}\label{prp:ratpart}
	Suppose $n\in\nats$ and $n\geq 2$ and let $P$ bet a set of $n-1$ primes $p\in\nats$.  Choose
	\begin{equation}
		D_p=\setbuild{q=\faktor{a}{p^k}}{a\in\ints\text{ and }k\in\ints\setminus\set{0}},\quad\text{for each }p\in P.
	\end{equation}
	If $P=\set{p_0,\dotsc,p_{n-2}}$ and $Q_i=\setbuild{D_p}{p\in P}$ when $i<n$ and $Q_n=\rats\setminus\bigcup_{i<n}Q_i$ then $\setbuild{Q_i}{0\leq i<n}$ is a partition of $\rats$ such that $Q_i$ is dense in $\eta$, for each $i=0,\dotsc,n-1$.
\end{prp}

\begin{dfn}[Canonical partition]
	The partition defined in proposition \ref{prp:ratpart} will be referred to as the \textbf{canonical partion} of the rationals into dense subsets.
\end{dfn}

\begin{cor}
	For every natural $n\geq 2$ there exists a partition $\mathcal{R}=\set{R_0,\dotsc,R_{n-1}}$ of $\reals$ such that:
	\begin{enumerate}
		\item $R_{n-1}=\irrats$,
		\item $R_k$ is a countable dense subset of $\lambda$, for each natural $k<n-1$.
	\end{enumerate}
\end{cor}

\begin{dfn}[The shuffle operation]
	Suppose $n\in\nats$ and let $F=\set{\alpha_0,\dotsc,\alpha_{n-1}}$ be an arbitrary set of $n$ linear orders.  Now suppose the following:
	\begin{enumerate}
		\item $\set{Q_0,\dotsc,Q_{n-1}}$ is the canonical partition of $\rats$,
		\item $h\colon\rats\to F$ is a colouring of $\eta$ such that $\inv{h}[\alpha_i]=Q_i$, for each natural $i<n$.
	\end{enumerate}
	We then define the \textbf{shuffle} $\sigma(F)$ of the set $F$ to be the linear order
	\begin{equation}
		\sigma(\set{\alpha_0,\dotsc,\alpha_{n-1}})=\sum_{q\in\eta}h(q),
	\end{equation}
	and we refer to the resulting (class) map $\sigma$ as the \textbf{shuffle operation} induced by the aforementioned partition.
\end{dfn}


\section{The L\"auchli and Leonard result for the class of linear orders}

\begin{dfn}[The class $\M$]
	The class $\M$ is the smallest class of linear orders such that the following holds:
	\begin{enumerate}
		\item	$\zero,\one\in\M$,
		\item	if $\alpha,\beta\in\M$ then $\alpha+\beta\in\M$,
		\item	if $\alpha\in\M$ then $\alpha\cdot\omega,\alpha\cdot\dual{\omega}\in\M$,
		\item	for every finite $F\subseteq\M\setminus\set{\zero}$, it holds that $\sigma(F)\in\M$.
	\end{enumerate}
\end{dfn}

\begin{prp}\label{prp:Msumint}
	If $\alpha_i\in\M$ for each $i\in\ints$ then there exists a $\beta\in\M$ such that
	\begin{equation}
		\sum_{i\in\zeta}\alpha_i\nequiv{n}\beta.
	\end{equation}
\end{prp}
\begin{proof}
	Since $\M$ is closed under finite sums as well as right multiplication by $\omega$ and $\dual{\omega}$, results analagous to propositions \ref{prp:M0sum} and \ref{prp:M0sumop} can be proven in the context of $\M$.  An argument similar to that which appears in the proof of proposition \ref{prp:M0sumint} then yields the desired result.
\end{proof}

\begin{thm}[L\"auchli and Leonard]
	For every countable linear order $\alpha$, and every $n\in\nats$, there exists some $\beta_n\in\M$ such that $\alpha\nequiv{n}\beta_n$.
\end{thm}
\begin{proof}
	Define a binary relation $R$ on $\alpha$ such that, for every $x,y\in\alpha$, $xRy$ iff $x\leq y$ and there exists a $\beta\in\M$ such that $[x,y]\nequiv{n}\beta$.  Since $R$ is clearly transitive (simply note $\M$ is closed under finite sums), it induces a congruence $\sim$ on $\alpha$.    We now claim the following:
	\begin{claim}
		For every $\gamma\in\faktor{\alpha}{\sim}$, there exists a $\beta\in\M$ such that $\gamma\nequiv{n}\beta$.
	\end{claim}
	\begin{proof}
		Fix any $\gamma\in\faktor{\alpha}{\sim}$ and note that $\M$ contains all finite ordinals so we may assume, without loss of generality, that $\gamma$ is infinite.  Consequently, at least one of $\omega$ and $\dual{\omega}$ is embeddable in $\gamma$.  There must then exist a family $\family{\gamma_i}{i\in\zeta}$ of (possibly empty) linear orders such that
		\begin{equation}
			\gamma\cong\sum_{i\in\zeta}\gamma_i.
		\end{equation}
		By definition of $\sim$ it now follows that there exists a $\gamma_i^\prime\in\M$, for each $i\in\ints$, such that $\gamma_i\nequiv{n}\gamma_i^\prime$.  Proposition \ref{prp:Msumint} then implies that there exists a $\gamma^\prime\in\M$ such that
		\begin{equation}
			\gamma^\prime\nequiv{n}\sum_{i\in\zeta}\gamma_i^\prime.
		\end{equation}
		From lemma \ref{lem:fvsum} we may conclude that
		\begin{equation}
			\gamma\nequiv{n}\sum_{i\in\zeta}\gamma_i^\prime\nequiv{n}\gamma^\prime,
		\end{equation}
		establishing the claim.
	\end{proof}

	Note that if $\card{\faktor{\alpha}{\sim}}=1$ then our task is complete so suppose, by way of contradiction, that $\card{\faktor{\alpha}{\sim}}>1$.
	\begin{claim}
		$\faktor{\alpha}{\sim}$ is dense.
	\end{claim}
	\begin{proof}
		With our goal being a contradiction, suppose that $\faktor{\alpha}{\sim}$ is not dense.  Therefore, there exists $a,b\in\alpha$ such that $a<b$, $a\not\sim b$ and for every $c\in(a,b)$ either $c\sim a$ or $c\sim b$ but not both.  Note that, since $a\not\sim b$, the interval $(a,b)$ cannot be finite.  Therefore, $(a,b)\cong\sum_{i\in\zeta}\gamma_i$ where $\gamma_i$ is $n$-equivalent to some member $\gamma_i^\prime$ of $\M$, for each $i\in\ints$.  Thus, by proposition \ref{prp:Msumint}, $(a,b)\nequiv{n}\gamma$ for some $\gamma\in\M$.  Since $\one+\gamma+\one\nequiv{n}[a,b]$ by lemma \ref{lem:fvsum} and $\one+\gamma+\one$ is a member of $\M$, it follows by definition that $a\sim b$ --- a contradiction.
	\end{proof}

	Now choose $K=\set{\chi_0,\dotsc,\chi_{m-1}}$, for some $m\in\posnats$, to be an $n$-spectrum for the class $\setbuild{\chi\in\M}{\chi\nequiv{n}\gamma,\text{ for some }\gamma\in\faktor{\alpha}{\sim}}$.
	\begin{claim}
		There exists a maximal $F\subseteq K$ and a nonempty interval $I=(A,B)$ of $\faktor{\alpha}{\sim}$ such that, for every $\chi\in F$, the set $D_\chi(I)=\setbuild{\gamma\in I}{\gamma\nequiv{n}\chi}$ is dense in $I$.
	\end{claim}
	\begin{proof}
		Argue by induction on $m$.  If $m=1$ then, since $\faktor{\alpha}{\sim}$ is dense we can choose any $A,B\in\faktor{\alpha}{\sim}$ such that $A<B$ and it will immediately follow that if $I=(A,B)$ then $D_{\chi_0}(I)$ is dense in $I$, since every member of $\faktor{\alpha}{\sim}$ is $n$-equivalent to $\chi_0$.

		Assume now that the claim holds for each $m<m^\prime\in\posnats$.  Let $m=m^\prime$ and suppose to the contrary that for every $F^\prime\subseteq K$ and every (nonempty) open interval $J\subseteq\faktor{\alpha}{\sim}$ there exists some $\chi_k\in F^\prime$ such that $0\leq k<m$ and $D_{\chi_k}(J)$ is not dense in $J$.  Note that, by assumption, there must exist a $\chi\in K$ such that $D_{\chi}(\faktor{\alpha}{\sim})$ is not dense in $\faktor{\alpha}{\sim}$.  Without loss of generality we may assume that $\chi=\chi_{m-1}$ and consequently there exists a (nonempty) open subset $J$ of $\faktor{\alpha}{\sim}$ such that $D_{\chi_{m-1}}(J)$ is not dense in $J$.  Note in particular that we can choose $J$ such that $\chi_{m-1}$ has no $n$-equivalent in $J$.

		By definition, there exists an $F^\prime\subseteq\set{\chi_0,\dotsc,\chi_{m-2}}$ which is an $n$-spectrum for the class $\setbuild{\chi^\prime\in\M}{\chi\nequiv{n}\gamma\text{ for some }\gamma\in J}$.  It then follows from the inductive hypothesis that there exists a maximal $F\subseteq F^\prime$ and some open interval $I=(A,B)\subseteq J$, where $A<B$, such that $D_{\chi}(I)$ is dense in $I$, for each $\chi\in F$.  Since $\chi_{m-1}\notin J\supseteq I$, it follows that $\chi_{m-1}$ is not dense in $I$ and, consequently, $F\subseteq K$ is maximal, as required.  This is the sought-after contradiction, thus establishing the claim.
	\end{proof}
	Note that $F$ is now an $n$-spectrum for the class
	\begin{equation}
		\setbuild{\chi\in\M}{\chi\nequiv{n}\gamma\text{ for some }\gamma\in I}.
	\end{equation}
	Consequently, since $D_{\chi}(I)$ is dense in $I$ for each $\chi\in F$, it follows that
	\begin{equation}
		\sigma(F)\nequiv{n}\sum I=\sum_{\gamma\in I}\gamma.
	\end{equation}
	Therefore, since $\sigma(F)\in\M$, it follows that $I$ has exactly one element.  This then contradicts the fact that $I$ is an open interval.  Hence, we may now conclude that $\faktor{\alpha}{\sim}$ has only one element, yielding the result.

\end{proof}

\begin{lem}
	The set $\setbuild{(\alpha,\sigma)}{\alpha\in\M\text{ and }\alpha\models\sigma}$ is recursively enumerable.
\end{lem}
\begin{proof}
	The proof is near identical to that of Lemma \label{lem:rescat}.  We shall only highlight the additions necessary to prove the current result.

	It is necessary to add an additional clause, respectively, to the defintions of $L_\alpha$, $S_\alpha$ and $T_\alpha$.  If, for some $n\in\nats$, there exists $\alpha_0,\dotsc,\alpha_{n-1}\in\M\setminus\set{\zero}$ such that $\alpha=\sigma(\alpha_0,\dotsc,\alpha_{n-1})$ then $L_\alpha$ is obtained from $L$ by adding new unary relation symbols $r_0,\dotsc,r_{n-1}$ and a new binary relation symbol $\theta$.

	We choose $S_\alpha$ to consist of sentences expressing that ``$\theta$ is a congruence relation'' as well as, for each $i<n$, the sentences:
	\begin{itemize}
		\item	$\exists x r_i(x)$,
		\item	$\forall x\big(r_0(x)\vee\dotsb\vee r_{n-1}(x)\big)$,
		\item	for each $j<i$: $\neg\exists x\big(r_i(x)\wedge r_j(x)\big)$,
		\item	$\forall x\forall y\big(x<y\rightarrow\exists z(r_i(z)\wedge x<z \wedge z<y))\big)$,
		\item	$\forall x\big(r_i(x)\rightarrow\forall y(\theta(y,x)\rightarrow r_i(y))\big)$.
		\item	$\forall x\big(\exists y(\neg\theta(x,y)\wedge y<x)\wedge\exists y(\neg\theta(x,y)\wedge x<y)\big)$
	\end{itemize}

	Define the $L_\alpha$-theory
	\begin{equation}
		T_\alpha=S_\alpha\cup\bigcup_{i<n}\setbuild{\forall x\left(r_i(x)\rightarrow\sigma^{\theta(v,x)}\right)}{\alpha_i\models\sigma}
	\end{equation}
	and note that, by definition, $T_\alpha$ is recursively enumerable.  Hence, we need only show that $T_\alpha$ is complete.  To this end, consider any $\mathfrak{M},\mathfrak{N}\models T_\alpha$.

	Fix any $k\in\nats$, so that our task is reduced to showing that $\mathfrak{M}\nequiv{k}\mathfrak{N}$.  This is facilitated by the following description of a winning strategy for player $\Right$ in the game $\EF_k(\mathfrak{M},\mathfrak{N})$.

	Should it be the case that $n\leq 1$ then it follows that $\mathfrak{M}\equiv\mathfrak{N}$ so we may suppose that $n>1$.  Let $a_0\in M$ be the first move of player $\Left$.  By definition of $T_\alpha$, there exists a natural $i_0<n$ such that $\mathfrak{M}\models r_{i_0}(a_0)$.

	It then follows from the definition of $T_\alpha$ that there exists a countermove $b_0\in N$ for player $\Right$ such that $\mathfrak{N}\models r_{i_0}(b_0)$.  Any such $b_0$ will suffice and a similar tactic is used if $\Left$ instead chooses his first move in $\mathfrak{N}$.

	Suppose now that $\bar{a}\in M^{k-1}$ and $\bar{b}\in N^{k-1}$ and we are in the position $(\bar{a},\bar{b})$ in the game.  Without loss of generality we may assume $a_i<a_{i+1}$, for $i<k$, and need only consider the case where $\Left$ never repeats any of his moves.

	Suppose, during the $k$-th round, $\Left$ plays some $c\in M$.  There must then exist, by definition of $T_\alpha$, some $i_k<n$ such that $\mathfrak{M}\models r_{i_{k-1}}(c)$.  If $c>a_{k-1}$ then there must exists some $d\in N$ such that $d>\bar{b}$ and $\mathfrak{N}\models r_{i_{k-1}}(d)$.  A similar scenario is encountered if $c<a_0$: there exists a $d<\bar{b}$ in $\mathfrak{M}$ such that $\mathfrak{N}\models r_{i_{k-1}}(d)$.

	Now, consider the case where $a_j<c<a_{j+1}$, for some $j<k$.  If $b_{j+1}\leq b_j$ then $\Right$ may play any element but otherwise there exists a $d\in N$ such that $b_j<d<b_{j+1}$ and, for $i<n$,
	\begin{equation}
		\mathfrak{M}\models r_i(c)\quad\iff\quad\mathfrak{N}\models r_i(d).
	\end{equation}

	Should $\Left$ instead play a move $d\in N$ then $\Right$ plays as follows.  If $b_{j+1}\leq b_j$, for some $j<k$ then $\Right$ may play any move in $\mathfrak{M}$ but otherwise $\Right$ plays similarly to the previous case where $\Left$ chose a move in $\mathfrak{M}$.  The roles of the two structures are merely interchanged.
\end{proof}

\begin{thm}
	The theory of linear orders is decidable.
\end{thm}


\section{Model-theoretic aspects of the class \text{$\dense$}}

\begin{thm}[Cantor's Theorem]\label{thm:cantor}
	If $\alpha\in\dense$ is countable then there exists an isomorphism $f\colon\alpha\to\eta$.
\end{thm}
\begin{proof}
	We will construct the required isomorphism by means of the game $\EF_\omega(\alpha,\eta)$. It is reasonable to assume that $\Left$ never plays an element more than once in the game $\EF_\omega(\alpha,\eta)$ as $\Right$ could then simply respond by copying his previous countermove to that element.

	Regardless of what initial move  $\Left$ makes, $e_0$ say, we let $\Right$ respond with $a_0$ if $e_0\in\rats$ and with $b_0$ if $e_0\in\alpha$.  Suppose now that we are on round $n$ and $\Left$ has just played the element $e_{n-1}$.  Label the elements in $\alpha$ that have already been played as $\nu_0,\dotsc,\nu_{n-2}$ and the elements that have been played in $\eta$ as $\xi_0,\dotsc,\xi_{n-2}$.  Without loss of generality we may assume $\nu_0<\dotsb<\nu_{n-2}$ and $\xi_0<\dotsb<\xi_{n-2}$.  We consider only the case $e_{n-1}\in\rats$ as the case $e_{n-1}\in\alpha$ is argued similarly.  Since $\alpha$ has no least or greatest element, if $e_{n-1}>\xi_{n-2}$ then $\Right$ plays the first entry in $\bar{a}$ that has yet to be played and is greater than $\nu_{n-2}$ and if $e_{n-1}<\xi_0$ then $\Right$ plays the first entry in $\bar{a}$ which is less than $\nu_0$ but has yet to be played.  Otherwise, there exists some $i_0<n-1$ such that $\nu_{i_0}<e_{n-1}<\nu_{i_0+1}$.  In this latter scenario we let $\Right$ play the first entry in $\bar{a}$ that has not yet been played and is a member of the interval $(\xi_{i_0},\xi_{i_0+1})$.

	Suppose the resulting play of the game is now $(\bar{c},\bar{d})$.  Since $\Left$ never played an element twice, we can define a function $f\colon\eta\to\alpha$ such that $f(c_i)=f(d_i)$, for each $i<\omega$.  Let $f_n$ denote the restriction $f_n=f\restriction\set{c_0,\dotsc,c_{n-1}}$, for each $n\in\nats$.  We aim now to show that $f$ is the desired isomorphism via the following claim:
	\begin{claim}
		For each $n\in\nats$, the map $f_n\colon\set{c_0,\dotsc,c_{n-1}}\to\set{d_0,\dotsc,d_{n-1}}$ is an isomorpshim.
	\end{claim}
	\begin{proof}
		The case $n=0$ is immediate from the definition.  Suppose now that $f_{n-1}$ is an isomorphism for some $n\in\nats$ and fix any $i<n-1$.  If $c_i<c_{n-1}$ then, by definition, $d_i=f(c_i)<d_{n-1}=f(c_{n-1})$. Similarly, if $c_{n-1}<c_i$ then $d_{n-1}=f(c_{n-1})<d_i=f(c_i)$.  It then follows necessarily from the inductive hypothesis that $f_n$ is a homomorphism.  From these observations one may also conclude that $f_n$ is bijective and thus in fact an isomorphism.
	\end{proof}
	It now remains but a simple exercise to show that, since each $f_n$ for $n\in\nats$ is an isomorphism, that $f$ itself is also an isomorphism.
\end{proof}

\begin{prp}\label{prp:xcantor}
	If $\alpha\in\dense$ is countable and, for some $n\in\nats$, we fix any $\bar{a}\in\domain{n}\alpha$ and any $\bar{b}\in\rats^n$ then there exists an isomorphism $f\colon\alpha\to\eta$ such that $f[\bar{a}]=\bar{b}$.
\end{prp}
\begin{proof}
	We prooceed by means of induction, nothing that the case $n=0$ is Cantor's Theorem (Theorem \ref{thm:cantor}).  Suppose now the result holds for each $k\leq n$ for some arbtrary $n\in\nats$ and note that we may assume $\bar{a}=(a_0,\dotsc,a_n)$ and $\bar{b}=(b_0,\dotsc,a_n)$ such that $a_i<a_j$ and $b_i<b_j$, whenever $i<j<\omega$.  It now follows from the inductive hypothesis that there exists isomoprhisms $f_0\colon\alpha^{<a_0}\to\eta^{<b_0}$ and $f_1\colon(\alpha^{>a_0},a_1,\dotsc,a_n)\to(\beta^{>a_0},b_1,\dotsc,b_n)$.  By identifying $f_0$ and $f_1$ with their (respective) underlying sets, we may define the new map $f\colon\alpha\to\beta$ such that $f=f_0\cup\set{(a_0,b_0)}\cup f_1$.  By definition, we have thusly obtained an order-preserving map $f\colon(\alpha,\bar{a})\to(\beta,\bar{b})$.  It is not difficult to see, simply from its definition, that $f$ is in fact an isomorphism $f\colon\alpha\to\beta$ satisfying $f[\bar{a}]=\bar{b}$, as required.
\end{proof}

As a consequence of the previous proposition we get the following:
\begin{prp}
	The linear order $\eta$ is saturated.
\end{prp}
\begin{proof}
	Choose any finite $A\subseteq\rats$.  Suppose that $\Phi(x)\subseteq L_1(A)$ is an $1$-type of $\eta$ over $A$.  By definition there exists a $\alpha\succcurlyeq\eta$ such that $\Phi$ is realised by some $b\in\alpha$ i.e.\ $\alpha\models\Phi(b)$.  By the Downwards L\"owhenheim-Skolem Theorem, however, we may assume that $\alpha$ is countable.  We may assume that $A=\set{a_0,\dotsc,a_{k-1}}$ and thus, from Proposition \ref{prp:xcantor}, if $\bar{a}=(a_0,\dotsc,a_{k-1})$ then there must exist an isomorphism $f\colon(\eta,\bar{a})\to(\alpha,\bar{a})$.  Therefore, it must follow that $\eta\models\varphi(f(b))$, for each $\varphi\in\Phi(x)$, and thus $\Phi(x)$ is realised by the element $f(b)\in\rats$.
\end{proof}


\begin{thm}
	The linear order $\eta$ is a prime model of the theory $\Th(\dense)$.
\end{thm}
\begin{proof}
	Suppose $\alpha\models\Th(\dense)$ then $\alpha$ must be infinite and thus, from the downward L\"owenheim-Skolem Theorem, there exists a countable $\alpha_0\preccurlyeq\alpha$.  From Cantor's theorem it follows that $\alpha_0\cong\eta$ and, therefore, there must exist an elementary embedding of $\eta$ into $\alpha$.
\end{proof}

\begin{thm}
	The theory $T=\Th(\dense)$ has quantifier elimination.
\end{thm}
\begin{proof}
	By Lemma \ref{lem:qelim}, it is enough to show that $T$ is substructure complete.  Suppose that $\alpha\models T$ and choose any $\alpha_0\subseteq\alpha$.  Define $A=\domain{}\alpha_0$ and suppose that $\mathfrak{M},\mathfrak{N}\models T\cup\diag{\alpha_0}$ are $L(A)$-structures.  Note that we may assume, without loss of generality, that $A\subseteq M,N$ such that $\mathfrak{M}=(\beta,A)$ and $\mathfrak{N}=(\gamma,A)$ for linear orders $\beta,\gamma\models T$.  We are now required to show that
	\begin{equation}
		\mathfrak{M}\equiv\mathfrak{N}.
	\end{equation}
	However, it is enough to simply prove that
	\begin{equation}
		(\beta,B)\equiv(\gamma,B),\quad\text{for every finite }B\subseteq A.
	\end{equation}

	Suppose that $B\subseteq A$ is finite then by the downwards L\"owenheim-Skolem Theorem there exists countable $\beta_0\elmsub\beta$ and $\gamma_0\elmsub\gamma$ such that $B\subseteq\beta_0,\gamma_0$.  We may assume that the elements of $B$, for some $n\in\nats$, are precisely $b_0<b_1<\dotsb<b_{n-1}$.  It now follows from Cantor's Theorem that each of the intervals $(b_0,b_1)_{\beta_0},\dotsc,(b_{n-2},b_{n-1})_{\beta_0}$ as well as $\beta_0^{<b_0}$ and $\beta_0^{>b_{n-1}}$ are isomorphic to $\eta$.  The same is true of the corresponding intervals in $\gamma_0$.  This then gives rise to isomorphisms $f_0,\dotsc,f_n$ (whose domains we will choose to be subsets of $\domain{}\beta_0$) between the corresponding open intervals of $\beta_0$ and $\gamma_0$ such that each member of the range of $f_i$ is less (in $\gamma_0$) than every member member of the range of $f_j$ whenever $i<j\leq n$.  One can then define (in an obvious manner) an isomorphism $f\colon\beta_0\to\gamma_0$ which is an extension of $f_i$ to all of $\beta_0$, for each $i\leq n$, such that $f(b_j)=b_j$ whenever $j<n$.  Therefore, since no element of $B$ is an endpoint of either $\beta_0$ or $\gamma_0$, it follows that $(\beta_0,B)\cong(\gamma_0,B)$ and thus $(\beta_0,B)\equiv(\gamma_0,B)$.  The proof is then concluded by recalling that $\beta_0\elmsub\beta$ and $\gamma_0\elmsub\gamma$.
\end{proof}

\begin{cor}
	The theory $T=\Th(\dense)$ is model-complete.
\end{cor}
\begin{proof}
	Immediate from the previous theorem and Corollary \ref{cor:qemc}.
\end{proof}

We've already seen that $\eta$ is saturated and can thus be thought of from an intuitive standpoint as being very ``thick'' among linear orders.  Contrary to what one might expect, though, the next result shows that $\eta$ is simultaneously also an atomic structure and can therefore be thought of as being very ``thin''!

\begin{prp}
	The linear order $\eta$ is atomic.
\end{prp}
\begin{proof}
	Let $T=\Th(\dense)$ then the result follows from the fact that $T$ admits quantifier elimination and the fact that there are only finitely many inequivalent formulas in finitely many variables in a finitary language.
\end{proof}

	\bibstyle{amsalpha}

\chapter{Complete linear orders and the reals}

	\section{Characteristics of complete linear orders}

	\begin{dfn}[Separable]
		A linear order $\alpha$ is \textbf{seperable} whenever there exists a coutable subset of $\domain{}\alpha$ which is dense in $\alpha$.
	\end{dfn}

	\begin{dfn}[Completeness]
		A linear order $\alpha$ is \textbf{complete} if every (nonempty) subset of $\domain\alpha$ which is bounded above in $\alpha$, has a least upper bound (supremum).
	\end{dfn}

	\begin{dfn}[Continuous linear order]
		If $\alpha\in\dense$ then we call $\alpha$ \textbf{continuous} precisely when it is complete.
	\end{dfn}

	\begin{prp}[Seperability]\label{prp:sep}
		A continuous linear order $\alpha$ is seperable iff there exists an embedding $h\colon\alpha\to\lambda$
	\end{prp}

	\begin{thm}[Characterising the reals]
		uppose $\alpha$ is a continuous linear order.  If $\alpha$ is also seperable then, ncessarily, we must have $\alpha\cong\lambda$.  Therefore a linear order is continuous and seperable iff it is isomorphic to the reals.
	\end{thm}
	\begin{proof}
		Suppose $\alpha$ is a seperable continuous linear order.  From proposition (\ref{prp:sep}) there exists an embedding $h\colon\alpha\to\lambda$.  Since $\alpha$ is seperable there exists some $\beta\subseteq\alpha$ which is dense in $\alpha$.  Note that we must have $\beta\cong\eta\cong h[\beta]$, since $\beta$ is countable and dense.  Now define $\alpha^\prime=\comp(h[\beta])$ so that we must have $\alpha\cong\alpha^\prime$, since $\alpha$ is complete and $\beta$ is dense in $\alpha$.

		Lastly, we will claim that $\alpha^\prime$ is convex.  Suppose to the contrary there exists $a,b\in\alpha^\prime$, with $a<b$ and a $c\in\reals$ such that $a<c<b$ but $c\notin\alpha^\prime$.  Define $c_0=\sup\setbuild{x\in\alpha^\prime}{x<c}$. Since, by definition, we have $c_0\leq c$.  If $c_0=c$ then the completeness of $\alpha^\prime$ guarantees $c=c_0\in\alpha^\prime$.  Assume now instead that $c_0<c$ then, since $h[\beta]$ must be dense in $\alpha^\prime$, there exists some $d\in\beta$ such that $c_0<d<c$.  However, since $h(d)\in\beta^\prime\subseteq\alpha^\prime$, this contradicts the definition of $c_0$ which requires that $h(d)\leq c_0$.  This then concludes the proof.
	\end{proof}

	\begin{thm}\label{thm:rchar}
		Every seperable continuous linear order is isomorphic to $\lambda$.
	\end{thm}


	\section{The  suslin property and the first order theory of the coloured reals}

	Note that in this section we will refer to \textit{coloured linear orders} (linear orders with finitely many relations defined on them) merely as linear orders.  Though all of it applies to "ordinary" monochromatic linear orders as well.  This is done to simplify the wording of some of the results.  The reader should, in particular,,take note how this setting differs from the monochromatic case.  For monochromatic linear orders some the results that follow, in fact, become either trivial or otherwise uninteresting.

	\begin{dfn}[Quasi-separable]
		A continuous linear order $\alpha$ is said to be $\textbf{quasi-separable}$ iff every condensation $\beta\in\dense$ of $\alpha$ contains, as a subset, a dense set of singletons of elements in $\alpha$.
	\end{dfn}

	\begin{dfn}[Suslin property]
		A continuous linear order $\alpha$ is said to posess the \textbf{Suslin property} iff every set of \textit{disjoint open intervals} in $\alpha$ is at most countable.
	\end{dfn}

	Clearly $\lambda$ has the suslin property as any set of open intervals of reals must contain a rational number, due to $\eta$ being dense in $\lambda$.  The more interesting question, posed by Suslin, was whether this property characterises the reals.

	The following lemma illustrates that quasi-separable linear orders are in abundance and among these is order type of the reals.

	\begin{prp}
		If $\alpha$ is a continuous linear order with the suslin property then $\alpha$ must be quasi-separable.
	\end{prp}

	\begin{dfn}[Definably quasi-separable]
		A linear oder $\alpha\in\dense$ is called \textbf{definably quasi-separable} iff each condensation induced by a definable congruence onf $\alpha$ has a dense set of singletons (of elements in $\alpha$).
	\end{dfn}
	In the following definition we call a linear order \textit{definably complete} whenever each of its definable subsets bounded above have a supremum.  Clearly, all complete linear orders are definably complete.

	\begin{prp}\label{prp:qdense}
		For every $n\in\posnats$, there exists a partition $Q_0,\dotsc,Q_{n-1}$ of $\rats$ such that, for each $i=0,\dotsc,n-1$, $Q_i$ is dense in $\eta$.
	\end{prp}

	\begin{prp}\label{prp:csums}
		If $\delta$ is a complete linear order and $\family{\alpha_i}{i\in\delta}$ is a family of complete linear orders then the sum
		\begin{equation}
			\sum_{i\in\delta}\alpha_i,
		\end{equation}
		is also a complete linear order.
	\end{prp}

	\begin{thm}
		Suppose $\alpha\in\dense$ is definably complete and definably quasi-separable. If $\card{\alpha}=\aleph_0$ then, for each $n\in\nats$, $\alpha$ has an $n$-equivalent of order type $\lambda$.
	\end{thm}
	\begin{proof}

		As in the statement, let $\alpha$ be countable, definably complete and definably quasi-separable.  Suppose also that there are $k$ colours defined on $\alpha$, for some fixed $k\in\nats$.  Now choose some $n\in\nats$ and define a (binary) relation $R$ on $\alpha$ such that, for every $a,b\in\alpha$:  $aRb$ iff $a\leq b$ and $(a,b)\nequiv{n}\lambda$ whenever $a\neq b$.  Since sums preserve $n$-equivalence, we may conclude that $R$ is transitive and therefore induces a congruence $\sim$ on $\alpha$.  Note that it follows from definition of $\sim$ that each $\beta\in\faktor{\alpha}{\sim}$ satisfies exactly one of: $\beta\nequiv{n}\one$, $\beta\nequiv{n}\one+\lambda$, $\beta\nequiv{n}\lambda+\one$, $\beta\nequiv{n}1+\lambda+1$ or $\beta\nequiv{n}\lambda$.

		\begin{claim}
			The congruence $\sim$ is a definable (binary) relation and the condensation $\faktor{\alpha}{\sim}$ is dense.
		\end{claim}
		If $\tau$ is the (finite) disjunction of characteristic sentences of $k$-colourings of $\lambda$, let $\varphi(x,y)$ be the formula given by
		\begin{equation}\label{eq:condef}
			x=y\vee(x<y\wedge\tau^{(x,y)})\vee(y<x\wedge\tau^{(y,x)}).
		\end{equation}
		Note, by definition of $\varphi$ as well as that of characteristic sentences, that $\varphi$ is a defining formula for the binary relation $\sim$.  It remains to be shown that $\faktor{\alpha}{\sim}$ is dense.  We suppose to the contrary that there exists $I,J\in\faktor{\alpha}{\sim}$ such that the interval $(I,J)$ is empty.  Therefore, there cannot exists an $c\in\alpha$ such that $I<c<J$ and thus there must exist some $a\in I$ and some $b\in J$ such that $(a,b)=\emptyset$, contradicting the density of $\alpha$.  We have thus established the aforementioned claim.

		Note that if $\card{\faktor{\alpha}{\sim}}=1$ then, since $\alpha$ is countable, it follows from lemma \ref{lem:fvsum} that $\alpha$ is $n$-equivalent to a $k$-colouring the order type
		\begin{equation}
			\sum_{k\in\dual{\omega}}(\one+\lambda)+\one+\sum_{k\in\omega}(\lambda+\one)\cong\lambda,
		\end{equation}
		as required.  Suppose then, by way of contradiction, that $\card{\faktor{\alpha}{\sim}}>1$.
		\begin{claim}
			There exists a (proper) open interval $D$ of $\faktor{\alpha}{\sim}$ and a finite set $\Sigma$ of coloured linear orders, each of which either has order type $\one$ or $\one+\lambda+\one$, such that the following holds:
			\begin{enumerate}[nosep]
				\item for each $\beta\in D$ there exists a $\sigma_\beta\in\Sigma$  such that $\beta\nequiv{n}\sigma_\beta$,

				\item if $\sigma\in\Sigma$ then the set $\setbuild{\beta\in D}{\beta\nequiv{n}\sigma}$ is dense in $D$.
			\end{enumerate}
		\end{claim}

		Define $C=\setbuild{\cha{\beta}{n}}{\beta\in\faktor{\alpha}{\sim}\text{ and }\beta\text{ is neither cofinal nor coinitial in }\alpha}$, then it immediately follows that $C$ is a finite set of sentences and, since $\card{\faktor{\alpha}{\sim}}>1$, $C$ is nonempty.  Now fix some set $\Sigma_C$ of coloured linear orders such that each of its members is at most countable and models precisely one sentence in $C$.  Note that, from its definition, each $\beta\in\Sigma_C$ either has order type $\one$ or, otherwise, order type $\one+\lambda+\one$.

		We will now argue, by means of a contradiction, that for some suitable $\Sigma\subseteq\Sigma_C$ there will exist an interval $D$ of $\faktor{\alpha}{\sim}$ satisfying properties 1 and 2 in the claim above.  Suppose that no such $D$ and $\Sigma\subseteq\Sigma_C$ exists.  We note that, since $\sim$ is definable and $\alpha$ is definably quasi-seperable, that $\card{C}\geq 1$ and then argue by induction on $\card{C}$.  If $\card{C}=1$ then every $\beta\in\faktor{\alpha}{\sim}$ has order type $\one$, since $\alpha$ is definably quasi-seperable and thus $\faktor{\alpha}{\sim}$ must contain a dense set of singletons.  If we now choose any fixed $\beta_0\in\faktor{\alpha}{\sim}$ as well as $D$ any (proper) open subinterval such that $\beta_0\in D$ then $\Sigma=\set{\beta_0}$ and $D$ satisfy properties 1 and 2 above.

		Assume now the claim holds whenever $\card{C}<m$, for some fixed $m\in\posnats$.  Suppose the claim fails when $\card{C}=m$ then, by definition, there exists a $\tau_0\in C$ and an open interval $D^\prime\subsetneq\faktor{\alpha}{\sim}$ such that $\beta\not\models\tau_0$ for any $\beta\in D^\prime$.  Since $\card{C\setminus\set{\tau_0}}<m$, it follows from the inductive hypothesis that there exists an open interval $D\subsetneq D^\prime$ and a $\Sigma\subseteq\Sigma_C$ satisfying properties 1 and 2.   Thus claim 2 holds.

		\smallskip	In order to attain the desired contradiction we will show that $\card{D}=1$, contradicting that $D$ is an open interval.  Since $\alpha$ is definably quasi-separable, there must exist a $\beta_1\in\faktor{\alpha}{\sim}$ which has order type $\one$ and such that $\setbuild{\beta\in\faktor{\alpha}{\sim}}{\beta\nequiv{n}\beta_1}$ is dense in the condensation $\faktor{\alpha}{\sim}$ and thus also dense in $D$.  We will now proceed to construct a coloured linear order $\delta$ of order type $\lambda$ which is $n$-equivalent to $\bigcup D$.  Suppose that $\Sigma=\set{\sigma_0,\dotsc,\sigma_{\ell-1}}$ then we may assume $\sigma_0\nequiv{n}\beta_1$, without loss of generality (as we are free to relabel the elements of $\Sigma$).  Define the map $h\colon\reals\to\Sigma$ such that $h(x)=\sigma_0\nequiv{n}\beta_1$, for each $x\in\irrats$, and $h^{-1}[\sigma]$ is dense in $\lambda$ for each $\sigma\in\Sigma\setminus\set{\beta_1}$.  The latter is possible due to proposition \ref{prp:qdense}.  We now define the (coloured) linear order
		\begin{equation}
			\delta=\sum_{x\in\lambda}h(x),
		\end{equation}
		where the elements of $\delta$ are coloured precisely as in the summands.  Note that, since $\sigma\in\set{\one,\one+\lambda+\one}$ for each $\sigma\in\Sigma$, it follows from proposition \ref{prp:csums} that $\delta$ is a complete linear order.  Additionally, since $h(x)$ has order type $\one$ whenever $x\in\irrats$ it follows that there exists a dense embedding of
		\begin{equation}
			\gamma=\sum_{x\in\eta}h(x)
		\end{equation}
		into $\delta$.  However, since each summand of $\gamma$ is seperable and $\eta$ is countable it follows immediately that $\delta$ is also seperable.  Consequently, from theorem \ref{thm:rchar}, we may conclude that $\delta$ has order type $\lambda$.  By definition of $\delta$, one can conclude via an $n$-game that we must have $\delta\nequiv{n}\bigcup D$.  Therefore, since the equivalence classes of $\sim$ must be convex we may conclude that $\card{D}=1$, which is the required contradiction.
	\end{proof}

	\bibstyle{amsalpha}
\chapter{Suslin lines}

\section{Fundamental concepts}

If $P$ is a partially ordered set and $a\in P$ then we denote by $\down{a}$ the
set
\begin{equation}
	\down{a}=\setbuild{x\in P}{x\leq a}.
\end{equation}
Similarly, we define
\begin{equation}
	\up{a}=\setbuild{x\in P}{x\leq a}.
\end{equation}

\begin{dfn}[Tree]
	A partial order $\mathfrak{T}=(T,<)$ is called a \textbf{tree} whenever it
	has a least element (referred to as a \textit{root}) and, for each $a\in T$, the
	set $\down{a}$ is well-ordered.
\end{dfn}

Equivalently, a tree is a well-founded partial order with a least element.

\begin{dfn}[Subtree]
	We call $\mathfrak{T}^{\prime}$ of a tree $\mathfrak{T}$, whenever it holds
	that $\mathfrak{T}^{\prime}$ is a partially ordered set satisfying
	$T^{\prime}\subseteq T$.
\end{dfn}

\begin{dfn}[Length]
	If $\beta$ is some chain in a tree $\mathfrak{T}$ then the ordinal
	$\length(\beta)$ in Proposition \ref{prp:length} is referred to as the
	\textbf{length} of the chain $\beta$.
\end{dfn}

\begin{dfn}[Branch]
	A \textbf{branch} $\beta$ in a tree $\mathfrak{T}$ is any maximal chain of nodes in $\mathfrak{T}$.
\end{dfn}

\begin{dfn}[Height]
	If $\mathfrak{T}=(T,<)$ is any tree and $a\in T$ then the \textbf{height of
	}$\bm{a}$\textbf{ in }$\bm{\mathfrak{T}}$ is the order type of
	$\mathfrak{T}^{<a}$.
\end{dfn}

Note that, by definition of a tree, any $\mathfrak{T}^{<a}$ is well-ordered and
is thus isomorphic to an ordinal.  Therefore, the height of any element (called
a \textit{node}) of a tree is an ordinal.  One can now also define the related
but distinct concept of height for trees themselves.

\begin{dfn}[Tree height]
	The height $\height(\mathfrak{T})$ of a tree $\mathfrak{T}$ is the least
	ordinal such that, for any ordinal $\alpha>\height(\mathfrak{T})$, there exists
	no embedding $f\colon\alpha\hookrightarrow\mathfrak{T}$.
\end{dfn}

The relationship between these notions of height is then given by the following
proposition:

\begin{prp}
	If $\mathfrak{T}$ is any tree and $\alpha$ is any ordinal then
	$\height(\mathfrak{T})=\alpha$ iff it holds that
	\begin{equation}
		\alpha=\sup_{a\in T}(\height(a)+1).
	\end{equation}
\end{prp}
\begin{proof}
	Consider the case where $\height(\mathfrak{T})$ is a limit ordinal.  It
	follows, for all $a\in T$, that
	\begin{equation}
		h(a)<h(a)+1<h(\mathfrak{T})\quad
	\end{equation}
	and
	\begin{equation}
		\sup_{b\in T}h(b)=h(\mathfrak{T}).
	\end{equation}
	Therefore, we may conclude that
	\begin{equation}
		\sup_{a\in T}\big(h(a)+1\big)=h(\mathfrak{T}).
	\end{equation}

	Suppose now that $\height(\mathfrak{T})$ is a successor ordinal, say
	$\height(\mathfrak{T})=\beta_{0}+1$, then there must exist a branch
	$B_{0}$ in $\mathfrak{T}$ of length $\beta_{0}+1$.  Note that no branch in
	$\mathfrak{T}$ may be strictly longer than $B_{0}$ and thus
	\begin{equation}
		\sup_{a\in T}\big(h(a)+1\big)=\sup_{a\in B_{0}}\big(h(a)+1\big).
	\end{equation}
	Since $B_{0}$ has a maximal element, $m$ say, it necessarily follows that
	\begin{equation}
		\beta_{0}=\height(m),
	\end{equation}
	and thus
	\begin{align}
		\sup_{a\in B_{0}}\big(\height(a)+1\big) & =\height(m)+1 \\
		                                        & =\beta_{0}+1,
	\end{align}
	as required.
\end{proof}

\begin{prp}\label{prp:length}
	If $\beta$ is any chain in some tree $\mathfrak{T}$ then $\beta$ is
	isomorphic to some ordinal $\length(\beta)\leq\height(\mathfrak{T})$.
\end{prp}
\begin{proof}
	Assume by way of contradiction that $\beta$ is not well-ordered.  There must
	then exist a strictly decreasing sequence
	\begin{equation}
		b_{0}>b_{1}>\dotsb
	\end{equation}
	in $\beta$.  However, this implies that $\down{b_{0}}$ contains an infinite
	decreasing sequence despite being well-ordered, the desired contradiction.
	Furthermore, it follows by definition of $h(\mathfrak{T})$ that we must have
	$\length(\beta)\leq h(\mathfrak{T})$, as required.
\end{proof}

\begin{dfn}[Levels]
	If $\mathfrak{T}$ is a tree and $\alpha<\height(\mathfrak{T})$ is an ordinal then the set
	\begin{equation}
		\level{\alpha}(\mathfrak{T})=\setbuild{a\in T}{\height(a)=\alpha}
	\end{equation}
	is referred to as the $\bm{\alpha}$\textbf{-th level} of $\mathfrak{T}$.
\end{dfn}

\begin{dfn}[Normality]
	A \textbf{normal }$\bm{\alpha}$\textbf{-tree} is a tree $\mathfrak{T}$ such that
	\begin{enumerate}
		\item	$\height(\mathfrak{T})=\alpha$,\label{dfn:n1}
		\item	for each $\gamma<\alpha$ it holds that
		      $\card{\level{\gamma}(\mathfrak{T})}\leq\aleph_0$,\label{dfn:n2}
		\item	if $x\in T$ is not maximal then $x$ has infinitely many
		      immediate successors,\label{dfn:n3}
		\item	for each $x\in T$ and each ordinal $\beta<\alpha$, if
		      $\beta>\height(x)$ then there exists some $y\in\level{\beta}(\mathfrak{T})$ such
		      that $y>x$,\label{dfn:n4}
		\item	for every limit ordinal $\beta<\alpha$ and every
		      $x,y\in\level{\beta}(\mathfrak{T})$, if $\mathfrak{T}^{<x}=\mathfrak{T}^{<y}$
		      then $x=y$.\label{dfn:n5}
	\end{enumerate}
	As one might expect, we refer to a tree $\mathfrak{T}$ simply as being
	\textbf{normal} whenever it is $\height(\mathfrak{T})$-normal.
\end{dfn}


\section{On the existence of Suslin lines and Suslin trees}

\begin{dfn}[Suslin line]
	A \textbf{Suslin line} is an inseparable dense linear order
	$\alpha\neq\zero$, without endpoints, such that every set of disjoint open
	subintervals of $\alpha$ is at most countable.
\end{dfn}

Succinctly, a linear order is a Suslin line whenever it is dense, inseparable
and satisfies the Suslin property.

\begin{prp}\label{prp:densesublambda}
	If $\alpha\subseteq\lambda$ is dense in $\lambda$ then $\alpha$ must be
	separable.
\end{prp}
\begin{proof}
	Let $I$ denote the set of open intervals $(q,r)\subseteq\reals$ such that
	$q,r\in\rats$ and $(q,r)\cap\alpha\neq\emptyset$.  Clearly $I$ is countable
	so we may fix an enumeration $\family{(q_{i},r_{i})}{i<\omega}$ of $I$.
	Furthermore, for each $i<\omega$, there exists an
	$s_{i}\in(q_{i},r_{i})\cap\alpha$.

	Let $S=\setbuild{s_{i}}{i<\omega}$ then it follows by definition that $S$ is
	countable.  The aim is now to show that, additionally, $S$ is dense in
	$\alpha$.  Fix any $a,b\in\alpha$ and suppose that $a<b$.  Since $\rats$ is
	dense in $\lambda$, we may now choose $q,r\in\rats$ such that
	\begin{equation}
		a<q<r<b.
	\end{equation}

	Since $\alpha$ is dense in $\lambda$, it follows that
	$(q,r)\cap\alpha\neq\emptyset$. Thus, by definition of $I$, there exists a
	$k<\omega$ such that $(q,r)=(q_{k},r_{k})$.  Therefore, since $s_{k}\in(q,r)$,
	we may conclude that $a<s_{k}<b$, as required.
\end{proof}

\begin{prp}\label{prp:completesuslin}
	If there exists a Suslin line then there exists a complete Suslin line.
\end{prp}
\begin{proof}
	Let $\beta$ be a Suslin line and choose $\alpha=\comp(\beta)$.  Our aim is
	now to show that $\alpha$ is a Suslin line in its own right.  We may assume,
	without loss of generality, that $\beta$ has no endpoints.  Note that $\alpha$
	cannot be separable as Theorem \ref{thm:rchar} would imply that
	$\alpha\cong\lambda$ and thus, from Proposition \ref{prp:densesublambda},
	$\beta$ is separable.  As this contradicts the Suslinity of $\beta$, we may
	conclude that $\alpha$ is inseparable.


	Since a dense linear order is necessarily dense in its completion, any
	family $\family{U_{i}}{i\in I}$ of pairwise disjoint, nonempty open
	subintervals of $\alpha$ gives rise to a corresponding family
	$\family{\beta\cap U_{i}}{i\in I}$ of pairwise disjoint, nonempty open
	subintervals of $\beta$.  Consequently, the Suslinity of $\beta$ implies
	that any such family is at most countable, as required.
\end{proof}

\begin{cor}
	There exists a Suslin line iff there exists a complete Suslin line.
\end{cor}
\begin{proof}
	\forward Refer to Proposition \ref{prp:completesuslin}.

	\backward Immediate, by definition.
\end{proof}

\begin{dfn}[Suslin tree]
	A tree $\mathfrak{T}$ is called a \textbf{Suslin tree} whenever
	$\height(\mathfrak{T})=\omega_1$, all the antichains of $\mathfrak{T}$ are at
	most countable and, for every branch $\beta$ in $\mathfrak{T}$, it holds that
	$\length(\beta)<\omega_1$.
\end{dfn}

\begin{dfn}[Branching point]
	A \textbf{branching point} in a tree $\mathfrak{T}$ is a non-maximal node
	$x$ in $\mathfrak{T}$ such that there exist (necessarily distinct) branches
	$\beta$ and $\gamma$ in $\mathfrak{T}$ satisfying $\beta\cap\gamma=\down{x}$.
\end{dfn}

\begin{dfn}[Siblings]
	Distinct nodes $x$ and $y$ in a tree $\mathfrak{T}$ are \textbf{siblings} of
	one another whenever it holds that $\mathfrak{T}^{<x}=\mathfrak{T}^{<y}$.
\end{dfn}

As the name might suggest, a node $x$ in a tree $\mathfrak{T}$ is a branching
point iff it has at least two distinct immediate successors.  Note that distinct
nodes $z$ and $z^{\prime}$ of non-limit height are siblings iff there exists a
branching point $y$ of $\mathfrak{T}$ which has $z$ and $z^{\prime}$ among its
immediate successors.

Henceforth, if $\mathfrak{T}$ is any tree and $X\subseteq T$, we use
$\mathfrak{T}^{<X}$ to denote the partial order whose domain is
\begin{equation}
	T^{<X}\define\setbuild{y\in T}{y<x\text{ for all }x\in X},
\end{equation}
and whose order relation is that of $\mathfrak{T}$ restricted to $T^{<X}$.  In
the special case $X=\set{x_{0}}$, we simply write $\mathfrak{T}^{<x_{0}}$
in place of $\mathfrak{T}^{<X}$.  Similar notation is employed for the dual
relation $>$ as well as the non-strict counterparts of these relations.

\begin{lem}\label{lem:norm}
	If there exists a Suslin tree then there exists a normal Suslin tree.
\end{lem}
\begin{proof}
	Suppose there exists a Suslin tree $\mathfrak{T}$.  We now construct, in
    stages, a normal Suslin tree from $\mathfrak{T}$.  It already follows by
    definition that $\height(\mathfrak{T})=\omega_1$ and the levels of
    $\mathfrak{T}$ are at most countable, since each level is also an antichain.
    Hence, $\mathfrak{T}$ satisfies both \ref{dfn:n1} and \ref{dfn:n2} in the
    definition of normality.

	We now choose $\mathfrak{T}_0$ to be the tree such that
	$T_0=\setbuild{x\in T}{\card{\up{x}}\geq\aleph_1}$.  That is to say,
	$\mathfrak{T}_0$ is obtained from $\mathfrak{T}$ by discarding all nodes $x$
	such that $\up{x}$ is at most countable.  Regarding $\mathfrak{T}_{0}$'s
	status as a tree, it is sufficient to note that any subset of well-ordered
	set is itself well-ordered.

	Since $\mathfrak{T}_{0}$ is a subtree of $\mathfrak{T}$, it follows that
	\begin{equation}
		\height(\mathfrak{T}_{0})\leq\height(\mathfrak{T})=\omega_{1}.
	\end{equation}
     In order to conclude that $\height(\mathfrak{T}_0)=\omega_1$, it is
     sufficient to note that $\mathfrak{T}_0$ has property \ref{dfn:n4}, in the
     definition of normality, 
    

	Since each branch in $\mathfrak{T}_{0}$ is simply a truncation of a branch
	in $\mathfrak{T}$, the branches in $\mathfrak{T}_{0}$ are certainly at most
    countable.  Hence, $\mathfrak{T}_0$ is a Suslin tree and satifies properties
    \ref{dfn:n1}, \ref{dfn:n2} and \ref{dfn:n4}.

    We now set out to build a Suslin tree $\mathfrak{T}_1$, from
    $\mathfrak{T}_0$, such that it is the minimal extension of
    $\mathfrak{T}_{1}$ satisfying both of the following:
	\begin{enumerate}
        \item If $x\in\mathfrak{T}_{0}$ is either of limit height or is the root
            of $\mathfrak{T}_{0}$, and $X=\mathfrak{T}^{<x}$, then add an
            $a_{x}\in\mathfrak{T}_{1}$, distinct from $x$, such that $a_{x}>X$
            and $a_{x}<\mathfrak{T}_{0}^{>X}$.
		\item Whenever $x$ and $x^{\prime}$ are siblings in $\mathfrak{T}_{0}$,
		      each of limit height, then it follows that $a_{x}=a_{x^{\prime}}$.
	\end{enumerate}

	We are now required to show that antichains in $\mathfrak{T}_1$ are at most
	countable.  Consider an arbitrary antichain $X$ in $\mathfrak{T}_1$.  For each $x\in X$,
	choose $y_{x}$ to be of some successor of $x$.  Define $X^{\prime}$
	to be the set of all such $y_{x}$ then it follows by definition that
	$\card{X}=\card{X^{\prime}}$. Since $X^{\prime}$ is clearly an antichain in
	$\mathfrak{T}_{0}$, it now follows that $X$ is at most countable.  Hence,
	$\mathfrak{T}_{1}$ is a Suslin tree. Furthermore, by definition,
	$\mathfrak{T}_1$ satisfies properties \ref{dfn:n4} and \ref{dfn:n5} in the
	definition of normality.

	We now discard all nodes that aren't branching points by letting
	$\mathfrak{T}_2$ be the subtree of $\mathfrak{T}_{1}$ with domain
	$T_2=\setbuild{x\in T_1}{x\text{ is a branching point of }\mathfrak{T}_1}$.
    Note, for every $x\in T_1$, that $\mathfrak{T}_1$ has uncountably many
    branches $\beta$ such that $x\in\beta$.  If, by way of contradiction, we
    suppose that $\mathfrak{T}_2$ does not have property \ref{dfn:n4} then there
    exists an $a\in T_2$ such that $\mathfrak{T}_2^{\geq a}$ is at most
    countable.  Therefore, there is at most countably many branches in
    $\mathfrak{T}_1$ containing $a$ as a member, a contradiction.

    We now argue that $\mathfrak{T}_2$ also has property 5.  Fix $a,b\in T_2$ of
    limit height in $\mathfrak{T}_2$ such that
    $\mathfrak{T}_2^{<a}=\mathfrak{T}_2^{<b}$.  Let $A=T_2^{<a}$ and
    $B=T_2^{<b}$ then it follows that their respective suprema in
    $\mathfrak{T}_1$ are $a$ and $b$.  Therefore, neither $a$ nor $b$ can have
    an immediate predecessor in $\mathfrak{T}_1$ and thus must have limit height
    in $\mathfrak{T}_1$ as well.  Since every nonbranching point in
    $\mathfrak{T}_1^{<a}$ sits below some $a^\prime\in A$ and, similarly, every
    nonbranching point in $\mathfrak{T}_1^{<b}$ sits below some $b^\prime\in B$
    we may conclude that $\mathfrak{T}_1^{<a}=\mathfrak{T}_1^{<b}$ and thus
    $a=b$, as required.

    Choose $\mathfrak{T}_3$ to be the tree such that $T_3=\setbuild{x\in
    T_2}{h(x)\text{ is a limit ordinal}}$. As $\mathfrak{T}_2$ had no
    uncountable branches or antichains it immediately follows that neither will
    $\mathfrak{T}_3$.  Note also that, for each $\alpha<\omega_{1}$,
    $\mathfrak{T}_{3}$ must have a branch of length at least $\alpha\cdot\omega$
    and, therefore, we must have $h(\mathfrak{T}_{3})=\omega_{1}$, thereby
    making $\mathfrak{T}_3$ a Suslin tree.

	By construction, it follows that each node in $\mathfrak{T}_3$ has
	infinitely many immediate successors, thus satisfying property \ref{dfn:n3}.
	It satisfies properties \ref{dfn:n1} (for $\alpha=\omega_1$) and \ref{dfn:n2},
	by virtue of being a Suslin tree, and inherits properties \ref{dfn:n4} and
    \ref{dfn:n5}, via the construction, from $\mathfrak{T}_2$.  This concludes
    the proof as $\mathfrak{T}_3$ is therefore the desired normal Suslin tree.
\end{proof}

\begin{lem}\label{lem:ltree}
	If there exists a Suslin line then there exists a Suslin tree.
\end{lem}
\begin{proof}
    Suppose $\sigma=(S,<)$ is a Suslin line.  We now construct, by transfinite
    recursion, a tree from closed intervals in $\sigma$, each consisting of at
    least two elements.  First, we choose $I_0=S$ and fix any $I_1=[a_1,b_1]$
    such that $a_1,b_1\in S$ and $a_1<b_1$ have been defined.

    Suppose now that $\alpha$ is an ordinal satisfying $0<\alpha<\omega_1$ and
    assume, for each ordinal $\beta$ satisfying $0<\beta<\alpha$, that
    $I_\beta=[a_\beta,b_\beta]$ and $a_\beta<b_\beta$ for some
    $a_\beta,b_\beta\in S$.  Define
    $E=\setbuild{a_\beta}{\beta<\alpha}\cup\setbuild{b_\beta}{\beta<\alpha}$ and
    note that $E$, by definition, must be countable.  Therefore, since $\sigma$
    is not separable there exist $a_\alpha,b_\alpha\in S$ such that
    $a_\alpha<b_\alpha$ and $I_\alpha=[a_\alpha,b_\alpha]$ is disjoint from $E$.

    Now, defining $T=\setbuild{I_\gamma}{\gamma<\omega_1}$, it follows that $T$
    is uncountable and partially ordered by $\supsetneq$.  We now argue, for
    every $\alpha<\omega_1$, that the set $\down{I_\alpha}$ is well-ordered.
    This will then allow us to conclude that $\mathfrak{T}=(T,\supsetneq)$ is a
    tree.

    It follows from the construction of $\mathfrak{T}$ that if
    $\alpha,\beta<\omega_1$ and $\alpha<\beta$ then either $I_\alpha\supsetneq
    I_\beta$ or $I_\alpha\cap I_\beta=\emptyset$.  Choose any $J\subseteq
    \down{I_{\alpha}}$ then there exists a set $\Gamma$ of ordinals smaller than
    $\omega_{1}$ such that \begin{equation}
    J=\setbuild{I_{\gamma}}{\gamma\in\Gamma}. \end{equation} Since $I_{\alpha}$
    is contained in the intersection of any pair in $\down{I_{\alpha}}$, it
    follows that $\down{I_{\alpha}}$, and thus $J$, is linearly ordered.  Hence,
    if $\gamma_{0}$ is the least member of $\Gamma$ then it necessarily follows
    that $I_{\gamma_{0}}$ is the least element of $\down{I_{\alpha}}$.
    Therefore, since $J$ was arbitrary, it follows by definition that
    $\down{I_{\alpha}}$ is well-ordered.

    We are now required to show that $\mathfrak{T}$ is in fact a Suslin tree.
    That is, we are required to prove that $\height(\mathfrak{T})=\omega_1$ and
    $\mathfrak{T}$ has neither an uncountable antichain nor an uncountable
    branch.

    Recall from earlier, if $\alpha,\beta<\omega_1$ and $\alpha<\beta$ then
    either $I_\alpha\supsetneq I_\beta$ or $I_\alpha$ and $ I_\beta$ are
    disjoint.  From this observation, it follows that if $X\subseteq T$ is an
    antichain then $X$ is a pairwise disjoint set of closed intervals in $S$.
    Choose $X^\prime$ to now be the corresponding set of open intervals i.e.\
    $X^\prime=\setbuild{(a,b)}{[a,b]\in X}$.

    Since $\sigma$ is a Suslin line it now follows that $X^\prime$, and thus
    also $X$, is at most countable.  In order to show that $\mathfrak{T}$ has no
    uncountable branch suppose the contrary: there exists a branch
    $\beta=(B,\supsetneq)$ in $\mathfrak{T}$ such that
    $\length(\beta)=\delta\geq\omega_1$.  We may now assume
    $B=\setbuild{I_{\gamma_{i}}}{i<\delta}$ and $I_{\gamma_i}\supsetneq
    I_{\gamma_j}$ whenever $i<j$.

    If, for each $\gamma<\delta$, we choose $x_\gamma$ to be the left endpoint
    of the interval $I_\gamma$ then it follows that
    $\setbuild{(x_\gamma,x_{\gamma+1})}{\gamma<\delta}$ is an uncountable set of
    pairwise disjoint open intervals in $\sigma$ --- contradicting the fact that
    $\sigma$ is a Suslin line.

    All that remains is to show that $\height(\mathfrak{T})=\omega_1$.  Since
    $\mathfrak{T}$ has no uncountable branch, however, it follows that
    $\height(\mathfrak{T})\leq\omega_1$.  Note also, since levels are
    antichains, each level of $\mathfrak{T}$ must be countable.  Consequently,
    as $\card{\mathfrak{T}}=\aleph_1$, we cannot have
    $\height(\mathfrak{T})<\omega_1$ and thus we may conclude that
    $\height(\mathfrak{T})=\omega_1$, as required.
\end{proof}

\begin{lem}\label{lem:tline}
	If there exists a Suslin tree then there exists a Suslin line.
\end{lem}
\begin{proof}
	Suppose there exists a Suslin tree.  It follows from Lemma \ref{lem:norm}
	that there exists a normal Suslin tree $\mathfrak{T}$.  Since $\mathfrak{T}$ is
	normal, each node $x\in T$ has $\aleph_0$ many immediate successors.  We may
	thus order the successors of every node like $\eta$.  Note now that every
	branch $\beta$ in $\mathfrak{T}$ uniquely determines some transfinite sequence
	$\sigma\in S=\rats^{<\omega_1}=\bigcup_{\alpha<\omega_1}\rats^\alpha$.

	Note, by the maximality of branches, it is never the case that some
	$\sigma_0\in S$ is an initial subsequence of any $\sigma_1\in S$.  Let
	$(S,\prec)$ now denote the set of all such $\sigma$ ordered lexicographically:
	if $\sigma_0,\sigma_1\in S$, $\sigma_0\neq\sigma_1$ and
	$\length(\sigma_0)\leq\length(\sigma_1)$ then we define $\sigma_0\prec\sigma_1$
	whenever, for the least $\alpha<\length(\sigma_0)$ such that
	$\sigma_0(\alpha)\neq\sigma_1(\alpha)$, it holds that
	$\sigma_0(\alpha)<_{\eta}\sigma_1(\alpha)$.

	It should be clear from its definition that $S$ is linearly ordered by
	$\prec$ and lacks endpoints.  It remains to be shown that $(S,\prec)$ is a
	Suslin line.  We now show that $S$ is in fact densely ordered.  If
	$\sigma_0,\sigma_1\in S$ and $\sigma_0\prec\sigma_1$ then it follows from the
	density of $\eta$ that there exists a $\tau\in S$ such that
	$\sigma_0\prec\tau\prec\sigma_1$.

    From here onwards, we identify each $\sigma\in S$ with its corresponding
    branch in $T$ and treat them as such.  We are required to show that $\sigma$
    has the Suslin property:  suppose $\mathcal{I}$ is a pairwise disjoint
    collection of open intervals $(A,B)$ in $S$.  For each $x\in T$, define
    $I_x\subseteq S$ to be the convex set:
	\begin{equation}
		I_x=\setbuild{\sigma\in S}{x\in \sigma}.
	\end{equation}
	Note now that, for $x,y\in T$, $I_x\cap I_y=\emptyset$ iff $x$ and $y$ are
	incomparable.

	Choose any $(\sigma_0,\sigma_1)\in\mathcal{I}$ and let
	$\alpha<\min\set{\length(\sigma_0),\length(\sigma_1)}$ be the least ordinal such
	that $\sigma_0(\alpha)<_{\eta}\sigma_1(\alpha)$.  Now choose any
	$\tau\in(\sigma_0,\sigma_1)$ such that
	$\sigma_0(\gamma)=\tau(\gamma)=\sigma_1(\gamma)$, for each $\gamma<\alpha$, and
	$\sigma_0(\alpha)<_{\eta}\tau(\alpha)<_{\eta}\sigma_1(\alpha)$.  Define $x=\tau(\alpha)$ and
	note that it then follows that $I_x\subseteq(\sigma_0,\sigma_1)$.  Thus, since
	$\sigma_0$ and $\sigma_1$ were arbitrary, we may conclude that for any open
	interval $J\in\mathcal{I}$ there exists a $\tau_J\in S$ and an $x_J\in\tau_J$
	such that $I_{x_J}\subseteq J$.

	By definition of $\mathcal{I}$, it follows that the set
	$\setbuild{I_{x_J}}{J\in\mathcal{I}}$ is pairwise disjoint. Consequently
	$X=\setbuild{x_J}{J\in\mathcal{I}}$ is a pairwise incomparable set of elements
	of $\mathfrak{T}$, i.e.\ $X$ is an antichain of $\mathfrak{T}$, and is therefore
	at most countable.  This implies that $\mathcal{I}$ is also at most countable.

    All that remains is to show that $(S,\prec)$ is not separable.  By way of
    contradiction, assume $D\subseteq S$ is dense in $S$.  It immediately
    follows that $D$ is cofinal in $S$.  From property \ref{dfn:n4} in the
    definition of normality, it follows that there are arbitrarily long branches
    of length less than $\omega_1$ in $\mathfrak{T}$.  Hence, by definition of
    $\prec$, we may conclude that $D$ is uncountable.
\end{proof}

\begin{thm}
	There exists a Suslin line iff there exists a Suslin tree
\end{thm}
\begin{proof}
	This result is the combination of lemmas \ref{lem:ltree} and \ref{lem:tline}.
\end{proof}

	\bibstyle{amsalpha}

\chapter{A La\"uchli and Leonard style result for continuous linear orders}


\section{Continuous linear orders}

\begin{dfn}[Continuous linear order]
	A \textbf{continuous linear order} is a complete linear order $\alpha$ such that $\alpha\in\dense$.  A \textit{coloured linear order} is said to be \textbf{continuous} whenever its monochromatic reduct is a continuous linear order.
\end{dfn}

\begin{dfn}[The class $\C$]\label{dfn:C}
	Let $\C$ be the smallest class of linear orders such that:
	\begin{enumerate}
		\item	$\lambda\in\C$\label{dfn:C1}
		\item	if $\alpha,\beta\in\C$ then $\alpha+\one+\beta\in\C$\label{dfn:C2}
		\item	if $\alpha\in\C$ then $(\alpha+\one)\cdot\omega,(\one+\alpha)\cdot\dual{\omega}\in\C$\label{dfn:C3}
		\item	if $\mathcal{F}\subseteq\C$ is finite and $h\colon\reals\to\mathcal{F}$ is a surjection such that, for each $\alpha\in\mathcal{F}$, the set $\inv{h}[\alpha]$ is dense in $\lambda$ then it follows that
		\begin{equation}
			\sum_{x\in\lambda}(\one+h(x)+\one)\in\C.
		\end{equation}\label{dfn:C4}
	\end{enumerate}
\end{dfn}

\begin{dfn}[The class $\C_k$]
	For each $k\in\nats$, let $\C_k$ denote the class of all $k$-coloured expansions of linear orders belonging to $\C$.
\end{dfn}

\begin{prp}\label{prp:cont}
	If $\alpha$ is a continuous linear order then there exists an embedding $h\colon\lambda\hookrightarrow\alpha$.  Furthermore, $h$ can be chosen such that $h[\reals]$ is an open interval in $\alpha$.
\end{prp}
\begin{proof}
	Since $\alpha$ is dense there exists an embedding $h_0\colon\eta\hookrightarrow\alpha$.  We now set out to extend $h_0$ to an embedding of $\lambda$ into $\alpha$.  Define the map $h\colon\reals\to\domain{}\alpha$ such that, for each $r\in\reals$,
	\begin{equation}
		h(r)=\sup\setbuild{h_0(q)}{q\in\rats\text{ and }q<r}.
	\end{equation}
	It clearly follows by definition that $h\restriction\rats=h_0$.  Note that, since $\eta$ is dense in $\lambda$ there exists a $q_0\in\rats$ such that $x<q_0<y$.  Again invoking the density of the rationals in the reals, there exists a $q_1\in(q_0,y)\cap\rats$.  Since $h_0$ is an embedding, it follows from the definition of $h$ that $h(x)\leq h(q_0)<h(q_1)\leq h(y)$ so that, in particular, it holds that $h(x)<h(y)$ and thus $h$ is an order-preserving map and, therefore, also an embedding $h\colon\lambda\hookrightarrow\alpha$.

	All that remains is to show that there now exists an embedding $h^\prime\colon\lambda\hookrightarrow\alpha$ such that $h^\prime[\reals]$ is a convex set in $\alpha$.  Consider now the open interval $I=(0,1)\subseteq\reals$ and its image $J=h[I]$ under $h$.  We now proceed to show that $J$ is convex.  Note that $h(0)<J<h(1)$ so we may therefore conclude that $J$ is bounded in $\alpha$.  Thus, since $\alpha$ is complete, we may define $a=\inf J$ and $b=\sup J$.  Now fix any $d\in(a,b)_\alpha$ and choose $\ell=\sup\setbuild{r\in\reals}{h(r)<d}$ and $u=\inf\setbuild{r\in\reals}{h(r)>d}$.  By definition, it must hold that $\ell\leq u$.  Suppose that $u<\ell$ then there exists some $r_0\in(\ell,u)$.  Since $h$ is an embedding, it follows that $h(\ell)<h(r_0)<h(u)$.  Now, if $h(r_0)<d$ then it follows by definition of $\ell$ that $r_0\leq\ell$, which is a contradiction.  Similarly, if $h(r_0)>d$ then it follows that $r_0\geq u$, also a contradiction.  Therefore, it follows that $h(r_0)=d$ from which we may conclude that, since $r_0\in(\ell,u)\subseteq I$ it follows that $d\in J$.  Note that if $J$ had a greatest element then its preimage under $h$ would be forced to be $1\in\reals$, contradicting the fact that $1\notin I=(0,1)$.  Similarly, since $0\notin I$, $J$ cannot have a least element either.  Therefore, since it holds that $(0,1)\cong\lambda$, $h$ is an embedding and $h\restriction(0,1)$ maps onto a convex subset of $\alpha$, the result follows.
\end{proof}

\begin{thm}\label{thm:Cll}
	For each $k,n\in\nats$, if $\alpha$ is a continuous $k$-coloured linear order then there exists a $\beta_n\in\C_k$ such that:
	\begin{equation}
		\alpha\nequiv{n}\beta_n
	\end{equation}
\end{thm}
\begin{proof}
	Fix any $k,n\in\nats$ and suppose $\alpha$ is a continuous $k$-coloured linear order.  Define a binary relation $R$ on $\domain{}\alpha$ such that, for every $a,b\in\reals$, $aRb$ iff  $a<b$ and there exists a $\beta\in\C_k$ such that $(a,b)\nequiv{n}\beta$.  It follows from clause \ref{dfn:C2} in definition \ref{dfn:C} that $R$ is transitive and, therefore, induces a congruence $\sim$ on $\alpha$.

	\begin{claim}
		The congruence $\sim$ is definable.
	\end{claim}
	\begin{proof}
		The defining formula $\varphi=\varphi(x,y)$ is of the same form as the formula in (\ref{eq:condef}), with the only differences being that $\varphi$ is a formulated in a language with a finite number of additional (unary) relation symbols and $\tau=\bigvee_{\delta\in\C_k}\cha{\delta}{n}$.
	\end{proof}

	\begin{claim}
		The linear order $\faktor{\alpha}{\sim}$ is dense.
	\end{claim}
	\begin{proof}
		Fix any $I,J\in\faktor{\alpha}{\sim}$ such that $I<J$ i.e.\ $x<y$ for each $x\in I$ and $y\in J$.  Since $\alpha$ is complete, there exists some $a,b\in\alpha$ such that $a=\sup I$ and $b=\inf J$.  Note, if $a=b$ then it follows from clause \ref{dfn:C2} in definition \ref{dfn:C} that $x\sim y$, for every $x\in I$ and $y\in J$, contradicting the definition of $I$ and $J$.  Hence, we may assume that $a<b$.  It now follows from the density of $\alpha$ that there exists a $c\in (a,b)$.  Consequently, it follows that $I<[c]<J$ --- as required.
	\end{proof}

	\begin{claim}
		The linear order $\faktor{\alpha}{\sim}$ is complete.
	\end{claim}
	\begin{proof}
		Suppose $X\subseteq\faktor{\alpha}{\sim}$ is bounded above.  Note that $\bigcup X\subseteq\alpha$ is now bounded above in $\alpha$.  It therefore follows from the completeness of $\alpha$ that there exists a $u\in\bigcup X$ such that
		\begin{equation}
			u=\sup\bigcup X.
		\end{equation}
		Define $I$ to now be the equivalence class $I=[u]$.  Suppose by way of contradiction that there exists a $J\in\faktor{\alpha}{\sim}$ such that $J\geq X$ and $J<I$.  Since $\faktor{\alpha}{\sim}$ is dense there must exist a $K\in\faktor{\alpha}{\sim}$ such that $J<K<I$.  Therefore, there exists an $x_0\in K$ such that $x<x_0<u$ for each $x\in J$.  This implies that $x_0$ is an upper bound of $\bigcup X$ less than $u$, a contradiction.  Therefore, $I$ is in fact the supremum of $X$.
	\end{proof}

	\begin{claim}\label{clm:contInt}
		For every $I\in\faktor{\alpha}{\sim}$ there exists a $\beta\in\C_k$ such that $\Int I\nequiv{n}\beta$.
	\end{claim}
	\begin{proof}
		Note that, by the Downwards L\"owenheim-Skolem Theorem, we may assume $I$ is at most countable.  Consider first the case where $I$ is bounded below but not above in $\alpha$ and choose a cofinal sequence $a=\family{a_i}{i<\omega}$ in $\alpha$ such that $a_0=\inf I$.  If $h$ denotes some colouring of $I$ such that, for $x,y\in I$, $h(x,y)\in\C_k$ and $(x,y)\nequiv{n}h(x,y)$ then it follows from Ramsey's Theorem that there exists a homogeneous subsequence $a^\prime=\family{a^\prime_i}{i<\omega}$ of $a$ for $h$.  It then follows by definition of $a^\prime$ that
		\begin{equation}
			\sum_{i<\omega}(a^\prime_i,a^\prime_{i+1}]\nequiv{n}\left(h(a^\prime_0,a^\prime_1)+\one\right)\cdot\omega.
		\end{equation}
		We now choose $\beta\in\C_k$ such that
		\begin{equation}
			\beta=\begin{cases}
				\left(h(a^\prime_0,a^\prime_1)+\one\right)\cdot\omega,&\text{if }a_0=a^\prime_0,\\
				h(a_0,a^\prime_0)+\one+\left(h(a^\prime_0,a^\prime_1)+\one\right)\cdot\omega,& \text{otherwise}.\\
			\end{cases}
		\end{equation}
		It follows by definition then that $\beta\in\C_k$ and
		\begin{equation}
			\Int I\nequiv{n}\beta,
		\end{equation}
		as required.  The case where $I$ is bounded above but not below in $\alpha$ is obtained dually i.e.\ by considering a cofinal sequence $b=\family{b_i}{i<\omega}$ in $\dual{I}\subseteq\dual{\alpha}$ and repeating the above argument.  In order to establish the claim, the remaining cases are obtained by similar means: considering the relevant cofinal, or coinitial, sequences and taking the appropriate sums.
	\end{proof}

	We now wish to show that $\card{\faktor{\alpha}{\sim}}=1$ so it follows from the previous claim that $\alpha\nequiv{n}\beta_n$, for some $\beta_n\in\C_k$.  To this end, we suppose to the contrary that $\card{\faktor{\alpha}{\sim}}>1$ and note that proposition \ref{prp:cont} implies the existence of an interval $I=(a,b)\subseteq\alpha$ and an embedding $g\colon\lambda\hookrightarrow\alpha$ such that $g[\reals]=I$.  Let $f$ be the surjective homomorphism $f\colon\alpha\to\faktor{\alpha}{\sim}$ induced by the congruence $\sim$ and define $S=\set{\delta_0,\dotsc,\delta_{m-1}}$ to be an $n$-spectrum for the class of $\chi\in\C_k$ such that $\chi\nequiv{n}X$ for some $X\in f[I]$.

	\begin{claim}
		There exists an open interval $J=(a^\prime,b^\prime)\subseteq I$ and a nonempty $S^\prime\subseteq S$ such that, for each $\sigma\in S$, the set
		\begin{equation}
			D_\delta=\setbuild{\faktor{d}{\sim}}{d\in J\text{ and }\faktor{d}{\sim}\nequiv{n}\delta}
		\end{equation}
		is dense in $f[J]$.
	\end{claim}
	\begin{proof}
		Proceed by induction on $m$ and note that, by definition of $I$, every nonempty open interval in $I$ has order type $\lambda$, implying that $I$ contains no singletons.  If $m=1$ then the result follows immediately from the definition of $S$ as then every member of $I$ is $n$-equivalent to the same (unique) linear order in $S$.  Assume the result holds whenever $m<m^\prime$, for some arbitrary $m^\prime\geq 1$.  Suppose to the contrary that $\delta\in S$ and $D_\delta$ is not dense in any open interval $(a^\prime,b^\prime)\subseteq I$.  By definition, there must then exist an open interval $J\subsetneq I$ such that no $X\in J$ satisfies $X\nequiv{n}\delta$.  If we now let $S^\prime=\setbuild{\delta^\prime\in S}{X\nequiv{n}\delta^\prime,\text{ for some }X\in J}$ then it follows by definition that $\delta\notin S^\prime$ and thus $\card{S^\prime}<\card{S}$.  Applying the induction hypothesis now to $J$ and $S^\prime$ yields the desired contradiction, thus establishing the claim.
	\end{proof}

	Note, since $I$ is bounded in $\alpha$, that each $X\in f[I]$ is also bounded in $\alpha$ and thus (by completeness of $\alpha$ and claim \ref{clm:contInt}) $X$ has a least and greatest element.  If we now choose $h$ to be the map $h\colon\reals\to S$ such that, for each $x\in\reals$, $h(x)$ is the interior of the (unique) $n$-equivalent of $fg(x)$ belonging to $S$ then it follows from lemma \ref{lem:fvsum} that
	\begin{equation}
		I\nequiv{n}\sum_{x\in\lambda}(\one+h(x)+\one)\in\C_k.\label{eq:rsum}
	\end{equation}
	Therefore, by definition of $I$, it follows that $a\sim b$ and thus $a,b\in I$, which is the desired contradiction.  Hence $\faktor{\alpha}{\sim}$ has only one element and the result follows from claim 4.
\end{proof}

It is inconvenient, however, that the class $\C$ as it stands, due to the final clause in its definition, is not countable let alone recursively enumerable.  This is due to the uncountability of $\reals$ itself as it guarantees the uncountability of at least one of it equivalence classes under any equivalence relation.  Should $A$ be any such uncountable equivalence class and $B$ any equivalence class distinct from it, uncountably many new partitions can be generated by substituting, for some $r\in A$, the set $A\setminus\set{r}$ for the equivalence class $A$ and the set $B\cup\set{r}$ for the equivalence class $B$ in the chosen partition of $\reals$.

\begin{prp}\label{prp:dsum}
	Suppose, for some $k\in\nats$, $\mathcal{S}$ is a set of $k$-coloured linear orders and fix any $\alpha\in\dense$.  If $h,h^\prime\colon\alpha\to\mathcal{S}$ are surjections such that, for every $\delta\in\mathcal{S}$, $\inv{h}[\delta]$ and $\inv{h^\prime}[\delta]$ are respectively dense in $\alpha$ then it follows, for each $n\in\nats$, that
	\begin{equation}
		\sum_{x\in\alpha}h(x)\nequiv{n}\sum_{x\in\alpha}h^\prime(x).
	\end{equation}
\end{prp}
\begin{proof}
	The case $n=0$ is immediate from the fact that the language of coloured linear orders has no constant symbols so we may assume $n>0$.  Define the linear orders $\beta_0$ and $\beta_1$ such that
	\begin{equation}
		\beta_0=\sum_{x\in\alpha}h(x)\quad\text{and}\quad\beta_1=\sum_{x\in\alpha}h^\prime(x).
	\end{equation}
	We now argue by describing the winning strategy of $\Right$ in the game $\EF_n(\beta_0,\beta_1)$.  On the first round of the game if $\Left$ plays $a_0\in\beta_0$ then $\Right$ responds with an element from a summand of $\beta_1$ which, leveraging the definition of $h^\prime$, corresponds to the summand of $\beta_0$ from which $a_0$ arose.  Should $\Left$ have played some $b_0\in\beta_1$ in stead a similar cuntermove would suffice.

	Consider now a position $(\bar{a},\bar{b})$ of length $\ell$ such that $0<\ell<n$.  Since $\alpha$ is dense, it follows from the definitions of $h$ and $h^\prime$ that (irrespective of $\Left$'s $\ell$-th move) $\Right$ always has available a response from an appropriate summand, of either $\beta_0$ or $\beta_1$, in the opposing $k$-coloured linear order.  If $\Left$ plays $d\in\beta_0$ and $d>\bar{a}$ then $\Right$ responds, from a corresponding summand, with an element $e\in\beta_1$ such that $d>\bar{b}$.  One argues dually when $d<\bar{a}$.  Otherwise, it must hold that $d\in(a_i,a_j)$, for some $i,j<\ell$, such that $a_{i^\prime}\notin(a_i,a_j)$ when $i^\prime<\ell$.  In this latter case it follows again from the definition of $h^\prime$ that $\Right$ has an ``appropriate'' response $e\in\beta_1$ from a summand corresponding to the one from which $d$ arose: meaning $e$ can be chosen such that $e\in(b_i,b_j)$ and $d_{i^\prime}\notin(d_i,d_j)$ when $i^\prime<\ell$.
\end{proof}

\begin{dfn}[Canonical partitions of $\reals$]
	Suppose $n\in\posnats$ then we refer to a partition $\mathcal{R}=\set{R_0,\dotsc,R_{n-1}}$ of $\reals$ as the \textbf{canonical partition of $\reals$ into $\bm{n}$ dense subsets} whenever:
	\begin{enumerate}
		\item	$n=1$ implies $\mathcal{R}=\set{\reals}$,
		\item	if $n>1$ then, for some $k\in\set{0,\dotsc,n-1}$, it holds that $R_k=\irrats$ and $\mathcal{R}\setminus\set{R_k}$ is the canonical partition of $\rats$ into $n-1$ dense subsets.
	\end{enumerate}
\end{dfn}

\begin{dfn}[The class $\Cast$]
	Let $\Cast$ be the smallest class of linear orders such that:
	\begin{enumerate}
		\item	$\lambda\in\Cast$\label{dfn:C1}
		\item	if $\alpha,\beta\in\Cast$ then $\alpha+\one+\beta\in\Cast$\label{dfn:C2}
		\item	if $\alpha\in\Cast$ then $(\alpha+\one)\cdot\omega,(\one+\alpha)\cdot\dual{\omega}\in\Cast$\label{dfn:C3}
		\item	if $\mathcal{F}\subseteq\Cast$ is nonempty and finite while $h\colon\reals\to\mathcal{F}$ is a surjection which induces the canonical partition of $\reals$ into $\card{\mathcal{F}}$ subsets then
			\begin{equation}
				\sum_{x\in\lambda}(\one+h(x)+\one)\in\Cast.
			\end{equation}\label{dfn:C4}
	\end{enumerate}
\end{dfn}

\begin{dfn}[The class $\Cast_k$]
	For each $k\in\nats$, let $\Cast_k$ denote the class of all $k$-coloured expansions of linear orders belonging to $\Cast$.
\end{dfn}

\begin{thm}
	For each $k,n\in\nats$, if $\alpha$ is a continuous $k$-coloured linear order then there exists a $\beta_n\in\Cast_k$ such that:
	\begin{equation}
		\alpha\nequiv{n}\beta_n
	\end{equation}
\end{thm}
\begin{proof}
	Near-identical to the proof of Theorem \ref{thm:Cll}.  The only variation is that one invokes Proposition \ref{prp:dsum} to choose the surjection $h$, occuring in (\ref{eq:rsum}), that induces the required canonical partition of $\reals$ into dense subsets.
\end{proof}


\section{Decidability: the coloured case}

\begin{dfn}[Definable completeness]
	A coloured linear order $\alpha$ is said to be \textbf{definably complete} whenever every nonempty definable subset of $\alpha$, which is bounded above, has a supremum.
\end{dfn}

Note that, for any subset $A$ of a definably complete $\alpha$, if there exists a tuple $\bar{a}$ of $\alpha$ such that $A$ is defined by the formula $\varphi(x,\bar{a})$ and $A$ is bounded bounded \textit{below} then the set of lower bounds of $A$ is defined by the formula
\begin{equation}
	\psi(x,\bar{a})=\forall y\big(\varphi(y,\bar{a})\rightarrow x<y\big).
\end{equation}

Additionally, the set defined by $\psi$ will necessarily be bounded above and thus have a supremum.  It can then be shown that the aforementioned supremum is the infimum of $A$.  Therefore, if a coloured linear order $\alpha$ is definably complete then every definable subset of $\alpha$, which is bounded below, has an infimum.  A similar argument shows that the converse also holds.

\begin{lem}\label{lem:compsum}
	Suppose $\beta$ is a coloured linear order and, for each $i\in\beta$, $\alpha_i$ is a complete coloured linear order then the following holds:
	\begin{enumerate}
		\item	If $\beta$ is complete and, for each $i\in\beta$, $\alpha_i$ has both a least and a greatest element then $\sum_{i\in\beta}\alpha_i$ is complete.
		\item	If $\beta$ is well-ordered and, for each $i\in\beta$, $\alpha_i$ has a least element then $\sum_{i\in\beta}\alpha_i$ is complete.
	\end{enumerate}
\end{lem}
\begin{proof}
	\begin{enumerate}[nosep]
		\item	Suppose $B$ is a nonempty subset of $\sum_{i\in\beta}\alpha_i$ which is bounded above.  By definition, there must then exist some $u\in\beta$ such that, for each $i\geq u$, $B$ is disjoint from the set $A_i=\setbuild{(a,i)}{a\in\alpha_i}$.  Since $\beta$ is complete we may choose $u$ to be the least element of $\beta$ with this property.  If $B\cap A_u$ is unbounded in $A_u$ then the supremum of $B$ is the greatest element of $A_u$.  Otherwise, the supremum of $B$ is $\sup A_u$.
		\item	Similar to 1 except one leverages the well-ordering property of $\beta$ to obtain the necessary $u\in\beta$.\qedhere
	\end{enumerate}
\end{proof}

\begin{prp}\label{prp:defcomp}
	If $\alpha$ is a definably complete coloured linear order without endpoints and $n\in\nats$ then there exists a complete coloured linear order $\beta_n$ such that $\alpha\nequiv{n}\beta_n$.
\end{prp}
\begin{proof}
	Let $n\in\nats$ be fixed but arbitrary and define a binary relation $r$ on $\domain{}\alpha$ such that $aRb$ iff $(a,b)$ is $n$-equivalent to a complete coloured linear order.  It follows from Lemma \ref{lem:IndCong} that $R$ induces a congruence $\sim$ on $\alpha$.

	Take note that, as in the proof of Theorem \ref{thm:Cll}, the congruence $\sim$ is definable.  Therefore, each equivalence class is also definable and we may prove the following:

	\begin{claim}
		If $I\in\faktor{\alpha}{\sim}$ then $I$ is $n$-equivalent to a complete coloured linear order.  Furthermore, if $I$ is bounded above (below) in $\alpha$ then $I$ has a greatest (least) element.
	\end{claim}
	\begin{proof}
		We first consider the case where $I$ is bounded above in $\alpha$.  Choose $u=\sup I$ and let $\family{a_\xi}{\xi<\beta}$ be a cofinal sequence in $I^{<u}$, for some ordinal $\beta\geq\omega$.

		By definition of $I$ it now follows, for each $\xi<\beta$, that the interval $[a_\xi,a_{\xi+1})$ is $n$-equivalent to some complete linear order $\delta_\xi$.  Define the linear order
		\begin{equation}
			\delta=\sum_{\xi<\beta}\delta_\xi
		\end{equation}
		so that it follows from Lemma \ref{lem:compsum} that $\delta$ is a complete coloured linear order.  Lemma \ref{lem:fvsum} then tells us that $\delta\nequiv{n}I\restriction(a_0,u)$ and thus, by definition, we have $a_0\sim u$.  Consequently, $u\in I$ and therefore $u$ is the greatest element of $I$.

		Were it the case that $I$ is unbounded in $\alpha$ then one, again, chooses $\family{a_\xi}{\xi<\beta}$ cofinal in $\alpha$ and a similar argument as before suffices in showing that $I^{>a_0}$ has a complete $n$-equivalent $\delta$.

		Similarly, one may obtain a $b_0\in I$ such that $b_0<a_0$ and $I^{<b_0}$ has a complete $n$-equivalent $\delta^\prime$.  Note, by definition, that $I\restriction (b_0,a_0)$ has a complete $n$-equivalent $\epsilon$.  Therefore, we may conclude that there exists a coloured expansion of the sum
		\begin{equation}
			\delta^\prime+\one+\epsilon+\one+\delta
		\end{equation}
		which is a complete $n$-equivalent of $I$, as required.
	\end{proof}

	\begin{claim}
		The coloured linear order $\faktor{\alpha}{\sim}$ is dense.
	\end{claim}
	\begin{proof}
		Choose any $I,J\in\faktor{\alpha}{\sim}$ such that $I<J$ and, aiming for a contradiction, suppopose there exists no $K\in\faktor{\alpha}{\sim}$ such that $I<K<J$.

		Note that, respectively, $I$ is bounded above and $J$ is bounded below in $\alpha$.  Since $I$ and $J$ are both definable, and $\alpha$ is definably complete, it now follows that there exists $a,b\in\alpha$ such that $a=\sup I$ and $b=\inf J$.

		From our assumption it now follows that $(a,b)=\emptyset$ so that $\alpha\restriction(a,b)$ is (vacuously) complete.  Therefore, since the previous claim yields $a\in I$ and $b\in J$, we may conclude that $a\sim b$ and thus $I=J$ --- the desired contradiction.
	\end{proof}

	Observe that if $\card{\faktor{\alpha}{\sim}}=1$ then the result follows immediately.  By way of contradiction, suppose now that $\card{\faktor{\alpha}{\sim}}>1$.  Let $S=\set{\alpha_0,\dotsc,\alpha_{k-1}}$ be an $n$-spectrum for the class of complete coloured linear orders. The contradiction desired is contained in the following claim:
	\begin{claim}
		There exists an $S^\prime\subseteq S$ and an open interval $D\subseteq\faktor{\alpha}{\sim}$ such that, for each $\delta\in S^\prime$, the set $\setbuild{I\in D}{I\nequiv{n}\delta}$ is dense in $D$.
	\end{claim}
	\begin{proof}
		Choose some open interval $D_0\subseteq\faktor{\alpha}{\sim}$ and define
		\begin{equation}
			S_0=\setbuild{\delta\in S}{\text{there exists an }I\in D_0\text{ such that }\delta\nequiv{n}I}.
		\end{equation}

		Recursively, whenever $0<k<\card{S_0}-1$, we obtain $S_{k+1}$ from $S_k$ by removing any order type $\delta_k\in S_k$ whose $n$-equivalents in the open interval $D_k\subseteq D_{k-1}$ are not dense in $D_k$.  Additionally, we choose $D_{k+1}\subseteq D_k$ to be some open interval, not having $\delta_k$ as a member.  Otherwise, if no such $\delta_k$ exists, we declare $S_{k+1}=S_k$ and $D_{k+1}=D_k$.

		Since $\faktor{\alpha}{\sim}$ is dense, if $m=\card{S_0}-1$ then it follows that $\card{S_m}>0$.  By definition, choosing $S^\prime=S_m$ and $D=D_m$ will establish the claim.
	\end{proof}

	We may assume that $D=(I,J)$ for some $I,J\in\faktor{\alpha}{\sim}$.  Now, let $a,b\in\bigcup D$ satisfy $a<b$.  Our aim is to show that the interval $(a,b)$ of $\alpha$ has some $n$-equivalent in $S$.

	Let $h\colon\reals\to S^\prime$ be the map that induces the canonical partition of $\reals$ into $\card{S^\prime}$ sets.  As in claim 1, note that $I^{>a}$ and $J^{<b}$ will have respective $n$-equivalents $\nu_a,\nu_b\in S$.

	Define a coloured linear order $\chi_{a,b}$ such that
	\begin{equation}
		\chi_{a,b}=\nu_a+\sum_{x\in\lambda}h(x)+\nu_b.
	\end{equation}
	As $\one+\lambda+\one$ is a complete order type, it follows from Lemma \ref{lem:fvsum} that $\chi_{a,b}$ is a complete coloured linear order.  Since the $n$-equivalents in $D$ of any member of $S^\prime$ form a dense subset of $D$, $\Right$ has a winning strategy in the $n$-game between the coloured linear orders $(a,b)$ and $\chi_{a,b}$.

	By definition, we now have $a\sim b$.  Since $a$ and $b$ were chosen arbitrarily, it follows by definition of $\faktor{\alpha}{\sim}$ that $\bigcup D$ is itself an equivalence class.  Thus $D$ has only one element, contradicting that $D$ is an open interval and thereby concluding the proof.
\end{proof}

For the purposes of the proof of the following proposition, given a fixed $k\in\nats$, let $\Sigma_k$ be the theory consisting of the following sentences which, together, express that its models are (nontrivially) dense coloured linear orders without endpoints and are definably complete:
\begin{itemize}
	\item	$\exists x(x=x)$
	\item	$\axmlin\wedge\axmden$,
	\item	$\forall x\exists y(x<y)\wedge\forall x\exists y(y<x)$,
	\item	for each formula $\varphi(x)$ the sentence given by:
		\begin{multline}
			\exists x\varphi(x)\wedge\exists y\forall x\big(\varphi(x)\rightarrow x<y\big)\rightarrow\\
			\exists z\Big(\forall x\big(\varphi (x)\rightarrow x<z\big)\wedge\forall y\big(\forall x(\varphi(x)\rightarrow x<y)\rightarrow z\leq y\big)\Big).
		\end{multline}
\end{itemize}

\begin{prp}
	For each $k\in\nats$, the theory $T$ of complete $k$-coloured linear orders is recursively enumerable.
\end{prp}
\begin{proof}
	Fix any $k\in\nats$ and note that it is enough to show that every model $\alpha$ of $\Sigma_k$ is also a model of $T$.  In line with this we suppose $\alpha\models\Sigma_k$ is fixed but arbitrary.

	Choose any $\sigma\in T$ and let $n=\qrank(\sigma)$.  It follows now from Proposition \ref{prp:defcomp} that there exists a complete coloured linear order $\beta_n\nequiv{n}\alpha$.  By definition, we must have $\beta_n\models T$ and therefore $\beta\models\sigma$.  In conclusion, since $\sigma$ has quantifier rank at most $n$ and $\beta_n\nequiv{n}\alpha$, it holds that $\alpha\models\sigma$.
\end{proof}


\iffalse
\begin{dfn}[regularity]
	We call a $k$-coloured linear order $\alpha$, with colours say $r_0,\dotsc,r_{k-1}$, \textbf{regular} whenever \textit{a subset of} its colours form a partition of $\alpha$ and no member of $\alpha$ posesses two distinct colours.  Equivalently, $\alpha$ is regular whenever it holds that:
	\begin{equation}
		\alpha\models\left(\forall x\bigvee_{0\leq i<n}r_i(x)\right)\wedge\left(\neg\exists x\bigvee_{i<j<k}(r_i(x)\wedge r_j(x))\right).
	\end{equation}
\end{dfn}

\begin{dfn}[$k$-Pattern]
	A $k$-coloured linear order $\alpha$ is called a $\bm{k}$\textbf{-pattern} whenever $\alpha$ is both regular and finite.  Should $k\in\nats$ be clear from the context we will refer to $\alpha$ as simply being a \textit{pattern}.
\end{dfn}
\fi

\begin{dfn}[Finitary]
	We call a $k$-coloured linear order $\alpha$ \textbf{finitary} whenever there exists some $n\in\nats$ such that, for every $k$-coloured linear order $\beta$, $\beta\nequiv{n}\alpha$ necessarily implies $\beta\equiv\alpha$.
\end{dfn}

\begin{prp}
	A $k$-coloured linear order $\alpha$ is finitary iff $\Th(\alpha)$ is finitely axiomatisable.
\end{prp}
\begin{proof}
	\forward	By assumption, there must exists a least $n\in\nats$ such that, for every $k$-coloured $\beta$, $\beta\nequiv{n}\alpha$ impies $\beta\equiv\alpha$.  Now, choose $T$ to be the theory consisting of all sentences $\sigma$ of quantifier rank at most $n$ such that $\alpha\models\sigma$.

	Note that every model $\beta$ of $T$ is forced to be an $n$-equivalent of $\beta$ and is therefore, by assumption, elementarily equivalent to $\alpha$.  Thus, we may conclude that $T$ is a complete theory.  Consequently, we have that $\dcl{T}=\Th(\alpha)$ so that $T$ is an axiomatization of the theory of $\alpha$.

	As there are, up logical equivalence, only finitely many sentences of quantifier rank $\sigma$ it follows that $\Th(\alpha)$ is, in fact, finitely axiomatisable.

	\backward	Choose $n$ to be the maximum quantifier rank of the sentences in the selected axiomatisation.  It immediately follows that $\beta\nequiv{n}\alpha$ implies $\beta\equiv\alpha$, as required.
\end{proof}

\begin{dfn}[$k$-Partitioned linear orders]
	A $\bm{k}$\textbf{-coloured} order $\alpha$ is said to be a $\bm{k}$\textbf{-partitioned} whenever each of the $k$ colours appears in $\alpha$ and every member of $\alpha$ has precisely one colour.
\end{dfn}

\begin{dfn}[Density of colours]
	If $\alpha$ is a $k$-partitioned linear order, $I$ is a subset of $\alpha$ and $C$ is a set of colours then a colour $r\in C$ is said to \textbf{occur densely in }$\bm{I}$, whenever $r^\alpha\cap I\neq\emptyset$ implies that $r^\alpha\cap I$ is a dense subset of $I$.
\end{dfn}

\begin{dfn}[Pigment]
	Suppose $\alpha$ is a $k$-partitioned linear order, the collours being $r_i$ for $i<k$, and $I$ is any convex subset of $\alpha$.  Should it be the case that $I$ is maximal with respect to the property that each colour, among those that occur in $I$, occur densely in $I$ then we refer to $I$ as being a \textbf{pigment} of $\alpha$.
\end{dfn}

\begin{prp}
	For a dense $k$-partitioned linear order $\alpha$, the pigments of $\alpha$ form a partition.  Furthermore, the induced equivalence relation is a congruence.
\end{prp}
\begin{proof}
	Since singletons are trivially dense, it follows that every element of $\alpha$ belongs to some pigment.  Hence, we only required to show that distinct pigments of $\alpha$ need be disjoint.

	Suppose $I$ and $J$ are both pigments of $\alpha$ and that $I\neq J$.  We now argue that the colours occuring in $I\cup J$ occur densely by fixing any $a,b\in I\cup J$ such that $a<b$.  The cases where $a,b\in I$ or $a,b\in J$ are trivial so we may assume, without loss of generality, that $a\in I$, $b\in J$ and $a,b\notin I\cap J$.

	If $I\cap J=\emptyset$ then there is nothing to prove so we may suppose that $I\cap J$ is nonempty.  By definition of $\alpha$, it now follows that $I\cap J\subseteq (a,b)$.  Consequently, we can find and element $c\in I\cap J\subseteq (a,b)$ which is of any arbitrary colour among those in $I\cup J$.  This now contradicts the maximality of $I,J$ and, therefore, it follows that $I$ and $J$ must be disjoint.

	From the definition of a pigment, it additionally follows that the equivalence relation induced by the aforementioned partition is, in fact, a congruence and not just a mere equivalence relation.
\end{proof}

\begin{dfn}[The class $\Mzero^k$]
	For each $k,n\in\nats$, we define the class $\Mzero^k(n)$ recursively as follows.  The class $\Mzero^k(0)$ consists of exactly the finite $k$-partitioned linear orders.

	If $\Mzero^k(n)$ has already been defined then we let $\Mzero^k(n+1)$ be the class of all finite sums $\sum_{i<m}\alpha_i$ such that each $\alpha_i$ is either a member of $\Mzero^k(n)$ or is a \textit{sum} of $\omega$ (or $\dual{\omega}$) copies of some fixed member of $\Mzero^k(n)$.

	Ultimately, as one might anticipate, we make the declaration that
	\begin{equation}
		\Mzero^k=\bigcup_{n<\omega}\Mzero^k(n).
	\end{equation}
\end{dfn}

Take note that this definition exhibits a striking resemblance to the previously defined class $\Mzero$ of (monochromatic) scattered linear orders.  In fact, $\Mzero$ is none other than the class $\Mzero^0$, as defined above.  We have merely broadened our perspective.

The following result will prove useful with regards to our analysis of the  theory of continuous $k$-partitioned linear orders:
\begin{thm}[Amit, Schmerl, Shelah]
	For every $k,n\in\nats$, and each $\alpha\in\Mzero^k$, there exists a finitary $k$-partitioned linear order $\beta_n$ such that $\alpha\nequiv{n}\beta_n$.
\end{thm}

\begin{prp}
	Every $k$-partitioned linear order $\alpha$ of order type $\lambda$, for each $n\in\nats$, has a decidable $n$-equivalent $\beta_n$ which is also of order type $\lambda$.
\end{prp}
\begin{proof}
\end{proof}

\begin{lem}
	For each $k\in\nats$, the set
	\begin{equation}
		R_k=\setbuild{(\alpha,\sigma)}{\alpha\in\Cast_k\text{ and }\alpha\models\sigma}
	\end{equation}
	is recursively enumerable.
\end{lem}
\begin{proof}
\end{proof}


	% backmatter
	\appendix
	\chapter{Computability}

	% print all glossaries
	\setglossarystyle{longheaderborder}
	\printglossaries

  	% print bibliography
	\nocite{*}
	\bibliography{references}

\end{document}
