%! TEX root = ./cos.tex
\documentclass{report}[12pt, A4paper]

% Set pdf page size
\pdfpagewidth=\paperwidth\pdfpageheight=\paperheight

% syntax only ---------
% comment out to build:
% \usepackage{syntonly}

% load packages
\usepackage[utf8]{inputenc}
% \usepackage[
%    language=english,
%    title={A L\"auchli and Leonard style result for complete linear orders},
%    author={Jean-Pierre de Villiers},
%    supervisor={Dr.\ Ruaan Kellerman},
%    examiner={},
%    type={Magister Scientiae},
%    institute={University of Pretoria},
%    course={MSc Mathematics},
%    startdate={2019--02--01},
%    enddate={2020--08--30}
% ]{scientific-thesis-cover}
\usepackage{mathtools}
\usepackage{amssymb}
\usepackage{amsfonts}
\usepackage{amsthm}
\usepackage{bm}
\usepackage{faktor}
\usepackage{enumitem}
\usepackage{tikz-cd}
\usepackage[sectionbib]{chapterbib}
\usepackage{stmaryrd}

% double spacing for easy annotation
% \usepackage[doublespacing]{setspace}

% packages that generally MUST be loaded last
% \usepackage{hyperref}

% packages that must be loaded after hyperref
\usepackage[nopostdot,symbols,nomain,nonumberlist]{glossaries}

% add definition theorem environments
\theoremstyle{definition}
\newtheorem{dfn}{Definition}[chapter]
\newtheorem{exm}[dfn]{Example}
\newtheorem{conv}[dfn]{Convention}
\newtheorem{assn}[dfn]{Assumption}

% add plain theorem environments
\theoremstyle{plain}
\newtheorem{thm}[dfn]{Theorem}
\newtheorem{lem}[dfn]{Lemma}
\newtheorem{cor}[dfn]{Corollary}
\newtheorem{prp}[dfn]{Proposition}

\newcounter{nclaim}[dfn]
\newcounter{ncase}[dfn]
% add remark theorem environments
\theoremstyle{remark}
\newtheorem{rem}[dfn]{Remark}
\newtheorem*{notation}{Notation}
\newtheorem{claim}[nclaim]{Claim}
\newtheorem{case}[ncase]{Case}

% common sets of numbers
\newcommand{\nats}{\mathbb{N}}
\newcommand{\posnats}{\mathbb{N}^{+}}
\newcommand{\ints}{\mathbb{Z}}
\newcommand{\rats}{\mathbb{Q}}
\newcommand{\reals}{\mathbb{R}}
\newcommand{\irrats}{\mathbb{I}}

% finite ordinals
\newcommand{\zero}{\mathbf{0}}
\newcommand{\one}{\mathbf{1}}
\newcommand{\two}{\mathbf{2}}
\newcommand{\n}{\mathbf{n}}

% special classes of linear orders
\newcommand{\VD}{\mathcal{VD}}
\newcommand{\Mzero}{\mathcal{M}_0}
\newcommand{\M}{\mathcal{M}}
\newcommand{\linear}{\mathcal{L}}
\newcommand{\scattered}{\mathcal{S}}
\newcommand{\C}{\mathcal{C}}
\newcommand{\Cast}{\mathcal{C}^\ast}
\newcommand{\dense}{\mathcal{D}}
\newcommand{\Dense}{\mathcal{D}^\ast}
\newcommand{\continuous}{\mathcal{L}_\lambda}
\newcommand{\ordinals}{\mathcal{O}\mathrm{rd}}

% important sentences
\newcommand{\axmlin}{\sigma_{\mathrm{L}}}
\newcommand{\axmden}{\sigma_{\mathrm{D}}}

% aliases
\newcommand{\define}{\coloneqq}
\newcommand{\eqsymb}{\approx}
\newcommand{\setbuild}[2]{\left\lbrace #1\colon #2 \right\rbrace}
\newcommand{\set}[1]{\left\lbrace #1\right\rbrace}
\newcommand{\dual}[1]{{#1}^\ast}
\newcommand{\up}[1]{\left\uparrow#1\right.}
\newcommand{\down}[1]{\left\downarrow#1\right.}
\newcommand{\powerset}[2][]{\mathcal{P}_{#1}(#2)}
\newcommand{\family}[2]{(#1)_{#2}}
\newcommand{\domain}[2][]{\partial^{#1}{#2}}
\newcommand{\del}{\partial}
\newcommand{\card}[1]{\left\lvert#1\right\rvert}
\newcommand{\subsets}[2]{\left[#2\right]^{#1}}
\newcommand{\nequiv}[1]{\equiv^{#1}}
\newcommand{\eq}[1]{{#1}^{\mathrm{eq}}}
\newcommand{\embed}{\preceq}
\newcommand{\elmsub}{\preccurlyeq}
\newcommand{\tto}{\twoheadrightarrow}
\newcommand{\dcl}[1]{{#1}^\mathrm{D}}
\newcommand{\forward}{\noindent$\Rightarrow$:\quad}
\newcommand{\backward}{\noindent$\Leftarrow$:\quad}
\newcommand{\cha}[2]{\llbracket#1\rrbracket^{#2}}
\newcommand{\godel}[1]{\lfloor #1\rfloor}
\newcommand{\pto}{\to}
\newcommand{\inv}[1]{{#1}^{-1}}
\newcommand{\gen}[2]{\langle #1\rangle_{#2}}
\newcommand{\Left}{\mathcal{L}}
\newcommand{\Right}{\mathcal{R}}
\newcommand{\pprime}{{\prime\prime}}
\newcommand{\A}[1]{\bm{\forall}_{#1}}
\newcommand{\E}[1]{\bm{\exists}_{#1}}
\newcommand{\length}{\ell}

% splittings
\newcommand{\fsplit}[1][]{\prescript{#1}{}{\pi}_\mathrm{F}}

% bold font operators
\newcommand{\cat}[1]{\bm{\mathbf{#1}}}

% roman font operators
\DeclareMathOperator{\Int}{Int}
\DeclareMathOperator{\cf}{cf}
\DeclareMathOperator{\range}{ran}
\DeclareMathOperator{\con}{Con}
\DeclareMathOperator{\defcon}{Con_d}
\DeclareMathOperator{\trclos}{Tr}
\DeclareMathOperator{\Th}{Th}
\DeclareMathOperator{\Mod}{Mod}
\DeclareMathOperator{\admis}{Admis}
\DeclareMathOperator{\kernel}{ker}
\DeclareMathOperator{\EF}{EF}
\DeclareMathOperator{\diag}{diag}
\DeclareMathOperator{\diagp}{{diag}^{+}}

% italic font operators
\newcommand{\rank}{\mathit{rank}}
\newcommand{\hrank}{\mathit{rank}_\mathrm{H}}
\newcommand{\vdrank}{\mathit{rank}_\mathcal{VD}}
\newcommand{\qrank}{\mathit{qr}}
\newcommand{\id}{\mathit{id}}
\newcommand{\height}{\mathit{h}}

% calligraphic operators
\newcommand{\comp}{\mathcal{C}}
\newcommand{\level}[1]{\mathcal{L}_{#1}}

% AMS-Latex customisation
\newcommand{\noqed}{\renewcommand{\qedsymbol}{}}

% define glossary entries
\loadglsentries{symbols}
\makeglossaries

% Bibliography style
\bibliographystyle{amsalpha}


% glossary entries
\glsadd{domain}
\glsadd{cofinality}
\glsadd{ksubsets}
\glsadd{positive integers}
\glsadd{reals}
\glsadd{rationals}
\glsadd{naturals}
\glsadd{initial ordinals}
\glsadd{range}


% build only the specified chapters
% \includeonly{prelim/pre, scattered/scatter, mono/mono}

\begin{document}
   	% \maketitle

   	% frontmatter
	\chapter{Introduction}


\section{Linear orders and coloured expansions}


\section{Basic operations on linear orders}


\section{Lattices}


\section{Model theory and logic}


\section{Decidability}

	\chapter{Preliminaries}

\section{Model theory and logic}

\begin{dfn}[Finitary languages]
	A first order language $L$ is called \textbf{finitary} if $L$ is a language over a \text{finite} signature.
\end{dfn}

\begin{prp}
	Suppose we fix any $n\in\nats$ and $L$ is a finitary first order language. If $v_0,\dotsc,v_{k-1}$ are distinct variables then, up to logical equivalence, there are only finitely many formulas $\varphi=\varphi(v_1,\dotsc,v_{k-1})$ such that $\qrank(\varphi)\leq n$.
\end{prp}

\begin{dfn}[Characteristic formulas]
	If $L$ is some finitary language, $\mathfrak{M}$ is some $L$-structure and $\bar{a}\in\domain{k}{\mathfrak{M}}$ then we define the \textbf{$\mathbf{n}$-characterstic formula} $\cha{\bar{a}}{n}$ of $\bar{a}$ relative to $\mathfrak{M}$, recursively, as follows:
	\begin{enumerate}
		\item $\cha{\bar{a}}{0}=\bigwedge\setbuild{\varphi(\bar{x})\in L_k}{\mathfrak{M}\models\varphi(\bar{a})}$;
		\item $\cha{\bar{a}}{n+1}=\bigwedge_{b\in\domain{}\mathfrak{M}}\exists v_n \cha{\bar{a}b}{n}\wedge\forall v_n\bigvee_{b\in\domain{}\mathfrak{M}}\cha{\bar{a}b}{n}$.
	\end{enumerate}
	If $\bar{a}$ is the empty tuple, and $n\in\nats$, then we write $\cha{\mathfrak{M}}{n}$ for $\cha{\bar{a}}{n}$ and call $\cha{\mathfrak{M}}{n}$ the \textit{$\mathit{n}$-characteristic sentence} of the structure $\mathfrak{M}$.
\end{dfn}

\begin{prp}
	The $n$-characterstic sentences,$n\in\nats$, of the structure $\mathfrak{M}$ are presicely the sentences, of quantifier rank $n$, that are pairwise inequivalent modulo the complete theory $T=\Th(\mathfrak{M})$.
\end{prp}


\section{Linear orders and coloured expansions}

\begin{dfn}[Dense linear order] A linear order $\alpha$ is called \textbf{dense} whenever, for every $a,b\in\alpha$, there exists $a\in\alpha$ such that $a<c<b$.  Additionally, $\alpha$ is said to be \textit{trivially dense} if the order type of alpha is either $\zero$ or $\one$.
\end{dfn}


\section{Basic operations on linear orders}

\begin{lem}\label{lem:fvsum}
	Suppose $\family{\alpha_i}{i\in I}$ and $\family{\beta}{i\in I}$ are families of (possibly coloured) linear orders, indexed by a linearly ordered set $I\neq\emptyset$.  Now fix some $n\in\nats$ and assume $\alpha_i\nequiv{n}\beta_i$, for each $i\in J$, then it follows that
	\begin{equation}
		\sum_{i\in I}\alpha_i\nequiv{n}\sum_{i\in I}\beta_i.
	\end{equation}
\end{lem}


\section{Lattices}

\begin{thm}[Knaster-Tarski Theorem]
	Suppose $\Lambda$ is a complete lattices.  If $h$ is some endomorphism on $\Lambda$ then there must exist a (least) $x_0\in \Lambda$ such that $x_0$ is a fix0ed point of $h$, i.e.\ $h(x_0)=x_0$.
\end{thm}
\begin{proof}
	We start by exhibiting a fixed point of $h$.  Since $\Lambda$ is nonempty there exists a top $1\in\Lambda$, since by completeness, taking the join of $\Lambda$ we get an element $1=\bigvee\Lambda$.  Since $h$ is isotone we may conclude that $h(1)=1$.

	We are now obligated to find a least fixed point of $h$ so define
	\begin{equation}
		S=\setbuild{x\in\Lambda}{h(x)=x}
	\end{equation}  then clearly $S\neq\emptyset$ since $1\in S$.  Define $x_0=\bigwedge S$ and note that, since $x_0\leq x$ for each $x\in S$, by definition of isotonicity it follows that $h(x_0)\leq x$, for $x\in S$. Therefore $h(x_0)\leq\bigwedge S=x_0$.  Consequently, since $h(x_0)$ is a lower bound of $S$.  Since $x_0$ is the infimum of $S$ we cannot have $h(x_0)<x_0$.  Thus the only remaining possibility is $h(x_0)=x_0$, implying that $x\in S$.
\end{proof}


\section{Categories and functors}

\section{Galois connections}



\section{Decidability}

	\begin{prp}
		If $\Sigma$ is a set of $L$-sentences then $\Sigma$ is decidable set of sentences iff $\Sigma$ and $L_0\setminus\Sigma$, each, are recursively enumerable.
	\end{prp}

	\begin{prp}\label{prp:sdth}
		Suppose $\mathcal{A}$ is a class of structures.  If $\Sigma$ is a recursively enumerable set of sentences and its deductive closure $\dcl{\Sigma}$ satisfies
		\begin{equation}
			\dcl{\Sigma}=\Th(\mathcal{A}),
		\end{equation}
		then $\Th(\mathcal{A})$ is recursively enumerable.
	\end{prp}

	\begin{dfn}[Encoding]
		An \textbf{encoding} $p$ of a countable set $X$ is an injective recursive partial function $p\colon X\pto\nats$.  If $X$ is a first order language then $p$ is in stead referred to as a \textit{G\"odel numbering}.
	\end{dfn}

	\begin{dfn}[Proof sequences]
		Suppose $X$ is a countable set of countable coloured linear orders and let $p\colon X\pto\nats$ be an encoding of $X$.  A \textbf{proof sequence} is then a finite sequence
		\begin{equation}
			(\alpha_0,\sigma_0),\dotsc,(\alpha_{n-1},\sigma_{n-1})\in X\times L_0
		\end{equation}
		and finite expansions $\alpha^\prime_0,\dotsc,\alpha^\prime_{n-1}$ of the (respective) linear orders $\alpha_0,\dotsc,\alpha_{n-1}$ so that, for each $i=1,\dotsc,n-1$, there exists an interpretation $\Gamma_i$ of $\alpha_{i-1}$ in $\alpha^\prime_i$.  Additionally, when $0\leq i<n$, we require that one of the following holds:
		\begin{enumerate}
			\item	$\sigma_i$ is a logical axiom,
			\item	$\sigma_i$ follows via an inference rule from $\sigma_0,\dotsc,\sigma_{i-1}$,
			\item	$\sigma_i\in\admis(\Gamma_i)$ or
			\item	$\sigma_i=\Gamma_i\Gamma_{i-1}\dotsb\Gamma_j\sigma_j$, for some natural $j<i$.
		\end{enumerate}
	\end{dfn}
\vfill
	\begin{verbatim}
		-------------------------------------------------------------------------
	\end{verbatim}
\noindent[\textbf{Alles in hierdie hoofstuk is tentatief.  Byvoegings en (veral) verwyderings sal gemaak word soos ek deur die tesis vorder}]

	\bibliographystyle{alpha}

\chapter{Ramsey theory}


\section{Classical Ramsey theorem}


    \begin{dfn}[Homogeneous set for a partition]
       If $A$ is a nonempty set and, for some $k\in\posnats$, $\mathcal{C}$ is a parition of $\subsets{k}{A}$ then we will call $X\subseteq A$ \textbf{homogeneous for the partition} $\mathcal{C}$ if there exists a $C\in \mathcal{C}$ such that $\subsets{k}{X}\subseteq C$.
    \end{dfn}

    \begin{thm}[Ramsey's theorem]
        Suppose $\mathcal{C}$ is a finite partition of $[\nats]^k$, for some $k\in\posnats$, then there exists an infinite set $X\subseteq\nats$ such that $X$ is homogeneous for $\mathcal{C}$.
    \end{thm}
    \begin{proof}
        We proceed by induction on $k$. If $k=1$ then there must exists an infinite $C\in\mathcal{C}$ and thus $X=C$ is homogeneous for $\mathcal{C}$.

        Suppose $\mathcal{C}=\set{C_0,\dotsc,C_{m-1}}$ and assume the result holds for some $k\in\posnats$. If we let $s\colon\nats\to\posnats$ be the successor map then $s$ is a bijection.  Now define
        \begin{equation}
            B_i=\setbuild{s^{-1}[X]}{X\in\subsets{k}{\posnats}\text{ and }X\cup\set{0}\in C_i},
        \end{equation}
        for each $i=0,\dotsc,m-1$.  Since $\mathcal{C}$ is a partition of $\subsets{k+1}{\nats}$, $s(x)\neq 0$ for any $x\in\nats$ and $s^{-1}$ is bijective it follows that $\mathcal{B}=\set{B_0,\dotsc,B_{m-1}}$ is a partition of $\subsets{k}{\nats}$.  By the inductive hypothesis there exists an infinite set $H\subseteq\nats$ which is homogeneous for $\mathcal{B}$.  Without loss of generality we may assume $\subsets{k}{H}\subseteq B_0$.  Therefore, by definition of $B_0$, if we define $H^\prime=\setbuild{s[X]\cup\set{0}}{X\in\subsets{k}{H}}$ then $H^\prime\subseteq C_0$, as required.
    \end{proof}

    \begin{dfn}[Colouring]\label{def:Col}
        If $S$ is a nonempty set and $\card{C}=k\in\posnats$ then a $\mathbf{k}$\textbf{-colouring} of $S$ is a surjective map $f\colon  S\to C$.  We call a surjection $f$ simply a \textbf{colouring} if it is a $k$-coulouring for some $k$.
    \end{dfn}

    Note that if $\mathcal{C}$ is some partition of a nonempty set $S$ then there exists a unique surjection $f\colon S\to \mathcal{C}$ such that $\mathcal{C}=\setbuild{f^{-1}[X]}{X\in \mathcal{C}}$. Since every surjection $f$ induces a partition of its domain, this justifies the following definition:

    \begin{dfn}[Homogeneous set for a colouring]
        Suppose $S$ is a nonempty set and $f\colon S\to C$ is a colouring.  We call $X\subseteq A$ \textbf{homogeneous for} $f$ whenever there exists a $c\in C$ such that $X\subseteq f^{-1}[c]$ or, equivalently, $f(x)=f(y)$ for every $x,y\in X$.
    \end{dfn}

    \begin{rem}[Conventions: colourings]\label{rem:Col}
        If $A$ is some linearly ordered set and, then for every $k\in\posnats$ then every colouring $f$ of $\subsets{k}{A}$ corresponds uniquely to a colouring $\tau_f$ of the set $\setbuild{(a_0,\dotsc,a_{k-1})\in A^k}{a_i<a_j\text{ whenever }i<j<k}$ such that if $a_i<a_j$ for each $i<j<k$ then
        \begin{equation}
            f(\set{a_0,\dotsc,a_{k-1}})=\tau_f(a_0,\dotsc,a_{k-1}).
        \end{equation}
        Therefore, we will make a habit of identifying each $f$ with the corresponding $\tau_f$ and simply write $f(a_0,\dotsc,a_{k-1})$ in stead of $f(\set{a_0,\dotsc,a_{k-1}})$ when $a_0<a_1<\dotsb<a_{k-1}$.  Furthermore, for the special case $k=2$ we will also identify each $f$ with the map $\upsilon_f$ that sends each (nonempty) open interval $(a,b)\subseteq A$ to $\tau_f(a,b)$.
    \end{rem}

    \begin{rem}
        Note that if $f$ is a colouring of $\subsets{k}{S}$, for some nonempty set $S$ and some $k\in\posnats$, then it follows by definition that $H\subseteq S$ is homogeneous for the partition $\setbuild{f^{-1}[c]}{c\in\range f}$ if and only if $\subsets{k}{H}$ is homogeneous for $f$.  This gives us the connection between the two notions of homogeneity.
    \end{rem}


    \begin{dfn}[Cofinality]
        If $\alpha$ is a linear order then $\cf(\alpha)$, called the \textbf{cofinality} of $\alpha$, is the least ordinal $\beta$ so that there exists an embedding $f\colon\beta\to\alpha$ such that $f[\beta]$ is not bounded above in $\alpha$.
    \end{dfn}

    \begin{dfn}[Cofinal sequence]
        A \textbf{cofinal sequence} in a linear order $\alpha$ is any (strictly) increasing transfinitite sequence $(a_\gamma)_{\gamma<\beta}$, where $\beta\geq\cf(\alpha)$, such that $\setbuild{a_\gamma}{\gamma<\beta}$ is an unbounded subset of $\alpha$
    \end{dfn}

    \begin{dfn}[Homogeneous sequence]
        If $\alpha$ is a linear order and $f$ is a colouring of $\subsets{2}{\domain\alpha}$ then a \textbf{homogeneous sequence for $f$} is a cofinal sequence $\family{a_\gamma}{\gamma<\beta}$ in $\alpha$ such that
        \begin{equation}
            H=\setbuild{\set{a_\gamma,a_\delta}}{\gamma<\delta<\beta}
        \end{equation}
        is homogeneous for $f$.
    \end{dfn}

    \begin{cor}[Existence: homogeneous sequences]\label{cor:Cofinal}
        Suppose $A$ is an infinite linearly ordered set without a greatest element. If $A$ has cofinality $\omega$ and $f$ is a colouring of $\subsets{2}{A}$ then there exists a homogeneous sequence $x=\family{a_i}{i<\omega}$ for $f$ in $A$.
    \end{cor}

    \begin{proof}
        Choose $y=(b_i)_{i<\omega}$ to be cofinal in $A$.  If we define $B=\setbuild{\set{b_i,b_j}}{i<j<\omega}$ then by Ramsey's theorem there exists a $H\subseteq B$ which is homogeneous for $f$.  Since $x$ is increasing, there exists a unique subsequence $x=(a_i)_{i<\omega}=(b_{k_i})_{i<\omega}$ of $y$ with image $H$.  By definition $x$ is the desired homogeneous sequence for the colouring $f$.
    \end{proof}

    \begin{exm}[Cofinal sequences in $\lambda$]
        Note that $\cf(\lambda)=\omega$ since $\lambda$ has no greatest element and $\family{k}{k<\omega}$ is a cofinal sequence in $\lambda$.  Now suppose $f$ is a colouring of $\subsets{2}{\reals}$ and let $x=(x_i)_{i<\omega}$ be a cofinal sequence in $\lambda$.  Now let $X=\setbuild{x_i}{i<\omega}$ and define $g=f\restriction_X$ then we may assume, without loss of generality, that $g$ is surjective and thus a colouring of $X$.  Since $x$ is clearly cofinal in $X$ it follows from corollary (\ref{cor:Cofinal}) that there exists a subsequence $y$ of $x$ which is homogeneous for $g$.  Since $y$ is homogeneous for $g$ and $X$ is a cofinal subset of $\lambda$, it follows by definition of $g$ that $y$ is a homogeneous sequence for $f$.
    \end{exm}



\section{Additive Ramsey theorem}


    A natural question to ask is whether corollary (\ref{cor:Cofinal}) also holds for order types of uncountable cofinality.  Unfortunately,  this is not the case.  This is illustrated in the following example:
    \begin{exm}[Sierpi\'nski colouring]
        Suppose $\alpha=(\reals,\prec)$ is a well-order.  Note that $\alpha\geq\omega_1$ and that every countable ordinal is bijectively equivalent to $\omega$.  Therefore, since $\omega+\alpha=\alpha$, we may assume $\cf(\alpha)\geq\omega_1$ because, informally speaking, we can move any countable tail to the front. The \textit{Sierpi\'nski colouring} $s\colon\subsets{2}{\reals}\to\set{0,1}$ is defined, for every $x,y\in\reals$, (recalling remark (\ref{rem:Col})) as
        \begin{equation}
            s(x,y)=
            \begin{cases}
                1,  &\text{if }x\prec y,\\
                0,   &\text{otherwise.}
            \end{cases}
        \end{equation}
        Since $\lambda$ is clearly not a well-order it follows that $s$ is a colouring of $\subsets{2}{\reals}$.  By way of contradiction, assume there exists a homogeneous sequence $x=(x_{\gamma})_{\gamma<\beta}$ for $s$ in $\alpha$.  By assumption we must have $\beta\geq\cf(\alpha)\geq\omega_1$.  However, since $\eta$ is dense in $\lambda$ we may shoose a rational $q_\gamma\in(x_\gamma,x_{\gamma+1})_{\lambda}\cap\rats$, for each $\gamma<\beta$, giving us the desired contradiction since the rationals are countable.
    \end{exm}

    \begin{dfn}[Additive colouring]
        An \textbf{additive colouring} of a linear order $\alpha$ is a colouring $f$ of $\subsets{2}{\domain\alpha}$ such that
        \begin{equation}
            f(x_0,y_0)=f(x_1,y_1)\text{ and }f(y_0,z_0)=f(y_1,z_1)
        \end{equation}
        imply $f(x_0,z_0)=f(x_1,z_1)$.
    \end{dfn}

    \begin{rem}[Addition of intervals]
        What this definition essentially says is that a colouring is called additive when $I_0,\dotsc, I_3$ are open intervals in $\alpha$ such that $I_0$ is the same colour as $I_1$ and $I_2$ the same colour as $I_3$ then by `joining together' $I_0$ with $I_2$ and $I_1$ with $I_3$ we obtain a pair of intervals which are of the same colour.  This is possible whenever there is exactly one element $x\in\alpha$ such that $I_0<x<I_2$ and exactly one element $y\in\alpha$ such that $I_1<y<I_3$.  In fact, if this is the case we may define a (partial) binary operation $+$, not to be confused with addition of order types, on the set of nonempty open intervals of $\alpha$ such that $I_0+I_2=I_0\cup\set{x}\cup I_2$ and, similarly $I_1+I_3=I_1\cup\set{y}\cup I_3$.  Consequently, one can restate the above definition as follows: the equivalence relation induced by $f$ on the set of (nonempty) open intervals in $\alpha$ is a congruence relation for the operation $+$.
    \end{rem}

    \begin{thm}[Additive Ramsey theorem \cite{ShelahOrder}]
        If $\delta$ is a limit ordinal, $\beta=\cf(\delta)$ and $f$ is an additive colouring of $\delta$ then there exists a homogeneous sequence $x=(\alpha_\gamma)_{\gamma<\beta}$ for $f$.
    \end{thm}

    \begin{proof}
        For every $\alpha,\alpha^\prime<\delta$, define $\alpha\sim\alpha^\prime$ whenever there exists a $\gamma_0<\delta$ such that $\alpha_0,\alpha_1<\gamma_0$ and $f(\alpha,\gamma_0)=f(\alpha^\prime,\gamma_0)$.  We now prove the following claim:
        \begin{claim}
            The binary relation $\sim$ is an equivalence relation
        \end{claim}
        \begin{proof}
            Fix any $\alpha_0,\alpha_1,\alpha_2<\delta$.  Since $f(\alpha_0,\alpha_0+1)=f(\alpha_0,\alpha_0+1)$ it follows that $\sim$ is reflexive.  If there exists a $\gamma_0>\alpha_0,\alpha_1$ such that $\gamma_0<\delta$ and $f(\alpha_0,\gamma_0)=f(\alpha_1,\gamma_0)$ then $f(\alpha_1,\gamma_0)=f(\alpha_0,\gamma_0)$ so that $\sim$ is symmetric.

            Lastly, to show transitivity, if there exists a $\gamma_0>\alpha_1,\alpha_0$ as well as a $\gamma_1>\alpha_1,\alpha_2$ such that $\gamma_0,\gamma_1<\delta$, $f(\alpha_0,\gamma_0)=f(\alpha_1,\gamma_0)$ and $f(\alpha_1,\gamma_1)=f(\alpha_2,\gamma_1)$. If $\gamma_0=\gamma_1$ then it follows that $f(\alpha_0,\gamma_0)=f(\alpha_2,\gamma_0)$ and thus $\alpha_0\sim\alpha_2$ so suppose $\gamma_0<\gamma_1$.  Since (trivially) $f(\gamma_0,\gamma_1)=f(\gamma_0,\gamma_1)$, it follows by additivity of $f$ that $f(\alpha_0,\gamma_1)=f(\alpha_1,\gamma_1)$ and therefore $f(\alpha_0,\gamma_1)=f(\alpha_2,\gamma_1)$ so that $\alpha_0\sim\alpha_2$, as required.  Since the case $\gamma_1<\gamma_0$ is similar, the relation $\sim$ is transitive and thus $\sim$ is an equivalence relation.
        \end{proof}

        Note that $\sim$ has at most $\card{\range f}<\aleph_0$ equivalence classes and there must exist an equivalence class $C$ under $\sim$ which is unbounded in $\delta$.  We will now define a cofinal sequence $x=(\alpha_{\gamma})_{\gamma<\beta}$, where $\beta=\cf(\delta)$, by means of transfinite recursion.  Let $\alpha_0$ be the least element of $C$ and define, for each $c\in\range f$, the set $I_c=\setbuild{\alpha\in C}{\alpha_0<\alpha\text{ and }f(\alpha_0,\alpha)=c}$.  It then follows, by definition, that
        \begin{equation}
            C\setminus\set{\alpha_0}=\bigcup_{c\in\range f}I_c.
        \end{equation}
        Since $I$ is unbounded in $\delta$ and $\card{\range f}<\card{\cf(\delta)}$ it follows that there exists a $d\in\range f$ such that $I_d$ is unbounded in $\delta$.  Note that there exists a cofinal sequence $(\delta_\gamma)_{\gamma<\beta}$ in $I_d$.  Now, continuing with the recursion, assume $\alpha_\gamma\in C$ has been defined for each $\gamma<\epsilon$ and some ordinal $\epsilon<\beta=\cf(\delta)$.  Invoking the definition of $\sim$, let $\alpha_\epsilon$ be the least ordinal $\alpha_\epsilon\in I_d$ such that $\alpha_\gamma<\alpha_\epsilon$, $\delta_\gamma<\alpha_\epsilon$ and
        \begin{equation}
            f(\alpha_0,\alpha_\epsilon)=f(\alpha_\gamma,\alpha_\epsilon),\quad\text{whenever }\gamma<\epsilon.
        \end{equation}
        By definition, it then follows that $x=\family{\alpha_\gamma}{\gamma<\beta}$ is a homogeneous sequence for the colouring $f$.
    \end{proof}

    \begin{rem}
        If in the above proof we substitute for the colouring $f$ any surjection with the property that $\card{\range f}<\card{\cf(\delta)}$ then the proof remains valid.  Hence, it is not necessary for the proof that $\card{\range f}<\aleph_0$ when $\cf(\delta)\geq\omega_1$.
    \end{rem}


    \begin{cor}[Existence: homogeneous sequences]
        Suppose $\alpha$ is a linear order without a greatest element and $f$ is an additive colouring of $\alpha$ then there exists a limit ordinal $\delta\geq\cf(\alpha)$ and a homogeneous sequence $(a_\gamma)_{\gamma<\delta}$ for $f$.
    \end{cor}

    \begin{proof}
        Let $(b_\gamma)_{\gamma<\delta^\prime}$ be some cofinal sequence in $\alpha$ such that $\delta^\prime=\cf(\alpha)$.  Note that, since $\alpha$ has no greatest element, $\delta^\prime=\cf(\alpha)$ is a limit ordinal.  Now define $B=\setbuild{b_\gamma}{\gamma<\delta}$ and let $g=f\restriction_B$ then we may assume, without loss of generality, that $g$ is surjective and hence a colouring of $B$.  Since $B\cong\delta^\prime=\cf(\alpha)$, it follows from the additive Ramsey theorem that there exists homogeneous sequence $x=(a_\gamma)_{\gamma<\delta}$, for some $\delta\geq\cf(\alpha)$, for $g$.  Since $B$ is not bounded above, $\delta$ must be a limit ordinal and, by definition of $g$, $x$ is a homogeneous sequence for $f$, as required.
    \end{proof}



\bibliography{references}

	\chapter{Scattered linear orders}


\section{Condensation maps and congruence lattices}

\begin{lem}[Congruence construction]\label{lem:IndCong}
        Suppose $\alpha$ is a linear order and $R$ is a transitive binary relation on $\alpha$.  Now define a another binary relation $\sim$ on $\alpha$ such that, for $a,b\in\alpha$, we have $a\sim b$ whenever one of the following is satisfied:
        \begin{enumerate}
            \item   $a=b$,
            \item   $a<b$ and $aRb$,
            \item   $b<a$ and $bRa$.
        \end{enumerate}
        Under these assumptions, $\sim$ is a congruence on $\alpha$.
\end{lem}

\begin{dfn}[Induced congruence]
        The congruence $\sim$ in lemma (\ref{lem:IndCong}) is referred to as the \textbf{congruence induced by $R$}.
\end{dfn}

\begin{dfn}[Factor maps]
        If $\alpha,\beta$ are linear orders then a \textbf{factor map} is a surjective homomorphism $\pi\colon\alpha\rightarrow\beta$.
\end{dfn}

\begin{dfn}[Splittings and Condensations]
        If $\sim$ is a congruence of the linear order $\alpha$ and $\pi\colon\alpha\to\faktor{\alpha}{\sim}$ is the (unique) factor map such that, for every $a,b\in\alpha$, it holds that:
        \begin{equation}
            a\sim b\iff \pi(a)=\pi(b),
        \end{equation}
        then we call $\pi$ the \textbf{splitting} of $\alpha$ \textit{induced by} $\sim$, refer to the quotient $\faktor{\alpha}{\sim}$ as a \textbf{condensation} of $\alpha$ and write $\pi\colon\alpha\tto\faktor{\alpha}{\sim}$.
\end{dfn}


\begin{dfn}[Finite splitting]
        Suppose $\alpha$ is a linear order and $\sim$ is the congruence on $\alpha$ induced by the relation $R$ defined by:
        \begin{equation}
            aRb\quad\iff\quad a<b\text{ and }[a,b]\text{ is finite}.
        \end{equation}
        For any linear order $\alpha$ we let $\fsplit$ denote the map $\fsplit\colon a\mapsto\faktor{a}{\sim}$ for $a\in\alpha$.  We refer to $\fsplit$ as the $\textbf{finite splitting}$.
\end{dfn}



\begin{prp}[Congruence lattices]
        Let $\con \alpha\subseteq\domain{2}\alpha$ be the set of all congruences on $\alpha$ and, for each $X\subseteq \con\alpha$, define:
        \begin{align}
            \bigwedge X &\coloneqq\bigcap X,\\
            \bigvee X   &\coloneqq\bigwedge\setbuild{a\in\con\alpha}{x\subseteq a,\forall x\in X}.
        \end{align}
        It then follows that $(\con\alpha,\vee,\wedge)$ is a complete lattic.  Furthermore, it holds what $\bigvee X=\trclos(\bigcup X)$ --- the latter being the \textit{transitive closure} of the (reflexive and symmetric) binary relation $\bigcup X$.
\end{prp}
\begin{proof}
	Let $X$ be an arbitrary (non-empty) subset of $\con\alpha$.  We now proceed to argue that $(\con\alpha,\vee,\wedge)$ is a complete lattice under the defined operations.  To achieve this, we are first required to show that (as defined above) the sets $\bigvee X$ and $\bigwedge X$ are in fact congruences.

	We first make the case for $\bigwedge X$.  Let $I$ denote the identity relation on $\domain{}\alpha$ then, since the members of $X$ are all reflexive, we must have $I\subseteq x$ for each $x\in X$ and thus $I\subseteq\bigcap X=\bigwedge X$.  Therefore $\bigwedge X$ is in fact a reflexive (binary) relation.  Also, since the members of $X$ are all symmetric it immediately follows that $\bigwedge X$ is also, since if $(a,b)\bigcap X$ then $(a,b)\in x$ for every $x\in X$ after which the symmetry of the members of $x$ yield the corresponding $(b,a)\in\bigcap X$.  In a somewhat similar fashion, an arbitrary intersection of transitive (binary) relations will again yield a transitive relation, all via first principles.

	Next we consider the case for $\bigvee X$.  Note that $I\subseteq \bigvee X$ since $X$ is non-empty and $I\subseteq x$ for each $x\in X$.  Therefore $\bigvee X$ is reflexive.  Note now that $\bigvee X$ is an intersection of symmetric (binary) relations and thus must itself also be symmetric.  In a similar fashion, since $\bigvee X$ is an intersection of transtive relations it must itself also be transitive.

	All that remains is to show that $\bigvee X=\trclos(\bigcup X)$.  Clearly we already have $\trclos(\bigcup X)\subseteq\bigvee X$, since $\bigvee X$ is transitive,  so we need only show that the reverse inclusion also holds.  Not that, by definition, we must have $x\subseteq\bigcup X$ for each $x\in X$.  Also, by definition of transitive closure, we also have $\bigcup X\subseteq \trclos(\bigcup X)$ and thus $x\subseteq\trclos(\bigcup X)$ for each $x\in X$, as required.
\end{proof}

\begin{prp}
         Suppose $f\colon\alpha\tto\beta$ is a factor map then there exists an unique congruence relation $\sim$ on $\alpha$ such that, if $\pi\colon\alpha\tto\faktor{\alpha}{\sim}$ is the splitting induced by $\sim$, there exists an unique isomorphism $\iota\colon\faktor{\alpha}{\sim}\to\beta$ making the diagram
        \begin{equation}
            \begin{tikzcd}
		\alpha \arrow[rr, "\pi",rightarrow]&&   \faktor{\alpha}{\sim}\\
		\\
		 &&   \beta \arrow[from=uu,"\iota", dashrightarrow] \arrow[from=uull,"f"']
            \end{tikzcd}
        \end{equation}
        commute.
\end{prp}
\begin{proof}
	The desired congruence relations is obtained in a manner familiar from algebra:  identify elements in $\alpha$ whenever they have the same image under $f$.  Borrowing a definition from universal algebra and lattice theory, we let $\sim$ be a binary relation whose underlying set of pairs is:
	\begin{equation}
		\kernel f=\setbuild{(a,b)\in\domain{2}\alpha}{f(a)=f(b)}.
	\end{equation}
	The required isomorphism $\iota\colon\faktor{\alpha}{\sim}\to\beta$ is then given by $\iota\colon[a]\mapsto f(a)$, for each $a\in\alpha$.  That $\iota$ is well-defined follows simply from the definition of ${\sim}=\kernel f$.

	We are now required to establish the uniquness of $\sim$ and $\iota$ in their respective roles.  By way of contradiction then, suppose $\sim_0$ is a congruence of $\alpha$ so that there exists an unique isomorphism $\iota_0$ that makes
        \begin{equation}
            \begin{tikzcd}
		\alpha \arrow[rr, "\pi_0",rightarrow]&&   \faktor{\alpha}{\sim_0}\\
		\\
		 &&   \beta \arrow[from=uu,"\iota_0", dashrightarrow] \arrow[from=uull,"f"']
            \end{tikzcd}
        \end{equation}
	commute.  This then clearly implies that ${\sim_0}=\ker f={\sim}$ and thus also $\iota_0=\iota$.
\end{proof}

\section{Hausdorff's characterisation of the countable scattered linear orders}


\begin{prp}[Operations on $\scattered$]\label{prp:OpScattered}
        The following properties hold:
        \begin{enumerate}
            \item   If $I\in\scattered$ and $\alpha_i\in\scattered$ for each $i\in I$ then $\sum_{i\in I}\alpha_i\in\scattered$,
            \item   If $\alpha,\beta\in\scattered$ then $\alpha+\beta,\alpha\cdot\beta\in\scattered$.
        \end{enumerate}
\end{prp}

\begin{proof}
        (1):  Suppose, by way of contradiction, that $\delta\subseteq\sum_{i\in I}\alpha_i$ is dense and define $\delta_i=\delta\cap\alpha_i$, for each $i\in I$.  Note that, since $\delta$ is dense, we cannot have $1<\card{\delta_i}<\aleph_0$ for any $i\in I$.  Therefore, each $\delta_i$ is either infinite or has at most one element.  If $\delta_i$ is finite for each $i\in I$ then $\delta\preceq I$, contradicting the definition of $I$.  Consequently, the must exists some $j\in I$ such that $\delta_j$ is infinite.  However, since $\delta_j\subseteq\alpha_j$, this implies that $\alpha_j$ has a dense subset --- the desired contradiction.

        (2):  Choosing $I=\two$, $\alpha_0=\alpha$ and $\alpha_1=\beta$ in (1) yields $\alpha+\beta\in\scattered$.  In stead, choosing $I=\beta$ and $\alpha_i=\alpha$ for each $i\in\beta$ we get $\alpha\cdot\beta\in\scattered$, as required.
\end{proof}

\begin{dfn}[The class $\VD$]
        By way of transfinite recursion, define for each ordinal $\gamma<\omega_1$ the class of linear orders $\VD_{\gamma}\subseteq\linear$ to be the smallest class which is \textit{closed under isomorphisms} while also satisfying:
        \begin{enumerate}
            \item   $\zero,\one\in\VD_0$,
            \item   if $\alpha_i\in\bigcup_{\beta<\gamma}\VD_{\beta}$, for each $i\in \zeta$, then $\sum_{i\in\zeta}\alpha_i\in\VD$.
        \end{enumerate}
        The class $\VD=\bigcup_{\gamma<\omega_1}\VD_\gamma$ called the class of $\textbf{(countable) very discrete}$ linear orders.
\end{dfn}

\begin{dfn}[$\VD$-rank]
        If $\alpha$ is a very discrete linear order then its $\bm{\mathcal{VD}}$\textbf{-rank} $\vdrank(\alpha)$ is the least ordinal $\beta$ such that $\alpha\in\VD_{\beta}$.
\end{dfn}

\begin{lem}\label{prp:vdsct}
	Every very discrete linear order is scattered.  That is to say that $\VD\subseteq\scattered$.
\end{lem}
\begin{proof}
	We argue by transfinite induction on $\gamma$ that $\VD_\gamma\subseteq\scattered$.  Finite linear orders are (trivially) scattered and thus $\zero,\one\in\scattered$.

	Assume now that for all ordinals $\gamma<\delta<\omega_1$ that $\VD_\gamma\subseteq\scattered$.  By definition, if $\vdrank(\alpha)=\delta$ then there exists for each $i\in\zeta$ an $\alpha_i\in\VD_{\gamma_i}$, for some ordinal $\gamma_i<\delta$, such that
	\begin{equation}
		\alpha=\sum_{i\in\zeta}\alpha_i.
	\end{equation}
	It now follows from proposition (\ref{prp:OpScattered}) and the inductive hypothesis that $\alpha$ is a scattered sum of scattered linear orders and therefore we must have $\alpha\in\scattered$.
\end{proof}

\iffalse\begin{lem}
        If $\alpha\in\VD$ and $\beta\subseteq\alpha$ is convex in $\alpha$ then $\beta\in\VD$.
\end{lem}

\begin{proof}
	 We argue by induction on  $\vdrank(\alpha)$.  The case $\vdrank(\alpha)=0$ is trivial so suppose that $\beta\subseteq\alpha$ and $\beta$ convex in $\alpha$ imply $\beta\in\VD$ whenever $\vdrank(\alpha)<\gamma<\omega_1$.  Suppose now that $\vdrank(\alpha)=\gamma$ and $\beta\subseteq\alpha$ is convex in $\alpha$.  By definition, for each $i\in\zeta$ there must exist $\alpha_i\in\VD_{\gamma_i}$, for some $\gamma_i<\gamma$, such that
	\begin{equation}
	    \alpha=\sum_{i\in\zeta}\alpha_i.
	\end{equation}
	Now define $\beta_i=\beta\cap\alpha_i$ for each $i\in I$.  For each $i\in\zeta$, since $\beta_i$ is convex in $\alpha_i$ and $\vdrank(\alpha_i)<\gamma$, it follows that $\beta_i\in\VD$.  However, note that
	\begin{equation}
	    \beta=\sum_{i\in\zeta}\beta_i.
	\end{equation}
	Therefore, by definition of $\VD$ it follows that $\beta\in\VD$.
\end{proof}\fi

    \begin{lem}\label{prp:sctvd}
        If $\alpha\in\scattered$ and $\card{\alpha}\leq\aleph_0$ then $\alpha$ is very discrete.
    \end{lem}

\begin{proof}
    Suppose $\alpha\in\scattered$ is countable and define a relation $R$ on $\domain{}\alpha$ such that, for every $a,b\in\alpha$, we have $aRb$ whenever $a\leq b$ and $[a,b]$ is very discrete.  Note that $R$ is transitive and thus, by lemma (\ref{lem:IndCong}), it induces a congruence relation $\sim$ on $\domain\alpha$.

        Define the condensation $\beta=\faktor{\alpha}{\sim}$.  If $\card{\beta}=1$ then there is nothing to prove so we may assume $\card{\beta}>1$.  By definition of $\VD$, it now suffices to show that $\beta\embed\zeta$ since every $\alpha\in\beta$ is very discrete simply by definition of $\sim$.  We will suppose to the contrary that $\beta\not\embed\zeta$.  Consequently, $\beta$ must be infinite.
\end{proof}

    	\iffalse\begin{prp}\label{prp:splitfix}
		Suppose $\Lambda$ is the lattice of condensations of the linear order $\alpha$.  Define $S$ to be the functor that takes each factor map $f\colon\alpha\to\beta$ to the splitting $S(f)\colon\alpha\to\faktor{\alpha}{\sim_\beta}$, for some congruence $\sim_\beta$ on $\alpha$, such that the unique isomorphism $\iota_\beta\colon\faktor{\alpha}{\sim_\beta}\to\beta$ makes the diagram
		\begin{equation}
			\begin{tikzcd}
				\alpha\arrow[rr,"S(f)"]&&\faktor{\alpha}{\sim_\beta}\arrow[dd,"\iota_\beta",dashrightarrow]\\
				\\
						       &&\beta\arrow[from=lluu,"f"]
			\end{tikzcd}
		\end{equation}
		commute.  If $\beta\in\Lambda$ is a minimal fixed point of $S^\prime=S\fsplit[-]$, and $\beta$ is not trivially dense then $\beta\in\Dense$.
		\begin{enumerate}
			.
		\end{enumerate}
	\end{prp}\fi

\begin{thm}[Hausdorff's Theorem]
	A linear order $\alpha$ is countable and  scattered iff it is very discrete.
\end{thm}
\begin{proof}

	\forward\	This is lemma (\ref{prp:sctvd}).

	\backward\	This is proposition (\ref{prp:vdsct}).
\end{proof}


\section{The first-order theory of linear orders}


\begin{prp}[Condensations and scatteredness]
        A linear order $\alpha$ is scattered iff there exists no condensation $\beta$ of $\alpha$ such that $\beta$ is dense.
    \end{prp}

    \begin{dfn}[Definably scattered]
	    A linear order $\alpha$ is \textbf{definably scattered} whenever, for every $n\in\nats$, there exists no $\bar{a}\in\domain{n}\alpha$ and no congruence formula $\varphi(x,y,\bar{z})$ of $\alpha$ such that, if $\sim$ is the congruence defined by $\varphi(x,y,\bar{a})$, then $\faktor{\alpha}{\sim}$ is a dense linear order.
    \end{dfn}

\begin{prp}
        If $\alpha$ is some linear order then the following are equivalent:
        \begin{enumerate}
            \item   $\alpha$ is definably scattered,
	    \item   there exists no dense linear order $\delta$ and no $n\in\nats$ such that, for some $\bar{b}\in\domain{n}\delta$ and $\bar{a}\in\domain{n}\alpha$, $(\delta,\bar{b})$ is interpretable in $(\alpha,\bar{a})$,
            \item   the lattice $\defcon\alpha$ is atomic.
        \end{enumerate}
\end{prp}
    	\begin{proof}

	\end{proof}

\begin{prp}\label{prp:dcform}
	For each formula $\varphi=\varphi(x,y,\bar{z})$, with $\bar{z}$ possibly empty, define the formula $\epsilon_\varphi(\bar{z})$ to be the conjunction of the following formulas:
	\begin{enumerate}
		\item	$\forall x\varphi(x,x,\bar{z})$,\hfill(reflexivity)
		\item	$\forall x\forall y(\varphi(x,y,\bar{z})\rightarrow\varphi(y,x,\bar{z}))$,\hfill (symmetry)
		\item 	$\forall x\forall y\forall w(\varphi(x,y,\bar{z})\wedge\varphi(y,w,\bar{z})\rightarrow\varphi(x,w,\bar{z}))$.\hfill (transitivity)
		\item 	$\forall x_0\forall x_1\forall y_0\forall y_1(x_0<x_1\wedge\neg\varphi(x_0,x_1,\bar{z})\wedge\varphi(x_0,y_0,\bar{z})\wedge\varphi(x_1,y_1,\bar{z})\rightarrow y_0<y_1)$.\phantom{}\hfill(compatibility)
	\end{enumerate}
	For every finite tuple $\bar{a}$ of $\alpha$ (of the same length as $\bar{z}$), it holds that $\alpha\models\epsilon_\varphi(\bar{a})$ iff $\varphi(x,y,\bar{a})$ defines a congruence of $\alpha$.
\end{prp}

\begin{prp}
	For each formula $\varphi(x,y,\bar{z})$ define $\theta_\varphi(\bar{z})$ to be the formula
	\begin{equation}
		\neg\forall x\forall y\big((x<y\wedge\neg\varphi(x,y,\bar{z}))\rightarrow\exists w(x<w<y\wedge\neg\varphi(x,w,\bar{z})\wedge\neg\varphi(w,y,\bar{z}))\big)
	\end{equation}
	which formalises the statement: ``the condensation which is induced by the congruence, defined by the formula $\varphi$, is not dense'' and let each $\epsilon_\varphi(\bar{z})$ be defined as in proposition \ref{prp:dcform}.  It then holds that:
	\begin{enumerate}
		\item the theory $\Sigma=\set{\axmlin}\cup\setbuild{\forall\bar{z}(\epsilon_\varphi(\bar{z})\rightarrow\theta_\varphi(\bar{z}))}{\varphi=\varphi(x,y,\bar{z})}$ axiomatises $\Th(\scattered)$, i.e.\ $\dcl{\Sigma}=\Th(\scattered)$,

		\item $\Sigma$ is recursively enumerable.
	\end{enumerate}
	\begin{proof}
		\begin{enumerate}[nosep]
			\item	$\alpha\models\Sigma$ iff $\alpha$ is definably scattered iff $\alpha\models\Th(\scattered)$.
			\item	Since our signature is finite it follows that the set of formulas in our language can be enumerated (via a G\"odel numbering $\godel{-}\colon L\to\nats$).  We can then enumerate $\Sigma$ by declaring a map acting on $\Sigma$ such that $\axmlin\mapsto 0$ and
			\begin{equation}
				\forall z(\epsilon_\varphi(\bar{z})\rightarrow\theta_\varphi(\bar{z})))\mapsto 2^{\godel{\epsilon_\varphi}}\cdot3^{\godel{\theta_\varphi}}.
			\end{equation}
		\end{enumerate}
	\end{proof}
\end{prp}

\begin{dfn}[The class $\Mzero$]
	The class $\Mzero$ is the smallest class of linear orders which satisfies the following:
	\begin{enumerate}
		\item	$\zero,\one\in\Mzero$,
		\item	if $\alpha,\beta\in\Mzero$ then $\alpha+\beta\in\Mzero$,
		\item	if $\alpha\in\Mzero$ then $\alpha\cdot\omega,\alpha\cdot\dual{\omega}\in\Mzero$,
	\end{enumerate}
\end{dfn}

\begin{thm}[L\"auchli and Leonard]
	For every countable $\alpha\in\scattered$ and every $n\in\nats$ there exists a $\beta_n\in\Mzero$ such that $\alpha\nequiv{n}\beta_n$.
\end{thm}
\begin{proof}
	Let $\alpha\in\scattered$ be countable and fix some $n\in\nats$.  Define $\tau=\bigvee_{\beta\in\M}\cha{\beta}{n}$ and let $\varphi=\varphi(x,y)$ bet the formula given by:
	\begin{equation}
		x=y\vee(x<y\wedge\tau^{(x,y)})\vee(y<x\wedge\tau^{(y,x)}).
	\end{equation}
	Note then that $\varphi(x,y)$ defines a reflexive, symmetric binary relation.  To see that the binary relation defined by $\varphi$ is also transitive, and therefore defines a congruence $\sim$, one need only note that the class $\Mzero$ is closed under (finite) sums and that $\one\in\Mzero$.

	We will now proceed to construct a suitable $\beta_n$ by employing Ramsey's theorem.  First, choose $\family{a_i}{i<\omega}$ to be a cofinal sequence in $\alpha$.  Now define a colouring $h$ of $\alpha$ such that $h(a_i,a_j)=\alpha_{k(i,j)}\in\Mzero$ is countable and $\alpha_{k(i,j)}\models\varphi(a_i,a_j)$ whenever $i<j<\omega$.  Recalling that our language is finite, we can choose the $\alpha_{k(i,j)}$ such that there are only finitely many, say $\alpha_0,\dotsc,\alpha_{m-1}$, and we can choose $m$ such that $\bigvee_{0\leq k<m}\cha{\alpha_k}{n}$ is logically equivalent to $\tau$.  From Ramsey's theorem there must then exist an homogeneous sequence $\family{a^\prime_i}{i<\omega}\subseteq\family{a_i}{i<\omega}$ for $h$.  Without loss of generality, however, we may assume that $h(a^\prime_i,a^\prime_j)=\alpha_0$ whever $i<j<\omega$.  We can similarly find a homogenous subsequence $\family{b^\prime_i}{i\in\dual{\omega}}$ of a coinitial sequence $\family{b_i}{i\in\dual{\omega}}$ in $\alpha$ for $h$.  We can then (in a similar fashion) assume that $h(b^\prime_j,b^\prime_i)=\gamma_0$ for some relabelling $\gamma_0,\dotsc,\gamma_{m-1}$ of the linear orders $\alpha_0,\dotsc,\alpha_{m-1}$.  Since $n$-equivalence is preserved under sums it follows that
	\begin{equation}
		\alpha\nequiv{n}(\one+\gamma_0)\cdot\dual{\omega}+\one+(\alpha_0+\one)\cdot\omega
	\end{equation}
	the latter of which belongs to $\Mzero$ by definition.
\end{proof}

\begin{thm}
	The theory $\Th(\scattered)$ is decidable.
\end{thm}
\begin{proof}
\end{proof}

	\bibstyle{amsalpha}

\chapter{Dense linear orders}

The prototypical dense linear orders are the rationals $\eta$ and the reals
$\lambda$.  The former is the unique (up to isomorphism) countable dense linear
order, whereas the latter is its Dedekind completion.

In this chapter we discuss the model theory surrounding the elementary class of
dense linear order orders.  We prove an analogue of the theorem of La\"uchli and
Leonard for this class by extending the class $\Mzero$ using an operation known
as the shuffle operation.  This lays the necessary ground work in order to prove
that the theory of \textit{all} linear orders is decidable.

\section{The shuffle operation}

\begin{prp}\label{prp:ratpart}
	Suppose $n\geq 2$ is a natural number and let $P\subseteq\nats$ bet the set
	of the first $n-1$ primes $p_0,\dotsc,p_{n-2}$.  Choose, for each $i<n-1$,
	the set
	\begin{equation}
		Q_i=\bigcup_{\substack{k\in\ints\\ k\neq
				0}}\setbuild{\faktor{a}{p_i^k}}{a\in\ints\text{ and } p_i\nmid a}.
	\end{equation}
	and define $Q_{n-1}=\rats\setminus\bigcup_{i<n-1}Q_i$.  Then
	$\mathcal{Q}=\setbuild{Q_i}{0\leq i<n}$ is a partition of $\rats$ into dense
	subsets of $\eta$.
\end{prp}
\begin{proof}
	By definition, it clearly follows that $\mathcal{Q}$ is a partition of
	$\rats$.  Hence, all that remains is to show that its equivalence classes
	are dense subset of the rationals.

	We commence by showing that $Q_{n-1}$ is such a dense subset.  Choose $p$ to
	be the least prime such that $p>p_{n-1}$ and define
	\begin{equation}
		S=\bigcup_{\substack{k\in\ints\\ k\neq
				0}}\setbuild{\faktor{a}{p^k}}{a\in\ints\text{ and } p\nmid a}.
	\end{equation}
	Since $S\subseteq Q_{n-1}$, it is sufficient to prove that $S$ is a dense
	subset of $\eta$.  If $s_0<s_1$ are elements of $S$ then there exists
	$a_0,a_1\in\ints$ and $k_0,k_1\in\posnats$ such that, for $i=0,1$, it holds
	that $p\nmid a$ and
	\begin{equation}
		s_i=\frac{a_i}{p^{k_i}}.
	\end{equation}

	Suppose first that $\abs{s_1-s_0}\geq 1$ and choose $m>k_0$ to be the least
	natural number such that $p^m>\abs{s_1-s_0}$.  Define
	\begin{equation}
		t=s_0+\frac{1}{p^m}.\label{eq:qpart}
	\end{equation}
	Then it follows that $s_0<t<s_1$ and
	\begin{equation}
		t=\frac{p^{m-k_0}a_0+1}{p^m}.
	\end{equation}
	Observe that, if $p$ divides the numerator, then we may conclude that $p\mid
		1$.  This is clearly a contradiction and, thus, $p$ cannot divide the
	aforementioned numerator.

	Suppose now, instead, that $\abs{s_1-s_0}<1$ and choose $m>k_0$ to be the
	least natural number such that $p^{-m}<\abs{s_1-s_0}$.  Define $t$ as
	in (\ref{eq:qpart}) then, similar to previously, it follows that
	\begin{equation}
		t=\frac{p^{m-k_0}a_0+1}{p^m}.
	\end{equation}
	Again, using a similar argument, we may conclude that $p$ does not divide
	the numerator above.

	An analagous methodology may be employed in order to prove that $Q_i$ is a
	dense subset of $\eta$ for $i<n$.  One need only substitute each $Q_i$ for
	$S$, and $p_i$ for $p$, in the above proof.
\end{proof}

As we will be revisiting it, we shall refer to the partition of $\rats$, defined
in the proposition above, as the \textit{canonical partition} of $\rats$ into
$n$ dense subsets.  Similarly, the \textit{canonical partition of the reals}
into $n$ dense subsets is laid out in the following corollary.

\begin{cor}
	For every natural $n\geq 2$, there exists a canonical partition
	$\mathcal{R}=\set{R_0,\dotsc,R_{n-1}}$ of $\reals$ such that:
	\begin{enumerate}
		\item $R_{n-1}=\irrats$,
		\item $R_k$ is a countable dense subset of $\lambda$, for each natural $k<n-1$.
	\end{enumerate}
\end{cor}
\begin{proof}
	Choose $R_0,\dotsc,R_{n-2}$ so that they correspond exactly to the
	equivalence classes of the canonical partition of $\rats$ into dense subsets
	of $\eta$.  Together with $R_{n-1}=\irrats$, these then form the equivalence
	classes of the desired partition $\mathcal{R}$ of $\reals$.
\end{proof}

\begin{dfn}[The shuffle operation]
	Let $n\in\nats$ and choose $F$ to be any finite set of linear orders.  Now
	define the \textbf{shuffle operation} on the class of linear orders as
	follows:
	\begin{enumerate}
		\item if $F=\emptyset$ then $\sigma(F)=\zero$;
		\item if $F=\set{\alpha}$ then $\sigma(F)=\alpha\cdot\eta$;
		\item if $n\geq 2$ and $h\colon\rats\to F$ is any colouring of $\eta$
		      whose induced partition on $\rats$ is the canonical one, define
		      \begin{equation}
			      \sigma(F)=\sum_{q\in\eta}h(q).
		      \end{equation}
	\end{enumerate}
	Furthermore, the image of $F$ under $\sigma$ is referred to as the
	\textbf{shuffle} of $F$.
\end{dfn}


\section{The L\"auchli and Leonard result for the class of linear orders}

\begin{dfn}[The class $\M$]
	The class $\M$ is the smallest class of linear orders such that the following holds:
	\begin{enumerate}
		\item	$\zero,\one\in\M$,
		\item	if $\alpha,\beta\in\M$ then $\alpha+\beta\in\M$,
		\item	if $\alpha\in\M$ then $\alpha\cdot\omega,\alpha\cdot\dual{\omega}\in\M$,
		\item	for every finite $F\subseteq\M\setminus\set{\zero}$, it holds that $\sigma(F)\in\M$.
	\end{enumerate}
\end{dfn}

\begin{prp}\label{prp:Msumint}
	If $\alpha_i\in\M$ for each $i\in\ints$ then there exists a $\beta\in\M$ such that
	\begin{equation}
		\sum_{i\in\zeta}\alpha_i\nequiv{n}\beta.
	\end{equation}
\end{prp}
\begin{proof}
	Since $\M$ is closed under finite sums as well as right multiplication by $\omega$ and $\dual{\omega}$, results analagous to propositions \ref{prp:M0sum} and \ref{prp:M0sumop} can be proven in the context of $\M$.  An argument similar to that which appears in the proof of proposition \ref{prp:M0sumint} then yields the desired result.
\end{proof}

\begin{thm}[L\"auchli and Leonard]\label{thm:LLlin}
	For every countable linear order $\alpha$, and every $n\in\nats$, there exists some $\beta_n\in\M$ such that $\alpha\nequiv{n}\beta_n$.
\end{thm}
\begin{proof}
	Define a binary relation $R$ on $\alpha$ such that, for every $x,y\in\alpha$, $xRy$ iff $x\leq y$ and there exists a $\beta\in\M$ such that $[x,y]\nequiv{n}\beta$.  Since $R$ is clearly transitive (simply note $\M$ is closed under finite sums), it induces a congruence $\sim$ on $\alpha$.    We now claim the following:
	\begin{claim}
		For every $\gamma\in\faktor{\alpha}{\sim}$, there exists a $\beta\in\M$ such that $\gamma\nequiv{n}\beta$.
	\end{claim}
	\begin{proof}
		Fix any $\gamma\in\faktor{\alpha}{\sim}$ and note that $\M$ contains all finite ordinals so we may assume, without loss of generality, that $\gamma$ is infinite.  Consequently, at least one of $\omega$ and $\dual{\omega}$ is embeddable in $\gamma$.  There must then exist a family $\family{\gamma_i}{i\in\zeta}$ of (possibly empty) linear orders such that
		\begin{equation}
			\gamma\cong\sum_{i\in\zeta}\gamma_i.
		\end{equation}
		By definition of $\sim$ it now follows that there exists a $\gamma_i^\prime\in\M$, for each $i\in\ints$, such that $\gamma_i\nequiv{n}\gamma_i^\prime$.  Proposition \ref{prp:Msumint} then implies that there exists a $\gamma^\prime\in\M$ such that
		\begin{equation}
			\gamma^\prime\nequiv{n}\sum_{i\in\zeta}\gamma_i^\prime.
		\end{equation}
		From lemma \ref{lem:fvsum} we may conclude that
		\begin{equation}
			\gamma\nequiv{n}\sum_{i\in\zeta}\gamma_i^\prime\nequiv{n}\gamma^\prime,
		\end{equation}
		establishing the claim.
	\end{proof}

	Note that if $\card{\faktor{\alpha}{\sim}}=1$ then our task is complete so suppose, by way of contradiction, that $\card{\faktor{\alpha}{\sim}}>1$.
	\begin{claim}
		$\faktor{\alpha}{\sim}$ is dense.
	\end{claim}
	\begin{proof}
		With our goal being a contradiction, suppose that $\faktor{\alpha}{\sim}$ is not dense.  Therefore, there exists $a,b\in\alpha$ such that $a<b$, $a\not\sim b$ and for every $c\in(a,b)$ either $c\sim a$ or $c\sim b$ but not both.  Note that, since $a\not\sim b$, the interval $(a,b)$ cannot be finite.  Therefore, $(a,b)\cong\sum_{i\in\zeta}\gamma_i$ where $\gamma_i$ is $n$-equivalent to some member $\gamma_i^\prime$ of $\M$, for each $i\in\ints$.  Thus, by proposition \ref{prp:Msumint}, $(a,b)\nequiv{n}\gamma$ for some $\gamma\in\M$.  Since $\one+\gamma+\one\nequiv{n}[a,b]$ by lemma \ref{lem:fvsum} and $\one+\gamma+\one$ is a member of $\M$, it follows by definition that $a\sim b$ --- a contradiction.
	\end{proof}

	Now choose $K=\set{\chi_0,\dotsc,\chi_{m-1}}$, for some $m\in\posnats$, to be an $n$-spectrum for the class $\setbuild{\chi\in\M}{\chi\nequiv{n}\gamma,\text{ for some }\gamma\in\faktor{\alpha}{\sim}}$.
	\begin{claim}
		There exists a maximal $F\subseteq K$ and a nonempty interval $I=(A,B)$ of $\faktor{\alpha}{\sim}$ such that, for every $\chi\in F$, the set $D_\chi(I)=\setbuild{\gamma\in I}{\gamma\nequiv{n}\chi}$ is dense in $I$.
	\end{claim}
	\begin{proof}
		Argue by induction on $m$.  If $m=1$ then, since $\faktor{\alpha}{\sim}$ is dense we can choose any $A,B\in\faktor{\alpha}{\sim}$ such that $A<B$ and it will immediately follow that if $I=(A,B)$ then $D_{\chi_0}(I)$ is dense in $I$, since every member of $\faktor{\alpha}{\sim}$ is $n$-equivalent to $\chi_0$.

		Assume now that the claim holds for each $m<m^\prime\in\posnats$.  Let $m=m^\prime$ and suppose to the contrary that for every $F^\prime\subseteq K$ and every (nonempty) open interval $J\subseteq\faktor{\alpha}{\sim}$ there exists some $\chi_k\in F^\prime$ such that $0\leq k<m$ and $D_{\chi_k}(J)$ is not dense in $J$.  Note that, by assumption, there must exist a $\chi\in K$ such that $D_{\chi}(\faktor{\alpha}{\sim})$ is not dense in $\faktor{\alpha}{\sim}$.  Without loss of generality we may assume that $\chi=\chi_{m-1}$ and consequently there exists a (nonempty) open subset $J$ of $\faktor{\alpha}{\sim}$ such that $D_{\chi_{m-1}}(J)$ is not dense in $J$.  Note in particular that we can choose $J$ such that $\chi_{m-1}$ has no $n$-equivalent in $J$.

		By definition, there exists an $F^\prime\subseteq\set{\chi_0,\dotsc,\chi_{m-2}}$ which is an $n$-spectrum for the class $\setbuild{\chi^\prime\in\M}{\chi\nequiv{n}\gamma\text{ for some }\gamma\in J}$.  It then follows from the inductive hypothesis that there exists a maximal $F\subseteq F^\prime$ and some open interval $I=(A,B)\subseteq J$, where $A<B$, such that $D_{\chi}(I)$ is dense in $I$, for each $\chi\in F$.  Since $\chi_{m-1}\notin J\supseteq I$, it follows that $\chi_{m-1}$ is not dense in $I$ and, consequently, $F\subseteq K$ is maximal, as required.  This is the sought-after contradiction, thus establishing the claim.
	\end{proof}
	Note that $F$ is now an $n$-spectrum for the class
	\begin{equation}
		\setbuild{\chi\in\M}{\chi\nequiv{n}\gamma\text{ for some }\gamma\in I}.
	\end{equation}
	Consequently, since $D_{\chi}(I)$ is dense in $I$ for each $\chi\in F$, it follows that
	\begin{equation}
		\sigma(F)\nequiv{n}\sum I=\sum_{\gamma\in I}\gamma.
	\end{equation}
	Therefore, since $\sigma(F)\in\M$, it follows that $I$ has exactly one element.  This then contradicts the fact that $I$ is an open interval.  Hence, we may now conclude that $\faktor{\alpha}{\sim}$ has only one element, yielding the result.

\end{proof}

\begin{lem}\label{lem:relin}
	The set $\setbuild{(\alpha,\sigma)}{\alpha\in\M\text{ and }\alpha\models\sigma}$ is recursively enumerable.
\end{lem}
\begin{proof}
	The proof is near identical to that of Lemma \label{lem:rescat}.  We shall only highlight the additions necessary to prove the current result.

	It is necessary to add an additional clause, respectively, to the defintions of $L_\alpha$, $S_\alpha$ and $T_\alpha$.  If, for some $n\in\nats$, there exists $\alpha_0,\dotsc,\alpha_{n-1}\in\M\setminus\set{\zero}$ such that $\alpha=\sigma(\alpha_0,\dotsc,\alpha_{n-1})$ then $L_\alpha$ is obtained from $L$ by adding new unary relation symbols $r_0,\dotsc,r_{n-1}$ and a new binary relation symbol $\theta$.

	We choose $S_\alpha$ to consist of sentences expressing that ``$\theta$ is a congruence relation'' as well as, for each $i<n$, the sentences:
	\begin{itemize}
		\item	$\exists x r_i(x)$,
		\item	$\forall x\big(r_0(x)\vee\dotsb\vee r_{n-1}(x)\big)$,
		\item	for each $j<i$: $\neg\exists x\big(r_i(x)\wedge r_j(x)\big)$,
		\item	$\forall x\forall y\big(x<y\rightarrow\exists z(r_i(z)\wedge x<z \wedge z<y))\big)$,
		\item	$\forall x\big(r_i(x)\rightarrow\forall y(\theta(y,x)\rightarrow r_i(y))\big)$.
		\item	$\forall x\big(\exists y(\neg\theta(x,y)\wedge y<x)\wedge\exists y(\neg\theta(x,y)\wedge x<y)\big)$
	\end{itemize}

	Define the $L_\alpha$-theory
	\begin{equation}
		T_\alpha=S_\alpha\cup\bigcup_{i<n}\setbuild{\forall x\left(r_i(x)\rightarrow\sigma^{\theta(v,x)}\right)}{\alpha_i\models\sigma}
	\end{equation}
	and note that, by definition, $T_\alpha$ is recursively enumerable.  Hence, we need only show that $T_\alpha$ is complete.  To this end, consider any $\mathfrak{M},\mathfrak{N}\models T_\alpha$.

	Fix any $k\in\nats$, so that our task is reduced to showing that $\mathfrak{M}\nequiv{k}\mathfrak{N}$.  This is facilitated by the following description of a winning strategy for player $\Right$ in the game $\EF_k(\mathfrak{M},\mathfrak{N})$.

	Should it be the case that $n\leq 1$ then it follows that $\mathfrak{M}\equiv\mathfrak{N}$ so we may suppose that $n>1$.  Let $a_0\in M$ be the first move of player $\Left$.  By definition of $T_\alpha$, there exists a natural $i_0<n$ such that $\mathfrak{M}\models r_{i_0}(a_0)$.

	It then follows from the definition of $T_\alpha$ that there exists a countermove $b_0\in N$ for player $\Right$ such that $\mathfrak{N}\models r_{i_0}(b_0)$.  Any such $b_0$ will suffice and a similar tactic is used if $\Left$ instead chooses his first move in $\mathfrak{N}$.

	Suppose now that $\bar{a}\in M^{k-1}$ and $\bar{b}\in N^{k-1}$ and we are in the position $(\bar{a},\bar{b})$ in the game.  Without loss of generality we may assume $a_i<a_{i+1}$, for $i<k$, and need only consider the case where $\Left$ never repeats any of his moves.

	Suppose, during the $k$-th round, $\Left$ plays some $c\in M$.  There must then exist, by definition of $T_\alpha$, some $i_k<n$ such that $\mathfrak{M}\models r_{i_{k-1}}(c)$.  If $c>a_{k-1}$ then there must exists some $d\in N$ such that $d>\bar{b}$ and $\mathfrak{N}\models r_{i_{k-1}}(d)$.  A similar scenario is encountered if $c<a_0$: there exists a $d<\bar{b}$ in $\mathfrak{M}$ such that $\mathfrak{N}\models r_{i_{k-1}}(d)$.

	Now, consider the case where $a_j<c<a_{j+1}$, for some $j<k$.  If $b_{j+1}\leq b_j$ then $\Right$ may play any element but otherwise there exists a $d\in N$ such that $b_j<d<b_{j+1}$ and, for $i<n$,
	\begin{equation}
		\mathfrak{M}\models r_i(c)\quad\iff\quad\mathfrak{N}\models r_i(d).
	\end{equation}

	Should $\Left$ instead play a move $d\in N$ then $\Right$ plays as follows.  If $b_{j+1}\leq b_j$, for some $j<k$ then $\Right$ may play any move in $\mathfrak{M}$ but otherwise $\Right$ plays similarly to the previous case where $\Left$ chose a move in $\mathfrak{M}$.  The roles of the two structures are merely interchanged in this case.

	Define the set
	\begin{equation}
		R^\prime=\setbuild{(\alpha,\sigma)}{\alpha\in\M\text{ and }\alpha\models\sigma}.
	\end{equation}
	We now adapt the algorithm given in the proof of Lemma \ref{lem:rescat} with the goal of listing the elements of $R^\prime$.  This is accomplished by also taking shuffles into account.

	Assume the entries $(\alpha_i,\sigma_i)$, for $i=0,\dotsc,n_0-1$, have been listed.  Continue the list sequentially with entries $(\alpha_i,\sigma_i)$, for $i\geq n_0$, as follows:
	\begin{enumerate}
		\item 	list all pairs of the form $(\alpha_i,\sigma)$ such that $\sigma\in S_{\alpha_i}$ and $i<n_0$ and let $n_1$ denote the resulting number of entries in the list;
		\item 	list all pairs of the form $(\alpha_i, \sigma)$, for $i<n_1$, where $\sigma$ is the direct consequence of an inference rule from $\sigma_{i_0},\dotsc,\sigma_{i_{k-1}}$ and, for each $j<k$, $(\alpha_i,\sigma_{i_j})$ has already appeared in the list and let $n_2$ denote the resulting number of entries;
		\item	list all pairs of the form $(\alpha,\sigma)$ where, for $i,j<n_2$, $\alpha=\alpha_i+\alpha_j$ and $\sigma=\sigma_i^{r_0(v)}\wedge\sigma_j^{r_1(v)}$ and let $n_3$ denote the resulting number of entries in the list;
		\item	list all pairs of the form $(\alpha_i\cdot\omega,\sigma)$, for $i<n_3$, such that $\sigma=\forall w\sigma_i^{\theta(w,v)}$ and let $n_4$ denote the resulting number of entries in the list;
		\item	list all pairs of the form $(\alpha_i\cdot\dual{\omega},\sigma)$, for $i<n_4$, such that $\sigma=\forall w\sigma_i^{\theta(w,v)}$ and let $n_5$ denote the resulting number of entries in the list;
		\item	for each $k<n_5$ and nonzero linear orders $\alpha_{i_0},\dotsc,\alpha_{i_k}$, add $(\alpha,\sigma)$ to the list whenever
		      \begin{equation}
			      \alpha=\sigma(\alpha_{i_0},\dotsc,\alpha_{i_{k-1}})
		      \end{equation}
		      and, for some $j<k$,
		      \begin{equation}
			      \sigma=\forall x\left(r_{i_j}(x)\rightarrow\sigma_{i_j}^{\theta(v,x)}\right);
		      \end{equation}
		\item	redefine $n_0$ to denote the current length of the list;
		\item	repeat steps 1 to 7.
	\end{enumerate}

	An argument by means of induction (on $\alpha\in\M$) will now show that the procedure described above will list any desired member of $R^\prime$ after a finite amount of time.  Consequently, the same may be said of $R$.
\end{proof}

\begin{thm}
	The theory of linear orders is decidable.
\end{thm}
\begin{proof}
	Since the theory of linear orders is finitely axiomatisable, say by $T=\set{\axmlin}$, there exists a machine $M_0$ that lists all of its consequences.  This is accomplished, using inference rules, by mechanically generating all proofs from the theory $T$.

	From Lemma \ref{lem:relin}, there exists a machine $M_1$ that mechanically lists all pairs $(\alpha,\sigma)$ such that $\alpha\in\M$ and $\alpha\models\sigma$.  Now consider a machine $M$ that, given a sentence $\sigma$, alternately lists entries from each of the two lists until either $\sigma$ or a pair of the form $(\alpha,\neg\sigma)$ appears on the list.  In the former case $M$ prints the string ``True'' and in the latter it prints ``False''.

	It follows from Theorem \ref{thm:LLlin} and the definitions of $M_0$ and $M_1$ that $M$ always halts, irrespective of the choice of $\sigma$.  This is as required.
\end{proof}


\section{Model-theoretic aspects of the class \text{$\dense$}}

The class $\dense$ of dense linear orders without endpoints is relatively
well-behaved, from a model-theoretic perspective.  In this section we show that,
up to isomorphism, $\Th(\dense)$ has only a single countable model, namely the
rationals $\eta$.  This is a standard result known as Cantor's Theorem and it
establishes the property it establishes is known as $\aleph_0$-categoricity of
$\Th(\dense)$.  Further, we take a brief look at some other, equally
interesting, model-theoretic properties exhibited by this theory: quantifier
elimination, model-completeness and posession of a so-called prime model.

With $\alpha$ and $\beta$ both countable, consider a modification
$G=G(\alpha,\beta)$ of the game $\EF_{\omega}(\alpha,\beta)$, defined as
follows.  A play $(\bar{c},\bar{d})$ in the game $G$ is winning for $\Left$ iff
the following holds:
\begin{enumerate}[noitemsep]
	\item The player $\Left$ never plays an element from the same structure on
	      two consecutive turns.
	\item At least one of the following is satisfied:
	      \begin{enumerate}
		      \item the play $(\bar{c},\bar{d})$ is winning for $\Left$ in
		            $\EF_\omega(\alpha,\beta)$,
		      \item there exists a member of $\alpha$ not occuring in $\bar{c}$ or
		            there exists a member of $\beta$ not occuring in $\bar{d}$.
	      \end{enumerate}
\end{enumerate}

Note, for $\alpha$ and $\beta$ both countable, it follows by definition
that $\Right$ has a winning strategy for $G$ iff there exists an isomorphism
$f\colon\alpha\to\beta$.  The forward implication can be shown by considering
any play of the game in which $\Left$ plays optimally by alternating between
$\alpha$ and $\beta$.

Conversely, any such isomorphism $f$ will determine a winning strategy for
$\Right$ in the following manner.  A move $a$ in $\alpha$ by $\Left$ is
countered by playing $f(a)$ while a move $b$ in $\beta$ is countered by
$\inv{f}(b)$.

\begin{thm}[Cantor's Theorem]\label{thm:cantor}
	If $\alpha,\beta\in\dense$ is countable then there exists an isomorphism
	$f\colon\alpha\to\beta$.
\end{thm}
\begin{proof}
	Let $G=G(a,b)$ be the game described previously and fix enumerations
	$\bar{a}$ and $\bar{b}$ of $\alpha$ and $\beta$, respectively.  Note that
	our task is achieved if we can exhibit a winning strategy $\sigma$ for
	$\Right$ in $G$.

	We now describe the necessary $\sigma$ to conclude the proof in a recursive
	fashion.  Any initial move by $\Left$ can be countered by any choice of
	element in the opposite structure.

	Suppose now that $n-1$ rounds have been played, for some $n\geq 2$.  As the
	cases are approached similarly, we may assume $\Left$ played his initial
	move in $\alpha$.  It is also reasonable that $\Left$ plays optimally in the
	sense that he never plays from the same structure in two of his consecutive
	turns.

	Now, if $n$ is odd then $\Left$'s last move was some $a_{k_{n-1}}\in\alpha$.
	The proper response for $\Right$ in this scenario would be to choose the
	first $b_{j_{n-1}}$ occuring in $\bar{b}$ such that
	\begin{equation}
		(\alpha,a_{0},\dotsc,a_{k_{n-1}})\nequiv{0}(\beta,b_{0},\dotsc,b_{j_{n-1}}),
	\end{equation}
	the existence of which is guaranteed by the density of $\beta$.  In other
	words, the (simple) expansions above should satisfy the same atomic
	sentences.

	If, instead, $n$ is even then $\Left$ has just played some $b_{k_{n-1}}$ so
	that the appropriate response is similarly determined.  The only difference
	being the reversal of the roles of the structures: $\Left$ chose from $\beta$
	and $\Right$ will respond with an element from $\alpha$.
\end{proof}

\begin{prp}\label{prp:xcantor}
	If $\alpha\in\dense$ is countable and, for some $n\in\nats$, we fix any
	$\bar{a}\in\domain{n}\alpha$ and any $\bar{b}\in\rats^n$ then there exists an
	isomorphism $f\colon\alpha\to\eta$ such that $f[\bar{a}]=\bar{b}$.
\end{prp}
\begin{proof}
	We prooceed by means of induction, nothing that the case $n=0$ is Cantor's
	Theorem (Theorem \ref{thm:cantor}).  Suppose now the result holds for each
	$k\leq n$ for some arbtrary $n\in\nats$ and note that we may assume
	$\bar{a}=(a_0,\dotsc,a_n)$ and $\bar{b}=(b_0,\dotsc,a_n)$ such that $a_i<a_j$
	and $b_i<b_j$, whenever $i<j<\omega$.  It now follows from the inductive
	hypothesis that there exists isomoprhisms $f_0\colon\alpha^{<a_0}\to\eta^{<b_0}$
	and $f_1\colon(\alpha^{>a_0},a_1,\dotsc,a_n)\to(\beta^{>a_0},b_1,\dotsc,b_n)$.
	By identifying $f_0$ and $f_1$ with their (respective) underlying sets, we may
	define the new map $f\colon\alpha\to\beta$ such that
	$f=f_0\cup\set{(a_0,b_0)}\cup f_1$.  By definition, we have thusly obtained an
	order-preserving map $f\colon(\alpha,\bar{a})\to(\beta,\bar{b})$.  It is not
	difficult to see, simply from its definition, that $f$ is in fact an isomorphism
	$f\colon\alpha\to\beta$ satisfying $f[\bar{a}]=\bar{b}$, as required.
\end{proof}

As a consequence of the previous proposition we get the following:
\begin{prp}
	The linear order $\eta$ is saturated.
\end{prp}
\begin{proof}
	Choose any finite $A\subseteq\rats$.  Suppose that $\Phi(x)\subseteq L_1(A)$
	is an $1$-type of $\eta$ over $A$.  By definition there exists a
	$\alpha\succcurlyeq\eta$ such that $\Phi$ is realised by some $b\in\alpha$ i.e.\
	$\alpha\models\Phi(b)$.  By the Downwards L\"owhenheim-Skolem Theorem, however,
	we may assume that $\alpha$ is countable.

	Note we may set $A=\set{a_0,\dotsc,a_{k-1}}$ and thus, from Proposition
	\ref{prp:xcantor}, if $\bar{a}=(a_0,\dotsc,a_{k-1})$ then there must exist an
	isomorphism $f\colon(\eta,\bar{a})\to(\alpha,\bar{a})$.  Therefore, it must
	follow that $\eta\models\varphi(f(b))$, for each $\varphi\in\Phi(x)$, and thus
	$\Phi(x)$ is realised by the element $f(b)\in\rats$.
\end{proof}


\begin{thm}
	The linear order $\eta$ is a prime model of the theory $\Th(\dense)$.
\end{thm}
\begin{proof}
	Suppose $\alpha\models\Th(\dense)$ then $\alpha$ must be infinite and thus, from the downward L\"owenheim-Skolem Theorem, there exists a countable $\alpha_0\preccurlyeq\alpha$.  From Cantor's theorem it follows that $\alpha_0\cong\eta$ and, therefore, there must exist an elementary embedding of $\eta$ into $\alpha$.
\end{proof}

\begin{thm}
	The theory $T=\Th(\dense)$ has quantifier elimination.
\end{thm}
\begin{proof}
	By Lemma \ref{lem:qelim}, it is enough to show that $T$ is substructure complete.  Suppose that $\alpha\models T$ and choose any $\alpha_0\subseteq\alpha$.  Define $A=\domain{}\alpha_0$ and suppose that $\mathfrak{M},\mathfrak{N}\models T\cup\diag{\alpha_0}$ are $L(A)$-structures.  Note that we may assume, without loss of generality, that $A\subseteq M,N$ such that $\mathfrak{M}=(\beta,A)$ and $\mathfrak{N}=(\gamma,A)$ for linear orders $\beta,\gamma\models T$.  We are now required to show that
	\begin{equation}
		\mathfrak{M}\equiv\mathfrak{N}.
	\end{equation}
	However, it is enough to simply prove that
	\begin{equation}
		(\beta,B)\equiv(\gamma,B),\quad\text{for every finite }B\subseteq A.
	\end{equation}

	Suppose that $B\subseteq A$ is finite then by the downwards
	L\"owenheim-Skolem Theorem there exists countable $\beta_0\elmsub\beta$ and
	$\gamma_0\elmsub\gamma$ such that $B\subseteq\beta_0,\gamma_0$.  We may assume
	that the elements of $B$, for some $n\in\nats$, are precisely
	$b_0<b_1<\dotsb<b_{n-1}$.  It now follows from Cantor's Theorem that each of the
	intervals $(b_0,b_1)_{\beta_0},\dotsc,(b_{n-2},b_{n-1})_{\beta_0}$ as well as
	$\beta_0^{<b_0}$ and $\beta_0^{>b_{n-1}}$ are isomorphic to $\eta$.  The same is
	true of the corresponding intervals in $\gamma_0$.  This then gives rise to
	isomorphisms $f_0,\dotsc,f_n$ (whose domains we will choose to be subsets of
	$\domain{}\beta_0$) between the corresponding open intervals of $\beta_0$ and
	$\gamma_0$ such that each member of the range of $f_i$ is less (in $\gamma_0$)
	than every member member of the range of $f_j$ whenever $i<j\leq n$.  One can
	then define (in an obvious manner) an isomorphism $f\colon\beta_0\to\gamma_0$
	which is an extension of $f_i$ to all of $\beta_0$, for each $i\leq n$, such
	that $f(b_j)=b_j$ whenever $j<n$.  Therefore, since no element of $B$ is an
	endpoint of either $\beta_0$ or $\gamma_0$, it follows that
	$(\beta_0,B)\cong(\gamma_0,B)$ and thus $(\beta_0,B)\equiv(\gamma_0,B)$.  The
	proof is then concluded by recalling that $\beta_0\elmsub\beta$ and
	$\gamma_0\elmsub\gamma$.
\end{proof}

\begin{cor}
	The theory $T=\Th(\dense)$ is model-complete.
\end{cor}
\begin{proof}
	Immediate from the previous theorem and Corollary \ref{cor:qemc}.
\end{proof}

We've already seen that $\eta$ is saturated and can thus be thought of from an intuitive standpoint as being very ``thick'' among linear orders.  Contrary to what one might expect, though, the next result shows that $\eta$ is simultaneously also an atomic structure and can therefore be thought of as being very ``thin''!

\begin{prp}
	The linear order $\eta$ is atomic.
\end{prp}
\begin{proof}
	Let $T=\Th(\dense)$ then the result follows from the fact that $T$ admits quantifier elimination and the fact that there are only finitely many inequivalent formulas in finitely many variables in a finitary language.
\end{proof}

	\bibstyle{amsalpha}

\chapter{Complete linear orders and the reals}

\section{Continuous linear orders and separability}

\begin{dfn}[Completeness]
	A linear order $\alpha$ is said to be \textbf{complete} if every (nonempty)
	subset of $\domain\alpha$, bounded above in $\alpha$, has a least upper
	bound.
\end{dfn}

Equivalently, $\alpha$ is complete whenever all of its nonempty subsets, bounded
below, have a greatest lower bound.

\begin{dfn}[Continuous linear order]
	If $\alpha\in\dense$ then we call $\alpha$ \textbf{continuous} precisely when it is complete.
\end{dfn}

\begin{dfn}[Separable]
	A linear order $\alpha$ is called \textbf{separable} whenever
	$\alpha\notin\set{\zero,\one}$ and there exists a countable subset of
	$\domain{}\alpha$ which is dense in $\alpha$.
\end{dfn}

A notable property of all separable linear orders is that they are necessarily
also dense.  Suppose, for instance, $\alpha$ is a separable linear order then
choose $D\subseteq\alpha$ to be countable dense in $\alpha$.  If $a<a^{\prime}$
are arbitrary elements of $\alpha$ then, by definition, there must exist some
$d\in D$ such that $a<d<a^{\prime}$.  Hence, the separable linear orders form a
subclass of the class $\Dense$ of all (nontrivially) dense linear orders.

\begin{prp}[Separability]\label{prp:sep}
	If $\alpha$ is a separable linear order then there exists an embedding
	$h\colon\alpha\to\lambda$.
\end{prp}
\begin{proof}
	Suppose $\alpha$ is separable.  By definition, there exists a countable
	$\beta\subseteq\alpha$ which is dense in $\alpha$.  From Cantor's Theorem, there
	exists an isomorphism $h_{0}\colon\beta\to\eta$. Define a map
	$h\colon\alpha\to\lambda$ such that, for $a\in\alpha$,
	\begin{equation}
		h(a)=\sup\setbuild{h_{0}(b)}{b\in\beta\text{ and }b<a}.
	\end{equation}
	It is now sufficient to prove that $h$ is an embedding.

	We first establish that $h$ is strictly increasing on $\alpha$, that is to
	say:
	\begin{equation}
		a<a^{\prime}\implies h(a)<h(a^{\prime}),
	\end{equation}
	for all $a,a^{\prime}\in\alpha$.  Fix any $a<a^{\prime}$ in $\alpha$. Since
	$\beta$ is dense in $\alpha$, there must exist a $b\in\beta$ such that
	\begin{equation}
		a<b<a^{\prime}.
	\end{equation}
	Consequently, by definition of $h$, we must have
	\begin{equation}
		h(a)<h(a^{\prime}).
	\end{equation}
	It follows that $h$ is a (strictly) increasing map and, thus, $h$ is
	an embedding of $\alpha$ into $\lambda$.
\end{proof}

\begin{thm}[Characterising the reals]\label{thm:rchar}
	Every separable continuous linear order is isomorphic to $\lambda$.
\end{thm}
\begin{proof}
	Suppose $\alpha$ is a separable continuous linear order.  From Proposition
	\ref{prp:sep}, there exists an embedding $h\colon\alpha\to\lambda$.  Since
	$\alpha$ is separable there exists some denumerable $\beta\subseteq\alpha$
	which is dense in $\alpha$.

	Since $h$ is an embedding and $\beta$ denumerable, dense and without
	endpoints, it follows from Cantor's theorem that we must have
	\begin{equation}
		\beta\cong\eta\cong h[\beta].
	\end{equation}
	Now define $\alpha^\prime=\comp(h[\beta])$ so that we must have
	$\alpha\cong\alpha^\prime$, since $\alpha$ is complete and $\beta$ is dense in
	$\alpha$.

	Lastly, we will claim that $\alpha^\prime$ is convex.  Suppose to the
	contrary there exists $a,b\in\alpha^\prime$, with $a<b$ and a $c\in\reals$ such
	that $a<c<b$ but $c\notin\alpha^\prime$.  Define
	$c_0=\sup\setbuild{x\in\alpha^\prime}{x<c}$. By definition, we have $c_0\leq c$.
	If $c_0=c$ then the completeness of $\alpha^\prime$ guarantees
	$c=c_0\in\alpha^\prime$.

	Assume now, instead, that $c_0<c$ then, since $h[\beta]$ must be dense in
	$\alpha^\prime$, there exists some $d\in\beta$ such that $c_0<h(d)<c$.  However,
	since $h(d)\in\beta^\prime\subseteq\alpha^\prime$, this contradicts the
	definition of $c_0$ which requires that $h(d)\leq c_0$.
\end{proof}


\section{The Suslin property and the first order theory of the coloured reals}

Note that in this section we will refer to \textit{coloured linear orders}
(linear orders with finitely many unary relations defined on them) simply as
linear orders.  This simplifies the language used but it should be noted that
all the following results also apply in the monochromatic case i.e.\ in the
absence of any colours.

\begin{dfn}[Quasi-separable]\label{dfn:quasiseparable}
	A linear order $\alpha$ is said to be $\textbf{quasi-separable}$ iff every
	densely ordered condensation $\beta\not\cong\one$ of $\alpha$ has a subset which
	is dense in $\beta$ and consists of only singleton subsets of $\alpha$.
\end{dfn}

\begin{dfn}[Suslin property]
	A continuous linear order $\alpha$ is said to posses the \textbf{Suslin
		property} iff every set of \textit{pairwise disjoint open intervals} in
	$\alpha$ is at most countable.
\end{dfn}

Clearly $\lambda$ has the Suslin property, since any open interval in $\lambda$
necessarily contains a rational number.  A question that immediately arises is
whether or not Suslinity could be a stand-in for separability in characterising
the real number line.  Unfortunately, this problem cannot be decided within
ordinary ZFC and additional set-theoretic assumptions are required in order to
give a definite yes or no answer.  The reader may refer to any of the standard
texts on set theory, should they decide to embark on this quest.

The following lemma illustrates that quasi-separable linear orders are in
abundance and that $\lambda$ is among them:

\begin{prp}
	If $\alpha$ is a complete linear order with the Suslin property then
	$\alpha$ must be quasi-separable.
\end{prp}
\begin{proof}
	Let $\alpha$ be any continuous linear order that has the Suslin property and
	choose $\beta\not\cong\one$ to be any dense condensation of $\alpha$, say
	$\beta=\faktor{\alpha}{\sim}$.  By way of contradiction, assume there exists no
	$D\subseteq\beta$ which is dense in $\beta$ and consists of only singleton
	subsets of $\alpha$.

	Given this setting, there must exist elements $a<b$ in $\alpha$ such that
	$\faktor{a}{\sim}<\faktor{b}{\sim}$ and no singleton subset of $\alpha$ is a
	member of the open interval
	\begin{equation}
		I_{a,b}=\left(\faktor{a}{\sim},\faktor{b}{\sim}\right).
	\end{equation}

	Since $I_{a,b}$ has no singleton members, each of its elements necessarily
	contains an open interval as a subset.  Therefore, by Suslinity, it follows
	that $I_{a,b}$ must be countable.  Since $\beta$ (and thus $I_{a,b}$) is dense,
	it follows by Cantor's Theorem that $I_{a,b}$ has order type $\eta$.

	Since $\eta$ is not complete, we may choose $G\subseteq I_{a,b}$ to be
	bounded above in $I_{{a,b}}$ but without a supremum in $I_{a,b}$.  We
	now argue that $\bigcup G$ has no supremum, thereby contradicting the
	completeness of $\alpha$.

	Suppose to the contrary that $\bigcup G$ has a supremum
	$u\in \bigcup I_{a,b}$.  By definition, it follows that $\faktor{u}{\sim}$ is an
	upper bound of $G$ in $I_{a,b}$.  Now choose any $u^{\prime}\in\bigcup I_{a,b}$
	such that $\faktor{u^{\prime}}{\sim}$ is an upper bound of $G$.  Consequently,
	by definition, $u^{\prime}$ is an upper bound of $\bigcup G$ and thus
	$u^{\prime}\geq u$.  By definition, we may conclude that
	\begin{equation}
		\faktor{u^{\prime}}{\sim}\geq\faktor{u^{\prime}}{\sim}
	\end{equation}
	so it follows that $\faktor{u}{\sim}$ is the supremum of $G$.  This
	contradicts the definition of $G$ and the result now follows.
\end{proof}

\begin{dfn}[Definably quasi-separable]
	A linear order $\alpha$ is called \textbf{definably quasi-separable} iff
	each densely ordered condensation $\beta\not\cong\one$ induced by a definable
	congruence of $\alpha$ has a dense set of singleton subsets of $\alpha$.
\end{dfn}

\begin{dfn}[Definable complete]
	A linear order $\alpha$ is \textbf{definably complete} whenever each of its
	definable subsets, bounded above, has a supremum.
\end{dfn}

Similar to before, $\alpha$ is definably complete in the sense just defined iff
each of its definable subsets, bounded below, has an infimum.  To this end,
suppose $\bar{a}$ is a finite tuple in $\alpha$ and $\varphi(x,\bar{a})$ defines
a subset $B_{\varphi}$ of $\alpha$, bounded below.

Define the formula
\begin{equation}
	\psi(x,\bar{a})\define\neg\exists y(\varphi(y,\bar{a})\wedge y<x).
\end{equation}
It follows immediately that $\psi=\psi(x,\bar{a})$ defines a subset $A_{\psi}$
of $\alpha$, bounded above by every element of $B_{\varphi}$.  Furthermore, we
have
\begin{equation}
	\sup A_{\psi}=\inf B_{\varphi}.
\end{equation}

The reverse direction of this implication follows similarly.  Hence, as
anticipated, the two alternate formulations of definable completeness are
equivalent.

\begin{prp}\label{prp:qdense}
	For each natural $n\geq 2$, a partition $Q_0,\dotsc,Q_{n-1}$ of $\rats$
	exists such that, for each $i=0,\dotsc,n-2$, $Q_i$ is dense in $\eta$.
\end{prp}
\begin{proof}
	Let $p_{0},\dotsc,p_{n-1}$ denote the first $n-1$ prime numbers.  Now, for each
	$i<n-1$, define the set
	\begin{equation}
		Q_{i}=\setbuild{\frac{m}{p_{i}^{k}}}{k,m\in\ints\setminus\set{0}\text{
				and  }\gcd(m,p_{i})=1},
	\end{equation}
	and let $Q_{n-1}=\rats\setminus\bigcup_{i<n-1}Q_{i}$.

	We now claim that each such $Q_{i}$, for $i<n$, is dense in $\eta$.  Fix any
	distinct $q,q^{\prime}\in\eta$ and first consider the case $i<n-1$.  Note
	that there must exist a least $k\in\posnats$ such that
	\begin{equation}
		\frac{1}{p_{i}^{k}}<\abs{q^{\prime} - q}.
	\end{equation}

	Since the distance between two elements of $\eta$ is preserved under the map
	$x\mapsto -x$, we may assume without loss of generality that
	$0\leq q<q^{\prime}$.  There now exists a least $m\in\nats$ such that
	\begin{equation}
		q<\frac{m}{p_{i}^{k}}.
	\end{equation}
	By definition of $m$, it follows that
	\begin{equation}
		\frac{m-1}{p_{i}^{k}}<q.
	\end{equation}
	Therefore, by definition of $k$, we may conclude that
	\begin{equation}
		q<\frac{m}{p_{i}^{k}}<q^{\prime}.
	\end{equation}
	Reducing the fraction above to its simplest form then yields the desired
	result.

	Let $p$ be the least prime number such that $p>p_{n-1}$.  Repeating the
	argument above, and substituting $p$ for $p_{i}$, will show that $Q_{n-1}$
	is also dense in $\eta$.
\end{proof}

\begin{prp}\label{prp:csums}
	If $\delta$ is a complete linear order and $\family{\alpha_i}{i\in\delta}$
	is a family of complete linear orders with endpoints then the sum
	\begin{equation}
		\alpha=\sum_{i\in\delta}\alpha_i,
	\end{equation}
	is also a complete linear order.
\end{prp}
\begin{proof}
	Suppose $B\subseteq\alpha$ is bounded above in $\alpha$ and identify each
	$\alpha_{i}$ with its corresponding image under the canonical embedding of
	$\alpha_{i}$ into $\alpha$. Choose $d\in\delta$ to be the supremum of
	all $i\in\delta$ such that $B\cap\alpha_{i}\neq\emptyset$.

	If $B\cap\alpha_{d}=\emptyset$ then let $u$ be the least element of
	$\alpha_{d}$, otherwise, choose $u$ to be the supremum of $B\cap\alpha_{d}$.
	Since $\alpha_{d}$ has a greatest element, it follows that $u\in\alpha_{d}$
	in either case.  Consequently, by definition, $u$ is an upper bound of $B$.

	All that remains is to demonstrate the minimality of $u$.  Suppose then that
	$u^{\prime}$ is any other upper bound of $B$.  If $u^{\prime}\in\alpha_{e}$,
	for some $e>d$, then we immediately have $u^{\prime}\geq u$ so assume the
	contrary.  By definition of $d$, we cannot have $u^{\prime}\in\alpha_{c}$
	for any $c<d$ so we may conclude that $u^{\prime}\in\alpha_{d}$.  It then
	follows immediately, by definition of $u$, that $u^{\prime}\geq u$, as required.
\end{proof}

\begin{prp}\label{prp:countdensumsep}
	Suppose $\gamma$ is a countable dense linear order and
	$\family{\alpha_{i}}{i\in\gamma}$ is a family of separable linear orders.
	The sum
	\begin{equation}
		\alpha=\sum_{i\in\gamma}\alpha_{i}
	\end{equation}
	must then also be a separable linear order.
\end{prp}
\begin{proof}
	Identify each $\alpha_{i}$, for $i\in\gamma$, with its image under the
	canonical embedding into $\alpha$.  By definition, for any $i\in\gamma$,
	there exists a countable $S_{i}\subseteq\alpha_{i}$ which is dense in
	$\alpha_{i}$.

	Define $S=\bigcup_{i\in\gamma}S_{i}$, then it follows by definition that
	\begin{equation}
		\card{S}= \aleph_0\cdot\aleph_0=\aleph_0.
	\end{equation}
	We are now required to show that $S$ is dense in $\alpha$ so fix any $a<b$
	in $\alpha$.  If there exists a $j\in\gamma$ such that $a,b\in\alpha_{j}$
	then there must exist a $c\in S_{j}\subseteq S$ such that $a<c<b$.

	Otherwise, there exists $j<k$ in $\gamma$ such that $a\in\alpha_{j}$ and
	$b\in\alpha_{k}$.  Since $\gamma$ is dense, there must exist an $m\in\gamma$
	such that $j<m<k$.  If we now choose any $c\in S_{m}\subseteq S$ then it
	follows that $a<c<b$, as required.
\end{proof}

\begin{thm}
	Suppose $\alpha\in\dense$ is countable, definably complete and definably
	quasi-separable.  For each $n\in\nats$, $\alpha$ has an $n$-equivalent of order
	type $\lambda$.
\end{thm}
\begin{proof}
	As in the statement, let $\alpha$ be countable, definably complete and
	definably quasi-separable.  Suppose also that there are $k$ colours defined on
	$\alpha$, for some fixed $k\in\nats$.  Now choose some $n\in\nats$ and define a
	(binary) relation $R$ on $\alpha$ such that, for every $a,b\in\alpha$:  $aRb$
	iff $a\leq b$ and $(a,b)\nequiv{n}\lambda$ whenever $a\neq b$.  Since sums
	preserve $n$-equivalence, we may conclude that $R$ is transitive and therefore
	induces a congruence $\sim$ on $\alpha$.  Note that it follows from the
	definition of $\sim$ that each $\beta\in\faktor{\alpha}{\sim}$ satisfies exactly
	one of: $\beta\nequiv{n}\one$, $\beta\nequiv{n}\one+\lambda$,
	$\beta\nequiv{n}\lambda+\one$, $\beta\nequiv{n}1+\lambda+1$ or
	$\beta\nequiv{n}\lambda$.

	\begin{claim}
		The congruence $\sim$ is a definable (binary) relation and the
		condensation $\faktor{\alpha}{\sim}$ is dense.
	\end{claim}
	Let $\tau$ be the (finite) disjunction of all characteristic sentences, of
	quantifier-rank at most $n$, of $k$-colourings of $\lambda$.  Define
	$\varphi(x,y)$ to be the formula given by
	\begin{equation}\label{eq:condef}
		x=y\vee(x<y\wedge\tau^{(x,y)})\vee(y<x\wedge\tau^{(y,x)}).
	\end{equation}
	By definition of $\varphi$, as well as that of aforementioned characteristic
	sentences, it follows that $\varphi$ is a defining formula for the
	congruence relation $\sim$.

	It now remains to be shown that $\faktor{\alpha}{\sim}$ is dense.  We
	suppose, to the contrary, that there exists $I,J\in\faktor{\alpha}{\sim}$ such
	that the interval $(I,J)$ is empty.  Therefore, there cannot exists a
	$c\in\alpha$ such that $I<c<J$.

	If we now let $a=\sup I$ and $b=\inf J$ then it must hold that
	$a,b\in I\cup J$ and $a\leq b$.  Since $\alpha$ is dense we cannot
	simultaneously have $a\in I$ and $b\in J$.  Therefore, either $a,b\in I$ or
	$a,b\in J$.

	As the cases are similar, we may suppose that $a,b\in I$.  Note that, since
	$a$ is the supremum of $I$, the interval $(a,b)$ must be empty and thus
	$a=b$.  Also, since $b\notin J$, it follows that $J$ cannot have order type
	$\one$.  From the countability of $\alpha$, it now follows that there exists
	a strictly decreasing sequence $(b_{i})_{i<\omega}\subseteq J$ such that
	\begin{equation}
		\inf_{i<\omega}b_{i}=b.
	\end{equation}

	For each $i<j<\omega$, by definition of $\sim$ the order type of the
	interval $[b_{j},b_{i})$ is $n$-equivalent to $\one+\lambda$.  From Lemma
	\ref{lem:fvsum}, it now follows that the order type of $(b,b_{0})$ is
	$n$-equivalent to $(\one+\lambda)\cdot\dual{\omega}\cong\lambda$.  Thus, by
	definition, it holds that $b\sim b_{0}$.  Since this contradicts that
	$b\notin J$, we have successfully established the above claim.  Furthermore,
	a similar argument may be used to show that each $I\in\faktor{\alpha}{\sim}$
	must contain their respective suprema and infima whenever they exist.

	Note that if $\card{\faktor{\alpha}{\sim}}=1$ then, since $\alpha$ is
	countable, it follows from Lemma \ref{lem:fvsum} that $\alpha$ is $n$-equivalent
	to a $k$-colouring of order type
	\begin{equation}
		(\one+\lambda)\cdot\dual{\omega}+\one+(\lambda+\one)\cdot\omega\cong\lambda,
	\end{equation}
	as required.  Suppose then, by way of contradiction, that
	$\card{\faktor{\alpha}{\sim}}>1$.
	\begin{claim}
		There exists a (proper) open interval $I$ of $\faktor{\alpha}{\sim}$ and
		a finite set $\Sigma$ of coloured linear orders, each of which either has order
		type $\one$ or $\one+\lambda+\one$, such that the following holds:
		\begin{enumerate}[nosep]
			\item for each $\beta\in I$ there exists a $\sigma_\beta\in\Sigma$
			      such that $\beta\nequiv{n}\sigma_\beta$,

			\item if $\sigma\in\Sigma$ then the set $\setbuild{\beta\in I}{\beta\nequiv{n}\sigma}$ is dense in $I$.
		\end{enumerate}
	\end{claim}

	Define $C$ to be the set of $n$-characteristic sentences $\cha{\beta}{n}$
	for $\beta\in\faktor{\alpha}{\sim}$. It follows that $C$ is a nonempty
	finite set of sentences. Fix some $n$-spectrum $\Sigma_{C}$ for the class
	$\faktor{\alpha}{\sim}$.  As noted previously, since equivalence classes
	under $\sigma$ must contain any existing suprema and infima, each
	$\beta\in\Sigma_C$ is of order type either $\one$ or $\one+\lambda+\one$.

	Note, since $\sim$ is definable and $\alpha$ is definably quasi-separable,
	that $\card{C}\geq 1$.  We then proceed by means of induction on $\card{C}$.
	If $\card{C}=1$ then every $\beta\in\faktor{\alpha}{\sim}$ has order type
	$\one$, since $\alpha$ is definably quasi-separable and thus
	$\faktor{\alpha}{\sim}$ must contain a dense set of singletons.  If we now
	choose any fixed $\beta_0\in\faktor{\alpha}{\sim}$, as well as any (proper)
	open subinterval $I$ of $\faktor{\alpha}{\sim}$ such that $\beta_0\in I$,
	then $\Sigma=\set{\beta_0}$ and $I$ satisfy properties 1 and 2 above.

	Assume that the claim holds whenever $\card{C}<m$, for some fixed
	$m\in\posnats$.  Suppose the claim fails when $\card{C}=m$ then, by definition,
	there exists a $\tau_0\in C$ and an open interval
	$I^\prime\subsetneq\faktor{\alpha}{\sim}$ such that
	$\beta\not\models\tau_0$, for any $\beta\in I^\prime$.  Since
	$\card{C\setminus\set{\tau_0}}<m$, it follows from the inductive hypothesis that
	there exists an open interval $I\subsetneq I^\prime$ and a
	$\Sigma\subseteq\Sigma_C$ satisfying properties 1 and 2.  This then
	establishes the desired claim.

	\smallskip	The goal is now to show that $\card{I}=1$, contradicting that
	$I$ is an open interval.  Once established, this contradiction will imply
	that the assumption $\card{\faktor{\alpha}{\sim}}>1$ was erroneous and,
	therefore, that $\faktor{\alpha}{\sim}$ is a singleton partition of
	$\alpha$.

	Since $\alpha$ is definably quasi-separable, there exists a
	$\beta_1\in\faktor{\alpha}{\sim}$, of order type $\one$, such that
	$S=\setbuild{\beta\in\faktor{\alpha}{\sim}}{\beta\nequiv{n}\beta_1}$ is dense in
	the condensation $\faktor{\alpha}{\sim}$.  Therefore, $S\cap I$ is dense in $I$.

	We now proceed to construct a coloured linear order $\delta$ of order type
	$\lambda$ which is $n$-equivalent to $\bigcup I$.  Let
	$\Sigma=\set{\sigma_0,\dotsc,\sigma_{\ell-1}}$ be an $n$-spectrum for
	$\faktor{\alpha}{\sim}$.  Since the labelling is irrelevant, we may assume
	without loss of generality that $\sigma_{0}\nequiv{n}\beta_{1}$.

	From Proposition \ref{prp:ratpart}, we may choose a surjection
	$h\colon\reals\to\Sigma$ such that $h(x)=\sigma_0\nequiv{n}\beta_1$, for each
	$x\in\irrats$, and $h^{-1}[\sigma]$ is dense in $\lambda$ for any
	$\sigma\in\Sigma\setminus\set{\beta_1}$. We now define the (coloured) linear
	order
	\begin{equation}
		\delta=\sum_{x\in\lambda}h(x).
	\end{equation}
	Since each $\sigma\in\Sigma$ has order type $\one$ or $\one+\lambda+\one$,
	it follows from Proposition \ref{prp:csums} that $\delta$ is a complete
	linear order.  Whenever $x\in\irrats$, it follows (by definition) that
	$h(x)$ has order type $\one$.  Hence, the sum
	\begin{equation}
		\gamma=\sum_{x\in\eta}h(x)
	\end{equation}
	embeds densely into $\delta$.

	Since each summand of $\gamma$ is separable, and $\eta$ is countable and
	dense, it follows form Proposition \ref{prp:countdensumsep} that $\delta$ is
	also separable.  Consequently, from theorem \ref{thm:rchar}, we may conclude
	that $\delta$ has order type $\lambda$.

	A winning strategy for $\Right$ in the $n$-round game
	$\EF_{n}(\delta,\bigcup I)$ may now be devised by leveraging the fact that each
	respective member of the $n$-spectrum $\Sigma$ is dense in $I$.  Hence, it
	may be concluded that $\delta\nequiv{n}\bigcup I$. Therefore, since the
	equivalence classes of $\sim$ must be convex, we may conclude that
	$\card{I}=1$.  This is the desired contradiction.
\end{proof}

	\chapter{Suslin lines}

\section{Fundamental concepts}

\begin{dfn}
	If $P$ is a partially ordered set and $a\in P$ then we denote by $\down{a}$ the set
	\begin{equation}
		\down{a}=\setbuild{x\in P}{x\leq a}.
	\end{equation}
\end{dfn}

\begin{dfn}[Tree]
	A partial order $\mathfrak{T}=(T,<)$ is called a \textbf{tree} whenever it has a least element (referred to as a \textit{root}) and, for each $a\in T$, the set $\down{a}$ is well-ordered.
\end{dfn}

\begin{dfn}[Height]
	If $\mathfrak{T}=(T,<)$ is any tree and $a\in T$ then \textbf{height of }$\mathbf{a}$\textbf{ in }$\bm{\mathfrak{T}}$ is the order type of $\mathfrak{T}^{<a}$.
\end{dfn}

Note that, by definition of a tree, any $\mathfrak{T}^{<a}$ is well-ordered and is thus isomorphic to an ordinal.  Therefore, the height of any element (called a \textit{node}) of a tree is an ordinal.  One can now also define the related but distinct concept of height for trees themselves.

\begin{dfn}[Tree height]
	The height $\height(\mathfrak{T})$ of a tree $\mathfrak{T}$ is the least ordinal such that, for any ordinal $\alpha>\height(T)$, there exists no embedding $f\colon\alpha\hookrightarrow\mathfrak{T}$.
\end{dfn}

The relation between these notions of height is then given by the following proposition:

\begin{prp}
	If $\mathfrak{T}$ is any tree and $\alpha$ is any ordinal then $\height(\mathfrak{T})=\alpha$ iff it holds that
	\begin{equation}
		\alpha=\sup_{a\in T}(\height(a)+1).
	\end{equation}
\end{prp}

\begin{dfn}[Length]
	For any chain $\beta$ in a tree $\mathfrak{T}$, the ordinal $\length(\beta)$ in Proposition \ref{prp:length} is referred to as the \textbf{length} of the chain $\beta$.
\end{dfn}

\begin{prp}\label{prp:length}
	If $\beta$ is a chain in a tree $\mathfrak{T}$ then $\beta$ is isomorphic to some ordinal $\length(\beta)\leq\height(\mathfrak{T})$.
\end{prp}

\begin{dfn}[Branch]
	A \textbf{branch} $\beta$ in a tree $\mathfrak{T}$ is any maximal chain of nodes in $\mathfrak{T}$.
\end{dfn}

\begin{dfn}[Levels]
	If $\mathfrak{T}$ is a tree and $\alpha<\height(\mathfrak{T})$ is an ordinal then the set
	\begin{equation}
		\level{\alpha}(\mathfrak{T})=\setbuild{a\in T}{\height(a)=\alpha}
	\end{equation}
	is referred to as the $\bm{\alpha}$\textbf{-th level} of $\mathfrak{T}$.
\end{dfn}

\begin{dfn}[Normality]
	A \textbf{normal }$\bm{\alpha}$\textbf{-tree} is a tree $\mathfrak{T}$ such that
	\begin{enumerate}
		\item	$\height(\mathfrak{T})=\alpha$,
		\item	for each $\gamma<\alpha$ it holds that $\card{\level{\gamma}(\mathfrak{T})}\leq\aleph_0$,
		\item	if $x\in T$ is not maximal then $x$ has infinitely many immediate successors,
		\item	for each $x\in T$ and each ordinal $\beta<\alpha$, if $\beta>\height(x)$ then there exists some $y\in\level{\beta}(\mathfrak{T})$ such that $y>x$,
		\item	for every $x,y\in T$, if $\mathfrak{T}^{<x}=\mathfrak{T}^{<y}$ then $x=y$.
	\end{enumerate}
	As one might expect, we refer to a tree $\mathfrak{T}$ simply as being \textbf{normal} whenever it is $\height(\mathfrak{T})$-normal.
\end{dfn}


\section{On the existence of Suslin lines and Suslin trees}

\begin{dfn}[Suslin line]
	A complete linear order $\alpha\in\dense$ is said to be a \textbf{Suslin line} whenever $\alpha$ is not seperable and every pairwise disjoint set of open intervals in $\alpha$ is at most countable.
\end{dfn}

Restated succintly:  A Suslin line is a complete inseperable linear order without endpoints posessing the \textit{Suslin property}.

\begin{prp}
	If there exists a Suslin line then there exists a complete Suslin line.
\end{prp}

\begin{cor}
	There exists a Suslin line iff there exists a complete Suslin line.
\end{cor}

\begin{dfn}[Suslin tree]
	A tree $\mathfrak{T}$ is called a \textbf{Suslin tree} whenever $\height(\mathfrak{T})=\omega_1$, all the anti-chains of $\mathfrak{T}$ are at most countable and, for every branch $\beta$ in $\mathfrak{T}$, it holds that $\length(\beta)<\omega_1$.
\end{dfn}

\begin{lem}
	If there exists a Suslin line then there exists a Suslin tree.
\end{lem}
\begin{proof}
	Suppose $\sigma=(S,<)$ is a Suslin line.  We now construct, by transifinite recursion, a tree from closed intervals in $\sigma$, each consisting of at least two elements.  First, we choose $I_0=S$ and fix any $I_1=[a_1,b_1]$ such that $a_1,b_1\in S$ and $a_1<b_1$.  Suppose now that $\alpha$ is an ordinal satisfying $0<\alpha<\omega_1$ and assume, for each ordinal $\beta$ satisfying $0<\beta<\alpha$, that $I_\beta=[a_\beta,b_\beta]$ and $a_\beta<b_\beta$ for some $a_\beta,b_\beta\in S$.  Define $E=\setbuild{a_\beta}{\beta<\alpha}\cup\setbuild{b_\beta}{\beta<\alpha}$ and note that $E$, by definition, must be countable.  Therefore, since $\sigma$ is not seperable there exists an $a_\alpha,b_\alpha\in S$ such that $a_\alpha<b_\alpha$ and $I_\alpha=[a_\alpha,b_\alpha]$ is disjoint from $E$.  Now, defining $T=\setbuild{I_\alpha}{\alpha<\omega_1}$, it follows that $T$ is uncountable and partially ordered by $\supsetneq$.  Note, for every $\alpha<\omega_1$, that the set $\down{I_\alpha}$ is well-ordered.  We may thus conclude that $\mathfrak{T}=(T,\supsetneq)$ is a tree.

	We are now required to show that $\mathfrak{T}$ is in fact a Suslin tree.  That is we are required to prove that $\height(\mathfrak{T})=\omega_1$ and $\mathfrak{T}$ has neither an uncountable antichain nor an uncountable branch.  It follows from the construction of $\mathfrak{T}$ that if $\alpha,\beta<\omega_1$ and $\alpha<\beta$ then either $I_\alpha\supsetneq I_\beta$ or $I_\alpha\cap I_\beta=\emptyset$.  From this observation it follows that if $X\subseteq T$ is an antichain then $X$ is a pairwise disjoint set of closed intervals in $X$.  Choose $X^\prime$ to now be the corresponding set of open intervals i.e.\ $X^\prime=\setbuild{\Int(I)}{I\in X}$.  Since $\sigma$ is a Suslin line it now follows that $X^\prime$, and thus also $X$, is at most countable.  In order to show that $\mathfrak{T}$ has no uncountable branch suppose the contrary: there exists a branch $\beta=(B,\supsetneq)$ in $\mathfrak{T}$ such that $\length(\beta)=\delta\geq\omega_1$.  We may now assume $B=\setbuild{I_\gamma}{\gamma<\delta}$ and $I_{\gamma_0}\supsetneq I_{\gamma_1}$ whenever $\gamma_0<\gamma_1$.  If, for each $\gamma<\delta$, we choose $x_\gamma$ to be the left endpoint of the interval $I_\gamma$ then it follows, since $B$ pairwise disjoint, that $\setbuild{(a_\gamma,a_{\gamma+1})}{\gamma<\delta}$ is an uncountable set of pairwise disjoint open intervals in $\sigma$ --- contradicting the fact that $\sigma$ is a Suslin line.

	All that remains is to show that $\height(\mathfrak{T})=\omega_1$.  Since $\beta$ has no uncountable branch, however, it follows that $\height(\mathfrak{T})\leq\omega_1$.  Note also, since levels are antichains. each level of $\mathfrak{T}$ must be countable.  Therefore, as $\card{\mathfrak{T}}=\aleph_1$, we cannot have $\height(\mathfrak{T})<\omega_1$ and thus we may conclude that $\height(\mathfrak{T})=\omega_1$, as required.
\end{proof}

\begin{lem}
	If there exists a Suslin Tree then there exists a normal Suslin tree.
\end{lem}

\begin{lem}
	If there exists a Suslin tree then there exists a Suslin line.
\end{lem}

\begin{thm}
	There exists a Suslin line iff there exists a Suslin tree
\end{thm}

\begin{prp}
	Let $\mathfrak{T}$ be a tree in which every node has at least two immediate successors.  If $\mathfrak{T}$ has no uncountable antichains then for every branch $\beta$ in $\mathfrak{T}$ it necessarily holds that $\length(\beta)<\omega_1$.
\end{prp}

\begin{thm}
	There exists a Suslin line iff there exists a regular Suslin tree.
\end{thm}

	\chapter{A La\"uchli and Leonard style result for continuous linear orders}


\section{Continuous linear orders}

\begin{dfn}[Continuous linear order]
	A \textbf{continuous linear order} is a complete linear order $\alpha$ such that $\alpha\in\dense$.  A \textit{coloured linear order} is said to be \textbf{continuous} whenever its monochromatic reduct is a continuous linear order.
\end{dfn}

\begin{dfn}[The class $\C$]\label{dfn:C}
	Let $\C$ be the smallest class of linear orders such that:
	\begin{enumerate}
		\item	$\lambda\in\C$\label{dfn:C1}
		\item	if $\alpha,\beta\in\C$ then $\alpha+\one+\beta\in\C$\label{dfn:C2}
		\item	if $\alpha\in\C$ then $(\alpha+\one)\cdot\omega,(\one+\alpha)\cdot\dual{\omega}\in\C$\label{dfn:C3}
		\item	if $\mathcal{F}\subseteq\C$ is finite and $h\colon\reals\to\mathcal{F}$ is a surjection such that, for each $\alpha\in\mathcal{F}$, the set $\inv{h}[\alpha]$ is dense in $\lambda$ then it follows that
			\begin{equation}
				\sum_{x\in\lambda}(\one+h(x)+\one)\in\C.
			\end{equation}\label{dfn:C4}
	\end{enumerate}
\end{dfn}

\begin{dfn}[The class $\C_k$]
	For each $k\in\nats$, let $\C_k$ denote the class of all $k$-coloured expansions of linear orders belonging to $\C$.
\end{dfn}

\begin{prp}\label{prp:cont}
	If $\alpha$ is a continuous linear order then there exists an embedding $h\colon\lambda\hookrightarrow\alpha$.  Furthermore, $h$ can be chosen such that $h[\reals]$ is an open interval in $\alpha$.
\end{prp}
\begin{proof}
	Since $\alpha$ is dense there exists an embedding $h_0\colon\eta\hookrightarrow\alpha$.  We now set out to extend $h_0$ to an embedding of $\lambda$ into $\alpha$.  Define the map $h\colon\reals\to\domain{}\alpha$ such that, for each $r\in\reals$,
	\begin{equation}
		h(r)=\sup\setbuild{h_0(q)}{q\in\rats\text{ and }q<r}.
	\end{equation}
	It clearly follows by definition that $h\restriction\rats=h_0$.  Note that, since $\eta$ is dense in $\lambda$ there exists a $q_0\in\rats$ such that $x<q_0<y$.  Again invoking the density of the rationals in the reals, there exists a $q_1\in(q_0,y)\cap\rats$.  Since $h_0$ is an embedding, it follows from the definition of $h$ that $h(x)\leq h(q_0)<h(q_1)\leq h(y)$ so that, in particular, it holds that $h(x)<h(y)$ and thus $h$ is an order-preserving map and, therefore, also an embedding $h\colon\lambda\hookrightarrow\alpha$.

	All that remains is to show that there now exists an embedding $h^\prime\colon\lambda\hookrightarrow\alpha$ such that $h^\prime[\reals]$ is a convex set in $\alpha$.  Consider now the open interval $I=(0,1)\subseteq\reals$ and its image $J=h[I]$ under $h$.  We now proceed to show that $J$ is convex.  Note that $h(0)<J<h(1)$ so we may therefore conclude that $J$ is bounded in $\alpha$.  Thus, since $\alpha$ is complete, we may define $a=\inf J$ and $b=\sup J$.  Now fix any $d\in(a,b)_\alpha$ and choose $\ell=\sup\setbuild{r\in\reals}{h(r)<d}$ and $u=\inf\setbuild{r\in\reals}{h(r)>d}$.  By definition, it must hold that $\ell\leq u$.  Suppose that $u<\ell$ then there exists some $r_0\in(\ell,u)$.  Since $h$ is an embedding, it follows that $h(\ell)<h(r_0)<h(u)$.  Now, if $h(r_0)<d$ then it follows by definition of $\ell$ that $r_0\leq\ell$, which is a contradiction.  Similarly, if $h(r_0)>d$ then it follows that $r_0\geq u$, also a contradiction.  Therefore, it follows that $h(r_0)=d$ from which we may conclude that, since $r_0\in(\ell,u)\subseteq I$ it follows that $d\in J$.  Note that if $J$ had a greatest element then its preimage under $h$ would be forced to be $1\in\reals$, contradicting the fact that $1\notin I=(0,1)$.  Similarly, since $0\notin I$, $J$ cannot have a least element either.  Therefore, since it holds that $(0,1)\cong\lambda$, $h$ is an embedding and $h\restriction(0,1)$ maps onto a convex subset of $\alpha$, the result follows.
\end{proof}

\begin{thm}\label{thm:Cll}
	For each $k,n\in\nats$, if $\alpha$ is a continuous $k$-coloured linear order then there exists a $\beta_n\in\C_k$ such that:
	\begin{equation}
		\alpha\nequiv{n}\beta_n
	\end{equation}
\end{thm}
\begin{proof}
	Fix any $k,n\in\nats$ and suppose $\alpha$ is a continuous $k$-coloured linear order.  Define a binary relation $R$ on $\domain{}\alpha$ such that, for every $a,b\in\reals$, $aRb$ iff  $a<b$ and there exists a $\beta\in\C_k$ such that $(a,b)\nequiv{n}\beta$.  It follows from clause \ref{dfn:C2} in definition \ref{dfn:C} that $R$ is transitive and, therefore, induces a congruence $\sim$ on $\alpha$.

	\begin{claim}
		The congruence $\sim$ is definable.
	\end{claim}
	\begin{proof}
		The defining formula $\varphi=\varphi(x,y)$ is of the same form as the formula in (\ref{eq:condef}), with the only differences being that $\varphi$ is a formulated in a language with a finite number of additional (unary) relation symbols and $\tau=\bigvee_{\delta\in\C_k}\cha{\delta}{n}$.
	\end{proof}

	\begin{claim}
		The linear order $\faktor{\alpha}{\sim}$ is dense.
	\end{claim}
	\begin{proof}
		Fix any $I,J\in\faktor{\alpha}{\sim}$ such that $I<J$ i.e.\ $x<y$ for each $x\in I$ and $y\in J$.  Since $\alpha$ is complete, there exists some $a,b\in\alpha$ such that $a=\sup I$ and $b=\inf J$.  Note, if $a=b$ then it follows from clause \ref{dfn:C2} in definition \ref{dfn:C} that $x\sim y$, for every $x\in I$ and $y\in J$, contradicting the definition of $I$ and $J$.  Hence, we may assume that $a<b$.  It now follows from the density of $\alpha$ that there exists a $c\in (a,b)$.  Consequently, it follows that $I<[c]<J$ --- as required.
	\end{proof}

	\begin{claim}
		The linear order $\faktor{\alpha}{\sim}$ is complete.
	\end{claim}
	\begin{proof}
		Suppose $X\subseteq\faktor{\alpha}{\sim}$ is bounded above.  Note that $\bigcup X\subseteq\alpha$ is now bounded above in $\alpha$.  It therefore follows from the completeness of $\alpha$ that there exists a $u\in\bigcup X$ such that
		\begin{equation}
			u=\sup\bigcup X.
		\end{equation}
		Define $I$ to now be the equivalence class $I=[u]$.  Suppose by way of contradiction that there exists a $J\in\faktor{\alpha}{\sim}$ such that $J\geq X$ and $J<I$.  Since $\faktor{\alpha}{\sim}$ is dense there must exist a $K\in\faktor{\alpha}{\sim}$ such that $J<K<I$.  Therefore, there exists an $x_0\in K$ such that $x<x_0<u$ for each $x\in J$.  This implies that $x_0$ is an upper bound of $\bigcup X$ less than $u$, a contradiction.  Therefore, $I$ is in fact the supremum of $X$.
	\end{proof}

	\begin{claim}\label{clm:contInt}
		For every $I\in\faktor{\alpha}{\sim}$ there exists a $\beta\in\C_k$ such that $\Int I\nequiv{n}\beta$.
	\end{claim}
	\begin{proof}
		Note that, by the Downwards L\"owenheim-Skolem Theorem, we may assume $I$ is at most countable.  Consider first the case where $I$ is bounded below but not above in $\alpha$ and choose a cofinal sequence $a=\family{a_i}{i<\omega}$ in $\alpha$ such that $a_0=\inf I$.  If $h$ denotes some colouring of $I$ such that, for $x,y\in I$, $h(x,y)\in\C_k$ and $(x,y)\nequiv{n}h(x,y)$ then it follows from Ramsey's Theorem that there exists a homogeneous subsequence $a^\prime=\family{a^\prime_i}{i<\omega}$ of $a$ for $h$.  It then follows by definition of $a^\prime$ that
		\begin{equation}
			\sum_{i<\omega}(a^\prime_i,a^\prime_{i+1}]\nequiv{n}\left(h(a^\prime_0,a^\prime_1)+\one\right)\cdot\omega.
		\end{equation}
		We now choose $\beta\in\C_k$ such that
		\begin{equation}
			\beta=\begin{cases}
				\left(h(a^\prime_0,a^\prime_1)+\one\right)\cdot\omega,&\text{if }a_0=a^\prime_0,\\
				h(a_0,a^\prime_0)+\one+\left(h(a^\prime_0,a^\prime_1)+\one\right)\cdot\omega,& \text{otherwise}.\\
			\end{cases}
		\end{equation}
		It follows by definition then that $\beta\in\C_k$ and
		\begin{equation}
			\Int I\nequiv{n}\beta,
		\end{equation}
		as required.  The case where $I$ is bounded above but not below in $\alpha$ is obtained dually i.e.\ by considering a cofinal sequence $b=\family{b_i}{i<\omega}$ in $\dual{I}\subseteq\dual{\alpha}$ and repeating the above argument.  In order to establish the claim, the remaining cases are obtained by similar means: considering the relevant cofinal, or coinitial, sequences and taking the appropriate sums.
	\end{proof}

	We now wish to show that $\card{\faktor{\alpha}{\sim}}=1$ so it follows from the previous claim that $\alpha\nequiv{n}\beta_n$, for some $\beta_n\in\C_k$.  To this end, we suppose to the contrary that $\card{\faktor{\alpha}{\sim}}>1$ and note that proposition \ref{prp:cont} implies the existence of an interval $I=(a,b)\subseteq\alpha$ and an embedding $g\colon\lambda\hookrightarrow\alpha$ such that $g[\reals]=I$.  Let $f$ be the surjective homomorphism $f\colon\alpha\to\faktor{\alpha}{\sim}$ induced by the congruence $\sim$ and define $S=\set{\delta_0,\dotsc,\delta_{m-1}}$ to be an $n$-spectrum for the class of $\chi\in\C_k$ such that $\chi\nequiv{n}X$ for some $X\in f[I]$.

	\begin{claim}
		There exists an open interval $J=(a^\prime,b^\prime)\subseteq I$ and a nonempty $S^\prime\subseteq S$ such that, for each $\sigma\in S$, the set
		\begin{equation}
			D_\delta=\setbuild{\faktor{d}{\sim}}{d\in J\text{ and }\faktor{d}{\sim}\nequiv{n}\delta}
		\end{equation}
		is dense in $f[J]$.
	\end{claim}
	\begin{proof}
		Proceed by induction on $m$ and note that, by definition of $I$, every nonempty open interval in $I$ has order type $\lambda$, implying that $I$ contains no singletons.  If $m=1$ then the result follows immediately from the definition of $S$ as then every member of $I$ is $n$-equivalent to the same (unique) linear order in $S$.  Assume the result holds whenever $m<m^\prime$, for some arbitrary $m^\prime\geq 1$.  Suppose to the contrary that $\delta\in S$ and $D_\delta$ is not dense in any open interval $(a^\prime,b^\prime)\subseteq I$.  By definition, there must then exist an open interval $J\subsetneq I$ such that no $X\in J$ satisfies $X\nequiv{n}\delta$.  If we now let $S^\prime=\setbuild{\delta^\prime\in S}{X\nequiv{n}\delta^\prime,\text{ for some }X\in J}$ then it follows by definition that $\delta\notin S^\prime$ and thus $\card{S^\prime}<\card{S}$.  Applying the induction hypothesis now to $J$ and $S^\prime$ yields the desired contradiction, thus establishing the claim.
	\end{proof}

	Note, since $I$ is bounded in $\alpha$, that each $X\in f[I]$ is also bounded in $\alpha$ and thus (by completeness of $\alpha$ and claim \ref{clm:contInt}) $X$ has a least and greatest element.  If we now choose $h$ to be the map $h\colon\reals\to S$ such that, for each $x\in\reals$, $h(x)$ is the interior of the (unique) $n$-equivalent of $fg(x)$ belonging to $S$ then it follows from lemma \ref{lem:fvsum} that
	\begin{equation}
		I\nequiv{n}\sum_{x\in\lambda}(\one+h(x)+\one)\in\C_k.\label{eq:rsum}
	\end{equation}
	Therefore, by definition of $I$, it follows that $a\sim b$ and thus $a,b\in I$, which is the desired contradiction.  Hence $\faktor{\alpha}{\sim}$ has only one element and the result follows from claim 4.
\end{proof}

It is inconvenient, however, that the class $\C$ as it stands, due to the final clause in its definition, is not countable let alone recursively enumerable.  This is due to the uncountability of $\reals$ itself as it guarantees the uncountability of at least one of it equivalence classes under any equivalence relation.  Should $A$ be any such uncountable equivalence class and $B$ any equivalence class distinct from it, uncountably many new partitions can be generated by substituting, for some $r\in A$, the set $A\setminus\set{r}$ for the equivalence class $A$ and the set $B\cup\set{r}$ for the equivalence class $B$ in the chosen partition of $\reals$.

\begin{prp}\label{prp:dsum}
	Suppose, for some $k\in\nats$, $\mathcal{S}$ is a set of $k$-coloured linear orders and fix any $\alpha\in\dense$.  If $h,h^\prime\colon\alpha\to\mathcal{S}$ are surjections such that, for every $\delta\in\mathcal{S}$, $\inv{h}[\delta]$ and $\inv{h^\prime}[\delta]$ are respectively dense in $\alpha$ then it follows, for each $n\in\nats$, that
	\begin{equation}
		\sum_{x\in\alpha}h(x)\nequiv{n}\sum_{x\in\alpha}h^\prime(x).
	\end{equation}
\end{prp}
\begin{proof}
	The case $n=0$ is immediate from the fact that the language of coloured linear orders has no constant symbols so we may assume $n>0$.  Define the linear orders $\beta_0$ and $\beta_1$ such that
	\begin{equation}
		\beta_0=\sum_{x\in\alpha}h(x)\quad\text{and}\quad\beta_1=\sum_{x\in\alpha}h^\prime(x).
	\end{equation}
	We now argue by describing the winning strategy of $\Right$ in the game $\EF_n(\beta_0,\beta_1)$.  On the first round of the game if $\Left$ plays $a_0\in\beta_0$ then $\Right$ responds with an element from a summand of $\beta_1$ which, leveraging the definition of $h^\prime$, corresponds to the summand of $\beta_0$ from which $a_0$ arose.  Should $\Left$ have played some $b_0\in\beta_1$ in stead a similar cuntermove would suffice.

	Consider now a position $(\bar{a},\bar{b})$ of length $\ell$ such that $0<\ell<n$.  Since $\alpha$ is dense, it follows from the definitions of $h$ and $h^\prime$ that (irrespective of $\Left$'s $\ell$-th move) $\Right$ always has available a response from an appropriate summand, of either $\beta_0$ or $\beta_1$, in the opposing $k$-coloured linear order.  If $\Left$ plays $d\in\beta_0$ and $d>\bar{a}$ then $\Right$ responds, from a corresponding summand, with an element $e\in\beta_1$ such that $d>\bar{b}$.  One argues dually when $d<\bar{a}$.  Otherwise, it must hold that $d\in(a_i,a_j)$, for some $i,j<\ell$, such that $a_{i^\prime}\notin(a_i,a_j)$ when $i^\prime<\ell$.  In this latter case it follows again from the definition of $h^\prime$ that $\Right$ has an ``appropriate'' response $e\in\beta_1$ from a summand corresponding to the one from which $d$ arose: meaning $e$ can be chosen such that $e\in(b_i,b_j)$ and $d_{i^\prime}\notin(d_i,d_j)$ when $i^\prime<\ell$.
\end{proof}

\begin{dfn}[Canonical partitions of $\reals$]
	Suppose $n\in\posnats$ then we refer to a partition $\mathcal{R}=\set{R_0,\dotsc,R_{n-1}}$ of $\reals$ as the \textbf{canonical partition of $\reals$ into $\bm{n}$ dense subsets} whenever:
	\begin{enumerate}
		\item	$n=1$ implies $\mathcal{R}=\set{\reals}$,
		\item	if $n>1$ then, for some $k\in\set{0,\dotsc,n-1}$, it holds that $R_k=\irrats$ and $\mathcal{R}\setminus\set{R_k}$ is the canonical partition of $\rats$ into $n-1$ dense subsets.
	\end{enumerate}
\end{dfn}

\begin{dfn}[The class $\Cast$]
	Let $\Cast$ be the smallest class of linear orders such that:
	\begin{enumerate}
		\item	$\lambda\in\Cast$\label{dfn:C1}
		\item	if $\alpha,\beta\in\Cast$ then $\alpha+\one+\beta\in\Cast$\label{dfn:C2}
		\item	if $\alpha\in\Cast$ then $(\alpha+\one)\cdot\omega,(\one+\alpha)\cdot\dual{\omega}\in\Cast$\label{dfn:C3}
		\item	if $\mathcal{F}\subseteq\Cast$ is nonempty and finite while $h\colon\reals\to\mathcal{F}$ is a surjection which induces the canonical partition of $\reals$ into $\card{\mathcal{F}}$ subsets then
			\begin{equation}
				\sum_{x\in\lambda}(\one+h(x)+\one)\in\Cast.
			\end{equation}\label{dfn:C4}
	\end{enumerate}
\end{dfn}

\begin{dfn}[The class $\Cast_k$]
	For each $k\in\nats$, let $\Cast_k$ denote the class of all $k$-coloured expansions of linear orders belonging to $\Cast$.
\end{dfn}

\begin{thm}
	For each $k,n\in\nats$, if $\alpha$ is a continuous $k$-coloured linear order then there exists a $\beta_n\in\Cast_k$ such that:
	\begin{equation}
		\alpha\nequiv{n}\beta_n
	\end{equation}
\end{thm}
\begin{proof}
	Near-identical to the proof of Theorem \ref{thm:Cll}.  The only variation is that one invokes Proposition \ref{prp:dsum} to choose the surjection $h$, occuring in (\ref{eq:rsum}), that induces the required canonical partition of $\reals$ into dense subsets.
\end{proof}


\section{Decidability: the coloured case}

\begin{dfn}[Definable completeness]
	A coloured linear order $\alpha$ is said to be \textbf{definably complete} whenever every nonempty definable subset of $\alpha$, which is bounded above, has a supremum.
\end{dfn}

Note that, for any subset $A$ of a definably complete $\alpha$, if there exists a tuple $\bar{a}$ of $\alpha$ such that $A$ is defined by the formula $\varphi(x,\bar{a})$ and $A$ is bounded bounded \textit{below} then the set of lower bounds of $A$ is defined by the formula
\begin{equation}
	\psi(x,\bar{a})=\forall y\big(\varphi(y,\bar{a})\rightarrow x<y\big).
\end{equation}

Additionally, the set defined by $\psi$ will necessarily be bounded above and thus have a supremum.  It can then be shown that the aforementioned supremum is the infimum of $A$.  Therefore, if a coloured linear order $\alpha$ is definably complete then every definable subset of $\alpha$, which is bounded below, has an infimum.  A similar argument shows that the converse also holds.

\begin{lem}\label{lem:compsum}
	Suppose $\beta$ is a coloured linear order and, for each $i\in\beta$, $\alpha_i$ is a complete coloured linear order then the following holds:
	\begin{enumerate}
		\item	If $\beta$ is complete and, for each $i\in\beta$, $\alpha_i$ has both a least and a greatest element then $\sum_{i\in\beta}\alpha_i$ is complete.
		\item	If $\beta$ is well-ordered and, for each $i\in\beta$, $\alpha_i$ has a least element then $\sum_{i\in\beta}\alpha_i$ is complete.
	\end{enumerate}
\end{lem}
\begin{proof}
	\begin{enumerate}[nosep]
		\item	Suppose $B$ is a nonempty subset of $\sum_{i\in\beta}\alpha_i$ which is bounded above.  By definition, there must then exist some $u\in\beta$ such that, for each $i\geq u$, $B$ is disjoint from the set $A_i=\setbuild{(a,i)}{a\in\alpha_i}$.  Since $\beta$ is complete we may choose $u$ to be the least element of $\beta$ with this property.  If $B\cap A_u$ is unbounded in $A_u$ then the supremum of $B$ is the greatest element of $A_u$.  Otherwise, the supremum of $B$ is $\sup A_u$.
		\item	Similar to 1 except one leverages the well-ordering property of $\beta$ to obtain the necessary $u\in\beta$.\qedhere
	\end{enumerate}
\end{proof}

\begin{prp}\label{prp:defcomp}
	If $\alpha$ is a definably complete coloured linear order without endpoints and $n\in\nats$ then there exists a complete coloured linear order $\beta_n$ such that $\alpha\nequiv{n}\beta_n$.
\end{prp}
\begin{proof}
	Let $n\in\nats$ be fixed but arbitrary and define a binary relation $r$ on $\domain{}\alpha$ such that $aRb$ iff $(a,b)$ is $n$-equivalent to a complete coloured linear order.  It follows from Lemma \ref{lem:IndCong} that $R$ induces a congruence $\sim$ on $\alpha$.

	Take note that, as in the proof of Theorem \ref{thm:Cll}, the congruence $\sim$ is definable.  Therefore, each equivalence class is also definable and we may prove the following:

	\begin{claim}
		If $I\in\faktor{\alpha}{\sim}$ then $I$ is $n$-equivalent to a complete coloured linear order.  Furthermore, if $I$ is bounded above (below) in $\alpha$ then $I$ has a greatest (least) element.
	\end{claim}
	\begin{proof}
		We first consider the case where $I$ is bounded above in $\alpha$.  Choose $u=\sup I$ and let $\family{a_\xi}{\xi<\beta}$ be a cofinal sequence in $I^{<u}$, for some ordinal $\beta\geq\omega$.

		By definition of $I$ it now follows, for each $\xi<\beta$, that the interval $[a_\xi,a_{\xi+1})$ is $n$-equivalent to some complete linear order $\delta_\xi$.  Define the linear order
		\begin{equation}
			\delta=\sum_{\xi<\beta}\delta_\xi
		\end{equation}
		so that it follows from Lemma \ref{lem:compsum} that $\delta$ is a complete coloured linear order.  Lemma \ref{lem:fvsum} then tells us that $\delta\nequiv{n}I\restriction(a_0,u)$ and thus, by definition, we have $a_0\sim u$.  Consequently, $u\in I$ and therefore $u$ is the greatest element of $I$.

		Were it the case that $I$ is unbounded in $\alpha$ then one, again, chooses $\family{a_\xi}{\xi<\beta}$ cofinal in $\alpha$ and a similar argument as before suffices in showing that $I^{>a_0}$ has a complete $n$-equivalent $\delta$.

		Similarly, one may obtain a $b_0\in I$ such that $b_0<a_0$ and $I^{<b_0}$ has a complete $n$-equivalent $\delta^\prime$.  Note, by definition, that $I\restriction (b_0,a_0)$ has a complete $n$-equivalent $\epsilon$.  Therefore, we may conclude that there exists a coloured expansion of the sum
		\begin{equation}
			\delta^\prime+\one+\epsilon+\one+\delta
		\end{equation}
		which is a complete $n$-equivalent of $I$, as required.
	\end{proof}

	\begin{claim}
		The coloured linear order $\faktor{\alpha}{\sim}$ is dense.
	\end{claim}
	\begin{proof}
		Choose any $I,J\in\faktor{\alpha}{\sim}$ such that $I<J$ and, aiming for a contradiction, suppopose there exists no $K\in\faktor{\alpha}{\sim}$ such that $I<K<J$.

		Note that, respectively, $I$ is bounded above and $J$ is bounded below in $\alpha$.  Since $I$ and $J$ are both definable, and $\alpha$ is definably complete, it now follows that there exists $a,b\in\alpha$ such that $a=\sup I$ and $b=\inf J$.

		From our assumption it now follows that $(a,b)=\emptyset$ so that $\alpha\restriction(a,b)$ is (vacuously) complete.  Therefore, since the previous claim yields $a\in I$ and $b\in J$, we may conclude that $a\sim b$ and thus $I=J$ --- the desired contradiction.
	\end{proof}

	Observe that if $\card{\faktor{\alpha}{\sim}}=1$ then the result follows immediately.  By way of contradiction, suppose now that $\card{\faktor{\alpha}{\sim}}>1$.  Let $S=\set{\alpha_0,\dotsc,\alpha_{k-1}}$ be an $n$-spectrum for the class of complete coloured linear orders. The contradiction desired is contained in the following claim:
	\begin{claim}
		There exists an $S^\prime\subseteq S$ and an open interval $D\subseteq\faktor{\alpha}{\sim}$ such that, for each $\delta\in S^\prime$, the set $\setbuild{I\in D}{I\nequiv{n}\delta}$ is dense in $D$.
	\end{claim}
	\begin{proof}
		Choose some open interval $D_0\subseteq\faktor{\alpha}{\sim}$ and define
		\begin{equation}
			S_0=\setbuild{\delta\in S}{\text{there exists an }I\in D_0\text{ such that }\delta\nequiv{n}I}.
		\end{equation}

		Recursively, whenever $0<k<\card{S_0}-1$, we obtain $S_{k+1}$ from $S_k$ by removing any order type $\delta_k\in S_k$ whose $n$-equivalents in the open interval $D_k\subseteq D_{k-1}$ are not dense in $D_k$.  Additionally, we choose $D_{k+1}\subseteq D_k$ to be some open interval, not having $\delta_k$ as a member.  Otherwise, if no such $\delta_k$ exists, we declare $S_{k+1}=S_k$ and $D_{k+1}=D_k$.

		Since $\faktor{\alpha}{\sim}$ is dense, if $m=\card{S_0}-1$ then it follows that $\card{S_m}>0$.  By definition, choosing $S^\prime=S_m$ and $D=D_m$ will establish the claim.
	\end{proof}

	We may assume that $D=(I,J)$ for some $I,J\in\faktor{\alpha}{\sim}$.  Now, let $a,b\in\bigcup D$ satisfy $a<b$.  Our aim is to show that the interval $(a,b)$ of $\alpha$ has some $n$-equivalent in $S$.

	Let $h\colon\reals\to S^\prime$ be the map that induces the canonical partition of $\reals$ into $\card{S^\prime}$ sets.  As in claim 1, note that $I^{>a}$ and $J^{<b}$ will have respective $n$-equivalents $\nu_a,\nu_b\in S$.

	Define a coloured linear order $\chi_{a,b}$ such that
	\begin{equation}
		\chi_{a,b}=\nu_a+\sum_{x\in\lambda}h(x)+\nu_b.
	\end{equation}
	As $\one+\lambda+\one$ is a complete order type, it follows from Lemma \ref{lem:fvsum} that $\chi_{a,b}$ is a complete coloured linear order.  Since the $n$-equivalents in $D$ of any member of $S^\prime$ form a dense subset of $D$, $\Right$ has a winning strategy in the $n$-game between the coloured linear orders $(a,b)$ and $\chi_{a,b}$.

	By definition, we now have $a\sim b$.  Since $a$ and $b$ were chosen arbitrarily, it follows by definition of $\faktor{\alpha}{\sim}$ that $\bigcup D$ is itself an equivalence class.  Thus $D$ has only one element, contradicting that $D$ is an open interval and thereby concluding the proof.
\end{proof}

For the purposes of the proof of the following proposition, given a fixed $k\in\nats$, let $\Sigma_k$ be the theory consisting of the following sentences which, together, express that its models are (nontrivially) dense coloured linear orders without endpoints and are definably complete:
\begin{itemize}
	\item	$\exists x(x=x)$
	\item	$\axmlin\wedge\axmden$,
	\item	$\forall x\exists y(x<y)\wedge\forall x\exists y(y<x)$,
	\item	for each formula $\varphi(x)$ the sentence given by:
		\begin{multline}
			\exists x\varphi(x)\wedge\exists y\forall x\big(\varphi(x)\rightarrow x<y\big)\rightarrow\\
			\exists z\Big(\forall x\big(\varphi (x)\rightarrow x<z\big)\wedge\forall y\big(\forall x(\varphi(x)\rightarrow x<y)\rightarrow z\leq y\big)\Big).
		\end{multline}
\end{itemize}

\begin{prp}
	For each $k\in\nats$, the theory $T$ of complete $k$-coloured linear orders is recursively enumerable.
\end{prp}
\begin{proof}
	Fix any $k\in\nats$ and note that it is enough to show that every model $\alpha$ of $\Sigma_k$ is also a model of $T$.  In line with this we suppose $\alpha\models\Sigma_k$ is fixed but arbitrary.

	Choose any $\sigma\in T$ and let $n=\qrank(\sigma)$.  It follows now from Proposition \ref{prp:defcomp} that there exists a complete coloured linear order $\beta_n\nequiv{n}\alpha$.  By definition, we must have $\beta_n\models T$ and therefore $\beta\models\sigma$.  In conclusion, since $\sigma$ has quantifier rank at most $n$ and $\beta_n\nequiv{n}\alpha$, it holds that $\alpha\models\sigma$.
\end{proof}

\begin{dfn}[Regularity]
	We call a $k$-coloured linear order $\alpha$, with colours say $r_0,\dotsc,r_{k-1}$, \textbf{regular} whenever \textit{a subset of} its colours form a partition of $\alpha$ and no member of $\alpha$ posesses two distinct colours.  Equivalently, $\alpha$ is regular whenever it holds that:
	\begin{equation}
		\alpha\models\left(\forall x\bigvee_{0\leq i<n}r_i(x)\right)\wedge\left(\neg\exists x\bigvee_{i<j<k}(r_i(x)\wedge r_j(x))\right).
	\end{equation}
\end{dfn}

\begin{dfn}[Normality]
	We call a $k$-coloured linear order $\alpha$, with colours say $r_0,\dotsc,r_{k-1}$, \textbf{normal} whenever \textit{all its colours} (together) form a partition of $\alpha$.  Equivalently, $\alpha$ is normal whenever it is regular and satisfies:
	\begin{equation}
		\alpha\models\bigwedge_{0\leq i<k}\exists xr_i(x).
	\end{equation}
\end{dfn}

\begin{dfn}[$k$-Pattern]
	A $k$-coloured linear order $\alpha$ is called a $\bm{k}$\textbf{-pattern} whenever $\alpha$ is both regular and finite.  Should $k\in\nats$ be clear from the context we will refer to $\alpha$ as simply being a \textit{pattern}.
\end{dfn}

\begin{prp}
	If $n\in\nats$ and $\alpha$ is a normal $k$-coloured linear order of order type $\omega$, with colours $r_0,\dotsc,r_{k-1}$, such that
	\begin{equation}
		i<k\quad\implies\quad\alpha\models\forall x\exists y\big(x<y\wedge r_i(y)\big),
	\end{equation}
	then there exists a $k$-coloured linear order $\beta_n$ such that $\alpha\nequiv{n}\beta_n$ and, for each $i<n$, $r_i^{\beta_n}$ is a recursively enumerable set of natural numbers.
\end{prp}
\begin{proof}
	The case $n=0$ is trivial so fix any natural $n>0$.  Choose any $\beta_n$ such that the initial segment of $\beta_n$ of length $2^n$ is coloured identically to the corresponding initial segment of $\alpha$, each of the colours $r_0^{\beta_n},\dotsc,r_{k-1}^{\beta_n}$ are recursively enumerable subsets of $\beta_n$ and there exists arbitrary large elements in $\beta_n$ of any given colour.

	We proceed by means of an $n$-game between $\alpha$ and $\beta_n$.  The winning strategy for the player $\Right$ is described as follows.  Should $\Left$ choose some $a_0\in\alpha$ such that $r_i^\alpha(a_0)$, for some $i<k$, and there are at most $2^n-1$ elements (strictly) below $a_0$ in $\alpha$ then $\Right$ responds with the corresponding element (per the definition of $\beta_n$) in $\beta_n$.

	On the other hand, if the elements (strictly) below $a_0$ in $\alpha$ number at least $2^n$ then $\Right$ chooses the least $b_0\in\beta$ such that $r^{\beta_n}_i(b_0)$ and every pattern, of length at most $n-1$, embeddable in $\alpha^{<a_0}$ can also be embedded in $\beta_n^{<b_0}$.  A similar approach is followed if, in stead, $\Left$ has some $b_0\in\beta_n$ as his first move.

	Suppose now that $0<m\leq n$ and it is round $m$, of the $n$-round game, and $\Left$ has just played some $a_{k-1}\in\alpha$.  Note that there is no harm in assuming $\Left$ never repeats a move and, for each $i<m-1$, that $a_i<a_{i+1}$.

	Additionally, we need only consider the case where the following holds: (1) the first $m-1$ rounds of the current game constitute a winning position for $\Right$ in the game $\EF_{m-1}(\alpha,\beta_n)$; (2) every pattern, of length at most $n-(m-1)$, which embeds in $\alpha^{<a_0}$ also embeds in $\beta_n^{<b_0}$ and (3) for each $i<m-1$ and every pattern, of length at most $n-(m-1)$, which can be embedded in $(a_i,a_{i+1})$ also embeds in $(b_i,b_{i+1})$.  Otherwise, $\Right$ would be free to pick any element from $\beta_n$ as his move.

	It now follows by assumption, for each $i<m-1$ and every $j<k$, that $b_i<b_{i+1}$ and
	\begin{equation}
		\alpha\models r_j(a_i)\quad\iff\quad\beta_n\models r_j(b_i).
	\end{equation}

	Should the length of $\alpha^{<a_{k-1}}$ be at most $2^{n-(m-1)}-1$ then $\Right$ simply responds with the corresponding element of $\beta_n$, according to the definition of $\beta_n$ by noting that $2^{n-(m-1)}\leq 2^n-1$.

	Now suppose the length of $\alpha^{<a_{k-1}}$ is at least $2^{n-(m-1)}$. In the case where $a_{k-1}>a_{k-2}$, $\Right$ responds with the least $b_{k-1}\in\beta_n$ such that every pattern, of length at most $2^{n-(m-1)}-1$, that embeds in $(a_{k-2},a_{k-1})$ also embeds in $(b_{k-2},b_{k-1})$.

	The only remaining case is if there exists some natural $\ell<m-1$ such that $a_k\in(a_\ell,a_{\ell+1})$.  The proper response is then for $\Right$ to play the least $b_k\in\beta_n$ such that every pattern embeddable in $(a_\ell,a_{k-1})$ also embeds in $(b_\ell,b_k)$.
\end{proof}

\begin{prp}
	Suppose $\alpha$ is a normal $k$-coloured linear order of order type $\omega$ such that each of the colours, say $r_0,\dotsc,r_{k-1}$, is cofinal in $\alpha$ i.e.\ for each $i<k$:
	\begin{equation}
		\alpha\models\forall x\exists y\big(x<y\wedge r_i(y)\big).
	\end{equation}
	It then necessarily follows that $\Th(\alpha)$ is a recursively enumerable theory.
\end{prp}

\begin{lem}
	For each $k\in\nats$, the set
	\begin{equation}
		R_k=\setbuild{(\alpha,\sigma)}{\alpha\in\Cast_k\text{ is normal and }\alpha\models\sigma}
	\end{equation}
	is recursively enumerable.
\end{lem}
\begin{proof}
	We proceed by induction on $k\in\nats$.  Since the order types in $\Cast$ are all dense and without endpoints, the result is trivial for the case $k=0$: each member of $\Cast$ is elementarily equivalent to order type of the rationals $\eta$ and $\Th(\eta)$ is recursively enumerable as it is both finitely axiomatisable and complete.

	Assume now that $R_k$ is recursively enumerable for some $k\in\nats$.  In what follows, we choose $\family{\sigma_i}{i<\omega}$ to be a (mechanically generated) enumeration of all instances of the first-order scheme which expresses the completeness property:
	\begin{equation}
		\forall\bar{z}\big(\exists x\varphi(x,\bar{z})\wedge\exists y\forall x(\varphi(x,\bar{z})\rightarrow x\leq y)\big).
	\end{equation}
	Additionally, if $\chi_\varphi$ denotes the sentence expressing that ``the binary relation defined by $\varphi(x,y,\bar{z})$ is a congruence``: choose $\family{\tau_i}{i\in\nats}$ be a (mechanically generated) enumeration of all instances of the first-order scheme expressing quasi-seperability:
	\begin{equation}
		\forall\bar{z}\big(\chi_\varphi\rightarrow\forall x_0\forall x_1\exists y_0(x_0<y_0<x_1\wedge\forall y_1(\varphi(y_0,y_1,\bar{z})\rightarrow y_0=y_1))\big)
	\end{equation}

	For each $\alpha\in\Cast$, we proceed to define theories $S_\alpha\subseteq T_\alpha$ such that $S_\alpha$ is finite and $T_\alpha$ is a complete recursively enumerable theory.

	Define $S_\lambda$ to consist of only the sentence which expresess that $\lambda$ is a dense linear order without endpoints:
	\begin{equation}
		\axmlin\wedge\axmden\wedge\forall x\exists y\exists z(y<x<z).
	\end{equation}
	Now choose $T_\lambda=S_\lambda$
\end{proof}

\begin{dfn}[$k$-Complete theory]
	A first-order theory $T$, for some $k\in\nats$, is called $\bm{k}$\textbf{-complete} whenever it holds, for every sentence $\sigma$ of quantifier rank at most $k$, that $T\models\sigma$ or $T\models\neg\sigma$.
\end{dfn}

\begin{dfn}[$k$-Decidability]
	A first-order theory $T$, for some $k\in\nats$, is called $\bm{k}$\textbf{-decidable} whenever it holds, for every sentence $\sigma$ of quantifier rank at most $k$, that there exists a machine that can decide the statement ``$T\models\sigma$''.
\end{dfn}


	% backmatter
	\appendix
	\bibstyle{amsalpha}

\chapter{Complete linear orders and the reals}

\section{Continuous linear orders and separability}

\begin{dfn}[Completeness]
	A linear order $\alpha$ is said to be \textbf{complete} if every (nonempty)
	subset of $\domain\alpha$, bounded above in $\alpha$, has a least upper
	bound.
\end{dfn}

Equivalently, $\alpha$ is complete whenever all of its nonempty subsets, bounded
below, have a greatest lower bound.

\begin{dfn}[Continuous linear order]
	If $\alpha\in\dense$ then we call $\alpha$ \textbf{continuous} precisely when it is complete.
\end{dfn}

\begin{dfn}[Separable]
	A linear order $\alpha$ is called \textbf{separable} whenever
	$\alpha\notin\set{\zero,\one}$ and there exists a countable subset of
	$\domain{}\alpha$ which is dense in $\alpha$.
\end{dfn}

A notable property of all separable linear orders is that they are necessarily
also dense.  Suppose, for instance, $\alpha$ is a separable linear order then
choose $D\subseteq\alpha$ to be countable dense in $\alpha$.  If $a<a^{\prime}$
are arbitrary elements of $\alpha$ then, by definition, there must exist some
$d\in D$ such that $a<d<a^{\prime}$.  Hence, the separable linear orders form a
subclass of the class $\Dense$ of all (nontrivially) dense linear orders.

\begin{prp}[Separability]\label{prp:sep}
	If $\alpha$ is a separable linear order then there exists an embedding
	$h\colon\alpha\to\lambda$.
\end{prp}
\begin{proof}
	Suppose $\alpha$ is separable.  By definition, there exists a countable
	$\beta\subseteq\alpha$ which is dense in $\alpha$.  From Cantor's Theorem, there
	exists an isomorphism $h_{0}\colon\beta\to\eta$. Define a map
	$h\colon\alpha\to\lambda$ such that, for $a\in\alpha$,
	\begin{equation}
		h(a)=\sup\setbuild{h_{0}(b)}{b\in\beta\text{ and }b<a}.
	\end{equation}
	It is now sufficient to prove that $h$ is an embedding.

	We first establish that $h$ is strictly increasing on $\alpha$, that is to
	say:
	\begin{equation}
		a<a^{\prime}\implies h(a)<h(a^{\prime}),
	\end{equation}
	for all $a,a^{\prime}\in\alpha$.  Fix any $a<a^{\prime}$ in $\alpha$. Since
	$\beta$ is dense in $\alpha$, there must exist a $b\in\beta$ such that
	\begin{equation}
		a<b<a^{\prime}.
	\end{equation}
	Consequently, by definition of $h$, we must have
	\begin{equation}
		h(a)<h(a^{\prime}).
	\end{equation}
	It follows that $h$ is a (strictly) increasing map and, thus, $h$ is
	an embedding of $\alpha$ into $\lambda$.
\end{proof}

\begin{thm}[Characterising the reals]\label{thm:rchar}
	Every separable continuous linear order is isomorphic to $\lambda$.
\end{thm}
\begin{proof}
	Suppose $\alpha$ is a separable continuous linear order.  From Proposition
	\ref{prp:sep}, there exists an embedding $h\colon\alpha\to\lambda$.  Since
	$\alpha$ is separable there exists some denumerable $\beta\subseteq\alpha$
	which is dense in $\alpha$.

	Since $h$ is an embedding and $\beta$ denumerable, dense and without
	endpoints, it follows from Cantor's theorem that we must have
	\begin{equation}
		\beta\cong\eta\cong h[\beta].
	\end{equation}
	Now define $\alpha^\prime=\comp(h[\beta])$ so that we must have
	$\alpha\cong\alpha^\prime$, since $\alpha$ is complete and $\beta$ is dense in
	$\alpha$.

	Lastly, we will claim that $\alpha^\prime$ is convex.  Suppose to the
	contrary there exists $a,b\in\alpha^\prime$, with $a<b$ and a $c\in\reals$ such
	that $a<c<b$ but $c\notin\alpha^\prime$.  Define
	$c_0=\sup\setbuild{x\in\alpha^\prime}{x<c}$. By definition, we have $c_0\leq c$.
	If $c_0=c$ then the completeness of $\alpha^\prime$ guarantees
	$c=c_0\in\alpha^\prime$.

	Assume now, instead, that $c_0<c$ then, since $h[\beta]$ must be dense in
	$\alpha^\prime$, there exists some $d\in\beta$ such that $c_0<h(d)<c$.  However,
	since $h(d)\in\beta^\prime\subseteq\alpha^\prime$, this contradicts the
	definition of $c_0$ which requires that $h(d)\leq c_0$.
\end{proof}


\section{The Suslin property and the first order theory of the coloured reals}

Note that in this section we will refer to \textit{coloured linear orders}
(linear orders with finitely many unary relations defined on them) simply as
linear orders.  This simplifies the language used but it should be noted that
all the following results also apply in the monochromatic case i.e.\ in the
absence of any colours.

\begin{dfn}[Quasi-separable]\label{dfn:quasiseparable}
	A linear order $\alpha$ is said to be $\textbf{quasi-separable}$ iff every
	densely ordered condensation $\beta\not\cong\one$ of $\alpha$ has a subset which
	is dense in $\beta$ and consists of only singleton subsets of $\alpha$.
\end{dfn}

\begin{dfn}[Suslin property]
	A continuous linear order $\alpha$ is said to posses the \textbf{Suslin
		property} iff every set of \textit{pairwise disjoint open intervals} in
	$\alpha$ is at most countable.
\end{dfn}

Clearly $\lambda$ has the Suslin property, since any open interval in $\lambda$
necessarily contains a rational number.  A question that immediately arises is
whether or not Suslinity could be a stand-in for separability in characterising
the real number line.  Unfortunately, this problem cannot be decided within
ordinary ZFC and additional set-theoretic assumptions are required in order to
give a definite yes or no answer.  The reader may refer to any of the standard
texts on set theory, should they decide to embark on this quest.

The following lemma illustrates that quasi-separable linear orders are in
abundance and that $\lambda$ is among them:

\begin{prp}
	If $\alpha$ is a complete linear order with the Suslin property then
	$\alpha$ must be quasi-separable.
\end{prp}
\begin{proof}
	Let $\alpha$ be any continuous linear order that has the Suslin property and
	choose $\beta\not\cong\one$ to be any dense condensation of $\alpha$, say
	$\beta=\faktor{\alpha}{\sim}$.  By way of contradiction, assume there exists no
	$D\subseteq\beta$ which is dense in $\beta$ and consists of only singleton
	subsets of $\alpha$.

	Given this setting, there must exist elements $a<b$ in $\alpha$ such that
	$\faktor{a}{\sim}<\faktor{b}{\sim}$ and no singleton subset of $\alpha$ is a
	member of the open interval
	\begin{equation}
		I_{a,b}=\left(\faktor{a}{\sim},\faktor{b}{\sim}\right).
	\end{equation}

	Since $I_{a,b}$ has no singleton members, each of its elements necessarily
	contains an open interval as a subset.  Therefore, by Suslinity, it follows
	that $I_{a,b}$ must be countable.  Since $\beta$ (and thus $I_{a,b}$) is dense,
	it follows by Cantor's Theorem that $I_{a,b}$ has order type $\eta$.

	Since $\eta$ is not complete, we may choose $G\subseteq I_{a,b}$ to be
	bounded above in $I_{{a,b}}$ but without a supremum in $I_{a,b}$.  We
	now argue that $\bigcup G$ has no supremum, thereby contradicting the
	completeness of $\alpha$.

	Suppose to the contrary that $\bigcup G$ has a supremum
	$u\in \bigcup I_{a,b}$.  By definition, it follows that $\faktor{u}{\sim}$ is an
	upper bound of $G$ in $I_{a,b}$.  Now choose any $u^{\prime}\in\bigcup I_{a,b}$
	such that $\faktor{u^{\prime}}{\sim}$ is an upper bound of $G$.  Consequently,
	by definition, $u^{\prime}$ is an upper bound of $\bigcup G$ and thus
	$u^{\prime}\geq u$.  By definition, we may conclude that
	\begin{equation}
		\faktor{u^{\prime}}{\sim}\geq\faktor{u^{\prime}}{\sim}
	\end{equation}
	so it follows that $\faktor{u}{\sim}$ is the supremum of $G$.  This
	contradicts the definition of $G$ and the result now follows.
\end{proof}

\begin{dfn}[Definably quasi-separable]
	A linear order $\alpha$ is called \textbf{definably quasi-separable} iff
	each densely ordered condensation $\beta\not\cong\one$ induced by a definable
	congruence of $\alpha$ has a dense set of singleton subsets of $\alpha$.
\end{dfn}

\begin{dfn}[Definable complete]
	A linear order $\alpha$ is \textbf{definably complete} whenever each of its
	definable subsets, bounded above, has a supremum.
\end{dfn}

Similar to before, $\alpha$ is definably complete in the sense just defined iff
each of its definable subsets, bounded below, has an infimum.  To this end,
suppose $\bar{a}$ is a finite tuple in $\alpha$ and $\varphi(x,\bar{a})$ defines
a subset $B_{\varphi}$ of $\alpha$, bounded below.

Define the formula
\begin{equation}
	\psi(x,\bar{a})\define\neg\exists y(\varphi(y,\bar{a})\wedge y<x).
\end{equation}
It follows immediately that $\psi=\psi(x,\bar{a})$ defines a subset $A_{\psi}$
of $\alpha$, bounded above by every element of $B_{\varphi}$.  Furthermore, we
have
\begin{equation}
	\sup A_{\psi}=\inf B_{\varphi}.
\end{equation}

The reverse direction of this implication follows similarly.  Hence, as
anticipated, the two alternate formulations of definable completeness are
equivalent.

\begin{prp}\label{prp:qdense}
	For each natural $n\geq 2$, a partition $Q_0,\dotsc,Q_{n-1}$ of $\rats$
	exists such that, for each $i=0,\dotsc,n-2$, $Q_i$ is dense in $\eta$.
\end{prp}
\begin{proof}
	Let $p_{0},\dotsc,p_{n-1}$ denote the first $n-1$ prime numbers.  Now, for each
	$i<n-1$, define the set
	\begin{equation}
		Q_{i}=\setbuild{\frac{m}{p_{i}^{k}}}{k,m\in\ints\setminus\set{0}\text{
				and  }\gcd(m,p_{i})=1},
	\end{equation}
	and let $Q_{n-1}=\rats\setminus\bigcup_{i<n-1}Q_{i}$.

	We now claim that each such $Q_{i}$, for $i<n$, is dense in $\eta$.  Fix any
	distinct $q,q^{\prime}\in\eta$ and first consider the case $i<n-1$.  Note
	that there must exist a least $k\in\posnats$ such that
	\begin{equation}
		\frac{1}{p_{i}^{k}}<\abs{q^{\prime} - q}.
	\end{equation}

	Since the distance between two elements of $\eta$ is preserved under the map
	$x\mapsto -x$, we may assume without loss of generality that
	$0\leq q<q^{\prime}$.  There now exists a least $m\in\nats$ such that
	\begin{equation}
		q<\frac{m}{p_{i}^{k}}.
	\end{equation}
	By definition of $m$, it follows that
	\begin{equation}
		\frac{m-1}{p_{i}^{k}}<q.
	\end{equation}
	Therefore, by definition of $k$, we may conclude that
	\begin{equation}
		q<\frac{m}{p_{i}^{k}}<q^{\prime}.
	\end{equation}
	Reducing the fraction above to its simplest form then yields the desired
	result.

	Let $p$ be the least prime number such that $p>p_{n-1}$.  Repeating the
	argument above, and substituting $p$ for $p_{i}$, will show that $Q_{n-1}$
	is also dense in $\eta$.
\end{proof}

\begin{prp}\label{prp:csums}
	If $\delta$ is a complete linear order and $\family{\alpha_i}{i\in\delta}$
	is a family of complete linear orders with endpoints then the sum
	\begin{equation}
		\alpha=\sum_{i\in\delta}\alpha_i,
	\end{equation}
	is also a complete linear order.
\end{prp}
\begin{proof}
	Suppose $B\subseteq\alpha$ is bounded above in $\alpha$ and identify each
	$\alpha_{i}$ with its corresponding image under the canonical embedding of
	$\alpha_{i}$ into $\alpha$. Choose $d\in\delta$ to be the supremum of
	all $i\in\delta$ such that $B\cap\alpha_{i}\neq\emptyset$.

	If $B\cap\alpha_{d}=\emptyset$ then let $u$ be the least element of
	$\alpha_{d}$, otherwise, choose $u$ to be the supremum of $B\cap\alpha_{d}$.
	Since $\alpha_{d}$ has a greatest element, it follows that $u\in\alpha_{d}$
	in either case.  Consequently, by definition, $u$ is an upper bound of $B$.

	All that remains is to demonstrate the minimality of $u$.  Suppose then that
	$u^{\prime}$ is any other upper bound of $B$.  If $u^{\prime}\in\alpha_{e}$,
	for some $e>d$, then we immediately have $u^{\prime}\geq u$ so assume the
	contrary.  By definition of $d$, we cannot have $u^{\prime}\in\alpha_{c}$
	for any $c<d$ so we may conclude that $u^{\prime}\in\alpha_{d}$.  It then
	follows immediately, by definition of $u$, that $u^{\prime}\geq u$, as required.
\end{proof}

\begin{prp}\label{prp:countdensumsep}
	Suppose $\gamma$ is a countable dense linear order and
	$\family{\alpha_{i}}{i\in\gamma}$ is a family of separable linear orders.
	The sum
	\begin{equation}
		\alpha=\sum_{i\in\gamma}\alpha_{i}
	\end{equation}
	must then also be a separable linear order.
\end{prp}
\begin{proof}
	Identify each $\alpha_{i}$, for $i\in\gamma$, with its image under the
	canonical embedding into $\alpha$.  By definition, for any $i\in\gamma$,
	there exists a countable $S_{i}\subseteq\alpha_{i}$ which is dense in
	$\alpha_{i}$.

	Define $S=\bigcup_{i\in\gamma}S_{i}$, then it follows by definition that
	\begin{equation}
		\card{S}= \aleph_0\cdot\aleph_0=\aleph_0.
	\end{equation}
	We are now required to show that $S$ is dense in $\alpha$ so fix any $a<b$
	in $\alpha$.  If there exists a $j\in\gamma$ such that $a,b\in\alpha_{j}$
	then there must exist a $c\in S_{j}\subseteq S$ such that $a<c<b$.

	Otherwise, there exists $j<k$ in $\gamma$ such that $a\in\alpha_{j}$ and
	$b\in\alpha_{k}$.  Since $\gamma$ is dense, there must exist an $m\in\gamma$
	such that $j<m<k$.  If we now choose any $c\in S_{m}\subseteq S$ then it
	follows that $a<c<b$, as required.
\end{proof}

\begin{thm}
	Suppose $\alpha\in\dense$ is countable, definably complete and definably
	quasi-separable.  For each $n\in\nats$, $\alpha$ has an $n$-equivalent of order
	type $\lambda$.
\end{thm}
\begin{proof}
	As in the statement, let $\alpha$ be countable, definably complete and
	definably quasi-separable.  Suppose also that there are $k$ colours defined on
	$\alpha$, for some fixed $k\in\nats$.  Now choose some $n\in\nats$ and define a
	(binary) relation $R$ on $\alpha$ such that, for every $a,b\in\alpha$:  $aRb$
	iff $a\leq b$ and $(a,b)\nequiv{n}\lambda$ whenever $a\neq b$.  Since sums
	preserve $n$-equivalence, we may conclude that $R$ is transitive and therefore
	induces a congruence $\sim$ on $\alpha$.  Note that it follows from the
	definition of $\sim$ that each $\beta\in\faktor{\alpha}{\sim}$ satisfies exactly
	one of: $\beta\nequiv{n}\one$, $\beta\nequiv{n}\one+\lambda$,
	$\beta\nequiv{n}\lambda+\one$, $\beta\nequiv{n}1+\lambda+1$ or
	$\beta\nequiv{n}\lambda$.

	\begin{claim}
		The congruence $\sim$ is a definable (binary) relation and the
		condensation $\faktor{\alpha}{\sim}$ is dense.
	\end{claim}
	Let $\tau$ be the (finite) disjunction of all characteristic sentences, of
	quantifier-rank at most $n$, of $k$-colourings of $\lambda$.  Define
	$\varphi(x,y)$ to be the formula given by
	\begin{equation}\label{eq:condef}
		x=y\vee(x<y\wedge\tau^{(x,y)})\vee(y<x\wedge\tau^{(y,x)}).
	\end{equation}
	By definition of $\varphi$, as well as that of aforementioned characteristic
	sentences, it follows that $\varphi$ is a defining formula for the
	congruence relation $\sim$.

	It now remains to be shown that $\faktor{\alpha}{\sim}$ is dense.  We
	suppose, to the contrary, that there exists $I,J\in\faktor{\alpha}{\sim}$ such
	that the interval $(I,J)$ is empty.  Therefore, there cannot exists a
	$c\in\alpha$ such that $I<c<J$.

	If we now let $a=\sup I$ and $b=\inf J$ then it must hold that
	$a,b\in I\cup J$ and $a\leq b$.  Since $\alpha$ is dense we cannot
	simultaneously have $a\in I$ and $b\in J$.  Therefore, either $a,b\in I$ or
	$a,b\in J$.

	As the cases are similar, we may suppose that $a,b\in I$.  Note that, since
	$a$ is the supremum of $I$, the interval $(a,b)$ must be empty and thus
	$a=b$.  Also, since $b\notin J$, it follows that $J$ cannot have order type
	$\one$.  From the countability of $\alpha$, it now follows that there exists
	a strictly decreasing sequence $(b_{i})_{i<\omega}\subseteq J$ such that
	\begin{equation}
		\inf_{i<\omega}b_{i}=b.
	\end{equation}

	For each $i<j<\omega$, by definition of $\sim$ the order type of the
	interval $[b_{j},b_{i})$ is $n$-equivalent to $\one+\lambda$.  From Lemma
	\ref{lem:fvsum}, it now follows that the order type of $(b,b_{0})$ is
	$n$-equivalent to $(\one+\lambda)\cdot\dual{\omega}\cong\lambda$.  Thus, by
	definition, it holds that $b\sim b_{0}$.  Since this contradicts that
	$b\notin J$, we have successfully established the above claim.  Furthermore,
	a similar argument may be used to show that each $I\in\faktor{\alpha}{\sim}$
	must contain their respective suprema and infima whenever they exist.

	Note that if $\card{\faktor{\alpha}{\sim}}=1$ then, since $\alpha$ is
	countable, it follows from Lemma \ref{lem:fvsum} that $\alpha$ is $n$-equivalent
	to a $k$-colouring of order type
	\begin{equation}
		(\one+\lambda)\cdot\dual{\omega}+\one+(\lambda+\one)\cdot\omega\cong\lambda,
	\end{equation}
	as required.  Suppose then, by way of contradiction, that
	$\card{\faktor{\alpha}{\sim}}>1$.
	\begin{claim}
		There exists a (proper) open interval $I$ of $\faktor{\alpha}{\sim}$ and
		a finite set $\Sigma$ of coloured linear orders, each of which either has order
		type $\one$ or $\one+\lambda+\one$, such that the following holds:
		\begin{enumerate}[nosep]
			\item for each $\beta\in I$ there exists a $\sigma_\beta\in\Sigma$
			      such that $\beta\nequiv{n}\sigma_\beta$,

			\item if $\sigma\in\Sigma$ then the set $\setbuild{\beta\in I}{\beta\nequiv{n}\sigma}$ is dense in $I$.
		\end{enumerate}
	\end{claim}

	Define $C$ to be the set of $n$-characteristic sentences $\cha{\beta}{n}$
	for $\beta\in\faktor{\alpha}{\sim}$. It follows that $C$ is a nonempty
	finite set of sentences. Fix some $n$-spectrum $\Sigma_{C}$ for the class
	$\faktor{\alpha}{\sim}$.  As noted previously, since equivalence classes
	under $\sigma$ must contain any existing suprema and infima, each
	$\beta\in\Sigma_C$ is of order type either $\one$ or $\one+\lambda+\one$.

	Note, since $\sim$ is definable and $\alpha$ is definably quasi-separable,
	that $\card{C}\geq 1$.  We then proceed by means of induction on $\card{C}$.
	If $\card{C}=1$ then every $\beta\in\faktor{\alpha}{\sim}$ has order type
	$\one$, since $\alpha$ is definably quasi-separable and thus
	$\faktor{\alpha}{\sim}$ must contain a dense set of singletons.  If we now
	choose any fixed $\beta_0\in\faktor{\alpha}{\sim}$, as well as any (proper)
	open subinterval $I$ of $\faktor{\alpha}{\sim}$ such that $\beta_0\in I$,
	then $\Sigma=\set{\beta_0}$ and $I$ satisfy properties 1 and 2 above.

	Assume that the claim holds whenever $\card{C}<m$, for some fixed
	$m\in\posnats$.  Suppose the claim fails when $\card{C}=m$ then, by definition,
	there exists a $\tau_0\in C$ and an open interval
	$I^\prime\subsetneq\faktor{\alpha}{\sim}$ such that
	$\beta\not\models\tau_0$, for any $\beta\in I^\prime$.  Since
	$\card{C\setminus\set{\tau_0}}<m$, it follows from the inductive hypothesis that
	there exists an open interval $I\subsetneq I^\prime$ and a
	$\Sigma\subseteq\Sigma_C$ satisfying properties 1 and 2.  This then
	establishes the desired claim.

	\smallskip	The goal is now to show that $\card{I}=1$, contradicting that
	$I$ is an open interval.  Once established, this contradiction will imply
	that the assumption $\card{\faktor{\alpha}{\sim}}>1$ was erroneous and,
	therefore, that $\faktor{\alpha}{\sim}$ is a singleton partition of
	$\alpha$.

	Since $\alpha$ is definably quasi-separable, there exists a
	$\beta_1\in\faktor{\alpha}{\sim}$, of order type $\one$, such that
	$S=\setbuild{\beta\in\faktor{\alpha}{\sim}}{\beta\nequiv{n}\beta_1}$ is dense in
	the condensation $\faktor{\alpha}{\sim}$.  Therefore, $S\cap I$ is dense in $I$.

	We now proceed to construct a coloured linear order $\delta$ of order type
	$\lambda$ which is $n$-equivalent to $\bigcup I$.  Let
	$\Sigma=\set{\sigma_0,\dotsc,\sigma_{\ell-1}}$ be an $n$-spectrum for
	$\faktor{\alpha}{\sim}$.  Since the labelling is irrelevant, we may assume
	without loss of generality that $\sigma_{0}\nequiv{n}\beta_{1}$.

	From Proposition \ref{prp:ratpart}, we may choose a surjection
	$h\colon\reals\to\Sigma$ such that $h(x)=\sigma_0\nequiv{n}\beta_1$, for each
	$x\in\irrats$, and $h^{-1}[\sigma]$ is dense in $\lambda$ for any
	$\sigma\in\Sigma\setminus\set{\beta_1}$. We now define the (coloured) linear
	order
	\begin{equation}
		\delta=\sum_{x\in\lambda}h(x).
	\end{equation}
	Since each $\sigma\in\Sigma$ has order type $\one$ or $\one+\lambda+\one$,
	it follows from Proposition \ref{prp:csums} that $\delta$ is a complete
	linear order.  Whenever $x\in\irrats$, it follows (by definition) that
	$h(x)$ has order type $\one$.  Hence, the sum
	\begin{equation}
		\gamma=\sum_{x\in\eta}h(x)
	\end{equation}
	embeds densely into $\delta$.

	Since each summand of $\gamma$ is separable, and $\eta$ is countable and
	dense, it follows form Proposition \ref{prp:countdensumsep} that $\delta$ is
	also separable.  Consequently, from theorem \ref{thm:rchar}, we may conclude
	that $\delta$ has order type $\lambda$.

	A winning strategy for $\Right$ in the $n$-round game
	$\EF_{n}(\delta,\bigcup I)$ may now be devised by leveraging the fact that each
	respective member of the $n$-spectrum $\Sigma$ is dense in $I$.  Hence, it
	may be concluded that $\delta\nequiv{n}\bigcup I$. Therefore, since the
	equivalence classes of $\sim$ must be convex, we may conclude that
	$\card{I}=1$.  This is the desired contradiction.
\end{proof}


	% print all glossaries
	\setglossarystyle{longheaderborder}
	\printglossaries

  	% print bibliography
	\nocite{*}
	\bibliography{references}

\end{document}
