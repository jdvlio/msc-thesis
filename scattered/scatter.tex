\bibliographystyle{amsalpha}

\chapter{Scattered linear orders}

The scattered linear orders are antithetical to the \textit{dense linear
orders}, the latter of which includes specimens such as $\eta$ and $\lambda$.
They are precisely the linear orders which do not embed any dense linear order
or, equivalently, do not embed a copy of the order type $\eta$ of the
rationals.

As we will see, the scattered linear orders can constructed in a systematic
fashion from the ground up by starting with the order types $\zero$ and $\one$
and iterating relatively ``simple'' operations.

From the perspective of first-order logic, the class $\scattered$ is also
well-behaved in that its theory $\Th(\scattered)$ is decidable.  A recursively
enumerable class $\Mzero\subseteq\scattered$ is presented in order to establish
this fact.  The argument proceeds as in \cite{RosLin}, owing to the techniques
of L\"auchli and Leonard.

\section{Condensation maps and congruence lattices}

The following definition is reminiscent of the analogous concept from algebra.

\begin{dfn}[Congruence]
	Suppose $\sim$ is an equivalence relation on the domain of the linear order
	$\alpha$.  We call $\sim$ a \textbf{congruence} on $\alpha$ whenever, for
	every $a,b\in\alpha$ such that $a<b$ and $a\nsim b$, the following
	holds:
	\begin{equation}
		a^\prime\sim a\text{ and }b^\prime\sim b\quad\implies\quad
		a^\prime<b^\prime,
	\end{equation}
	for all $a^\prime,b^\prime\in\alpha$.
\end{dfn}

Essentially, congruences are the equivalence relations that are compatible with
the order structure of $\alpha$.

Upon careful inspection, the above definition can be seen to be equivalent to
the definition of a lattice-theoretic congruence.  One need only view $\alpha$
as a lattice.  The lattice operations are derived in the usual fashion from the
order relation.

We will, ofcourse, now need a means of constructing various condensations along
our travels.  The following Lemma accomplished this feat by starting with an
arbitrary transitive relation.

\begin{lem}[Congruence construction]
	\label{lem:IndCong}
	Suppose $\alpha$ is a linear order and $R$ is a transitive binary relation
	on $\alpha$.  Now define a another binary relation $\sim$ on $\alpha$ such
	that, for $a,b\in\alpha$, we have $a\sim b$ whenever one of the following
	is satisfied:
	\begin{enumerate}
		\item   $a=b$,
		\item   $a<b$ and $aRb$,
		\item   $b<a$ and $bRa$.
	\end{enumerate}
	Under these assumptions, $\sim$ is a congruence on $\alpha$.
\end{lem}
\begin{proof}
	By definition, $\sim$ must be both reflexive and transitive.  To establish
	transitivity, fix any $a,b,c\in\domain{}\alpha$.

	Without loss of generality we may assume that $a\leq b\leq c$.  However,
	sine the cases $b=c$ and $a=b$ are trivial, it suffices to consider only the
	case $a<b<c$.  The result then follows by trasitivity of the linear order
	relation.
\end{proof}

\begin{dfn}[Induced congruence]
	The congruence $\sim$ in Lemma (\ref{lem:IndCong}) is referred to as the
	\textbf{congruence induced by $R$}.
\end{dfn}

\begin{dfn}[Splitting]
	If $\alpha,\beta$ are linear orders then a \textbf{splitting} is a
	surjective homomorphism $\pi\colon\alpha\rightarrow\beta$.
\end{dfn}

The definition of a splitting should be somewhat reminiscent of the
\textit{factor maps}, also called quotient maps, from algebra and topology.  Our
analog of a quotient structure is the following:

\begin{dfn}[Condensations]
	If $\sim$ is a congruence of the linear order $\alpha$ and
	$\pi_\sim\colon\alpha\to\faktor{\alpha}{\sim}$ is the (unique) splitting such
	that, for every $a,b\in\alpha$, it holds that:
	\begin{equation}
		a\sim b\iff\pi_\sim(a)=\pi_\sim(b),
	\end{equation}
	then we call $\pi_\sim$ the \textit{splitting of $\alpha$ induced
	by} $\sim$, refer to the quotient $\faktor{\alpha}{\sim}$ as a
	\textbf{condensation}.
\end{dfn}

We single out the following splitting in particular.  It will play a crucial
role in some of the results to come.

\begin{dfn}[Finite splitting]
	Suppose $\alpha$ is a linear order and $\sim$ is the congruence on $\alpha$
	induced by the relation $R$ defined by:
	\begin{equation} aRb\quad\iff\quad
		a<b\text{ and }[a,b]\text{ is finite}.
	\end{equation} For any linear order
	$\alpha$ we let $\fsplit[\alpha]$ denote the map $\fsplit[\alpha]\colon
	a\mapsto\faktor{a}{\sim}$ for $a\in\alpha$.  We refer to $\fsplit[\alpha]$
	as the $\textbf{finite splitting}$ on $\alpha$ and we omit the prescript
	whenever $\alpha$ is clear from the context.
\end{dfn}

In the result that follows, for any binary relation $R$, we use the notation
$\trclos(R)$ to denote the \textit{transitive closure} of $R$.  In other words,
$\trclos(R)$ is the smallest (w.r.t.\ set inclusion) transitive binary relation
containing $R$ as a subset.

\begin{prp}[Congruence lattices]
	Let $\con \alpha\subseteq\powerset{\domain{2}\alpha}$ be the set of all
	congruences on $\alpha$ and, for each $X\subseteq \con\alpha$, define:
	\begin{align}
		\bigwedge X &\coloneqq\bigcap X,\\ \bigvee X
					&\coloneqq\bigwedge\setbuild{a\in\con\alpha}{x\subseteq
					a,\forall x\in X}.
	\end{align}
	It then follows that $(\con\alpha,\vee,\wedge)$ is a complete lattice.
	Furthermore, for any $X\subseteq\con\alpha$, it holds that $\bigvee
	X=\trclos(\bigcup X)$.
\end{prp}
\begin{proof}
	Let $X$ be an arbitrary (non-empty) subset of $\con\alpha$.  We now proceed
	to argue that $(\con\alpha,\vee,\wedge)$ is a complete lattice under the
	defined operations.  To achieve this, we are first required to show that (as
	defined above) the sets $\bigvee X$ and $\bigwedge X$ are in fact
	congruences.

	We first make the case for $\bigwedge X$.  Let $I$ denote the identity
	relation on $\domain{}\alpha$ then, since the members of $X$ are all
	reflexive, we must have $I\subseteq x$ for each $x\in X$ and thus
	$I\subseteq\bigcap X=\bigwedge X$.

	Therefore $\bigwedge X$ is in fact a reflexive (binary) relation.  Also,
	since the members of $X$ are all symmetric it immediately follows that
	$\bigwedge X$ is also, since $(a,b)\bigcap X$ implies $(a,b)\in x$ for every
	$x\in X$.  The symmetry of the members of $x$ now yield the
	corresponding $(b,a)\in\bigcap X$.

	In a somewhat similar fashion, an arbitrary intersection of transitive
	(binary) relations will again yield a transitive relation, all from first
	principles.

	Next we consider the case for $\bigvee X$.  Note that $I\subseteq \bigvee X$
	since $X$ is non-empty and $I\subseteq x$ for each $x\in X$.  Therefore
	$\bigvee X$ is reflexive.  Note now that $\bigvee X$ is an intersection of
	symmetric (binary) relations and thus must itself also be symmetric.  In a
	similar fashion, since $\bigvee X$ is an intersection of transtive relations
	it must itself also be transitive.

	All that remains is to show that $\bigvee X=\trclos(\bigcup X)$.  Clearly,
	since $\bigvee X$ is transitive, we already have $\trclos(\bigcup
	X)\subseteq\bigvee X$.  Noting then that $\trclos(\bigcup X)\in X$, we may
	conclude $\bigvee X\leq\trclos(\bigcup X)$, as required.
\end{proof}

The sublattice of $\con\alpha$ generated by the set of $0$\textit{-definable}
congruences on $\alpha$ will be denoted as $\defcon{\alpha}$.  It should be clear
that, in general, $\defcon{\alpha}$ will not be complete as it might lack
certain \textit{infinite} joins or meets.

Ideally, one would want every condensation of a member of $\con\alpha$ to again
belong to $\con\alpha$.  Up to isomorphism this is in fact the case, as
suggested by the following ``homomorphism'' theorem.  Take note that we use
$\pi_\sim$ to denote the splitting
$\pi_\sim\colon\alpha\to\faktor{\alpha}{\sim}$ induced by the congruence $\sim$
on $\alpha$.

\begin{prp}[Homomorphisms]
	 Suppose $f\colon\alpha\to\beta$ is a splitting. There exists a unique
	 congruence relation $\sim$ on $\alpha$ and a unique order isomorphism
	 $\iota\colon\faktor{\alpha}{\sim}\to\beta$ that makes the following diagram
	 commute:
	\begin{equation}
		\begin{tikzcd}
			\alpha \arrow[r, "\pi_\sim",rightarrow]&   \faktor{\alpha}{\sim}\\
			 &   \beta\arrow[from=u,"\iota", dashrightarrow]
			 \arrow[from=ul,"f"']
		\end{tikzcd}
	\end{equation}
\end{prp}
\begin{proof}
	The desired congruence relations is obtained in a manner familiar from
	algebra:  identify elements in $\alpha$ whenever they have the same image
	under $f$.  Borrowing a definition from universal algebra and lattice
	theory, we let $\sim$ be a binary relation such that:
	\begin{equation}
		\sim{}=\kernel f=\setbuild{(a,b)\in\domain{2}\alpha}{f(a)=f(b)}.
	\end{equation}

	It is readily verified that the required isomorphism is given by
	$\iota([a])=f(a)$, for each $a\in\alpha$.  That $\iota$ is well-defined
	follows simply from the definition of ${\sim}=\kernel f$.

	We are now required to establish the uniqueness of $\sim$ and $\iota$ in
	their respective roles.  Suppose $\sim_0$ is a congruence on $\alpha$ and
	there exists a unique isomorphism $\iota_0$ that makes
	\begin{equation}
		\begin{tikzcd}
			\alpha\arrow[r, "\pi_{\sim_0}",rightarrow]&
			\faktor{\alpha}{\sim_0}\\
			 &   \beta\arrow[from=u,"\iota", dashrightarrow]
			 \arrow[from=ul,"f"']
		\end{tikzcd}
	\end{equation}
	commute.  This then clearly implies that ${\sim_0}=\ker f={\sim}$ and,
	consequently, also $\iota_0=\iota$.
\end{proof}


\section{Hausdorff's characterisation of the countable scattered linear orders}

\begin{prp}[Operations on $\scattered$]\label{prp:OpScattered}
        The following properties hold:
        \begin{enumerate}
			\item   If $I\in\scattered$ and $\alpha_i\in\scattered$ for each
				$i\in I$ then $\sum_{i\in I}\alpha_i\in\scattered$;
			\item   if $\alpha,\beta\in\scattered$ then
				$\alpha+\beta,\alpha\cdot\beta\in\scattered$.
        \end{enumerate}
\end{prp}
\begin{proof}
		(1):  Suppose, by way of contradiction, that $\delta\subseteq\sum_{i\in
		I}\alpha_i$ is countable and dense.  Define
		$\delta_i=\delta\cap\alpha_i$, for each $i\in I$.

		Note that, since $\delta$ is dense, we cannot have
		$1<\card{\delta_i}<\aleph_0$ for any $i\in I$.  Therefore, each
		$\delta_i$ is either infinite or has at most one element.

		If $\delta_i$ is finite for each $i\in I$ then $\delta\preceq I$,
		contradicting the scatteredness of $I$.  Consequently, the must exist
		some $j\in I$ such that $\delta_j$ is infinite.  However, since
		$\delta_j\subseteq\alpha_j$, this implies that $\alpha_j$ has a
		countable dense subset --- the desired contradiction.

		(2):  Choosing $I=\two$, $\alpha_0=\alpha$ and $\alpha_1=\beta$ in (1)
		yields $\alpha+\beta\in\scattered$.  Instead, choosing $I=\beta$ and
		$\alpha_i=\alpha$ for each $i\in\beta$ we get
		$\alpha\cdot\beta\in\scattered$, as required.
\end{proof}

The result above gives us the mantra: \textit{scattered sums} of scattered
linear orders are themselves scattered.  In particular, products and (finite)
sums of scattered linear orders are also scattered.

This observation suggests the possibility that one could generate the class of
all scattered linear orders recursively as sums of previously constructed
(scattered) linear orders.  We do this for the countable case but the approach
readily extends to the class of \text{all} scattered linear orders.

\begin{dfn}[The class $\VD$]
		By way of transfinite recursion define, for each ordinal
		$\gamma<\omega_1$, the class of linear orders
		$\VD_{\gamma}\subseteq\linear$ to be the smallest class which is
		\textit{closed under isomorphisms} and satisfies:
        \begin{enumerate}
            \item   $\zero,\one\in\VD_0$;
			\item   if $\alpha_i\in\bigcup_{\beta<\gamma}\VD_{\beta}$, for each
				$i\in \zeta$, then $\sum_{i\in\zeta}\alpha_i\in\VD$.
        \end{enumerate}
		The class $\VD=\bigcup_{\gamma<\omega_1}\VD_\gamma$ is called the class
		of $\textbf{(countable) very discrete}$ linear orders.
\end{dfn}

\begin{dfn}[$\VD$-rank]
		If $\alpha$ is a very discrete linear order then its
		$\bm{\mathcal{VD}}$\textbf{-rank} $\vdrank(\alpha)$ is the least ordinal
		$\beta$ such that $\alpha\in\VD_{\beta}$.
\end{dfn}

\begin{lem}\label{prp:vdsct}
	Every very discrete linear order is scattered.  That is to say
	$\VD\subseteq\scattered$.
\end{lem}
\begin{proof}
	We argue by transfinite induction on $\gamma$ that
	$\VD_\gamma\subseteq\scattered$.  Finite linear orders are (trivially)
	scattered and thus $\VD_0\subseteq\scattered$.

	Fix $\delta<\omega_1$ and assume $\VD_\gamma\subseteq\scattered$, for each
	$\gamma<\delta$.  By definition, if $\vdrank(\alpha)=\delta$ then there
	exists, for each $i\in\zeta$, an $\alpha_i\in\VD_{\gamma_i}$, for some
	ordinal
	$\gamma_i<\delta$, such that
	\begin{equation}
		\alpha=\sum_{i\in\zeta}\alpha_i.
	\end{equation}

	It follows from Proposition \ref{prp:OpScattered}, as well as the
	inductive hypothesis, that $\alpha$ is a scattered sum of scattered linear
	orders.  Therefore we must have $\alpha\in\scattered$, as requred.
\end{proof}

\iffalse\begin{lem}
        If $\alpha\in\VD$ and $\beta\subseteq\alpha$ is convex in $\alpha$ then $\beta\in\VD$.
\end{lem}

\begin{proof}
	 We argue by induction on  $\vdrank(\alpha)$.  The case $\vdrank(\alpha)=0$ is trivial so suppose that $\beta\subseteq\alpha$ and $\beta$ convex in $\alpha$ imply $\beta\in\VD$ whenever $\vdrank(\alpha)<\gamma<\omega_1$.  Suppose now that $\vdrank(\alpha)=\gamma$ and $\beta\subseteq\alpha$ is convex in $\alpha$.  By definition, for each $i\in\zeta$ there must exist $\alpha_i\in\VD_{\gamma_i}$, for some $\gamma_i<\gamma$, such that
	\begin{equation}
	    \alpha=\sum_{i\in\zeta}\alpha_i.
	\end{equation}
	Now define $\beta_i=\beta\cap\alpha_i$ for each $i\in I$.  For each $i\in\zeta$, since $\beta_i$ is convex in $\alpha_i$ and $\vdrank(\alpha_i)<\gamma$, it follows that $\beta_i\in\VD$.  However, note that
	\begin{equation}
	    \beta=\sum_{i\in\zeta}\beta_i.
	\end{equation}
	Therefore, by definition of $\VD$ it follows that $\beta\in\VD$.
\end{proof}\fi

    \begin{lem}\label{prp:sctvd}
        If $\alpha\in\scattered$ and $\card{\alpha}\leq\aleph_0$ then $\alpha$ is very discrete.
    \end{lem}

\begin{proof}
    Suppose $\alpha\in\scattered$ is countable and define a relation $R$ on $\domain{}\alpha$ such that, for every $a,b\in\alpha$, we have $aRb$ whenever $a\leq b$ and $[a,b]$ is very discrete.  Note that $R$ is transitive and thus, by Lemma (\ref{lem:IndCong}), it induces a congruence relation $\sim$ on $\domain\alpha$.

        Define the condensation $\beta=\faktor{\alpha}{\sim}$.  If $\card{\beta}=1$ then there is nothing to prove so we may assume $\card{\beta}>1$.  By definition of $\VD$, it now suffices to show that $\beta\embed\zeta$ since every $\alpha\in\beta$ is very discrete simply by definition of $\sim$.  We will suppose to the contrary that $\beta\not\embed\zeta$.  Consequently, $\beta$ must be infinite.
\end{proof}

    	\iffalse\begin{prp}\label{prp:splitfix}
		Suppose $\Lambda$ is the lattice of condensations of the linear order $\alpha$.  Define $S$ to be the functor that takes each factor map $f\colon\alpha\to\beta$ to the splitting $S(f)\colon\alpha\to\faktor{\alpha}{\sim_\beta}$, for some congruence $\sim_\beta$ on $\alpha$, such that the unique isomorphism $\iota_\beta\colon\faktor{\alpha}{\sim_\beta}\to\beta$ makes the diagram
		\begin{equation}
			\begin{tikzcd}
				\alpha\arrow[rr,"S(f)"]&&\faktor{\alpha}{\sim_\beta}\arrow[dd,"\iota_\beta",dashrightarrow]\\
				\\
						       &&\beta\arrow[from=lluu,"f"]
			\end{tikzcd}
		\end{equation}
		commute.  If $\beta\in\Lambda$ is a minimal fixed point of $S^\prime=S\fsplit[-]$, and $\beta$ is not trivially dense then $\beta\in\Dense$.
		\begin{enumerate}
			.
		\end{enumerate}
	\end{prp}\fi

\begin{thm}[Hausdorff's Theorem]
	A linear order $\alpha$ is countable and  scattered iff it is very discrete.
\end{thm}
\begin{proof}

	\forward\	This is Lemma (\ref{prp:sctvd}).

	\backward\	This is Proposition (\ref{prp:vdsct}).
\end{proof}


\section{The first-order theory of scattered linear orders}


\begin{prp}[Condensations and scatteredness]
        A linear order $\alpha$ is scattered iff there exists no condensation $\beta$ of $\alpha$ such that $\beta$ is dense.
    \end{prp}

    \begin{dfn}[Definably scattered]
	    A linear order $\alpha$ is \textbf{definably scattered} whenever, for every $n\in\nats$, there exists no $\bar{a}\in\domain{n}\alpha$ and no congruence formula $\varphi(x,y,\bar{z})$ of $\alpha$ such that, if $\sim$ is the congruence defined by $\varphi(x,y,\bar{a})$, then $\faktor{\alpha}{\sim}$ is a dense linear order.
    \end{dfn}

\begin{prp}
        If $\alpha$ is some linear order then the following are equivalent:
        \begin{enumerate}
            \item   $\alpha$ is definably scattered,
	    \item   there exists no dense linear order $\delta$ and no $n\in\nats$ such that, for some $\bar{b}\in\domain{n}\delta$ and $\bar{a}\in\domain{n}\alpha$, $(\delta,\bar{b})$ is interpretable in $(\alpha,\bar{a})$,
            \item   the lattice $\defcon\alpha$ is atomic.
        \end{enumerate}
\end{prp}
    	\begin{proof}

	\end{proof}

\begin{prp}\label{prp:dcform}
	For each formula $\varphi=\varphi(x,y,\bar{z})$, with $\bar{z}$ possibly empty, define the formula $\epsilon_\varphi(\bar{z})$ to be the conjunction of the following formulas:
	\begin{enumerate}
		\item	$\forall x\varphi(x,x,\bar{z})$,\hfill(reflexivity)
		\item	$\forall x\forall y(\varphi(x,y,\bar{z})\rightarrow\varphi(y,x,\bar{z}))$,\hfill (symmetry)
		\item 	$\forall x\forall y\forall w(\varphi(x,y,\bar{z})\wedge\varphi(y,w,\bar{z})\rightarrow\varphi(x,w,\bar{z}))$.\hfill (transitivity)
		\item 	$\forall x_0\forall x_1\forall y_0\forall y_1(x_0<x_1\wedge\neg\varphi(x_0,x_1,\bar{z})\wedge\varphi(x_0,y_0,\bar{z})\wedge\varphi(x_1,y_1,\bar{z})\rightarrow y_0<y_1)$.\phantom{}\hfill(compatibility)
	\end{enumerate}
	For every finite tuple $\bar{a}$ of $\alpha$ (of the same length as $\bar{z}$), it holds that $\alpha\models\epsilon_\varphi(\bar{a})$ iff $\varphi(x,y,\bar{a})$ defines a congruence of $\alpha$.
\end{prp}

\begin{thm}
	For every countable, definably scattered linear order $\alpha$ and every  $n\in\nats$ there exists a $\beta_n\in\scattered$ such that $\alpha\nequiv{n}\beta_n$.
\end{thm}
\begin{proof}
	Suppose $\alpha\models\Th(\scattered)$ is countable and fix any $n\in\nats$.  Now define a binary relation $R$ on $\domain{}\alpha$ such that $aRb$ iff there exists a $\beta\in\scattered$ such that $(a,b)\nequiv{n}\beta$ in the language of $k$-coloured linear orders (for some $k\in\nats$).  Since $R$ is clearly transitive, it induces a cogruence $\sim$ on $\alpha$.  Moreover, $\sim$ is defined by the $L_{\omega_1\omega}$ formula $\varphi(x,y)$ given by:
	\begin{equation}
		x=;y\vee(x<y\wedge\bigvee_{\beta\in\scattered}(\cha{\beta}{n})^{(x,y)})\vee(y<x\wedge\bigvee_{\beta\in\scattered}(\cha{\beta}{n})^{(y,x)}).
	\end{equation}
	Note that, since our language is finitary, we need only choose finitely many characterstics $\alpha_i\in\scattered$ such that $\alpha_0,\dotsc,\alpha_{m-1}$ is a traversal of the $n$-equivalence classes of $\scattered$. Thus we may identify $\varphi$ with its first-order equivalent which, has finite length.

	\begin{claim}\label{clm:LL1}
		Each $\gamma\in\faktor{\alpha}{\sim}$ has a scattered $n$-equivalent.
	\end{claim}
	\begin{proof}
		Choose any $\gamma\in\faktor{\alpha}{\sim}$.  If there exists a least element $a_0\in\gamma$ then, since $\alpha$ and thus $\gamma$ is countable, there exists a cofinal sequence $\family{a_i}{i<\omega}$.  By Ramsey's Theorem, there then exists a homogeneous subsequence $\family{a^\prime_i}{i<\omega}\subseteq\family{a^\prime_i}{i<\omega}$ for the colouring $h$ that sends intervals $(a_i,a_j)$, with $i<j$, to a characteristic among $\alpha_0,\dotsc,\alpha_{m-1}$.  Consequently, $\gamma$ is an $\omega$-sum
		\begin{equation}
			\gamma=\sum_{i<\omega}[a^\prime_i,a^\prime_{i+1})
		\end{equation}
		where all of the summands are $n$-equivlent to the same $\one+\alpha_{i_0}$ for some $i_0\in\set{0,\dotsc,m-1}$ so that, since summation preserves $n$-equivalence, it follows that $\gamma$ is $n$-equivalent to the scattered linear order:
		\begin{equation}
			\gamma^\prime=(\one+\alpha_{i_0})\cdot\omega.
		\end{equation}
		Since $\gamma^{<a^\prime_0}=[a_0,a^\prime_0))$, by defintion of $\sim$, has an $n$-equivalent (($\gamma^\prime_0$ say) it follows that $\beta_n=\gamma_0+\gamma^\prime$ is the desired scatterred $n$-equivalent of $\gamma$.

		A similar argument can be obtained for when $\gamma$ has a greatest element by dualising the argument above (i.e. respectively swap around $<$ and $>$ as well as $\leq$ and $\geq$.  In te case where $\gamma$ has neither a largest nor a smallest element choose some fixed $a\in\gamma$ and apply a similar argument in the respective suborderings $\gamma^{\geq a}$ and $\gamma^{<a}$.  The sum of their $n$-equivalents will then yield a scattered $n$-equivalent of the equivalence class $\gamma$, as required.\noqed
	\end{proof}

	\begin{claim}
		The equivalence classes are dense under the induced order.
	\end{claim}
	\begin{proof}
		Suppose $\epsilon_0,\epsilon_1\in\faktor{\alpha}{\sim}$ satisfy the property $\epsilon_0<\epsilon_1$ and and then assume, by way of contradiction, that for no $c\in\alpha$ does it hold that:
		\begin{equation}
			\epsilon_0<c<\epsilon_1.
		\end{equation}
		However, since $n$-equivalence is preserved under sums it follows that $\epsilon_0+\epsilon_1$ is $n$-equivalent to some scattered linear order which implies tha by choosing an $a_0\in\epsilon_1$ and an $a_1\in\epsilon_1$ that $(a_0,a_1)$ posess a scattered $n$-equivalent.  This is the desired contradiction as then, by defition, it would have to hold that $a_0\sim a_1$: a contradiction.\noqed
	\end{proof}
	Since $\faktor{\alpha}{\sim}$ is dense and $\alpha$ is definably scattered, the only possible conclusion is that $\faktor{\alpha}{\sim}\cong\one$.  From claim \ref{clm:LL1} the result then follows.
\end{proof}

\begin{prp}\label{prp:rescat}
	For each formula $\varphi(x,y,\bar{z})$ define $\theta_\varphi(\bar{z})$ to be the formula
	\begin{equation}
		\forall x\forall y\big((x<y\wedge\neg\varphi(x,y,\bar{z}))\rightarrow\exists w(x<w<y\wedge\neg\varphi(x,w,\bar{z})\wedge\neg\varphi(w,y,\bar{z}))\big)
	\end{equation}
	which formalises the statement: ``the condensation which is induced by the congruence, defined by the formula $\varphi$, is dense'' and let each $\epsilon_\varphi(\bar{z})$ be defined as in Proposition \ref{prp:dcform}.  It then holds that:
	\begin{enumerate}
		\item the theory $\Sigma=\set{\axmlin}\cup\setbuild{\forall\bar{z}(\epsilon_\varphi(\bar{z})\rightarrow\neg\theta_\varphi(\bar{z}))}{\varphi=\varphi(x,y,\bar{z})}$ axiomatises $\Th(\scattered)$, i.e.\ $\dcl{\Sigma}=\Th(\scattered)$,

		\item $\Sigma$ is recursively enumerable.
	\end{enumerate}
	\begin{proof}
		\begin{enumerate}[nosep]
			\item	$\alpha\models\Sigma$ iff $\alpha$ is definably scattered iff $\alpha\models\Th(\scattered)$.
			\item	Since our signature is finite it follows that the set of formulas in our language can be enumerated (via a G\"odel numbering $\godel{-}\colon L\to\nats$).  We can then enumerate $\Sigma$ by declaring a map acting on $\Sigma$ such that $\axmlin\mapsto 0$ and, for every tuple $\bar{z}$ of variables,
			\begin{equation}
				\forall \bar{z}(\epsilon_\varphi(\bar{z})\rightarrow\theta_\varphi(\bar{z})))\mapsto 2^{\godel{\varphi(x,y,\bar{z})}}.
			\end{equation}
		\end{enumerate}
	\end{proof}
\end{prp}


\section{Theorems of L\"auchli and Leonard}

\begin{dfn}[The class $\Mzero$]
	The class $\Mzero$ is the smallest class of linear orders which satisfies the following:
	\begin{enumerate}
		\item	$\zero,\one\in\Mzero$,
		\item	if $\alpha,\beta\in\Mzero$ then $\alpha+\beta\in\Mzero$,
		\item	if $\alpha\in\Mzero$ then $\alpha\cdot\omega,\alpha\cdot\dual{\omega}\in\Mzero$,
	\end{enumerate}
\end{dfn}

\begin{prp}\label{prp:M0sum}
	Let $n\in\nats$ and suppose for each $i<\omega$ it holds that $\alpha_i\in\Mzero$ then there exists a $\beta\in\Mzero$ such that
	\begin{equation}
		\sum_{i<\omega}\alpha_i\nequiv{n}\beta.
	\end{equation}
\end{prp}
\begin{proof}
	For each $i,j<\omega$, if $i<j$ then define $\alpha_{ij}=\alpha_{i+1}+\alpha_{i+2}+\dotsb+\alpha_j$ and note that $\alpha_{ij}\in\Mzero$ since $\Mzero$ is closed under finite sums.  Choose $S=\set{\beta_0,\dotsc,\beta_{k-1}}\subseteq\Mzero$ to be an $n$-spectrum for $\Mzero$.  Consequently, there exists a colouring $h$ of $\omega$ whose codomain is $\set{0,\dotsc,k-1}$ such that $h(i,j)=\ell$ iff $\alpha_{ij}\nequiv{n}\beta_\ell$.

	It now follows from Ramsey's theorem that there exists a homogeneous sequence $\family{i_j}{j<\omega}$ for $h$.  By definition, there exists a natural $m$ such that $0\leq m<k$ and $\alpha_{i_ji_{j+1}}\nequiv{n}\beta_m$, for every $j<\omega$.  Note, from Lemma \ref{lem:fvsum}, it follows that
	\begin{align}
		\sum_{i<\omega}\alpha_i&\cong\alpha_0+\alpha_1+\dotsb+\alpha_{i_0}+\sum_{j<\omega}\alpha_{i_ji_{j+1}}\\
		&\nequiv{n}\alpha_0+\dotsb+\alpha_{i_0}+\beta_m\cdot\omega,
	\end{align}
	Since $\Mzero$ is closed under finite sums, we may thus conclude that if $\beta=\alpha_0+\dotsb+\alpha_{i_0}+\beta_m\cdot\omega$ then $\sum_{i<\omega}\alpha_i\nequiv{n}\beta$, as required.
\end{proof}

\begin{prp}\label{prp:M0sumop}
	Let $n\in\nats$ and suppose for each $i\in\dual{\omega}$ it holds that $\alpha_i\in\Mzero$ then there exists a $\beta\in\Mzero$ such that
	\begin{equation}
		\sum_{i\in\dual{\omega}}\alpha_i\nequiv{n}\beta.
	\end{equation}
\end{prp}
\begin{proof}
	Dualise the argument in the proof of Proposition \ref{prp:M0sum}.
\end{proof}

\begin{prp}\label{prp:M0sumint}
	Let $n\in\nats$ and suppose for each $i\in\ints$ it holds that $\alpha_i\in\Mzero$ then there exists a $\beta\in\Mzero$ such that
	\begin{equation}
		\sum_{i\in\zeta}\alpha_i\nequiv{n}\beta.
	\end{equation}
\end{prp}
\begin{proof}
	Choose any fixed $a_0\in\alpha_0$ and define $\alpha=\sum_{i\in\zeta}\alpha_i$ apply Propositions \ref{prp:M0sum} and \ref{prp:M0sum} respectively to $\alpha^{>a_0}$ and $\alpha^{<a_0}$ to obtain their (respective $n$-equivelants $\beta^+\in\Mzero$ and $\beta^-\in\Mzero$.  It then follows, since $\Mzero$ is closed under finite sums, that if $\beta=\beta^-+\one+\beta^+$ then $\alpha\nequiv{n}\beta$, as required.
\end{proof}

\begin{thm}[L\"auchli and Leonard]
	For every countable $\alpha\in\scattered$, and each $n\in\nats$, there exists a $\beta_n\in\Mzero$ such that $\alpha\nequiv{n}\beta_n$.
\end{thm}
\begin{proof}
	Suppose $\alpha\in\scattered$ is countable.  From Hausdorff's theorem it follows that $\alpha$ is very discrete so we may argue by induction on the $\VD$-rank of $\alpha$. Since $\zero,\one\in\Mzero$, the result holds when $\vdrank(\alpha)=0$.

	Assume now there exists an ordinal $\gamma<\omega_1$ such that if $\vdrank(\alpha)<\gamma$ then, for every $n\in\nats$, there exists a $\beta_n\in\Mzero$ such that $\alpha\nequiv{n}\beta_n$.  Now fix any $n\in\nats$ and  suppose $\vdrank(\alpha)=\gamma$ then, by definition of $\VD$, there exists an $\alpha_i\in\VD$ such that $\vdrank(\alpha_i)<\gamma$, for each $i\in\ints$, such that
	\begin{equation}
		\alpha=\sum_{i\in\zeta}\alpha_i.
	\end{equation}
	By the induction hypothesis there exists, for each $i\in\ints$, some $\alpha^\prime_i\in\Mzero$ such that $\alpha_i\nequiv{n}\alpha_i^\prime$.  It now follows from Lemma \ref{lem:fvsum}, if we choose $\alpha^\prime=\sum_{i\in\zeta}\alpha_i^\prime$, that $\alpha\nequiv{n}\alpha^\prime$.  According to Proposition \ref{prp:M0sumint} there exists a $\beta\in\Mzero$ such that $\alpha^\prime\nequiv{n}\beta$ and, therefore, $\alpha\nequiv{n}\beta$, as required.
\end{proof}

\begin{prp}
	Suppose $\alpha$ and $\beta$ are linear orders and consider the families $\family{\mathfrak{A}_i}{i\in\alpha}$ and $\family{\mathfrak{B}_j}{j\in\beta}$ such that, for each $i\in\alpha$ and $j\in\beta$, $\mathfrak{A}_i,\mathfrak{B}_j$ are regular $L^\prime$-expansions of linear orders.  If $n\in\nats$, $\alpha\nequiv{n}\beta$ and, for each $i\in\alpha$ and $j\in\beta$,
	\begin{equation}
		\mathfrak{A}_i\nequiv{n}\mathfrak{B}_j
	\end{equation}
	then it necessarily follows that
	\begin{equation}
		\sum_{i\in\alpha}\mathfrak{A}_i\nequiv{n}\sum_{j\in\beta}\mathfrak{B}_j.
	\end{equation}
\end{prp}
\begin{proof}
	We describe a winning strategy for $\Right$ in the $n$-round game
	\begin{equation}
		\EF_n\left(\sum_{i\in\alpha}\mathfrak{A}_i,\sum_{j\in\beta}\mathfrak{B}_j\right).
	\end{equation}
	A move $(a,i)$ in $\sum_{i\in\alpha}\mathfrak{A}_i$ by $\Left$ is countered by $\Right$ by first employing his winning strategy in the game $\EF_n(\alpha,\beta)$ to obtain some $j\in\beta$ followed by an application of his winning strategy in the game $\EF_n(\mathfrak{A}_i,\mathfrak{B}_j)$ in order to acquire a $b\in B_j$.  The countermove by $\Right$ is then $(b,j)$.  A similar choice is made in the case that $\Left$ instead plays some $(b^\prime,j^\prime)$ in the opposing sum.
\end{proof}

\begin{lem}\label{lem:rescat}
	The following set is recursively enumerable:
	\begin{equation}
		R\coloneqq\setbuild{(\alpha,\sigma)}{\alpha\in\Mzero\text{ and }\alpha\models\sigma}.
	\end{equation}
\end{lem}
\begin{proof}
	Recursively on $\Mzero$, for each $\alpha\in\Mzero$, by adding finitely many (new) relation symbols to $L$ we shall define a language $L_\alpha\supseteq L$ and first-order theories $S_\alpha$, $T_\alpha$ such that $T_\alpha\supseteq S_\alpha$ is a recursively enumerable complete theory.  The role of each $S_\alpha$ will be that of a finitely axiomatisable theory whose models interpret a finitely axiomatisable linear order.

	If $\alpha=\zero$ then we choose $L_\alpha$ to be the language of linear orders and we define $T_\alpha=S_\alpha$ to consist only of the sentence $\neg\exists x(x=x)$.  While, when $\alpha=\one$, we choose the same $L_\alpha$ as before and define $T_\alpha=S_\alpha$ to consist only of the sentence $\axmlin\wedge\exists x\forall y(x=y)$.

	In both cases above, $T_\alpha$ is recursively enumerable (since it is finite) as well as complete:  $T_\alpha$ determines its models up to isomorphism.

	Assume now that $L_\alpha$, $T_\alpha$ and $S_\alpha$ have been defined for $\alpha\in\set{\alpha_0,\alpha_1}$ and that $T_{\alpha_0},T_{\alpha_1}$ are recursively enumerable (complete) theories.  Suppose that $\alpha=\alpha_0+\alpha_1$ and expand $L$ by adding the unary relation symbols $r_0$ and $r_1$ in order to obtain $L_\alpha$.  These new symbols will serve to ``identify'' the linear orders $\alpha_0$ and $\alpha_1$, respectively, within $\alpha$.  Let $S_\alpha$ consist of exactly the $L_\alpha$-sentences $\axmlin$, $\forall x(r_0(x)\vee r_1(x))$ and $\forall x\forall y(r_0(x)\wedge r_1(y)\rightarrow x<y)$ while also choosing
	\begin{equation}
		T_\alpha=S_\alpha\cup\setbuild{\sigma^{r_0(v)}}{\alpha_0\models\sigma}\cup\setbuild{\sigma^{r_1(v)}}{\alpha_1\models\sigma}.
	\end{equation}

	We now argue that $T_\alpha$ is complete.  Note that if $\mathfrak{M},\mathfrak{N}\models T_\alpha$ then, by definition, their respective order types, $\beta$ and $\delta$, can be decomposed as $\beta=\beta_0+\beta_1$ and $\delta=\delta_0+\delta_1$ such that,
	\begin{equation}
		\beta_i,\delta_i\models\Th(\alpha_i), \quad\text{for }i=0,1.
	\end{equation}
	Therefore $\beta_0\equiv\delta_0$ and $\beta_1\equiv\delta_1$ so that, for each $n\in\nats$, $\Right$ has the (respective) winning strategies $s_n$ and $t_n$ for the games $\EF_n(\beta_0,\delta_0)$ and $\EF_n(\beta_1,\delta_1)$.

	The winning strategy for $\Right$ in the game $\EF_n(\mathfrak{M},\mathfrak{N})$ is laid out as follows.  A move in the $\beta_0$-part of $\mathfrak{M}$ is countered using the strategy $s_n$, as if playing the game $\EF_n(\beta_0,\delta_0)$ while a move in the $\beta_1$-part is similarly countered but using the strategy $t_n$.  If instead $\Left$ plays a member of $\mathfrak{N}$ then a similar methodology may be employed.

	If $\alpha=\alpha_0\cdot\omega$ then we obtain $L_\alpha$ from $L$ by adding a single binary relation symbol $\theta$.  Choose $\Gamma$ to be the (unique) interpretation of $L$ in $L_\alpha$ such that $\Gamma(x<y)$ is the formula $x<y\wedge\neg\theta(x,y)$.

	Let $\Sigma$ consist of the (finitely many) $L_\alpha$-sentences which, together, express the assertion ``$\theta$ is an equivalence relation'' as well as the sentence
	\begin{equation}
		\forall x_0\forall x_1\forall y_0\forall y_1( x_0<y_0\wedge\neg\theta(x_0,y_0)\wedge\theta(x_1,x_0)\wedge\theta(y_1,y_0)\rightarrow x_1<y_1).
	\end{equation}
	so that, in fact, $\theta$ is a congruence and not simply a mere equivalence relation.

	Define $S_\alpha$ to be a finite axiomatisation of the $L_\alpha$-theory
	\begin{equation}
		\Gamma\left[\Th(\omega)\right]\cup\Sigma\cup\set{\axmlin}
	\end{equation}
	and let $T_\alpha=S_\alpha\cup\setbuild{\forall w\sigma^{\theta(w,v)}}{\alpha_0\models\sigma}$.  The latter set of sentences expresses the sentiment that each $\theta$-equivalence class is elementarily equivalent to $\alpha_0$.

	Note, it follows from the induction hypothesis that $T_\alpha$ is recursively enumerable.  The burden is now on us to show that the $L_\alpha$-theory $T_\alpha$ is complete.  Consider any $\mathfrak{M},\mathfrak{N}\models T_\alpha$ then the respective order types, $\beta$ and $\delta$, of $\mathfrak{M}$ and $\mathfrak{N}$ can be decomposed as
	\begin{equation}
		\beta=\beta_0\cdot\gamma_0\qquad\text{and}\qquad\delta=\delta_0\cdot\gamma_1,
	\end{equation}
	such that $\gamma_0\equiv\omega\equiv\gamma_1$ and $\beta_0,\delta_0\models\Th(\alpha_0)$.

	Fix any $n\in\nats$ then we describe a winning strategy for $\Right$ in the game $\EF_n(\mathfrak{M},\mathfrak{N})$.  If $\Left$ plays the move $(a,i)\in\domain{}\beta\times\domain{}\gamma_0$ then $\Right$ chooses a $j\in\gamma_1$ using his winning strategy in the game $\EF_n(\gamma_0,\gamma_1)$.  This is then followed by a choice of $b\in\delta_0$ using his winning strategy in the game $\EF(\beta_0,\delta_0)$.  This determines the countermove $(b,j)$.  If $\Left$ instead plays a move in the opposite structure then a countermove be obtained by $\Right$ in a similar fashion.

	The case when $\alpha=\alpha_0\cdot\dual{\omega}$ is approached in a similar manner: let $\theta$, $L_\alpha$, $\Gamma$ and $\Sigma$ be defined as before.  Choose $S_\alpha$ to be a finite axiomatisation of the theory
	\begin{equation}
		\Gamma[\Th(\dual{\omega})]\cup\Sigma\cup\set{\axmlin}
	\end{equation}
	and define $T_\alpha$ as in the previous case.  It again follows that $T_\alpha$ is recursively enumerable and a similar argument as before will suffice to show that $T_\alpha$ is complete.

	All that remains is that a mechanical listing of the set
	\begin{equation}
		R^\prime\coloneqq\setbuild{(\alpha,\sigma)}{\alpha\in\Mzero\text{ and }\sigma\text{ is an }L_\alpha\text{-sentence such that }T_\alpha\models\sigma}
	\end{equation}
	need be given.  Once this procedure has been established it will follow that $R$ is also recursively enumerable since, for any $\alpha\in\Mzero$ and $\sigma\in L_\alpha$, the problem of whether or not $\sigma$ is a member of $L$ is decidable.

	The list starts with all pairs of the form $(\alpha,\sigma)$ such that $\alpha\in\set{\zero,\one}$ and $\sigma\in S_\alpha$, noting that the aforementioned pairs are finite in number.  Now assume the entries $(\alpha_i,\sigma_i)$, for $i=0,\dotsc,n_0-1$, have been listed.  Continue the list sequentially with entries $(\alpha_i,\sigma_i)$, for $i\geq n_0$, as follows:
	\begin{enumerate}
		\item 	list all pairs of the form $(\alpha_i,\sigma)$ such that $\sigma\in S_{\alpha_i}$ and $i<n_0$ and let $n_1$ denote the resulting number of entries in the list;
		\item 	list all pairs of the form $(\alpha_i, \sigma)$, for $i<n_1$, where $\sigma$ is the direct consequence of an inference rule from $\sigma_{i_0},\dotsc,\sigma_{i_{k-1}}$ and, for each $j<k$, $(\alpha_i,\sigma_{i_j})$ has already appeared in the list and let $n_2$ denote the resulting number of entries;
		\item	list all pairs of the form $(\alpha,\sigma)$ where, for $i,j<n_2$, $\alpha=\alpha_i+\alpha_j$ and $\sigma=\sigma_i^{r_0(v)}\wedge\sigma_j^{r_1(v)}$ and let $n_3$ denote the resulting number of entries in the list;
		\item	list all pairs of the form $(\alpha_i\cdot\omega,\sigma)$, for $i<n_3$, such that $\sigma=\forall w\sigma_i^{\theta(w,v)}$ and and let $n_4$ denote the resulting number of entries in the list;
		\item	list all pairs of the form $(\alpha_i\cdot\dual{\omega},\sigma)$, for $i<n_4$, such that $\sigma=\forall w\sigma_i^{\theta(w,v)}$ and and redefine $n_0$ to be the resulting number of entries in the list;
		\item	repeat steps 1 to 5.
	\end{enumerate}

	By way of induction (on $\alpha\in\Mzero$) it can be shown that each member of the set $R^\prime$ does indeed appear as some entry in the above list.  Consequently, we may conclude that the set $R$ is in fact recursively enumerable, as required.
\end{proof}

\begin{thm}
	The theory $\Th(\scattered)$ is decidable.
\end{thm}
\begin{proof}
	From Proposition \ref{prp:rescat} and Lemma \ref{lem:rescat}, there exists machines $M_0$ and $M_1$ which, respectively, list the members of $\Th(\scattered)$ and the set of pairs
	\begin{equation}
		R=\setbuild{(\alpha,\sigma)}{\alpha\in\Mzero)\text{ and }\alpha\models\sigma}.
	\end{equation}

	Now consider a machine $M$ which, when given a sentence $\sigma$, alternately produces entries from each list until either $\sigma$ or, for some $\alpha\in\Mzero$, $(\alpha,\neg\sigma)$ appears in the list, at which point the machine halts and returns the value "True" if $\sigma$ appeared on the list as member of $\Th(\scattered)$ and returns "False" otherwise.

	We now need to argue that $M$ always halts irrespective of the choice of input sentence.  The case where $\sigma\in\Th(\scattered)$ is trivial so suppose that $\sigma\notin\Th(\scattered)$.  By definition, there must exist an $\alpha\in\scattered$ such that $\alpha\models\neg\sigma$.

	Choose $n=\qrank(\sigma)$ and recall it follows from the theorem of L\"auchli and Leonard that, for some $\beta_n\in\Mzero$, it holds that $\beta_n\nequiv{n}\alpha$ and thus $\beta_n\models\neg\sigma$.  Consequently, $(\beta_n,\neg\sigma)\in R$ and thus $(\beta_n,\neg\sigma)$ will appear in the list and $M$ will return "False".  Therefore, the machcine $M$ always halts --- as required.
\end{proof}

\bibliography{references}
