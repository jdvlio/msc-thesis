\bibliographystyle{amsalpha}

\chapter{Scattered linear orders}

The scattered linear orders are antithetical to the \textit{dense linear
orders}, the latter of which includes specimens such as $\eta$ and $\lambda$.
They are precisely the linear orders which do not embed any dense linear order
or, equivalently, do not embed a copy of the order type $\eta$ of the
rationals.

As we will see, the scattered linear orders can constructed in a systematic
fashion from the ground up by starting with the order types $\zero$ and $\one$
and iterating relatively ``simple'' operations.

From the perspective of first-order logic, the class $\scattered$ is also
well-behaved in that its theory $\Th(\scattered)$ is decidable.  A recursively
enumerable class $\Mzero\subseteq\scattered$ is presented in order to establish
this fact.  The argument proceeds as in \cite{RosLin}, owing to the techniques
of L\"auchli and Leonard.

\section{Splittings and congruence lattices}

The following definition is reminiscent of the analogous concept from algebra
and, in particular, lattice theory.
\begin{dfn}[Congruence]
	Suppose $\sim$ is an equivalence relation on the domain of the linear order
	$\alpha$.  We call $\sim$ a \textbf{congruence} on $\alpha$ whenever, for
	every $a,b\in\alpha$ such that $a<b$ and $a\nsim b$, the following
	holds:
	\begin{equation}
		a^\prime\sim a\text{ and }b^\prime\sim b\quad\implies\quad
		a^\prime<b^\prime,
	\end{equation}
	for all $a^\prime,b^\prime\in\alpha$.
\end{dfn}

Essentially, congruences are the equivalence relations that are compatible with
the order structure of $\alpha$.

Upon careful inspection, the above definition can be seen to be equivalent to
the definition of a lattice-theoretic congruence.  One need only view $\alpha$
as a lattice.  The lattice operations are derived in the usual fashion from the
order relation.

We will, ofcourse, now need a means of constructing various condensations along
our travels.  The following Lemma accomplished this feat by starting with an
arbitrary transitive relation.

\begin{lem}[Congruence construction]
	\label{lem:IndCong}
	Suppose $\alpha$ is a linear order and $R$ is a transitive binary relation
	on $\alpha$.  Now define a another binary relation $\sim$ on $\alpha$ such
	that, for $a,b\in\alpha$, we have $a\sim b$ whenever one of the following
	is satisfied:
	\begin{enumerate}
		\item   $a=b$,
		\item   $a<b$ and $aRb$,
		\item   $b<a$ and $bRa$.
	\end{enumerate}
	Under these assumptions, $\sim$ is a congruence on $\alpha$.
\end{lem}
\begin{proof}
	By definition, $\sim$ must be both reflexive and transitive.  To establish
	transitivity, fix any $a,b,c\in\domain{}\alpha$.

	Without loss of generality we may assume that $a\leq b\leq c$.  However,
	sine the cases $b=c$ and $a=b$ are trivial, it suffices to consider only the
	case $a<b<c$.  The result then follows by trasitivity of the linear order
	relation.
\end{proof}

\begin{dfn}[Induced congruence]
	The congruence $\sim$ in Lemma (\ref{lem:IndCong}) is referred to as the
	\textbf{congruence induced by $R$}.
\end{dfn}

\begin{dfn}[Splitting]
	If $\alpha,\beta$ are linear orders then a \textbf{splitting} is a
	surjective homomorphism $\pi\colon\alpha\rightarrow\beta$.
\end{dfn}

The definition of a splitting should be somewhat reminiscent of the
\textit{factor maps}, also called quotient maps, from algebra and topology.  Our
analog of a quotient structure is the following:

\begin{dfn}[Condensations]
	If $\sim$ is a congruence of the linear order $\alpha$ and
	$\pi_\sim\colon\alpha\to\faktor{\alpha}{\sim}$ is the (unique) splitting such
	that, for every $a,b\in\alpha$, it holds that:
	\begin{equation}
		a\sim b\iff\pi_\sim(a)=\pi_\sim(b),
	\end{equation}
	then we call $\pi_\sim$ the \textit{splitting of $\alpha$ induced
	by} $\sim$, refer to the quotient $\faktor{\alpha}{\sim}$ as a
	\textbf{condensation}.
\end{dfn}

We single out the following splitting in particular.  It will play a crucial
role in some of the results to come.

\begin{dfn}[Finite splitting]
	Suppose $\alpha$ is a linear order and $\sim$ is the congruence on $\alpha$
	induced by the relation $R$ defined by:
	\begin{equation} aRb\quad\iff\quad
		a<b\text{ and }[a,b]\text{ is finite}.
	\end{equation} For any linear order
	$\alpha$ we let $\fsplit[\alpha]$ denote the map $\fsplit[\alpha]\colon
	a\mapsto\faktor{a}{\sim}$ for $a\in\alpha$.  We refer to $\fsplit[\alpha]$
	as the $\textbf{finite splitting}$ on $\alpha$ and we omit the prescript
	whenever $\alpha$ is clear from the context.
\end{dfn}

In the result that follows, for any binary relation $R$, we use the notation
$\trclos(R)$ to denote the \textit{transitive closure} of $R$.  In other words,
$\trclos(R)$ is the smallest (w.r.t.\ set inclusion) transitive binary relation
containing $R$ as a subset.

\begin{prp}[Congruence lattices]
	\label{prp:conlat}
	Let $\con \alpha\subseteq\powerset{\domain{2}\alpha}$ be the set of all
	congruences on $\alpha$ and, for each $X\subseteq \con\alpha$, define:
	\begin{align}
		\bigwedge X &\coloneqq\bigcap X,\\ \bigvee X
					&\coloneqq\bigwedge\setbuild{a\in\con\alpha}{x\subseteq
					a,\forall x\in X}.
	\end{align}
	It then follows that $(\con\alpha,\vee,\wedge)$ is a complete lattice.
	Furthermore, for any $X\subseteq\con\alpha$, it holds that $\bigvee
	X=\trclos(\bigcup X)$.
\end{prp}
\begin{proof}
	Let $X$ be an arbitrary (non-empty) subset of $\con\alpha$.  We now proceed
	to argue that $(\con\alpha,\vee,\wedge)$ is a complete lattice under the
	defined operations.  To achieve this, we are first required to show that (as
	defined above) the sets $\bigvee X$ and $\bigwedge X$ are in fact
	congruences.

	We first make the case for $\bigwedge X$.  Let $I$ denote the identity
	relation on $\domain{}\alpha$ then, since the members of $X$ are all
	reflexive, we must have $I\subseteq x$ for each $x\in X$ and thus
	$I\subseteq\bigcap X=\bigwedge X$.

	Therefore $\bigwedge X$ is in fact a reflexive (binary) relation.  Also,
	since the members of $X$ are all symmetric it immediately follows that
	$\bigwedge X$ is also, since $(a,b)\bigcap X$ implies $(a,b)\in x$ for every
	$x\in X$.  The symmetry of the members of $x$ now yield the
	corresponding $(b,a)\in\bigcap X$.

	In a somewhat similar fashion, an arbitrary intersection of transitive
	(binary) relations will again yield a transitive relation, all from first
	principles.

	Next we consider the case for $\bigvee X$.  Note that $I\subseteq \bigvee X$
	since $X$ is non-empty and $I\subseteq x$ for each $x\in X$.  Therefore
	$\bigvee X$ is reflexive.  Note now that $\bigvee X$ is an intersection of
	symmetric (binary) relations and thus must itself also be symmetric.  In a
	similar fashion, since $\bigvee X$ is an intersection of transtive relations
	it must itself also be transitive.

	All that remains is to show that $\bigvee X=\trclos(\bigcup X)$.  Clearly,
	since $\bigvee X$ is transitive, we already have $\trclos(\bigcup
	X)\subseteq\bigvee X$.  Noting then that $\trclos(\bigcup X)\in X$, we may
	conclude $\bigvee X\leq\trclos(\bigcup X)$, as required.
\end{proof}

The sublattice of $\con\alpha$ generated by the set of $0$\textit{-definable}
congruences on $\alpha$ will be denoted as $\defcon{\alpha}$.  It should be clear
that, in general, $\defcon{\alpha}$ will not be complete as it might lack
certain \textit{infinite} joins or meets.

Ideally, one would want every condensation of a member of $\con\alpha$ to again
belong to $\con\alpha$.  Up to isomorphism this is in fact the case, as
suggested by the following ``homomorphism'' theorem.  Take note that we use
$\pi_\sim$ to denote the splitting
$\pi_\sim\colon\alpha\to\faktor{\alpha}{\sim}$ induced by the congruence $\sim$
on $\alpha$.

\begin{prp}[Homomorphisms]
	 Suppose $f\colon\alpha\to\beta$ is a splitting. There exists a unique
	 congruence relation $\sim$ on $\alpha$ and a unique order isomorphism
	 $\iota\colon\faktor{\alpha}{\sim}\to\beta$ that makes the following diagram
	 commute:
	\begin{equation}
		\begin{tikzcd}
			\alpha \arrow[r, "\pi_\sim",rightarrow]&   \faktor{\alpha}{\sim}\\
			 &   \beta\arrow[from=u,"\iota", dashrightarrow]
			 \arrow[from=ul,"f"']
		\end{tikzcd}
	\end{equation}
\end{prp}
\begin{proof}
	The desired congruence relations is obtained in a manner familiar from
	algebra:  identify elements in $\alpha$ whenever they have the same image
	under $f$.  Borrowing a definition from universal algebra and lattice
	theory, we let $\sim$ be a binary relation such that:
	\begin{equation}
		\sim{}=\kernel f=\setbuild{(a,b)\in\domain{2}\alpha}{f(a)=f(b)}.
	\end{equation}

	It is readily verified that the required isomorphism is given by
	$\iota([a])=f(a)$, for each $a\in\alpha$.  That $\iota$ is well-defined
	follows simply from the definition of ${\sim}=\kernel f$.

	We are now required to establish the uniqueness of $\sim$ and $\iota$ in
	their respective roles.  Suppose $\sim_0$ is a congruence on $\alpha$ and
	there exists a unique isomorphism $\iota_0$ that makes
	\begin{equation}
		\begin{tikzcd}
			\alpha\arrow[r, "\pi_{\sim_0}",rightarrow]&
			\faktor{\alpha}{\sim_0}\\
			 &   \beta\arrow[from=u,"\iota", dashrightarrow]
			 \arrow[from=ul,"f"']
		\end{tikzcd}
	\end{equation}
	commute.  This then clearly implies that ${\sim_0}=\ker f={\sim}$ and,
	consequently, also $\iota_0=\iota$.
\end{proof}


\section{Hausdorff's characterisation of the countable scattered linear orders}

\begin{prp}[Operations on $\scattered$]
	\label{prp:OpScattered}
	The following properties hold:
	\begin{enumerate}
		\item   If $I\in\scattered$ and $\alpha_i\in\scattered$ for each
			$i\in I$ then $\sum_{i\in I}\alpha_i\in\scattered$;
		\item   if $\alpha,\beta\in\scattered$ then
			$\alpha+\beta,\alpha\cdot\beta\in\scattered$.
	\end{enumerate}
\end{prp}
\begin{proof}
		(1):  Suppose, by way of contradiction, that $\delta\subseteq\sum_{i\in
		I}\alpha_i$ is countable and dense.  Define
		$\delta_i=\delta\cap\alpha_i$, for each $i\in I$.

		Note that, since $\delta$ is dense, we cannot have
		$1<\card{\delta_i}<\aleph_0$ for any $i\in I$.  Therefore, each
		$\delta_i$ is either infinite or has at most one element.

		If $\delta_i$ is finite for each $i\in I$ then $\delta\preceq I$,
		contradicting the scatteredness of $I$.  Consequently, the must exist
		some $j\in I$ such that $\delta_j$ is infinite.  However, since
		$\delta_j\subseteq\alpha_j$, this implies that $\alpha_j$ has a
		countable dense subset --- the desired contradiction.

		(2):  Choosing $I=\two$, $\alpha_0=\alpha$ and $\alpha_1=\beta$ in (1)
		yields $\alpha+\beta\in\scattered$.  Instead, choosing $I=\beta$ and
		$\alpha_i=\alpha$ for each $i\in\beta$ we get
		$\alpha\cdot\beta\in\scattered$, as required.
\end{proof}

The result above gives us the mantra: \textit{scattered sums} of scattered
linear orders are themselves scattered.  In particular, products and (finite)
sums of scattered linear orders are also scattered.

This observation suggests the possibility that one could generate the class of
all scattered linear orders recursively as sums of previously constructed
(scattered) linear orders.  We do this for the countable case but the approach
readily extends to the class of \text{all} scattered linear orders.

\begin{dfn}[The class $\VD$]
		By way of transfinite recursion define, for each ordinal
		$\gamma<\omega_1$, the class of linear orders
		$\VD_{\gamma}\subseteq\linear$ to be the smallest class which is
		\textit{closed under isomorphisms} and satisfies:
        \begin{enumerate}
            \item   $\zero,\one\in\VD_0$;
			\item   if $\alpha_i\in\bigcup_{\beta<\gamma}\VD_{\beta}$, for each
				$i\in \zeta$, then $\sum_{i\in\zeta}\alpha_i\in\VD_\gamma$.
        \end{enumerate}
		The class $\VD=\bigcup_{\gamma<\omega_1}\VD_\gamma$ is called the class
		of $\textbf{(countable) very discrete}$ linear orders.
\end{dfn}

\begin{dfn}[$\VD$-rank]
		If $\alpha$ is a very discrete linear order then its
		$\bm{\mathcal{VD}}$\textbf{-rank} $\vdrank(\alpha)$ is the least ordinal
		$\beta$ such that $\alpha\in\VD_{\beta}$.
\end{dfn}

\begin{lem}\label{prp:vdsct}
	Every very discrete linear order is scattered.  That is to say
	$\VD\subseteq\scattered$.
\end{lem}
\begin{proof}
	We argue by transfinite induction on $\gamma$ that
	$\VD_\gamma\subseteq\scattered$.  Finite linear orders are (trivially)
	scattered and thus $\VD_0\subseteq\scattered$.

	Fix $\delta<\omega_1$ and assume $\VD_\gamma\subseteq\scattered$, for each
	$\gamma<\delta$.  By definition, if $\vdrank(\alpha)=\delta$ then there
	exists, for each $i\in\zeta$, an $\alpha_i\in\VD_{\gamma_i}$, for some
	ordinal
	$\gamma_i<\delta$, such that
	\begin{equation}
		\alpha=\sum_{i\in\zeta}\alpha_i.
	\end{equation}

	It follows from Proposition \ref{prp:OpScattered}, as well as the
	inductive hypothesis, that $\alpha$ is a scattered sum of scattered linear
	orders.  Therefore we must have $\alpha\in\scattered$, as requred.
\end{proof}

Suppose $\pi\colon\alpha\to\faktor{\alpha}{\sim}$ is a splitting. Define $\pi^0$
to be the splitting $a\mapsto\set{a}$ and let $\pi^1=\pi$.  If $\pi^\gamma$ has
been defined and $\pi^\prime$ is the finite splitting of
$\faktor{\alpha}{\sim_\gamma}$ then let
$\pi^{\gamma+1}=\pi^\prime\circ\pi^\gamma$.  Now choose $\delta$ to be some
limit ordinal and assume $\pi^\gamma$ has been defined for all $\gamma<\delta$.

Recall from Proposition \ref{prp:conlat} that, given any linear order $\alpha$,
the lattice $\con\alpha$ is complete.  Hence, $\sim_\gamma$ is the congruence of
$\alpha$ induced by $\pi_\gamma$, for each $\gamma<\delta$, we may define
$\sim_\delta=\bigvee_{\gamma<\delta}\sim_\gamma$ and choose $\pi_\delta$ to be
the (corresponding) induced splitting of $\alpha$.

Note that there necessarily exists a least ordinal $\beta$ with the property:
for every ordinal $\delta\geq\beta$, there exists an isomorphism $\iota$ such
that the diagram
\begin{equation}
	\begin{tikzcd}
		\alpha\arrow[dd,"{\pi^\beta}"']\arrow[rdd,"{\pi^\delta}"]&\\
									 &\\
		\faktor{\alpha}{\sim_\beta}\arrow[r,"{\iota}"]&\faktor{\alpha}{\sim_\delta}
	\end{tikzcd}
\end{equation}
commutes.  In the special case $\pi=\fsplit$ we refer to $\beta$ as the
\textit{Hausdorff rank} of $\alpha$ and denote it by $\hrank(\alpha)$.

\begin{prp}
	If $\alpha$ is a linear order, $\beta<\hrank(\alpha)$ is a limit ordinal
	and $\pi=\fsplit$ then it follows, for every $\gamma<\beta$, that there
	exists a unique splitting
	$h_\gamma\colon\faktor{\alpha}{\sim_\gamma}\to\faktor{\alpha}{\sim_\beta}$ so that
	\begin{equation}
		\begin{tikzcd}
			\alpha\arrow[rd,"{\pi^\gamma}"]\arrow[dd,"{\pi^\beta}"']&\\
			&\faktor{\alpha}{\sim_{\gamma}}\arrow[ld,dashrightarrow,"{h_\gamma}"]\\
			\faktor{\alpha}{\sim_\beta}&
		\end{tikzcd}
	\end{equation}
	commutes.
\end{prp}
\begin{proof}
	Fix any $\gamma<\beta$ then note that, for any $a,b\in\alpha$,
	\begin{equation}
		a\sim_\gamma b\implies a\sim_\beta b.
	\end{equation}
	Denoting by $a_\gamma$ the equivalence class $[a]_{\sim_\gamma}$, for
	$a\in\alpha$, it follows that the map $a_\gamma\mapsto\pi_\beta(a)$ is
	well-defined.

	Let $h_\gamma\colon\faktor{\alpha}{\sim_\gamma}\to\faktor{\alpha}{\sim_\beta}$ be
	the map with the above action then the diagram in question commutes.  Since
	$\pi_\gamma$ is surjective, it also follows that $h_\gamma$ is unique.

	All that remains is to establish the surjectivity of $h_\gamma$.  To this
	effect, it is sufficient to note: for every $a\in\alpha$ it holds that
	$h_\gamma(a_{\gamma})=a_\beta$.

\end{proof}

Note, for a scattered linear order $\alpha$ and any ordinal
$\beta\geq\hrank(\alpha)$, we necessarily have
$\faktor{\alpha}{\sim_\beta}\cong\one$.  It is enough to argue that this is the
case when $\beta=\hrank(\alpha)$ so suppose the contrary.

By assumption there must exist some $a,b\in\alpha$ such that $a_\beta<b_\beta$.
Since $\pi_\beta$ is surjective, it follows from the definition of $\beta$ that
there exists a unique isomorphism
$\iota\colon\faktor{\alpha}{\sim_\beta}\to\faktor{\alpha}{\sim_{\beta+1}}$ such
that
\begin{equation}
	\begin{tikzcd}
		\alpha\arrow[rd,"{\pi^\beta}"]\arrow[dd,"{\pi^{\beta+1}}"']&\\
		&\faktor{\alpha}{\sim_{\beta}}\arrow[ld,dashrightarrow,"\iota"]\\
		\faktor{\alpha}{\sim_{\beta+1}}&
	\end{tikzcd}
\end{equation}
commutes.

Since $\iota$ is injective it follows that $a_{\beta+1}<b_{\beta+1}$.  By
definition of $\pi_{\beta+1}$, we may now conclude that the interval
$[a_\beta,b_\beta]$, and thus also $I=(a_\beta,b_\beta)$, is infinite.

Working towards a contradiction, we now argue that the interval $I$ is a dense
linear order.  If $c_\beta<d_\beta$ both belong to $I$, for some $c,d\in\alpha$,
then it follows from the injectivity of $\iota$ that $c_{\beta+1}<d_{\beta+1}$
and, thus, $[c_\beta,d_\beta]$ is infinite.  Consequently, the open interval
$(c_\beta,d_\beta)$ is nonempty, as required.

This is the desired contradiction.  Therefore, we may conclude that
$\faktor{\alpha}{\sim_\beta}\cong\one$ whenever $\alpha$ is scattered.

\begin{lem}
	\label{lem:consplit}
	Suppose $\alpha$ is a countable linear order then the following holds:
	\begin{enumerate}
		\item if $\alpha$ is very discrete and $\delta$ is a convex subset of
			$\alpha$ then $\delta$ is a very discrete linear order.
		\item if $\alpha$ is scattered, $\hrank(\alpha)$ is a limit
			ordinal $\rho>0$ and
			\begin{equation}
				\label{eq:limrank}
				\lim_{i<\omega}\rho_i=\rho
			\end{equation} then,
			for every $a<b$ in $\alpha$, there exists an $n<\omega$ such that
			$\fsplit^{\rho_i}(a)=\fsplit^{\rho_i}(b)$ whenever $i\geq n$.
	\end{enumerate}
\end{lem}
\begin{proof}\leavevmode
	\begin{enumerate}[topsep=0pt]
		\item If $\vdrank(\alpha)=0$ then $\alpha$, and thus $\delta$, must be
			finite so that the result holds trivially.

			Suppose now that $\vdrank(\alpha)>0$ and assume the result holds for
			all very discrete linear orders of $\VD$-rank less than
			$\vdrank(\alpha)$.  It follows by definition that there exists
			a family $\family{\alpha_i}{i\in\zeta}$ of very discrete linear
			orders such that
			\begin{equation}
				\alpha=\sum_{i\in\zeta}\alpha_i.
			\end{equation}

			For each $i\in\zeta$, define $\delta_i=\delta\cap\alpha_i$.  We may
			then conclude that
			\begin{equation}
				\delta=\sum_{i\in\zeta}\delta_i
			\end{equation}
			and, for each $i\in\zeta$, it holds that
			$\delta_i\subseteq\alpha_i$.  From the inductive hypothesis the
			result then readily follows.
			\smallskip

		\item Choose $\pi=\fsplit$ and let $\sim$ denote the induced congruence.
			Define $\family{\rho_i}{i<\omega}$ to be a strictly increasing
			sequence of ordinals such that (\ref{eq:limrank}) is
			satisfied.

			Choose the least $n<\omega$ such that
			\begin{equation}
				\hrank\left([a,b]\right)\leq\rho_n,
			\end{equation}
			and fix any $\delta=\rho_m$ such that $m\geq n$.  Note, if
			$\sim^\prime$ denotes the congruence
			$\sim_\delta\restriction[a,b]$ then it follows that
			\begin{equation}
				\faktor{[a,b]}{\sim^\prime}\cong
				\left[\faktor{a}{\sim_\delta}
				\mathpunct{,}\faktor{b}{\sim_\delta}\right].
			\end{equation}
			However, since $\alpha$ is scattered, it follows from part 1 that
			the subinterval $[a,b]$ is also scattered.  Hence, we may conclude
			that
			\begin{equation}
				\left[\faktor{a}{\sim_\delta},\faktor{b}{\sim_\delta}\right]
				\cong\one.
			\end{equation}
			Therefore, it follows that $\pi^\delta(a)=\pi^\delta(b)$, as
			required.
	\end{enumerate}
\end{proof}

\begin{lem}\label{prp:sctvd}
	If $\alpha\in\scattered$ and $\card{\alpha}\leq\aleph_0$ then $\alpha$ is
	very discrete.
\end{lem}
\begin{proof}
	Suppose $\alpha\in\scattered$ is at most countable and define a relation $R$
	on $\domain{}\alpha$ such that, for every $a,b\in\alpha$, we have $aRb$
	whenever $a\leq b$ and $[a,b]$ is finite.  Note that $R$ is transitive and
	thus, by Lemma \ref{lem:IndCong}, it induces a congruence relation $\sim$ on
	$\domain{}\alpha$.

	Define the condensation $\beta=\faktor{\alpha}{\sim}$ and argue by
	transfinite induction on $\hrank(\alpha)$.  If $\hrank(\alpha)=0$ then it
	follows immediately that $\alpha\embed\zeta$ and thus $\alpha\in\VD$.

	Assume the result holds for each $\gamma<\rho=\hrank(\alpha)$ and consider
	the case where $\rho=\rho_0+1$.  It follows that
	$\pi_{\rho_0}[\alpha]\embed\zeta$ and thus $\alpha$ can be written as a
	$\zeta$-sum of members of $\beta$.  This implies that $\alpha$ is very
	discrete, since $\hrank(\xi)<\rho$ whenever $\xi\in\beta$.

	Suppose now that $\rho$ is a limit ordinal then there exists a (strictly)
	increasing sequence
	$\family{\rho_i}{i<\omega}$ of ordinals such that
	$\lim_{i<\omega}\rho_i=\rho$.  Fix some $a_0\in\alpha$ and construct a
	sequence $\family{a_i}{i<\omega}$ as follows.  If $a_i$ has been defined for
	some $i<\omega$ then choose $a_{i+1}$ such that $a_{i+1}=a_i$ whenever
	\begin{equation}
		\up{a_0}\cap\pi^{\rho_{i+1}}(a_i)=\up{a_0}\cap\pi^{\rho_i}(a_i),
	\end{equation}
	otherwise, let
	$a_{i+1}\in\up{a_0}\cap(\pi^{\rho_{i+1}}(a_i)\setminus\pi^{\rho_i}(a_i))$.

	In a similar fashion, construct a sequence $\family{a_{-i}}{i<\omega}$ in
	$\down{a_0}$.  Define the indexed family $\family{I_j}{j\in\zeta}$
	such that
	\begin{equation}
		I_j=\begin{cases}
			\set{a_0}&\text{if }j=0,\\
			(a_{j-1},a_j]&\text{if }j>0,\\
			[a_j,a_{j+1})&\text{if }j<0.
		\end{cases}
	\end{equation}
	It follows from Lemma \ref{lem:consplit} that each $I_j$, for $i\in\ints$,
	is very discrete and we must have $\bigcup_{j\in\ints} I_j=\domain{}\alpha$.
	We may therefore write
	\begin{equation}
		\alpha=\sum_{j\in\zeta}I_j
	\end{equation}
	and thus conclude, by definition, that $\alpha$ is indeed a very discrete
	linear order.
\end{proof}

\begin{thm}[Hausdorff's Theorem]
	A linear order $\alpha$ is countable and scattered iff it is very discrete.
\end{thm}
\begin{proof}

	\forward\	Lemma \ref{prp:sctvd}.

	\backward\	Proposition \ref{prp:vdsct}.
\end{proof}


\section{The first-order theory of scattered linear orders}

A technical, but useful, feature of splittings is that they necessarily always
have \textit{right} inverses: if $s\colon\alpha\to\beta$ is a splitting then
there exists a homomorphism $h\colon\beta\to\alpha$ such that $sh=\id_\alpha$.
Additionally, $h$ must be an embedding of linear orders.  This latter remark
follows from the fact that if $b<b^\prime$ are members of $\beta$ then we must
have $sh(b)<sh(b^\prime)$ and thus $h(b)\neq h(b^\prime)$.

Splittings turn out to be an instance of what a category theorist would refer to
as a \textit{split epimorphism}.  This serves to justify our choice of
terminology as splittings are therefore, in fact, split!  Henceforth, we will
use this property without explicit mention of it.

From an ontological standpoint, the definition of scatteredness is
\textit{negative} in nature.  That is to say, scattered linear orders are
defined by stating what they are \textit{not}.  We now seek to remedy this in
the result that follows.

\begin{prp}[Condensations and scatteredness]
	\label{prp:cscat}
	The following are equivalent for any linear order $\alpha$:
	\begin{enumerate}
		\item  $\alpha$ is scattered,
		\item  no (nontrivial) condensation of $\alpha$ is dense,
		\item  the lattice $\con\alpha$ is atomic.
	\end{enumerate}
\end{prp}
\begin{proof}
	1$\Rightarrow$2:  If $\beta$ is a nontrivial condensation of $\alpha$ and
	$s\colon\alpha\to\beta$ is the induced splitting then there exists a right
	inverse $h$ of $s$ which is an embedding of $\beta$ in $\alpha$.  Hence,
	$\beta$ cannot be dense as this would contradict the scatteredness of
	$\alpha$.

	\smallskip\noindent 2$\Rightarrow$3:  Suppose ${\sim}\in\con\alpha$ is not
	the identity congruence.  By assumption, $\faktor{\alpha}{=}$ cannot be
	dense.  Hence, there exists $a<a^\prime$ in $\alpha$ such that $a\sim
	a^\prime$ and $a^\prime$ is an immediate successor of $a$.

	Now define the congruence $\sim_0$ on $\alpha$ such that $x\sim_0 y$ iff
	either $x=y$ or $x,y\in\set{a,a^\prime}$.  It follows immediately, by
	definition, that ${\sim_0}\subseteq{\sim}$ and no congruence $\sim_1$ on
	$\alpha$ satisfies ${\sim_1}\subsetneq{\sim_0}$ other than the identity
	congruence.  Thus, $\con\alpha$ is indeed atomic.

	\smallskip\noindent 3$\Rightarrow$1:  Suppose, with an aim to a
	contradiction, that there exists an embedding $h\colon\eta\to\alpha$.  We
	may then assume, without loss of generality, that $f[\beta]=\rats$.  Now, if
	$\sim$ is an atomic element of $\con\alpha$ then it follows that
	${\sim}\cap\rats\times\rats$ must be an atomic element of $\con\eta$,
	contradicting the density of $\eta$.
\end{proof}

The implication 1$\Rightarrow$2, in the previous proposition, will serve to aid
in the choice of a first-order approximation of the property of scatteredness.
This is done by requiring that 2 holds for \textit{definable} congruences only.

\begin{dfn}[Congruence formula]
	If $\alpha$ is any linear order and, for some $\bar{b}\in\domain[n]{\alpha}$
	the formula $\varphi(x,y,\bar{b})$ defines a congruence $\sim$ on $\alpha$
	then we call the formula $\varphi=\varphi(x,y,\bar{z})$, with $\bar{z}$ an
	$n$-tuple of variables, a \textbf{congruence formula}.
\end{dfn}

Furthermore, we will use the notation $\faktor{\alpha}{\varphi}$ to refer
to condensation $\faktor{\alpha}{\sim}$ and, for any $a\in\alpha$,
$a_\varphi=\faktor{a}{\varphi}$ will mean the equivalence class
$\faktor{a}{\sim}$.  In this given scenario, we will refer to
$\faktor{\alpha}{\varphi}$ as being a \textit{definable condensation} of
$\alpha$.

\begin{dfn}[Definably scattered]
	A linear order $\alpha$ is \textbf{definably scattered} whenever none of its
	(infinite) definable condensations are dense.
\end{dfn}

Na\"ively, one would expect that scatteredness and its ``definable
approximation'' do not coincide.  This is suggessted by the fact that ordinary
scatteredness is itself a \textit{dyadic} second order property: if $X$ is a
binary relation variable, $\varphi(X)$ expresses that $X$ is a (nontrivial)
congruence and $\psi(X)$ expresses that the induced condensation is not dense
then it follows that $\alpha$ is scattered iff
\begin{equation}
	\alpha\models\forall X\big(\varphi(X)\rightarrow\psi(X)\big).
\end{equation}

It turns out that this expectation is in fact correct!  The demonstration of
this fact is a routine application of the Diagram Lemma and Compactness Theorem.

\begin{exm}[A definably scattered linear order which is not scattered]
	\label{exm:dscat}
	Consider the language $L$ which is obtained, from the language of linear
	orders, by adding a new constant symbol $c_q$, for each $q\in\rats$.  Also,
	define $T$ to be the first-order theory of definably scattered linear
	orders.

	Since $\zeta$ is scattered, it follows that $\zeta\models T$.  Furthermore,
	if $\Sigma\subseteq\diag\eta$ is finite then, for each $q\in\rats$,
	$z_q={c_q}^\zeta\in\ints$ may be chosen in a manner such that
	$(\zeta,\family{z_q}{q\in\rats})\models T\cup\Sigma$.  This is possible due
	to the fact that only a finite number of the new constant symbols will have
	an occurence in some member of $\Sigma$.

	From the Compactness Theorem, it follows that $T\cup\diag\eta$ must have
	some model $\mathfrak{M}$.  Therefore, from the Diagram Lemma, there exists
	a linear order $\alpha$ and an embedding $h\colon\eta\to\alpha$ such that
	$(\alpha,h[\rats])=\mathfrak{M}$.

	By construction, $\alpha$ cannot be scattered but, nonetheless, must
	be a model of the first-order theory $T$.  Consequently, $\alpha$ is
	definably scattered but not scattered.
\end{exm}

\begin{prp}\label{prp:dcform}
	For each formula $\varphi=\varphi(x,y,\bar{z})$, with $\bar{z}$ possibly
	empty, define the formula $\epsilon_\varphi(\bar{z})$ to be the conjunction
	of the following formulas:
	\begin{enumerate}
		\item	$\forall x\varphi(x,x,\bar{z})$,\hfill(reflexivity)
		\item	$\forall x\forall
			y(\varphi(x,y,\bar{z})\rightarrow\varphi(y,x,\bar{z}))$,\hfill
			(symmetry)
		\item 	$\forall x\forall y\forall
			w(\varphi(x,y,\bar{z})\wedge\varphi(y,w,\bar{z})
			\rightarrow\varphi(x,w,\bar{z}))$.\hfill
			(transitivity)
		\item 	$\forall x_0\forall x_1\forall y_0\forall
			y_1(x_0<x_1\wedge\neg\varphi(x_0,x_1,\bar{z})
			\wedge\varphi(x_0,y_0,\bar{z})
			\wedge\varphi(x_1,y_1,\bar{z})\rightarrow
			y_0<y_1)$\phantom{}\hfill(compatibility).
	\end{enumerate}
	For every finite tuple $\bar{a}$ of $\alpha$ (of the same length as
	$\bar{z}$), it holds that $\alpha\models\epsilon_\varphi(\bar{a})$ iff
	$\varphi(x,y,\bar{a})$ defines a congruence of $\alpha$.
\end{prp}
\begin{proof}
	Immediate, follows from the definition of a congruence of on $\alpha$.
\end{proof}

\begin{thm}
	\label{thm:nscat}
	For every countable definably scattered linear order $\alpha$, and every
	$n\in\nats$, there exists a $\beta_n\in\scattered$ such that
	$\alpha\nequiv{n}\beta_n$.
\end{thm}
\begin{proof}
	Suppose $\alpha$ is countable and definably scattered and fix any
	$n\in\nats$.  Now define a binary relation $R$ on $\domain{\alpha}$ such
	that $aRb$ iff there exists a $\beta\in\scattered$ such that
	$(a,b)\nequiv{n}\beta$.

	Since $R$ is clearly transitive, it induces a congruence $\sim$ on $\alpha$.
	Moreover, $\sim$ is defined by the first-order formula $\varphi(x,y)$ as
	follows.  Choose an $n$-spectrum $\chi=\set{\alpha_0,\dotsc,\alpha_{m-1}}$
	for $\scattered$ and let $\varphi(x,y)$ be the formula:
	\begin{equation}
		x=y\vee\left(x<y\wedge\bigvee_{i<m}(\cha{\alpha_i}{n})^{(x,y)}\right)
		\vee\left(y<x\wedge\bigvee_{i<m}(\cha{\alpha_i}{n})^{(y,x)} \right).
	\end{equation}

	\begin{claim}\label{clm:LL1}
		Each $\gamma\in\faktor{\alpha}{\sim}$ has a scattered $n$-equivalent.
	\end{claim}
	\begin{proof}[Proof of claim]
		Choose any $\gamma\in\faktor{\alpha}{\sim}$ and first consider the case
		where $\gamma$ has a least element element $a_0$ but no greatest
		element.  Since $\gamma$ is countable, there exists a cofinal sequence
		$\family{a_i}{i<\omega}$.

		By Ramsey's Theorem, if $h$ is the colouring of $\alpha$ that sends each
		interval $(a,b)$ to its characteristic in $\chi$, then there exists
		a homogeneous subsequence
		$\family{a^\prime_i}{i<\omega}\subseteq\family{a_i}{i<\omega}$ for the
		colouring $h$.

		Consequently, $\gamma$ is an $\omega$-sum
		\begin{equation}
			\gamma=[a_0,a^\prime_0)+\sum_{i<\omega}[a^\prime_i,a^\prime_{i+1}),
		\end{equation}
		the summands of which are all $n$-equivalent to the same $\one+\beta$,
		for some $\beta\in\chi$.  It now follows that $\gamma$ is $n$-equivalent
		to the scattered linear order:
		\begin{equation}
			\gamma^\prime=(\one+\beta)\cdot\omega.
		\end{equation}
		Since $[a_0,a^\prime_0)$ is $n$-equivalent to some $\gamma_0\in\chi$,
		it follows that $\beta_n=\gamma_0+\gamma^\prime$ is the desired
		scatterred $n$-equivalent of $\gamma$.

		If $\gamma$ has both a least and a greatest element we simply add a
		greatest element to $\gamma^\prime$ in the argument above.  That is to
		say, $\gamma$ will be $n$-equivalent to
		\begin{equation}
			\gamma^\prime=(\one+\beta)\cdot\omega+\one.
		\end{equation}

		A similar methodology may be applied in the case where $\gamma$ has a
		greatest element. One simply dualises the argument above: systematically
		and consistently replace each linear order with its reverse order.  As a
		result, each instance of $<$ will become $>$ and ``least element'' will
		become ``greatest element'' (or vice-versa).

		In the case where $\gamma$ has neither a largest nor a smallest element,
		choose some fixed $b<a$ in $\gamma$ and apply the argument above to the
		respective suborderings $\gamma^{\geq a}$ and $\gamma^{\leq b}$.  This
		yields the respective $n$-equivalents $\gamma_a,\gamma_b\in\chi$.  Thus,
		if $\delta\in\chi$ is the $n$-equivalent of $(a,b)$ then $\gamma$ is
		$n$-equivalent to the scattered linear order
		\begin{equation}
			\gamma_b+\delta+\gamma_a,
		\end{equation}
		as required.\noqed
	\end{proof}

	\begin{claim}
		The equivalence classes are dense under the induced order.
	\end{claim}
	\begin{proof}[Proof of claim]
		Suppose $\epsilon_0,\epsilon_1\in\faktor{\alpha}{\sim}$ satisfy the
		relation $\epsilon_0\ll\epsilon_1$ and assume, by way of contradiction,
		that no $c\in\alpha$ satisfies:
		\begin{equation}
			\epsilon_0\ll c\ll\epsilon_1.
		\end{equation}

		Since $n$-equivalence is preserved under sums it follows that
		$\epsilon_0+\epsilon_1$ is $n$-equivalent to some scattered linear order
		$\sigma$.  Choosing an $a_0\in\epsilon_1$ and an $a_1\in\epsilon_1$,
		note that the interval $(a_0,a_1)$ in $\sigma$ necessarily posesses a
		scattered $n$-equivalent.  This is the desired contradiction as then, by
		definition, it would have to hold that $a_0\sim a_1$.\noqed
	\end{proof}
	Since $\faktor{\alpha}{\sim}$ is dense and $\alpha$ is definably scattered,
	it follows by definition that $\faktor{\alpha}{\sim}\cong\one$.  Therefore,
	it follows from the first claim that $\alpha$ has a scattered
	$n$-equivalent.
\end{proof}

\begin{prp}\label{prp:rescat}
	For each formula $\varphi(x,y,\bar{z})$ define $\theta_\varphi(\bar{z})$ to
	be the formula
	\begin{equation}
		\exists x\exists y\big((x<y\wedge\neg\varphi(x,y,\bar{z}))\wedge\forall
		w(x<w<y\rightarrow\varphi(x,w,\bar{z})\vee\varphi(w,y,\bar{z}))\big)
	\end{equation}
	which formalises the statement: ``$\varphi(x,y,\bar{z})$ defines a
	congruence of which the equivalence classes are not dense''.  Let each
	$\epsilon_\varphi(\bar{z})$ be defined as in Proposition \ref{prp:dcform}
	then the following holds:
	\begin{enumerate}
		\item the theory
			$\Sigma=\set{\axmlin}
			\cup\setbuild{\forall\bar{z}(\epsilon_\varphi(\bar{z})
			\rightarrow\theta_\varphi(\bar{z}))}{\varphi=\varphi(x,y,\bar{z})}$
			axiomatises $\Th(\scattered)$ i.e.\ $\dcl{\Sigma}=\Th(\scattered)$,

		\item $\Sigma$ is recursively enumerable.
	\end{enumerate}
	\begin{proof}
		\begin{enumerate}
			\item	Since $\alpha\models\Sigma$ iff $\alpha$ is definably
				scattered, it is enough to prove that $\alpha$ is definably
				scattered iff $\alpha\models\Th(\scattered)$.  By the
				L\"owenheim-Skolem Theorem, we need only consider the case where
				$\alpha$ is countable.

				If $\alpha$ is scattered then it follows from Proposition
				\ref{prp:cscat} that no nontrivial condensation of $\alpha$ is
				dense and thus, as this observation includes definable
				condensations, $\alpha$ is necessarily definably scattered.

				Suppose now, instead, that $\alpha$ is definably scattered so
				that we are required to show that
				$\alpha\models\Th(\scattered)$.  Choose any
				$\sigma\in\Th(\scattered)$ and suppose it holds that
				$\qrank(\sigma)=n\in\nats$.  It now follows from Theorem
				\ref{thm:nscat} that there exists a scattered $\beta$ satisfying
				$\beta\nequiv{n}\alpha$.  Certainly, we must have
				$\beta\models\sigma$ and, therefore, the $n$-equivalence
				guarantees that $\alpha\models\sigma$.

			\item	Since our signature is finite, it follows that the set of
				formulas in our language can be enumerated (via a G\"odel
				numbering $\godel{-}\colon L\to\nats$).  We can then enumerate
				$\Sigma$ by declaring a map, from $\Sigma$ to $\nats$, such that
				$\axmlin\mapsto 0$ and, for every tuple $\bar{z}$ of variables,
				\begin{equation}
					\forall\bar{z}(\epsilon_\varphi(\bar{z})
					\rightarrow\theta_\varphi(\bar{z})))\mapsto
					2^{\godel{\varphi(x,y,\bar{z})}}.\label{eq:regod}
				\end{equation}

				In other words, a listing of the members of $\Sigma$ may be
				produced by simultaneously generating an auxiliary list of
				formulas of the form $\varphi=\varphi(x,y,\bar{z})$.  For each
				entry $\varphi$ in the auxiliary list one creates an entry,
				having the form of (\ref{eq:regod}), in the principal
				list.\qedhere
		\end{enumerate}
	\end{proof}
\end{prp}


\section{Theorems of L\"auchli and Leonard}

A subtle feature of first-order model theory is that, given a theory $T$, the
class $\Mod T$ is hopelessly oversized and may, in a certain sense, be trimmed
down in a manner dependent on the cardinality of the underlying language of the
theory.

In particular, and with more precision, if $T$ is a theory in a countable
language then there exists a set $\mathcal{K}\subseteq\Mod T$ of cardinality at
most $2^{\aleph_0}$ such that $\Th\mathcal{K}=\dcl{T}$.  This may be expressed
by saying that such a set $\mathcal{K}$ \textit{approximates} the class of
models of $T$.  Furthermore, if every sentence $\sigma$ consistent with $T$ has
a model in $\mathcal{K}$ then it is said to be a \textit{full} approximation or
that it \textit{fully approximates} the class $\Mod T$.

Certain natural questions arise in this context.  Is it possible to choose the
approximating class $\mathcal{K}$ to be countable or, better yet, recursively
enumerable?  Could this be done in manner that only models with decidable
theories occur in $\mathcal{K}$?  Henceforth, we will concern ourselves with
questions of precisely this nature, for specified classes of linear orders.

\begin{dfn}[The class $\Mzero$]
	The class $\Mzero$ is the smallest class of linear orders which satisfies
	the following:
	\begin{enumerate}
		\item	$\zero,\one\in\Mzero$,
		\item	if $\alpha,\beta\in\Mzero$ then $\alpha+\beta\in\Mzero$,
		\item	if $\alpha\in\Mzero$ then
			$\alpha\cdot\omega,\alpha\cdot\dual{\omega}\in\Mzero$,
	\end{enumerate}
\end{dfn}

From Proposition \ref{prp:OpScattered}, and the above definition, it follows
that $\Mzero$ is a countable set of countable scattered linear orders.  This is
readily verified by induction on $\Mzero$.

Note, however, that $\Mzero$ need not contain arbitrary $\zeta$-sums of its
elements.  Hence, as we will argue, it is not the case that every countable
scattered linear order need be isomorphic to a member of $\Mzero$.  It is
sufficient to note that any ordinal $\alpha\in\Mzero$ must have finite degree.
In other words, there must exist some $n\in\nats$ so that, in Cantor normal
form,
\begin{equation}
	\alpha=\omega^n\cdot\bk_{m-1}+\dotsb+\bk_0.
\end{equation}
Therefore, we cannot have $\omega^\omega\in\Mzero$ as $\omega^\omega$ has degree
$\omega$.

As we will see, in the propositions that follow, $\Mzero$ is nonetheless closed
under $\zeta$-sums when considered only up to $n$-equivalence, for any fixed
$n\in\nats$.  This will prove useful when establishing the more general fact
that $\Mzero$ contains, for each $n\in\nats$, $n$-equivalents for every choice
of scattered linear order.

\begin{prp}\label{prp:M0sum}
	Let $n\in\nats$ and suppose for each $i<\omega$ it holds that
	$\alpha_i\in\Mzero$.  Then there exists a $\beta\in\Mzero$ such that
	\begin{equation}
		\sum_{i<\omega}\alpha_i\nequiv{n}\beta.
	\end{equation}
\end{prp}
\begin{proof}
	For each $i,j<\omega$, if $i<j$ then define
	$\alpha_{ij}=\alpha_{i+1}+\alpha_{i+2}+\dotsb+\alpha_j$ and note that
	$\alpha_{ij}\in\Mzero$ since $\Mzero$ is closed under finite sums.  Choose
	$S=\set{\beta_0,\dotsc,\beta_{k-1}}\subseteq\Mzero$ to be an $n$-spectrum
	for $\Mzero$.  Consequently, there exists a colouring
	$h\colon\subsets{k}{\nats}\to\set{0,\dotsc,k-1}$ of $\omega$ such that
	$h(i,j)=\ell$ iff $\alpha_{ij}\nequiv{n}\beta_\ell$.

	It now follows from Ramsey's theorem that there exists a homogeneous
	sequence $\family{i_j}{j<\omega}$ for $h$.  By definition, there exists a
	natural $m$ such that $0\leq m<k$ and
	$\alpha_{i_ji_{j+1}}\nequiv{n}\beta_m$, for every $j<\omega$.  Note, from
	Lemma \ref{lem:fvsum}, it follows that
	\begin{align}
		\sum_{i<\omega}\alpha_i&\cong\phantom{^n}\alpha_0+\dotsb+\alpha_{i_0}
		+\sum_{j<\omega}\alpha_{i_ji_{j+1}}\\
		&\nequiv{n}\alpha_0+\dotsb+\alpha_{i_0}+\beta_m\cdot\omega.
	\end{align}
	Since $\Mzero$ is closed under finite sums, we may conclude that
	$\beta=\alpha_0+\dotsb+\alpha_{i_0}+\beta_m\cdot\omega$ implies
	$\beta\in\Mzero$ and $\sum_{i<\omega}\alpha_i\nequiv{n}\beta$, as required.
\end{proof}

\begin{prp}\label{prp:M0sumop}
	Let $n\in\nats$ and suppose for each $i\in\dual{\omega}$ it holds that
	$\alpha_i\in\Mzero$.  Then there exists a $\beta\in\Mzero$ such that
	\begin{equation}
		\sum_{i\in\dual{\omega}}\alpha_i\nequiv{n}\beta.
	\end{equation}
\end{prp}
\begin{proof}
	Dualise the argument in the proof of Proposition \ref{prp:M0sum}.
\end{proof}

\begin{prp}\label{prp:M0sumint}
	Let $n\in\nats$ and suppose for each $i\in\ints$ it holds that
	$\alpha_i\in\Mzero$.  Then there exists a $\beta\in\Mzero$ such that
	\begin{equation}
		\sum_{i\in\zeta}\alpha_i\nequiv{n}\beta.
	\end{equation}
\end{prp}
\begin{proof}
	Choose any fixed $a_0\in\alpha_0$ and define
	$\alpha=\sum_{i\in\zeta}\alpha_i$ apply Propositions \ref{prp:M0sum} and
	\ref{prp:M0sum} respectively to $\alpha^{>a_0}$ and $\alpha^{<a_0}$ to
	obtain their respective $n$-equivalents $\beta^+$ and $\beta^-$ in $\Mzero$.
	Since $\Mzero$ is closed under finite sums, it necessarily follows that
	$\beta=\beta^-+\one+\beta^+$ implies $\alpha\nequiv{n}\beta$, as required.
\end{proof}

\begin{thm}[L\"auchli and Leonard]
	For every countable $\alpha\in\scattered$, and each $n\in\nats$, there
	exists a $\beta_n\in\Mzero$ such that $\alpha\nequiv{n}\beta_n$.
\end{thm}
\begin{proof}
	Suppose $\alpha\in\scattered$ is countable.  From Hausdorff's theorem it
	follows that $\alpha$ is very discrete so we may argue by induction on the
	$\VD$-rank of $\alpha$. Since $\zero,\one\in\Mzero$, the result holds when
	$\vdrank(\alpha)=0$.

	Assume, now, there exists an ordinal $\gamma<\omega_1$ such that if $\alpha$
	is any linear order and $\vdrank(\alpha)<\gamma$ then, for every
	$n\in\nats$, there exists a $\beta_n\in\Mzero$ such that
	$\alpha\nequiv{n}\beta_n$.  Now fix any $n\in\nats$ and  suppose
	$\vdrank(\alpha)=\gamma$.  Then, by definition of $\VD$, there exists for
	each $i\in\ints$ an $\alpha_i\in\VD$ such that $\vdrank(\alpha_i)<\gamma$
	and
	\begin{equation}
		\alpha=\sum_{i\in\zeta}\alpha_i.
	\end{equation}

	By the induction hypothesis there exists, for each $i\in\ints$, some
	$\alpha^\prime_i\in\Mzero$ such that $\alpha_i\nequiv{n}\alpha_i^\prime$.
	It now follows from Lemma \ref{lem:fvsum}, if we choose
	$\alpha^\prime=\sum_{i\in\zeta}\alpha_i^\prime$, that
	$\alpha\nequiv{n}\alpha^\prime$.  According to Proposition
	\ref{prp:M0sumint} there exists a $\beta\in\Mzero$ such that
	$\alpha^\prime\nequiv{n}\beta$ and, therefore, $\alpha\nequiv{n}\beta$, as
	required.
\end{proof}

\begin{lem}\label{lem:rescat}
	The following set is recursively enumerable:
	\begin{equation}
		R\coloneqq\setbuild{(\alpha,\sigma)}{\alpha\in\Mzero\text{ and }\alpha\models\sigma}.
	\end{equation}
\end{lem}
\begin{proof}
	Recursively on $\Mzero$, for each $\alpha\in\Mzero$, by adding finitely many (new) relation symbols to $L$ we shall define a language $L_\alpha\supseteq L$ and first-order theories $S_\alpha$, $T_\alpha$ such that $T_\alpha\supseteq S_\alpha$ is a recursively enumerable complete theory.  The role of each $S_\alpha$ will be that of a finitely axiomatisable theory whose models interpret a finitely axiomatisable linear order.

	If $\alpha=\zero$ then we choose $L_\alpha$ to be the language of linear orders and we define $T_\alpha=S_\alpha$ to consist only of the sentence $\neg\exists x(x=x)$.  While, when $\alpha=\one$, we choose the same $L_\alpha$ as before and define $T_\alpha=S_\alpha$ to consist only of the sentence $\axmlin\wedge\exists x\forall y(x=y)$.

	In both cases above, $T_\alpha$ is recursively enumerable (since it is finite) as well as complete:  $T_\alpha$ determines its models up to isomorphism.

	Assume now that $L_\alpha$, $T_\alpha$ and $S_\alpha$ have been defined for $\alpha\in\set{\alpha_0,\alpha_1}$ and that $T_{\alpha_0},T_{\alpha_1}$ are recursively enumerable (complete) theories.  Suppose that $\alpha=\alpha_0+\alpha_1$ and expand $L$ by adding the unary relation symbols $r_0$ and $r_1$ in order to obtain $L_\alpha$.  These new symbols will serve to ``identify'' the linear orders $\alpha_0$ and $\alpha_1$, respectively, within $\alpha$.  Let $S_\alpha$ consist of exactly the $L_\alpha$-sentences $\axmlin$, $\forall x(r_0(x)\vee r_1(x))$ and $\forall x\forall y(r_0(x)\wedge r_1(y)\rightarrow x<y)$ while also choosing
	\begin{equation}
		T_\alpha=S_\alpha\cup\setbuild{\sigma^{r_0(v)}}{\alpha_0\models\sigma}\cup\setbuild{\sigma^{r_1(v)}}{\alpha_1\models\sigma}.
	\end{equation}

	We now argue that $T_\alpha$ is complete.  Note that if $\mathfrak{M},\mathfrak{N}\models T_\alpha$ then, by definition, their respective order types, $\beta$ and $\delta$, can be decomposed as $\beta=\beta_0+\beta_1$ and $\delta=\delta_0+\delta_1$ such that,
	\begin{equation}
		\beta_i,\delta_i\models\Th(\alpha_i), \quad\text{for }i=0,1.
	\end{equation}
	Therefore $\beta_0\equiv\delta_0$ and $\beta_1\equiv\delta_1$ so that, for each $n\in\nats$, $\Right$ has the (respective) winning strategies $s_n$ and $t_n$ for the games $\EF_n(\beta_0,\delta_0)$ and $\EF_n(\beta_1,\delta_1)$.

	The winning strategy for $\Right$ in the game $\EF_n(\mathfrak{M},\mathfrak{N})$ is laid out as follows.  A move in the $\beta_0$-part of $\mathfrak{M}$ is countered using the strategy $s_n$, as if playing the game $\EF_n(\beta_0,\delta_0)$ while a move in the $\beta_1$-part is similarly countered but using the strategy $t_n$.  If instead $\Left$ plays a member of $\mathfrak{N}$ then a similar methodology may be employed.

	If $\alpha=\alpha_0\cdot\omega$ then we obtain $L_\alpha$ from $L$ by adding a single binary relation symbol $\theta$.  Choose $\Gamma$ to be the (unique) interpretation of $L$ in $L_\alpha$ such that $\Gamma(x<y)$ is the formula $x<y\wedge\neg\theta(x,y)$.

	Let $\Sigma$ consist of the (finitely many) $L_\alpha$-sentences which, together, express the assertion ``$\theta$ is an equivalence relation'' as well as the sentence
	\begin{equation}
		\forall x_0\forall x_1\forall y_0\forall y_1( x_0<y_0\wedge\neg\theta(x_0,y_0)\wedge\theta(x_1,x_0)\wedge\theta(y_1,y_0)\rightarrow x_1<y_1).
	\end{equation}
	so that, in fact, $\theta$ is a congruence and not simply a mere equivalence relation.

	Define $S_\alpha$ to be a finite axiomatisation of the $L_\alpha$-theory
	\begin{equation}
		\Gamma\left[\Th(\omega)\right]\cup\Sigma\cup\set{\axmlin}
	\end{equation}
	and let $T_\alpha=S_\alpha\cup\setbuild{\forall w\sigma^{\theta(w,v)}}{\alpha_0\models\sigma}$.  The latter set of sentences expresses the sentiment that each $\theta$-equivalence class is elementarily equivalent to $\alpha_0$.

	Note, it follows from the induction hypothesis that $T_\alpha$ is recursively enumerable.  The burden is now on us to show that the $L_\alpha$-theory $T_\alpha$ is complete.  Consider any $\mathfrak{M},\mathfrak{N}\models T_\alpha$ then the respective order types, $\beta$ and $\delta$, of $\mathfrak{M}$ and $\mathfrak{N}$ can be decomposed as
	\begin{equation}
		\beta=\beta_0\cdot\gamma_0\qquad\text{and}\qquad\delta=\delta_0\cdot\gamma_1,
	\end{equation}
	such that $\gamma_0\equiv\omega\equiv\gamma_1$ and $\beta_0,\delta_0\models\Th(\alpha_0)$.

	Fix any $n\in\nats$ then we describe a winning strategy for $\Right$ in the game $\EF_n(\mathfrak{M},\mathfrak{N})$.  If $\Left$ plays the move $(a,i)\in\domain{}\beta\times\domain{}\gamma_0$ then $\Right$ chooses a $j\in\gamma_1$ using his winning strategy in the game $\EF_n(\gamma_0,\gamma_1)$.  This is then followed by a choice of $b\in\delta_0$ using his winning strategy in the game $\EF(\beta_0,\delta_0)$.  This determines the countermove $(b,j)$.  If $\Left$ instead plays a move in the opposite structure then a countermove be obtained by $\Right$ in a similar fashion.

	The case when $\alpha=\alpha_0\cdot\dual{\omega}$ is approached in a similar manner: let $\theta$, $L_\alpha$, $\Gamma$ and $\Sigma$ be defined as before.  Choose $S_\alpha$ to be a finite axiomatisation of the theory
	\begin{equation}
		\Gamma[\Th(\dual{\omega})]\cup\Sigma\cup\set{\axmlin}
	\end{equation}
	and define $T_\alpha$ as in the previous case.  It again follows that $T_\alpha$ is recursively enumerable and a similar argument as before will suffice to show that $T_\alpha$ is complete.

	All that remains is that a mechanical listing of the set
	\begin{equation}
		R^\prime\coloneqq\setbuild{(\alpha,\sigma)}{\alpha\in\Mzero\text{ and }\sigma\text{ is an }L_\alpha\text{-sentence such that }T_\alpha\models\sigma}
	\end{equation}
	need be given.  Once this procedure has been established it will follow that $R$ is also recursively enumerable since, for any $\alpha\in\Mzero$ and $\sigma\in L_\alpha$, the problem of whether or not $\sigma$ is a member of $L$ is decidable.

	The list starts with all pairs of the form $(\alpha,\sigma)$ such that $\alpha\in\set{\zero,\one}$ and $\sigma\in S_\alpha$, noting that the aforementioned pairs are finite in number.  Now assume the entries $(\alpha_i,\sigma_i)$, for $i=0,\dotsc,n_0-1$, have been listed.  Continue the list sequentially with entries $(\alpha_i,\sigma_i)$, for $i\geq n_0$, as follows:
	\begin{enumerate}
		\item 	list all pairs of the form $(\alpha_i,\sigma)$ such that $\sigma\in S_{\alpha_i}$ and $i<n_0$ and let $n_1$ denote the resulting number of entries in the list;
		\item 	list all pairs of the form $(\alpha_i, \sigma)$, for $i<n_1$, where $\sigma$ is the direct consequence of an inference rule from $\sigma_{i_0},\dotsc,\sigma_{i_{k-1}}$ and, for each $j<k$, $(\alpha_i,\sigma_{i_j})$ has already appeared in the list and let $n_2$ denote the resulting number of entries;
		\item	list all pairs of the form $(\alpha,\sigma)$ where, for $i,j<n_2$, $\alpha=\alpha_i+\alpha_j$ and $\sigma=\sigma_i^{r_0(v)}\wedge\sigma_j^{r_1(v)}$ and let $n_3$ denote the resulting number of entries in the list;
		\item	list all pairs of the form $(\alpha_i\cdot\omega,\sigma)$, for $i<n_3$, such that $\sigma=\forall w\sigma_i^{\theta(w,v)}$ and and let $n_4$ denote the resulting number of entries in the list;
		\item	list all pairs of the form $(\alpha_i\cdot\dual{\omega},\sigma)$, for $i<n_4$, such that $\sigma=\forall w\sigma_i^{\theta(w,v)}$ and and redefine $n_0$ to be the resulting number of entries in the list;
		\item	repeat steps 1 to 5.
	\end{enumerate}

	By way of induction (on $\alpha\in\Mzero$) it can be shown that each member of the set $R^\prime$ does indeed appear as some entry in the above list.  Consequently, we may conclude that the set $R$ is in fact recursively enumerable, as required.
\end{proof}

\begin{thm}
	The theory $\Th(\scattered)$ is decidable.
\end{thm}
\begin{proof}
	From Proposition \ref{prp:rescat} and Lemma \ref{lem:rescat}, there exists machines $M_0$ and $M_1$ which, respectively, list the members of $\Th(\scattered)$ and the set of pairs
	\begin{equation}
		R=\setbuild{(\alpha,\sigma)}{\alpha\in\Mzero)\text{ and }\alpha\models\sigma}.
	\end{equation}

	Now consider a machine $M$ which, when given a sentence $\sigma$, alternately produces entries from each list until either $\sigma$ or, for some $\alpha\in\Mzero$, $(\alpha,\neg\sigma)$ appears in the list, at which point the machine halts and returns the value "True" if $\sigma$ appeared on the list as member of $\Th(\scattered)$ and returns "False" otherwise.

	We now need to argue that $M$ always halts irrespective of the choice of input sentence.  The case where $\sigma\in\Th(\scattered)$ is trivial so suppose that $\sigma\notin\Th(\scattered)$.  By definition, there must exist an $\alpha\in\scattered$ such that $\alpha\models\neg\sigma$.

	Choose $n=\qrank(\sigma)$ and recall it follows from the theorem of L\"auchli and Leonard that, for some $\beta_n\in\Mzero$, it holds that $\beta_n\nequiv{n}\alpha$ and thus $\beta_n\models\neg\sigma$.  Consequently, $(\beta_n,\neg\sigma)\in R$ and thus $(\beta_n,\neg\sigma)$ will appear in the list and $M$ will return "False".  Therefore, the machcine $M$ always halts --- as required.
\end{proof}

\bibliography{references}
