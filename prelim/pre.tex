\chapter{Preliminaries}

\section{Model theory and logic}

\begin{dfn}[Finitary languages]
	A language $L$ is called \textbf{finitary} if $L$ is a language over a \text{finite} signature.
\end{dfn}

\begin{prp}\label{prp:finequiv}
	Suppose we fix any $n\in\nats$ and $L$ is a finitary first order language. If $v_0,\dotsc,v_{k-1}$ are distinct variables then, up to logical equivalence, there are only finitely many formulas $\varphi=\varphi(v_1,\dotsc,v_{k-1})$ such that $\qrank(\varphi)\leq n$.
\end{prp}

\begin{dfn}[Characteristic formulas]
	If $L$ is some finitary language, $\mathfrak{M}$ is some $L$-structure and $\bar{a}\in\domain{k}{\mathfrak{M}}$ then we define the \textbf{$\mathbf{n}$-characterstic formula} $\cha{\bar{a}}{n}$ of $\bar{a}$ relative to $\mathfrak{M}$, recursively, as follows:
	\begin{enumerate}
		\item $\cha{\bar{a}}{0}=\bigwedge\setbuild{\varphi(\bar{x})\in L_k}{\mathfrak{M}\models\varphi(\bar{a})}$;
		\item $\cha{\bar{a}}{n+1}=\bigwedge_{b\in\domain{}\mathfrak{M}}\exists v_n \cha{\bar{a}b}{n}\wedge\forall v_n\bigvee_{b\in\domain{}\mathfrak{M}}\cha{\bar{a}b}{n}$, for each $n\in\nats$.
	\end{enumerate}
	If $\bar{a}$ is the empty tuple, and $n\in\nats$, then we write $\cha{\mathfrak{M}}{n}$ for $\cha{\bar{a}}{n}$ and call $\cha{\mathfrak{M}}{n}$ the \textit{$\mathit{n}$-characteristic sentence} of the structure $\mathfrak{M}$.
\end{dfn}

\begin{prp}
	If $L_{\omega\omega}$ is a finitary language, $\mathfrak{M}$ is an $L$-structure and $n\in\nats$ then the $L_{\omega_1\omega}$-sentence $\cha{\mathfrak{M}}{n}$ is logically equivalent to an $L_{\omega\omega}$-sentence $\sigma$.
\end{prp}
\begin{proof}
	From proposition \ref{prp:finequiv} it follows that there exists sentences $\sigma_0,\dotsc,\sigma_{k-1}$, of quantifier rank at most $n$, such that for every sentence $\sigma\in\Th(\mathfrak{M})$ it holds that $\qrank(\sigma)\leq n$ implies that $\sigma$ is logically equivalent to exactly one of $\sigma_0,\dotsc,\sigma_{k-1}$.  Without loss of generality we may assume that $\mathfrak{M}\models\sigma_i$ for $i=0,\dotsc,k-1$ as we could otherwise simply discard sentences false in $\mathfrak{M}$. By definition of a characteristic sentence, it then follows that $\cha{\mathfrak{M}}{n}$ is logically equivalent to $\sigma=\bigwedge_{0\leq i<k_0}\sigma_i$, which is the required sentence.
\end{proof}

In light of the previous proposition, when $L$ is a finitary first-order language and $n\in\nats$, for a class $\mathcal{S}$ of $L$-structures we will identify the $L_{\omega_1\omega}$-sentences $\bigvee_{\alpha\in\mathcal{S}}\cha{\alpha}{n}$ and $\bigwedge_{\alpha\in\mathcal{S}}\cha{\alpha}{n}$ with any of their respective first-order equivalents.


\section{Games between structures}

In what follows we will describe, for some arbitrary ordinal $\gamma$, the \textbf{Ehrenfeucht-Fra\"iss\'e game} $\EF_\gamma(\mathfrak{A},\mathfrak{B})$ of length $\gamma$ on $L$-structures $\mathfrak{A}$ and $\mathfrak{B}$.  In this game there are two players $\Left$ and $\Right$ (respectively pronounced ``Left'' and ``Right'').  This first move of the game is always made by $\Left$ and consists of some choice of element from either one of the structures $\mathfrak{A}$ or $\mathfrak{B}$.  The players alternate turns and on each of  $\Right$'s turns he is obliged to choose an element from the opposing structure i.e.\ from $\mathfrak{B}$ if $\Left$'s last move was an element in $\mathfrak{A}$ or vice-versa.

\begin{dfn}[Play of a game]
	A pair $(\bar{a},\bar{b})$, consisting of transfinite sequences $\bar{a}=\family{a_i}{i<\gamma}$ and $\bar{b}=\family{b_i}{i<\gamma}$, is called a \textbf{play} of the aforementioned game whenever, for each $i<\gamma$, $a_i\in\domain{}\mathfrak{A}$ and $b_i\in\domain{}\mathfrak{B}$ are the elements played during the $i$-th round of the game $\EF_\gamma(\mathfrak{A},\mathfrak{B})$.
\end{dfn}

\begin{dfn}[Winning play]
	A play $(\bar{a},\bar{b})$ of the game $\EF_\gamma(\mathfrak{A},\mathfrak{B})$ is said to be \textbf{a winning play for} $\bm{\mathcal{R}}$ iff there exists an isomorphism $f\colon\gen{\bar{a}}{\mathfrak{A}}\to\gen{\bar{b}}{\mathfrak{B}}$ such that $f[\bar{a}]=f[\bar{b}]$.
\end{dfn}

\begin{dfn}[Back-and-forth equivalence]
	We define $\mathfrak{A}\sim_\gamma\mathfrak{B}$ for an ordinal $\gamma$ if there exists a winning strategy for $\Right$ in the game $\EF_\gamma(\mathfrak{A},\mathfrak{B})$.  For the special case $\gamma=\omega$: if $\mathfrak{A}\sim_\omega\mathfrak{B}$ then it is said that $\mathfrak{A}$ and $\mathfrak{B}$ are \textbf{back-and-forth equivalent}.
\end{dfn}




\section{Linear orders and coloured expansions}

\begin{assn}
	From this section onwards all languages $L$ will be assumed to be first-order unless explicitly stated otherwise.
\end{assn}

\begin{dfn}[Dense linear order]
	A linear order $\alpha$ is called \textbf{dense} whenever, for every $a,b\in\alpha$, there exists $a\in\alpha$ such that $a<c<b$.  Additionally, $\alpha$ is said to be \textit{trivially dense} if the order type of alpha is either $\zero$ or $\one$.
\end{dfn}

\begin{dfn}[Coloured linear order]
	If $\alpha$ is a linear order and $\bar{r}$ is a (finite) tuple $(r_0,\dotsc,r_{k-1})$ of unary relations on $\alpha$ then the expansion $(\alpha,\bar{r})$ is said to be a $\mathbf{k}$\textbf{-coloured linear order}.
\end{dfn}

\begin{dfn}[Spectrum]
	Suppose $\mathcal{A}$ is a class of $k$-coloured linear orders.  We call $F$ an $\mathbf{n}$\textbf{-spectrum} for the class $\mathcal{A}$ when $F$ is a minimal set of linear orders with the property that for every $\alpha\in\mathcal{A}$ there exists a $\chi\in F$ such that $\alpha\nequiv{n}\chi$.
\end{dfn}


\section{Basic operations on linear orders}

\begin{lem}\label{lem:fvsum}
	Suppose $\family{\alpha_i}{i\in I}$ and $\family{\beta}{i\in I}$ are families of (possibly coloured) linear orders, indexed by a linearly ordered set $I\neq\emptyset$.  Now fix some $n\in\nats$ and assume $\alpha_i\nequiv{n}\beta_i$, for each $i\in J$, then it follows that
	\begin{equation}
		\sum_{i\in I}\alpha_i\nequiv{n}\sum_{i\in I}\beta_i.
	\end{equation}
\end{lem}


\section{Lattices}

\begin{thm}[Knaster-Tarski Theorem]
	Suppose $\Lambda$ is a complete lattices.  If $h$ is some endomorphism on $\Lambda$ then there must exist a (least) $x_0\in \Lambda$ such that $x_0$ is a fix0ed point of $h$, i.e.\ $h(x_0)=x_0$.
\end{thm}
\begin{proof}
	We start by exhibiting a fixed point of $h$.  Since $\Lambda$ is nonempty there exists a top $1\in\Lambda$, since by completeness, taking the join of $\Lambda$ we get an element $1=\bigvee\Lambda$.  Since $h$ is isotone we may conclude that $h(1)=1$.

	We are now obligated to find a least fixed point of $h$ so define
	\begin{equation}
		S=\setbuild{x\in\Lambda}{h(x)=x}
	\end{equation}  then clearly $S\neq\emptyset$ since $1\in S$.  Define $x_0=\bigwedge S$ and note that, since $x_0\leq x$ for each $x\in S$, by definition of isotonicity it follows that $h(x_0)\leq x$, for $x\in S$. Therefore $h(x_0)\leq\bigwedge S=x_0$.  Consequently, since $h(x_0)$ is a lower bound of $S$.  Since $x_0$ is the infimum of $S$ we cannot have $h(x_0)<x_0$.  Thus the only remaining possibility is $h(x_0)=x_0$, implying that $x\in S$.
\end{proof}


\section{Categories and functors}


\section{Galois connections}



\section{Decidability}

	\begin{prp}
		If $\Sigma$ is a set of $L$-sentences then $\Sigma$ is decidable set of sentences iff $\Sigma$ and $L_0\setminus\Sigma$, each, are recursively enumerable.
	\end{prp}

	\begin{prp}\label{prp:sdth}
		Suppose $\mathcal{A}$ is a class of structures.  If $\Sigma$ is a recursively enumerable set of sentences and its deductive closure $\dcl{\Sigma}$ satisfies
		\begin{equation}
			\dcl{\Sigma}=\Th(\mathcal{A}),
		\end{equation}
		then $\Th(\mathcal{A})$ is recursively enumerable.
	\end{prp}

	\begin{dfn}[Encoding]
		An \textbf{encoding} $p$ of a countable set $X$ is an injective recursive partial function $p\colon X\pto\nats$.  If $X$ is a first order language then $p$ is in stead referred to as a \textit{G\"odel numbering}.
	\end{dfn}

	\begin{dfn}[Frameworks]
		Suppose for some $A\subseteq X$, where $X$ is of (coloured) linear orders, there exists a family $\family{\Gamma_{alpha,\beta}}{(\alpha,\beta)\in A\times X}$
		\begin{enumerate}
			\item	$p\colon X\to\nats$ is an encoding of the (countable) set of coloured linear orders,
			\item 	the binary relation $\preceq$ is a well-founded preorder defined on $X$,
			\item	the map $\rho\colon X\to\nats$ is the rank function associated with $(X,\preceq)$.  If, for each $x,y\in X$, it holds that
		\end{enumerate}
	\end{dfn}

	\begin{dfn}[Proof sequences]
		Suppose $X$ is a countable set of countable coloured linear orders and let $p\colon X\pto\nats$ be an encoding of $X$.  A \textbf{proof sequence} is then a finite sequence
		\begin{equation}
			(\alpha_0,\sigma_0),\dotsc,(\alpha_{n-1},\sigma_{n-1})\in X\times L_0
		\end{equation}
		and finite expansions $\alpha^\prime_0,\dotsc,\alpha^\prime_{n-1}$ of the (respective) linear orders $\alpha_0,\dotsc,\alpha_{n-1}$ so that, for each $i=1,\dotsc,n-1$, there exists an interpretation $\Gamma_i$ of $\alpha_{i-1}$ in $\alpha^\prime_i$.  Additionally, when $0\leq i<n$, we require that one of the following holds:
		\begin{enumerate}
			\item	$\sigma_i$ is a logical axiom,
			\item	$\sigma_i$ follows via an inference rule from $\sigma_0,\dotsc,\sigma_{i-1}$,
			\item	$\sigma_i\in\admis(\Gamma_i)$ or
			\item	$\sigma_i=\Gamma_i\Gamma_{i-1}\dotsb\Gamma_j\sigma_j$, for some natural $j<i$.
		\end{enumerate}
	\end{dfn}
\vfill
	\begin{verbatim}
		-------------------------------------------------------------------------
	\end{verbatim}
\noindent[\textbf{Alles in hierdie hoofstuk is tentatief.  Byvoegings en (veral) verwyderings sal gemaak word soos ek deur die tesis vorder}]
