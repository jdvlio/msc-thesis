\chapter{Preliminaries}


\section{Model theory and logic}

Our primitive (logical) symbols are $\bot$, $\rightarrow$, $\wedge$ and $\exists$ with their usual semantics.  We will also make use of the following symbols to abbreviate certain formulas in the way one would normally expect in (classical) first-order logic: $\top$, $\neg$, $\vee$ and $\forall$.  Since there is more than one way to do this

On occasion we might encounter infinite conjunctions and disjuctions by employing the (respective) symbols $\bigwedge$ and $\bigvee$, with or without subscripts.  These do not occur in the language of first-order logic as such but rather in its infinitary extensions $L_{\kappa\omega}$ with $\kappa$ an initial ordinal and $L$ ranging over all possible signatures.  For a fixed signature $L$ the notation $L_{\infty\omega}$ is also encountered in the literature to denote the (smallest) language containg $L_{\kappa\omega}$, for each initial ordinal $\kappa$.  An example of a minimalistic but infinitary extension of a first-order language would be the language $L_{\omega_1\omega}$ which also allows formulas to have denumerable conjunctions and disjunctions such as in $\bigvee_{i<\omega}\bigwedge_{j<\omega}\varphi_{ij}$.

\begin{conv}[Signatures and languages]
	We will make a habit of identifying the language $L_{\omega\omega}$ (the first-order language over the signature $L$) with its underlying signature as these uniquely determine each other.
\end{conv}

\begin{conv}
	We indentify any language $L_{\kappa\omega}$ with its corresponding set of formulas, allowing us to make assertions such as ``$\varphi\in L$'' in the case $L=L_{\omega\omega}$.
\end{conv}

\begin{dfn}[Finitary languages]
	A language $L$ is called \textbf{finitary} if $L$ is a language over a \text{finite} signature.
\end{dfn}

\begin{prp}\label{prp:finequiv}
	Suppose we fix any $n\in\nats$ and $L$ is a finitary first order language. If $v_0,\dotsc,v_{k-1}$ are distinct variables then, up to logical equivalence, there are only finitely many formulas $\varphi=\varphi(v_1,\dotsc,v_{k-1})$ such that $\qrank(\varphi)\leq n$.
\end{prp}

\begin{dfn}[Characteristic formulas]
	If $L$ is some finitary language, $\mathfrak{M}$ is some $L$-structure and $\bar{a}\in\domain{k}{\mathfrak{M}}$ then we define the \textbf{$\mathbf{n}$-characterstic formula} $\cha{\bar{a}}{n}$ of $\bar{a}$ relative to $\mathfrak{M}$, recursively, as follows:
	\begin{enumerate}
		\item $\cha{\bar{a}}{0}=\bigwedge\setbuild{\varphi(\bar{x})\in L_k}{\mathfrak{M}\models\varphi(\bar{a})}$;
		\item $\cha{\bar{a}}{n+1}=\bigwedge_{b\in\domain{}\mathfrak{M}}\exists v_n \cha{\bar{a}b}{n}\wedge\forall v_n\bigvee_{b\in\domain{}\mathfrak{M}}\cha{\bar{a}b}{n}$, for each $n\in\nats$.
	\end{enumerate}
	If $\bar{a}$ is the empty tuple, and $n\in\nats$, then we write $\cha{\mathfrak{M}}{n}$ for $\cha{\bar{a}}{n}$ and call $\cha{\mathfrak{M}}{n}$ the \textit{$\mathit{n}$-characteristic sentence} of the structure $\mathfrak{M}$.
\end{dfn}

\begin{prp}
	If $L_{\omega\omega}$ is a finitary language, $\mathfrak{M}$ is an $L$-structure and $n\in\nats$ then the $L_{\omega_1\omega}$-sentence $\cha{\mathfrak{M}}{n}$ is logically equivalent to an $L_{\omega\omega}$-sentence $\sigma$.
\end{prp}
\begin{proof}
	From proposition \ref{prp:finequiv} it follows that there exists sentences $\sigma_0,\dotsc,\sigma_{k-1}$, of quantifier rank at most $n$, such that for every sentence $\sigma\in\Th(\mathfrak{M})$ it holds that $\qrank(\sigma)\leq n$ implies that $\sigma$ is logically equivalent to exactly one of $\sigma_0,\dotsc,\sigma_{k-1}$.  Without loss of generality we may assume that $\mathfrak{M}\models\sigma_i$ for $i=0,\dotsc,k-1$ as we could otherwise simply discard sentences false in $\mathfrak{M}$. By definition of a characteristic sentence, it then follows that $\cha{\mathfrak{M}}{n}$ is logically equivalent to $\sigma=\bigwedge_{0\leq i<k_0}\sigma_i$, which is the required sentence.
\end{proof}

In light of the previous proposition, when $L$ is a finitary first-order language and $n\in\nats$, for a class $\mathcal{S}$ of $L$-structures we will identify the $L_{\omega_1\omega}$-sentences $\bigvee_{\alpha\in\mathcal{S}}\cha{\alpha}{n}$ and $\bigwedge_{\alpha\in\mathcal{S}}\cha{\alpha}{n}$ with any of their respective first-order equivalents.

\begin{lem}\label{lem:diagH}
	Suppose $\mathfrak{A}$ and $\mathfrak{B}$ are $L$-structures.  If $\bar{c}$ is a (possibly infinite) tuple of constant symbols and $\bar{a}\in A^{\ell(\bar{c})}$ and $\bar{b}\in B^{\ell(\bar{c})}$ then
	\begin{enumerate}
		\item 	for every atomic $\sigma\in L_0(\bar{c})$, if $(\mathfrak{A},\bar{a})\models\sigma$ then $(\mathfrak{B},\bar{b})\models\sigma$,
		\item  there exists a homomorphism $h\colon\gen{\bar{a}}{\mathfrak{A}}\to\mathfrak{B}$ such that $f[\bar{a}]=f[\bar{b}]$.
	\end{enumerate}
	Additionally, if $h$ exists then it is unique and, furthermore, $h$ is an embedding iff $(\mathfrak{A},\bar{a})\nequiv{0}(\mathfrak{B},\bar{b})$.
\end{lem}

\begin{lem}[Positive Diagram Lemma]\label{lem:diagp}
	Suppose $\mathfrak{A}$ and $\mathfrak{B}$ are $L$-structures.  If $f$ is a map $f\colon A\to B$ then the following are equivalent:
	\begin{enumerate}
		\item	$h$ is a homomorphism $h\colon\mathfrak{A}\to\mathfrak{B}$,
		\item	for each $n\in\nats$ and every atomic formula $\varphi\in L_n$ it holds that, whenever $\bar{a}\in A^n$,
			\begin{equation}
				\mathfrak{A}\models\varphi(\bar{a}) \quad\implies\quad \mathfrak{B}\models\varphi(h[\bar{a}])
			\end{equation}
		\item	it holds that $(\mathfrak{B},h[A])\models\diagp\mathfrak{A}$.
	\end{enumerate}
\end{lem}

\begin{lem}[Diagram Lemma]\label{lem:diag}
	Suppose $\mathfrak{A}$ and $\mathfrak{B}$ are $L$-structures.  If $f$ is a map $f\colon A\to B$ then the following are equivalent:
	\begin{enumerate}
		\item	$h$ is an embedding $h\colon\mathfrak{A}\hookrightarrow\mathfrak{B}$,
		\item	for each $n\in\nats$ and every quantifier-free formula $\varphi\in L_n$ it holds that, whenever $\bar{a}\in A^n$,
			\begin{equation}
				\mathfrak{A}\models\varphi(\bar{a}) \quad\iff\quad \mathfrak{B}\models\varphi(h[\bar{a}])
			\end{equation}
		\item	it holds that that $(\mathfrak{B},h[A])\models\diag\mathfrak{A}$.
	\end{enumerate}
\end{lem}

\begin{dfn}[Quantifier elimination]
	An $L$-theory $T$ is said to \textbf{admit quantifier elimination} (or \textit{has} quantifier elimination) if for every $k\in\nats$ and every $\varphi\in L_k$ there exists a quantifier-free formula $\psi\in L_k$ such that $\varphi$ is $T$-equivalent (or equivalent modulo $T$) to $\psi$ i.e.\ $T\models\forall\bar{x}(\varphi(\bar{x})\leftrightarrow\psi(\bar{x}))$.
\end{dfn}

\begin{dfn}[Substructure-completeness]
	A first-order theory $T$, in some language $L$, is said to be \textbf{substructure-complete} whenever for every model $\mathfrak{M}$ of $T$ and every substructure $\mathfrak{A}$ of $\mathfrak{M}$ it necessarily holds that $T\cup\diag(\mathfrak{A})$ is a complete $L(A)$-theory.
\end{dfn}

\begin{dfn}[Literal]
	A formula $\varphi\in L$ is called a \textbf{literal} whenever it is either an atomic $L$-formula or, otherwise, the negation of such an atomic formula.
\end{dfn}

\begin{dfn}[Simply primitive]
	A formula $\varphi(\bar{x})\in L_n$, for some $n\in\posnats$, is said to be \textbf{simply primitive} whenever there exists literals $\varphi_0(\bar{x},y),\dotsc,\varphi_{n-1}(\bar{x},y)\in L_n$ such that $\varphi(\bar{x})=\exists y\bigwedge_{0\leq i<n}\varphi_i(\bar{x},y)$.  We denote the class of simply primitive formulas as $\E{}^\ast$.
\end{dfn}

\begin{prp}\label{prp:sprm}
	An $L$-theory $T$ admits quantifier elimination iff every simply primitive $L$-formula is $T$-equivalent to a quantifier-free formula.
\end{prp}
\begin{proof}
	\forward Trivial.

	\backward We argue on the complexity of the formula $\varphi\in L$ that it is $T$-equivalent to a quantifier-free formula.  If $\varphi$ is atomic then it trivially follows that $\varphi$ is quantifier-free.  Note $\bot$ is quantifier-free as it is atomic and suppose now that $\varphi_0$ and $\varphi_1$ are, respectively, $T$-equivalent to quantifier-free formulas $\psi_0,\psi_1\in L$.  It follows  by definition that the formulas $\varphi_0\rightarrow\varphi_1$ and $\varphi_0\wedge\varphi_1$ are $T$-equivalent to the respective quantifier-free formulas $\psi_0\rightarrow\psi_1$ and $\psi_0\wedge\psi_1$.  Lastly, note that $\exists y\varphi_0$, by the induction hypothesis, is $T$-equivalent to the simply primitive formula $\exists y\psi_0$.  As simply primitive $L$-formulas (by assumption) are $T$-equivalent to quantifier-free formulas, we may conclude that $\exists y\psi_0$, and thus $\exists y\varphi_0$, is $T$-equivalent to a quantifier-free formula.
\end{proof}

\begin{dfn}[Boolean closure]
	If $\Phi\subseteq L$ is a set of $L$-formulas then its \textbf{boolean closure} (denoted as $\widetilde{\Phi}$) is the smallest set of $L$-formulas such that $\widetilde{\Phi}\supseteq\Phi\cup\set{\bot,\top}$ and is closed under negation, conjunction and disjunction i.e.\ if $\varphi,\psi\in\widetilde{\Phi}$ then it must follow that $\set{\neg\varphi, \varphi\wedge\psi,\varphi\vee\psi}\subseteq\widetilde{\Phi}$.
\end{dfn}

\begin{rem}
	Note in the definition of a boolean closure that the condition that $\widetilde{\Phi}$ be closed under disjunctions is redundant.
\end{rem}

\begin{lem}\label{lem:bcls}
	Let $T$ be an $L$-theory and fix some $\sigma\in L_0$.  If $\Sigma$ is an $L$-theory then $\sigma$ is $T$-equivalent to a sentence from $\widetilde{\Sigma}$ iff it holds for every $\mathfrak{M},\mathfrak{N}\models T$ that:
	\begin{equation}
		\mathfrak{M}\equiv_\Sigma\mathfrak{N} \quad\implies\quad \mathfrak{M}\equiv_\sigma\mathfrak{N}
	\end{equation}
\end{lem}
\begin{proof}
	\forward  Note that no generality is lost by assuming that $\sigma\in\widetilde{\Sigma}$.  As the case $\sigma\in\set{\bot,\top}$ is trivial, suppose not.  We argue by induction on the complexity of $\sigma$.  If $\sigma\in\Sigma$ then it follows by definition that $\mathfrak{M}\equiv_\Sigma\mathfrak{N}$.  Assume now that $\sigma_0,\sigma_1\in\widetilde{\Sigma}$ and either $\sigma=\neg\sigma_0$, $\sigma=\sigma_0\wedge\sigma_1$, where (for $i=0,1$) it holds, for $\mathfrak{M},\mathfrak{N}\models T$, that
	\begin{equation}
		\mathfrak{M}\equiv_\Sigma\mathfrak{N} \quad\implies\quad \mathfrak{M}\equiv_{\sigma_i}\mathfrak{N}.
	\end{equation}
	In both cases it necessarily then follows from the usual definition of Tarski semantics that, for $\mathfrak{M},\mathfrak{N}\models T$, we must have
	\begin{equation}
		\mathfrak{M}\equiv_\Sigma\mathfrak{N} \quad\implies\quad \mathfrak{M}\equiv_\sigma\mathfrak{N}.
	\end{equation}

	\backward  Consider the set $S_L$ of complete $L$-theories, under its usual Stone topology, and define the sets $S=\gen{\sigma}{}\cap\bigcap_{\sigma^\prime\in T}\gen{\sigma^\prime}{}$ and $S^\prime=\gen{\neg\sigma}{}\cap\bigcap_{\sigma^\prime\in T}\gen{\sigma^\prime}{}$.  Note, if $T\cup\set{\sigma}$ is inconsistent then it follows that $T\models\neg\sigma$ and thus $\sigma$ is $T$-equivalent to $\bot$ so that the result follows trivially.  Similarly,  if $T\cup\set{\neg\Sigma}$ is inconsistent then $\sigma$ is $T$-equivalent to $\top$ and the result again follows trivially.  Now assume both $T\cup\set{\sigma}$ and $T\cup\set{\neg\sigma}$ are consistent so that $S\neq\emptyset\neq S^\prime$.

	Note now that, for every $T_0\in S$ and $T_1\in S^\prime$, it follows our assumption (by contraposition) that if $\mathfrak{M}_0\models T_0$ and $\mathfrak{M}_1\models T_1$ then there exists a $\sigma_{T_0,T_1}\in\Sigma$ such that $\mathfrak{M}_0\models\sigma_{T_0,T_1}$ but $\mathfrak{M}_1\models\neg\sigma_{T_0,T_1}$.  Note that, for every $T_0\in S$, the set $\setbuild{\gen{\neg\sigma_{T_0,T^\prime}}{}}{{T^\prime\in S^\prime}}$ is an open cover of $S^\prime$.  Since $S^\prime$ is an intersection of closed sets, it now follows that $S^\prime$ itself is closed and therefore must be compact as it is a closed subset of the compact topological space $S_L$.  By definition of compactness, there must then exist $T^\prime_0,\dotsc,T^\prime_{k-1}\in S^\prime$ (for some $k\in\posnats$) such that the set $\set{\gen{\neg\sigma_{T_0,T^\prime_0}}{},\dotsc,\gen{\neg\sigma_{T_0,T^\prime_{k-1}}}{}}$ is an open cover of $S^\prime$.  We may now conclude that
	\begin{equation}
		T\cup\set{\neg\sigma}\models\bigvee_{i<k}\neg\sigma_{T_0,T^\prime_i},
	\end{equation}
	and therefore
	\begin{equation}
		T\models\neg\sigma\rightarrow\bigvee_{i<k}\neg\sigma_{T_0,T^\prime_i}.\label{eq:tmnsr}
	\end{equation}
	Now, for each $T_0\in S$, define $\sigma_{T_0}=\bigwedge_{i<k}\sigma_{T_0,T^\prime_i}$ and note that $\sigma_{T_0}\in\widetilde{\Sigma}$ as it is a conjunction of sentences from $\Sigma\subseteq\widetilde{\Sigma}$.  Note also, by contraposition, that
	\begin{equation}
		T\models\sigma_{T_0}\rightarrow\sigma.
	\end{equation}
	Note now for every $T_0\in S$, and each natural $i<k$, that we necessarily have $T_0\models\sigma_{T_0,T_i^\prime}$.  Consequently $T_0\models\sigma_{T_0}$, for each $T_0\in S$, and therefore the set $\setbuild{\gen{\sigma_{T_0}}{}}{T_0\in S}$ is an open cover of $S$.  Since $S$ is a closed, and therefore compact, subset of $S_L$ it follows that there exists a finite subcover $\set{\gen{\sigma_{T^\pprime_0}}{},\dotsc,\gen{\sigma_{T^\pprime_{m-1}}}{}}$ of the aforementioned open cover.  Consequently,
	\begin{equation}
		T\cup\set{\sigma}\models\bigvee_{i<m}\sigma_{T^\pprime_i}.
	\end{equation}
	We may now assume, without loss of generality, it holds that $T\cup\set{\sigma}\models\sigma_{T^\pprime_0}$ and therefore
	\begin{equation}
		T\models\sigma\rightarrow\sigma_{T^\pprime_0}.
	\end{equation}
	Since it has already been determined that $T\models\sigma_{T_0^\pprime}\rightarrow\sigma$ and it holds that $\sigma_{T^\pprime_0}\in\widetilde{\Sigma}$, the result then follows.
\end{proof}

\begin{prp}\label{prp:bcls}
	Let $T$ be an $L$-theory and, for any $n\in\nats$, fix some $\varphi\in L_n$.  If $\Delta\subseteq L_n$ then $\varphi$ is $T$-equivalent to a formula from $\widetilde{\Delta}$ iff the following holds for every $\mathfrak{M},\mathfrak{N}\models T$ and each $\bar{a}\in M^n$ and $\bar{b}\in N^n$:
	\begin{equation}
		 \mathfrak{M}\models\delta(\bar{a})\text{ iff }\mathfrak{N}\models\delta(\bar{b})\text{, for all $\delta\in\Delta$,} \implies \mathfrak{M}\models\varphi(\bar{a})\text{ iff }\mathfrak{N}\models\varphi(\bar{b}).
	\end{equation}
\end{prp}
\begin{proof}
	Let $n\in\nats$ and suppose that $\varphi=\varphi(\bar{x})\in L_n$ and $\Delta=\Delta(\bar{x})\subseteq L_n$.  Consider now a tuple $\bar{c}$ of constant symbols which is of length $n$.  After convincing oneself that $\widetilde{\Delta(\bar{c})}=\widetilde{\Delta}(\bar{c})$, that is to say that the operation of taking a boolean closure of a set of formulas and that of substition by constant symbols commute, it follows from lemma \ref{lem:bcls} that $\varphi(\bar{c})$ is equivalent to a sentence from $\widetilde{\Delta}(\bar{c})$ iff
	\begin{equation}
		\mathfrak{M}^\prime\equiv_{\Delta(\bar{c})}\mathfrak{N}^\prime \quad\implies\quad \mathfrak{M}^\prime\equiv_{\varphi(\bar{c})}\mathfrak{N}^\prime,
	\end{equation}
	for all $L(\bar{c})$-expansions $\mathfrak{M}^\prime$ and $\mathfrak{N}^\prime$ of the respective $L$-structures $\mathfrak{M},\mathfrak{N}\models T$.  Redressing this equivalence purely in terms of $L$-formulas and $L$-structures (by taking their $L$-reducts) then yields the desired result.
\end{proof}

\begin{dfn}
	Recursively define the following classes of $L_{\infty\omega}$ formulas, with $L$ ranging over all signatures:
	\begin{enumerate}
		\item	$\E{0}=\A{0}$ is the class of formulas $\varphi$ such that $\varphi$ is a quantifier-free formula in some first-order language $L$,
		\item	if $n\in\nats$, $\E{n+1}\supseteq\A{n}$ is defined to the smallest class of formulas, closed under $\bigwedge$ and $\bigvee$, such that $\exists\bar{x}\varphi\in\E{n+1}$ whenever $\bar{x}$ is finite tuple of variables and $\varphi\in\E{n+1}$,
		\item	if $n\in\nats$, $\A{n+1}\supseteq\E{n}$ is defined to be the smallest class of formulas, closed under $\bigwedge$ and $\bigvee$, such that $\forall\bar{x}\varphi\in\A{n+1}$ whenever $\bar{x}$ is finite tuple of variables and $\varphi\in\A{n+1}$.
	\end{enumerate}
	In particular, members of $\E{1}$ are known as the \textbf{existential formulas} while the members of $\A{1}$ are referred to as \textbf{universal formulas}.
\end{dfn}

\begin{lem}\label{lem:qelim}
	The following are equivalent for a consistent $L$-theory $T$:
	\begin{enumerate}
		\item	$T$ admits quantifier elimination,
		\item	$T$ is a substructure-complete theory,
		\item	if $\mathfrak{A}$ is finitely generated, $f\colon\mathfrak{A}\to\mathfrak{M}$ and $g\colon\mathfrak{A}\to\mathfrak{N}$ are embeddings and $\mathfrak{M},\mathfrak{N}\models T$ then, for every $n\in\nats$ and each simply primitive $\varphi\in L_n(A)$, it holds for every $\bar{a}\in A^n$ that
		\begin{equation}
			\mathfrak{M}\models\varphi(f[\bar{a}]) \quad\iff\quad \mathfrak{N}\models\varphi(g[\bar{a}]).\label{eq:qelim}
		\end{equation}
	\end{enumerate}
\end{lem}
\begin{proof}
	1$\Rightarrow$2:  Choose any $\mathfrak{M}\models T$ and any $\mathfrak{A}\subseteq\mathfrak{M}$.  Note, since $T$ has quantifier elimination, it follows from the diagram lemma (choosing $h$ as the inclusion map) that
	\begin{equation}
		\Th(\mathfrak{M},A)\subseteq T\cup\diag\mathfrak{A}.
	\end{equation}
	Since the former is complete it immediately follows that the latter is complete as well.

	2$\Rightarrow$3:  Suppose $\mathfrak{A}$ is a finitely generated structure and let $f\colon\mathfrak{A}\hookrightarrow\mathfrak{M}$ and $g\colon\mathfrak{A}\hookrightarrow\mathfrak{N}$ be embeddings into the respective models $\mathfrak{M}$ and $\mathfrak{N}$ of $T$.  Without loss of generality we may assume $f$ and $g$ are inclusion maps and therefore $\mathfrak{A}\subseteq\mathfrak{M},\mathfrak{N}$.  By assumption the $L(A)$-theory $T\cup\diag\mathfrak{A}$ is complete.  Therefore, since $(\mathfrak{M},A)\models T\cup\diag\mathfrak{A}$ and $(\mathfrak{N},A)\models T\cup\diag\mathfrak{A}$, it follows that $(\mathfrak{M},A)\equiv(\mathfrak{N},A)$.  By definition, this is equivalent to (\ref{eq:qelim}).

	3$\Rightarrow$1:  By Proposition \ref{prp:sprm}, it is enough to show that every simply primitive formula is $T$-equivalent to a quantifier-free formula.  In turn, Proposition \ref{prp:bcls} reduces this to showing, for each $n\in\nats$, that if $\bar{c}$ is an $n$-tuple of constant symbols and $\mathfrak{M},\mathfrak{N}\models T$ are $L$-structures then all of their (respective) $L(\bar{c})$-expansions $\mathfrak{M}^\prime$ and $\mathfrak{N}^\prime$ satisfy
	\begin{equation}
		\mathfrak{M}^\prime\equiv_{\A{0}}\mathfrak{N}^\prime \quad\implies\quad \mathfrak{M}^\prime\equiv_{\E{}^\ast}\mathfrak{N}^\prime.
	\end{equation}

	Let $\mathfrak{M},\mathfrak{N}\models T$ be $L$-structures and choose, for an arbitrary $n\in\nats$, any $\bar{a}\in\domain{n}\mathfrak{M}$ and any $\bar{b}\in\domain{n}\mathfrak{N}$ such that
	\begin{equation}
		(\mathfrak{M},\bar{a})\equiv_{\A{0}}(\mathfrak{N},\bar{b}).
	\end{equation}
	Define the structures $\mathfrak{A}=\gen{\bar{a}}{\mathfrak{M}}$ and $\mathfrak{B}=\gen{\bar{b}}{\mathfrak{N}}$ and note, by definition, that both of them are finitely generated.  From lemma \ref{lem:diagH}, it follows that there exists an embedding $h\colon\mathfrak{A}\to\mathfrak{N}$ such that $h[\bar{a}]=h[\bar{b}]$.  Since $\mathfrak{B}$ is generated by $\bar{b}=h[\bar{a}]$, we may consider $h$ to be an isomorphism $h\colon\mathfrak{A}\to\mathfrak{B}$.  Therefore, we may identify the structures $\mathfrak{A}$ and $\mathfrak{B}$ so that the desired conlusion then follows from (3) by choosing $f$ and $g$ to be the respective inclusion maps $f\colon\mathfrak{A}\hookrightarrow\mathfrak{M}$ and $g\colon\mathfrak{A}\hookrightarrow\mathfrak{N}$.
\end{proof}

\begin{dfn}[Model-completeness]
	An $L$-theory $T$ is said to be \textbf{model-complete} if every embedding $f\colon\mathfrak{M}\to\mathfrak{N}$, between models $\mathfrak{M}$ and $\mathfrak{N}$ of $T$, is elementary.
\end{dfn}

\begin{prp}
	The following are equivalent for an $L$-theory $T$:
	\begin{enumerate}
		\item	$T$ is model-complete,
		\item	for all $\mathfrak{N}\models T$, if $\mathfrak{M}\subseteq\mathfrak{N}$ and $\mathfrak{M}\models T$ then $T\cup\diag(\mathfrak{M})$ is a complete $L(M)$-theory.
	\end{enumerate}
\end{prp}
\begin{proof}
	1$\Rightarrow$2: Suppose $\mathfrak{M},\mathfrak{N}\models T$ and $\mathfrak{M}\subseteq\mathfrak{N}$.  Note now that for any $L(M)$-structure $\mathfrak{A}^\prime\models T\cup\diag(\mathfrak{M})$ if $\mathfrak{A}$ is the $L$-reduct of $\mathfrak{A}^\prime$ then there exists an embedding $h\colon\mathfrak{M}\hookrightarrow\mathfrak{A}$ such that $\mathfrak{A}^\prime=(\mathfrak{A},h[M])$.  From (1) it follows that $h$ is elementary and thus $(\mathfrak{M},M)\equiv\mathfrak{A}^\prime$.  Since $\mathfrak{A}^\prime$ was arbitrary it follows that any two models of $T\cup\diag(\mathfrak{M})$ will be elementarily equivalent to $(\mathfrak{M},M)$ and thus to each other.

	2$\Rightarrow$1:  Suppose $\mathfrak{M},\mathfrak{N}\models T$ and let $f\colon\mathfrak{M}\hookrightarrow\mathfrak{N}$ be an embedding.  We are now required to show that $f$ is elementary.  By the Elementary Diagram Lemma this amounts to showing that
	\begin{equation}\label{eq:mc2}
		(\mathfrak{N},f[M])\models\Th(\mathfrak{M},M).
	\end{equation}
	Noting that, since $f$ is an embedding, it follows from the Diagram Lemma that $(\mathfrak{N},f[M])\models T\cup\diag{\mathfrak{M}}$.  Also, since $(\mathfrak{M},M)\models T\cup\diag{\mathfrak{M}}$ and $T\cup\diag{\mathfrak{M}}$ is complete it follows that the deductive closure of $T\cup\diag{\mathfrak{M}}$ is the complete theory $\Th(\mathfrak{M},M)$.  As a consequence of these observations, (\ref{eq:mc2}) follows immediately.
\end{proof}

\begin{cor}\label{cor:qemc}
	If $T$ is a first-order theory that admits quantifier elmination then $T$ is model-complete.
\end{cor}
\begin{proof}
	Follows immediately from the definition of model-completeness and the implication (2)$\Rightarrow$(1) in the previous proposition.
\end{proof}

\begin{dfn}[$n$-Type]
	Let $\mathfrak{A}$ be an $L$-structure and $n\in\nats$ while supposing, for some $A_0\subseteq A$, that $\Phi=\Phi(\bar{x})\subseteq L_n(A_0)$ is deductively closed and closed under finite conjunctions.  We call $\Phi$ an $\mathbf{n}$\textbf{-type of }$\bm{\mathfrak{A}}$\textbf{ over }$\bm{A_0}$ iff there exists some $\mathfrak{B}\succcurlyeq\mathfrak{A}$ and a $\bar{b}\in B^n$ such that $\mathfrak{B}\models\Phi(\bar{b})$ i.e.\ $\mathfrak{B}\models\varphi(\bar{b})$, for every $\varphi\in\Phi(\bar{x})$.  Additionally, if $A_0=\emptyset$ then $\Phi$ is also referred to as an $\mathbf{n}$\textbf{-type of the complete theory }$T=\Th(\mathfrak{A})$.
\end{dfn}

\begin{dfn}[Realisation]
	A tuple $\bar{b}$ of length $n$ in a structure $\mathfrak{B}\succcurlyeq\mathfrak{A}$ is said to \textbf{realise} a type $\Phi(\bar{x})\subseteq L_n(A)$ of a structure $\mathfrak{A}$ iff $\mathfrak{B}\models\Phi(\bar{b})$.
\end{dfn}

\begin{dfn}[Isolated and principal types]
	An $n$-type $\Phi(\bar{x})\subseteq L_n(A)$ of an $L$-structure $\mathfrak{A}$ is called \textbf{isolated} whenever there exists a formula $\varphi_0\in L_n(A)$, satisfiable in $\mathfrak{A}$, such that $(\mathfrak{A},A)\models\forall\bar{x}(\varphi_0(\bar{x})\rightarrow\varphi(\bar{x}))$, for each $\varphi\in\Phi(\bar{x})$.  If $\varphi_0$ can be chosen such that $\varphi_0\in\Phi(\bar{x})$ then $\Phi$ is also said to be \textbf{principal}.
\end{dfn}

\begin{dfn}[Saturation]
	Given any infinite cardinal $\kappa$, an $L$-structure $\mathfrak{A}$ is called $\bm{\kappa}$\textbf{-saturated} whenever $\kappa\leq\card{\mathfrak{A}}$ and each $1$-type of $\mathfrak{A}$, over any $A_0\subseteq A$ such that $\card{A_0}<\kappa$, is realised by some element in $\mathfrak{A}$.  We call $\mathfrak{A}$ \textbf{saturated} iff it is $\card{\mathfrak{A}}$-saturated.
\end{dfn}

\begin{prp}\label{prp:sat0}
	If $\mathfrak{A}$ is an $\aleph_0$-saturated $L$-structure then, for each $n\in\posnats$, $\mathfrak{A}$ realises all $n$-types over subsets of finite cardinality.
\end{prp}
\begin{proof}
	We argue by induction on $n$.  The case $n=1$ follows immediately from the definition of $\aleph_0$-saturation so assume for every $k\leq n$ it holds that $\mathfrak{A}$ realises all $k$-types of $\mathfrak{A}$ over any of its finite subsets.  Let $\Phi(\bar{x})\subseteq L_{n+1}(A_0)$ be an $(n+1)$-type of $\mathfrak{A}$, for some finite $A_0\subseteq A$.  By definition, there must exist an elementary extension $\mathfrak{B}\succcurlyeq\mathfrak{A}$ and some $\bar{b}\in B^{n+1}$ such that $\mathfrak{B}\models\Phi(\bar{b})$.  Now define $\Phi^\prime(x_n)\subseteq L_1(A_0)$ to be the deductive closure of the set
	\begin{equation}
		\setbuild{\exists x_0\dotso\exists x_{n-1}\varphi(x_0,\dotsc,x_n)}{\varphi\in\Phi}.
	\end{equation}

	Supposing that $\bar{b}=(b_0,\dotsc,b_n)$, note that $\mathfrak{B}\models\Phi^\prime(b_n)$ and thus $\Phi^\prime$ is a $1$-type of $\mathfrak{A}$ over $A_0$ and consequently, since $\mathfrak{A}$ is $\aleph_0$-saturated, there exists an $a_n\in A$ such that $\mathfrak{A}\models\Phi^\prime(a_n)$.  Aiming to show that $\Phi(x_0,\dotsc,x_{n-1},a_n)$ is an $n$-type of $\mathfrak{A}$ over $A_1\coloneqq A_0\cup\set{a_n}$, let $\bar{c}$ be an $n$-tuple of constant symbols.  Note that $\Phi(x_0,\dotsc,x_{n-1},a_n)$ is an $n$-type of $\mathfrak{A}$ iff $\Phi(\bar{c},a_n)\cup\Th(\mathfrak{A})$ is a consistent $L(A_1)$-theory.  Also, since $\Phi$ is closed under conjunctions, if $\psi=\psi(x_0,\dotsc,x_n)$ is a finite conjunction of formulas from $\Phi$ then $\psi\in\Phi(\bar{x})$ and thus $\exists x_0\dotso\exists x_{n-1}\psi\in\Phi^\prime(x_n)$.  Therefore, since $\psi$ was arbitrary and $\mathfrak{A}\models\Phi^\prime(a_n)$, it follows that $T\coloneqq\Th(\mathfrak{A})\cup\Phi(\bar{c},a_n)$ is a finitely satisfiable $L(A_1)$-theory so that, by the compactness theorem, $T$ must have an $L(A_1\cup\bar{c})$-model $(\mathfrak{M},\bar{d}\string^a_n)$, for some $L(A_0)$-structure $\mathfrak{M}$ and some $\bar{d}\in M^n$.  By definition it follows that $\mathfrak{M}\succcurlyeq\mathfrak{A}$ is an $L(A_0)$-structure such that $(\mathfrak{M},a_n)\models\Phi(\bar{d}\string^a_n)$.  Therefore, $\Phi(x_0,\dotsc,x_{n-1},a_n)$ is an $n$-type over $A_1$ which, by the induction hypothesis, is realised by some $(a_0,\dotsc,a_{n-1})\in A^n$ and thus if $\bar{a}=(a_0,\dotsc,a_{n-1},a_n)$ then $\mathfrak{A}\models\Phi(\bar{a})$.
\end{proof}

\begin{prp}
	For any ordinal $\alpha$, if $\mathfrak{A}$ is an $\aleph_\alpha$-saturated $L$-structure then, for each $n\in\posnats$ and any ordinal $\beta<\alpha$, $\mathfrak{A}$ realises all $n$-types over subsets of cardinality $\aleph_\beta$.
\end{prp}
\begin{proof}
	The case $\alpha=0$ is given by Proposition \ref{prp:sat0}.  Consider now the case $\alpha>0$ and note that if $A_0\subseteq A$ and $\card{A_0}=\aleph_\beta$, for some ordinal $\beta<\alpha$, then it follows that $\card{A_0\cup\set{a}}=\card{A_0}=\aleph_\beta$.  By leveraging this property of infinite cardinals, an induction argument similar to that in the proof of Proposition \ref{prp:sat0} yields the desired result.
\end{proof}

\begin{prp}
	If $\Phi(\bar{x})\subseteq L_n(A)$ is an isolated type of the nonempty $L$-structure $\mathfrak{A}$ then there exists an $\bar{a}\in A^n$ which realises $\Phi$.
\end{prp}
\begin{proof}
	Suppose $\Phi(\bar{x})$ is an $n$-type of $\mathfrak{A}$ over $A$, for some $n\in\posnats$.  Since $\Phi$ is isolated, there exists a $\varphi_0\in L_n(A)$ such that $\varphi_0(\bar{x})$ is satisfiable in $\mathfrak{A}$ and $\mathfrak{A}\models\forall\bar{x}(\varphi_0(\bar{x})\rightarrow\varphi(\bar{x}))$ for each $\varphi\in\Phi(\bar{x})$.  By definition, there must exist some $\bar{a}\in A^n$ such that $\mathfrak{A}\models\varphi_0(\bar{a})$.  Consequently, it follows from the definition of $\varphi_0$ that $\mathfrak{A}\models\Phi(\bar{a})$, as required.
\end{proof}

\begin{dfn}[Atomicity]
	An $L$-structure $\mathfrak{A}$ is called \textbf{atomic} whenever the only types over $\emptyset$ which are realised in $\mathfrak{A}$ are isolated.
\end{dfn}


\section{Games on structures}

In what follows we will describe, for some arbitrary ordinal $\gamma$, the \textbf{Ehrenfeucht-Fra\"iss\'e game} $\EF_\gamma(\mathfrak{A},\mathfrak{B})$ of length $\gamma$ on $L$-structures $\mathfrak{A}$ and $\mathfrak{B}$.  In this game there are two players $\Left$ and $\Right$ (respectively pronounced ``Left'' and ``Right'').  This first move of the game is always made by $\Left$ and consists of some choice of element from either one of the structures $\mathfrak{A}$ or $\mathfrak{B}$.  The players alternate turns and on each of  $\Right$'s turns he is obliged to choose an element from the opposing structure i.e.\ from $\mathfrak{B}$ if $\Left$'s last move was an element in $\mathfrak{A}$ or vice-versa.

\begin{dfn}[Position]
	A pair $(\bar{a},\bar{b})$, consisting of transfinite sequences $\bar{a}=\family{a_i}{i<\gamma}$ and $\bar{b}=\family{b_i}{i<\gamma}$, is called a \textbf{position} of the game $\EF_\gamma(\mathfrak{A},\mathfrak{B})$ whenever, for each $i<\gamma$, $a_i\in\domain{}\mathfrak{A}$ and $b_i\in\domain{}\mathfrak{B}$ are the elements played during the $i$-th round of the game $\EF_\gamma(\mathfrak{A},\mathfrak{B})$.  We refer to $\gamma$ as the \textit{length} of the position.
\end{dfn}

\begin{dfn}[Play]
	A \textbf{play} in the game $\EF_\gamma(\mathfrak{A},\mathfrak{B})$ is any maximal position in the game.
\end{dfn}

\begin{dfn}[Winning play]
	A position $(\bar{a},\bar{b})$ of the game $\EF_\gamma(\mathfrak{A},\mathfrak{B})$ is said to be \textbf{a winning play for} $\bm{\mathcal{R}}$ iff there exists some isomorphism $f\colon\gen{\bar{a}}{\mathfrak{A}}\to\gen{\bar{b}}{\mathfrak{B}}$ such that $f[\bar{a}]=f[\bar{b}]$.
\end{dfn}

\begin{dfn}[Back-and-forth equivalence]
	We define $\mathfrak{A}\sim_\gamma\mathfrak{B}$ for an ordinal $\gamma$ if there exists a winning strategy for $\Right$ in the game $\EF_\gamma(\mathfrak{A},\mathfrak{B})$.  For the special case $\gamma=\omega$: if $\mathfrak{A}\sim_\omega\mathfrak{B}$ then it is said that $\mathfrak{A}$ and $\mathfrak{B}$ are \textbf{back-and-forth equivalent}.
\end{dfn}

\begin{dfn}[Winning position]
	In the back-and-forth game $\EF_\omega(\mathfrak{A},\mathfrak{B})$ a position $(\bar{c},\bar{d})$ of length $n\in\nats$ is called a $\textbf{winning position}$ for a particular player if whenever that player can win the $\EF_\gamma(\mathfrak{A},\mathfrak{B})$ when starting from position $(\bar{c},\bar{d})$.
\end{dfn}

\begin{prp}\label{prp:bfiso}
	Suppose $\mathfrak{A}$ and $\mathfrak{B}$ are countable $L$-structures, for some language $L$, and let $(\bar{c},\bar{d})$ be a winning position for $\Right$ in $\EF_\omega(\mathfrak{A},\mathfrak{B})$.  If $\mathfrak{A}\sim_\omega\mathfrak{B}$ then there exists an isomorphism $f\colon\mathfrak{A}\to\mathfrak{B}$ such that $f[\bar{c}]=f[\bar{d}]$ i.e.\ an isomorphism $f\colon(\mathfrak{A},\bar{c})\to(\mathfrak{B},\bar{d})$.
\end{prp}
\begin{proof}
	Let $\bar{a}=\family{a_i}{i<\omega}$ and $\bar{a}=\family{a_i}{i<\omega}$, respectively, be enumerations of the structures $\mathfrak{A}$ and $\mathfrak{B}$.  In what follows we describe the remainder of the game $\EF_\omega(\mathfrak{A},\mathfrak{B}$.  On every odd round $\Left$ plays the first element occuring in $\bar{a}$ which he has not played at any previous position of the game and does not occur in $\bar{c}$.  Similarly, on every even round $\Left$ plays the first element occuring in $\bar{b}$ which he has not played at a previous position of the game and does not occur in $\bar{d}$.  Naturally, $\Right$ simply responds using his winning strategy on every round.

	Note, given any element from either one of the two structures, that $\Left$ will play that element after finitely many rounds.  Hence all the elements from both $\mathfrak{A}$ and $\mathfrak{B}$ make an appearance at some point in the game.  By definition of a winning play, there then exists an isomorphism $f\colon\gen{\bar{c}\string^\bar{a}}{\mathfrak{A}}\to\gen{\bar{d}\string^\bar{b}}{\mathfrak{B}}$ such that $f[\bar{c}\string^\bar{a}]=f[\bar{d}\string^\bar{b}]$ and hence $f[\bar{c}]=f[\bar{d}]$.  Since $\gen{\bar{a}}{\mathfrak{A}}=\mathfrak{A}$ and $\gen{\bar{b}}{\mathfrak{B}}=\mathfrak{B}$, the result follows as intended.
\end{proof}


\section{Linear orders and coloured expansions}

\begin{assn}
	From this section onwards all languages $L$ will be assumed to be first-order unless explicitly stated otherwise.
\end{assn}

\begin{dfn}[Dense linear order]
	A linear order $\alpha$ is called \textbf{dense} whenever, for every $a,b\in\alpha$, there exists $a\in\alpha$ such that $a<c<b$.  Additionally, $\alpha$ is said to be \textit{trivially dense} if the order type of alpha is either $\zero$ or $\one$.
\end{dfn}

\begin{dfn}[Coloured linear order]
	If $\alpha$ is a linear order and $\bar{r}$ is a (finite) tuple $(r_0,\dotsc,r_{k-1})$ of unary relations on $\alpha$ then the expansion $(\alpha,\bar{r})$ is said to be a $\mathbf{k}$\textbf{-coloured linear order}.
\end{dfn}

\begin{dfn}[Spectrum]
	Suppose $\mathcal{A}$ is a class of $k$-coloured linear orders.  We call $F$ an $\mathbf{n}$\textbf{-spectrum} for the class $\mathcal{A}$ when $F$ is a minimal set of linear orders with the property that for every $\alpha\in\mathcal{A}$ there exists a $\chi\in F$ such that $\alpha\nequiv{n}\chi$.
\end{dfn}


\section{Basic operations on linear orders}

\begin{lem}\label{lem:fvsum}
	Suppose $\family{\alpha_i}{i\in I}$ and $\family{\beta}{i\in I}$ are families of (possibly coloured) linear orders, indexed by a linearly ordered set $I\neq\emptyset$.  Now fix some $n\in\nats$ and assume $\alpha_i\nequiv{n}\beta_i$, for each $i\in J$, then it follows that
	\begin{equation}
		\sum_{i\in I}\alpha_i\nequiv{n}\sum_{i\in I}\beta_i.
	\end{equation}
\end{lem}


\section{Lattices}

\begin{thm}[Knaster-Tarski Theorem]
	Suppose $\Lambda$ is a complete lattices.  If $h$ is some endomorphism on $\Lambda$ then there must exist a (least) $x_0\in \Lambda$ such that $x_0$ is a fix0ed point of $h$, i.e.\ $h(x_0)=x_0$.
\end{thm}
\begin{proof}
	We start by exhibiting a fixed point of $h$.  Since $\Lambda$ is nonempty there exists a top $1\in\Lambda$, since by completeness, taking the join of $\Lambda$ we get an element $1=\bigvee\Lambda$.  Since $h$ is isotone we may conclude that $h(1)=1$.

	We are now obligated to find a least fixed point of $h$ so define
	\begin{equation}
		S=\setbuild{x\in\Lambda}{h(x)=x}
	\end{equation}  then clearly $S\neq\emptyset$ since $1\in S$.  Define $x_0=\bigwedge S$ and note that, since $x_0\leq x$ for each $x\in S$, by definition of isotonicity it follows that $h(x_0)\leq x$, for $x\in S$. Therefore $h(x_0)\leq\bigwedge S=x_0$.  Consequently, since $h(x_0)$ is a lower bound of $S$.  Since $x_0$ is the infimum of $S$ we cannot have $h(x_0)<x_0$.  Thus the only remaining possibility is $h(x_0)=x_0$, implying that $x\in S$.
\end{proof}


\section{Categories and functors}


\section{Galois connections}



\section{Decidability}

	\begin{prp}
		If $\Sigma$ is a set of $L$-sentences then $\Sigma$ is decidable set of sentences iff $\Sigma$ and $L_0\setminus\Sigma$, each, are recursively enumerable.
	\end{prp}

	\begin{prp}\label{prp:sdth}
		Suppose $\mathcal{A}$ is a class of structures.  If $\Sigma$ is a recursively enumerable set of sentences and its deductive closure $\dcl{\Sigma}$ satisfies
		\begin{equation}
			\dcl{\Sigma}=\Th(\mathcal{A}),
		\end{equation}
		then $\Th(\mathcal{A})$ is recursively enumerable.
	\end{prp}

	\begin{dfn}[Encoding]
		An \textbf{encoding} $p$ of a countable set $X$ is an injective recursive partial function $p\colon X\pto\nats$.  If $X$ is a first order language then $p$ is in stead referred to as a \textit{G\"odel numbering}.
	\end{dfn}

	\begin{dfn}[Frameworks]
		Suppose for some $A\subseteq X$, where $X$ is of (coloured) linear orders, there exists a family $\family{\Gamma_{alpha,\beta}}{(\alpha,\beta)\in A\times X}$
		\begin{enumerate}
			\item	$p\colon X\to\nats$ is an encoding of the (countable) set of coloured linear orders,
			\item 	the binary relation $\preceq$ is a well-founded preorder defined on $X$,
			\item	the map $\rho\colon X\to\nats$ is the rank function associated with $(X,\preceq)$.  If, for each $x,y\in X$, it holds that
		\end{enumerate}
	\end{dfn}

	\begin{dfn}[Proof sequences]
		Suppose $X$ is a countable set of countable coloured linear orders and let $p\colon X\pto\nats$ be an encoding of $X$.  A \textbf{proof sequence} is then a finite sequence
		\begin{equation}
			(\alpha_0,\sigma_0),\dotsc,(\alpha_{n-1},\sigma_{n-1})\in X\times L_0
		\end{equation}
		and finite expansions $\alpha^\prime_0,\dotsc,\alpha^\prime_{n-1}$ of the (respective) linear orders $\alpha_0,\dotsc,\alpha_{n-1}$ so that, for each $i=1,\dotsc,n-1$, there exists an interpretation $\Gamma_i$ of $\alpha_{i-1}$ in $\alpha^\prime_i$.  Additionally, when $0\leq i<n$, we require that one of the following holds:
		\begin{enumerate}
			\item	$\sigma_i$ is a logical axiom,
			\item	$\sigma_i$ follows via an inference rule from $\sigma_0,\dotsc,\sigma_{i-1}$,
			\item	$\sigma_i\in\admis(\Gamma_i)$ or
			\item	$\sigma_i=\Gamma_i\Gamma_{i-1}\dotsb\Gamma_j\sigma_j$, for some natural $j<i$.
		\end{enumerate}
	\end{dfn}
\vfill
	\begin{verbatim}
		-------------------------------------------------------------------------
	\end{verbatim}
\noindent[\textbf{Alles in hierdie hoofstuk is tentatief.  Byvoegings en (veral) verwyderings sal gemaak word soos ek deur die tesis vorder}]
