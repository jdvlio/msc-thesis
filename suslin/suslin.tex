\bibstyle{amsalpha}
\chapter{Suslin lines}

\section{Fundamental concepts}

If $P$ is a partially ordered set and $a\in P$ then we denote by $\down{a}$ the
set
\begin{equation}
	\down{a}=\setbuild{x\in P}{x\leq a}.
\end{equation}
Similarly, we define
\begin{equation}
	\up{a}=\setbuild{x\in P}{x\leq a}.
\end{equation}

\begin{dfn}[Tree]
	A partial order $\mathfrak{T}=(T,<)$ is called a \textbf{tree} whenever it
	has a least element (referred to as a \textit{root}) and, for each $a\in T$, the
	set $\down{a}$ is well-ordered.
\end{dfn}

Equivalently, a tree is a well-founded partial order with a least element.

\begin{dfn}[Subtree]
	We call $\mathfrak{T}^{\prime}$ of a tree $\mathfrak{T}$, whenever it holds
	that $\mathfrak{T}^{\prime}$ is a partially ordered set satisfying
	$T^{\prime}\subseteq T$.
\end{dfn}

\begin{dfn}[Length]
	If $\beta$ is some chain in a tree $\mathfrak{T}$ then the ordinal
	$\length(\beta)$ in Proposition \ref{prp:length} is referred to as the
	\textbf{length} of the chain $\beta$.
\end{dfn}

\begin{dfn}[Branch]
	A \textbf{branch} $\beta$ in a tree $\mathfrak{T}$ is any maximal chain of nodes in $\mathfrak{T}$.
\end{dfn}

\begin{dfn}[Height]
	If $\mathfrak{T}=(T,<)$ is any tree and $a\in T$ then the \textbf{height of
	}$\bm{a}$\textbf{ in }$\bm{\mathfrak{T}}$ is the order type of
	$\mathfrak{T}^{<a}$.
\end{dfn}

Note that, by definition of a tree, any $\mathfrak{T}^{<a}$ is well-ordered and
is thus isomorphic to an ordinal.  Therefore, the height of any element (called
a \textit{node}) of a tree is an ordinal.  One can now also define the related
but distinct concept of height for trees themselves.

\begin{dfn}[Tree height]
	The height $\height(\mathfrak{T})$ of a tree $\mathfrak{T}$ is the least
	ordinal such that, for any ordinal $\alpha>\height(\mathfrak{T})$, there exists
	no embedding $f\colon\alpha\hookrightarrow\mathfrak{T}$.
\end{dfn}

The relationship between these notions of height is then given by the following
proposition:

\begin{prp}
	If $\mathfrak{T}$ is any tree and $\alpha$ is any ordinal then
	$\height(\mathfrak{T})=\alpha$ iff it holds that
	\begin{equation}
		\alpha=\sup_{a\in T}(\height(a)+1).
	\end{equation}
\end{prp}
\begin{proof}
	Consider the case where $\height(\mathfrak{T})$ is a limit ordinal.  It
	follows, for all $a\in T$, that
	\begin{equation}
		h(a)<h(a)+1<h(\mathfrak{T})\quad
	\end{equation}
	and
	\begin{equation}
		\sup_{b\in T}h(b)=h(\mathfrak{T}).
	\end{equation}
	Therefore, we may conclude that
	\begin{equation}
		\sup_{a\in T}\big(h(a)+1\big)=h(\mathfrak{T}).
	\end{equation}

	Suppose now that $\height(\mathfrak{T})$ is a successor ordinal, say
	$\height(\mathfrak{T})=\beta_{0}+1$, then there must exist a branch
	$B_{0}$ in $\mathfrak{T}$ of length $\beta_{0}+1$.  Note that no branch in
	$\mathfrak{T}$ may be strictly longer than $B_{0}$ and thus
	\begin{equation}
		\sup_{a\in T}\big(h(a)+1\big)=\sup_{a\in B_{0}}\big(h(a)+1\big).
	\end{equation}
	Since $B_{0}$ has a maximal element, $m$ say, it necessarily follows that
	\begin{equation}
		\beta_{0}=\height(m),
	\end{equation}
	and thus
	\begin{align}
		\sup_{a\in B_{0}}\big(\height(a)+1\big) & =\height(m)+1 \\
		                                        & =\beta_{0}+1,
	\end{align}
	as required.
\end{proof}

\begin{prp}\label{prp:length}
	If $\beta$ is any chain in some tree $\mathfrak{T}$ then $\beta$ is
	isomorphic to some ordinal $\length(\beta)\leq\height(\mathfrak{T})$.
\end{prp}
\begin{proof}
	Assume by way of contradiction that $\beta$ is not well-ordered.  There must
	then exist a strictly decreasing sequence
	\begin{equation}
		b_{0}>b_{1}>\dotsb
	\end{equation}
	in $\beta$.  However, this implies that $\down{b_{0}}$ contains an infinite
	decreasing sequence despite being well-ordered, the desired contradiction.
	Furthermore, it follows by definition of $h(\mathfrak{T})$ that we must have
	$\length(\beta)\leq h(\mathfrak{T})$, as required.
\end{proof}

\begin{dfn}[Levels]
	If $\mathfrak{T}$ is a tree and $\alpha<\height(\mathfrak{T})$ is an ordinal then the set
	\begin{equation}
		\level{\alpha}(\mathfrak{T})=\setbuild{a\in T}{\height(a)=\alpha}
	\end{equation}
	is referred to as the $\bm{\alpha}$\textbf{-th level} of $\mathfrak{T}$.
\end{dfn}

\begin{dfn}[Normality]
	A \textbf{normal }$\bm{\alpha}$\textbf{-tree} is a tree $\mathfrak{T}$ such that
	\begin{enumerate}
		\item	$\height(\mathfrak{T})=\alpha$,\label{dfn:n1}
		\item	for each $\gamma<\alpha$ it holds that
		      $\card{\level{\gamma}(\mathfrak{T})}\leq\aleph_0$,\label{dfn:n2}
		\item	if $x\in T$ is not maximal then $x$ has infinitely many
		      immediate successors,\label{dfn:n3}
		\item	for each $x\in T$ and each ordinal $\beta<\alpha$, if
		      $\beta>\height(x)$ then there exists some $y\in\level{\beta}(\mathfrak{T})$ such
		      that $y>x$,\label{dfn:n4}
		\item	for every limit ordinal $\beta<\alpha$ and every
		      $x,y\in\level{\beta}(\mathfrak{T})$, if $\mathfrak{T}^{<x}=\mathfrak{T}^{<y}$
		      then $x=y$.\label{dfn:n5}
	\end{enumerate}
	As one might expect, we refer to a tree $\mathfrak{T}$ simply as being
	\textbf{normal} whenever it is $\height(\mathfrak{T})$-normal.
\end{dfn}


\section{On the existence of Suslin lines and Suslin trees}

\begin{dfn}[Suslin line]
	A \textbf{Suslin line} is an inseparable dense linear order
	$\alpha\neq\zero$, without endpoints, such that every set of disjoint open
	subintervals of $\alpha$ is at most countable.
\end{dfn}

Succinctly, a linear order is a Suslin line whenever it is dense, inseparable
and satisfies the Suslin property.

\begin{prp}\label{prp:densesublambda}
	If $\alpha\subseteq\lambda$ is dense in $\lambda$ then $\alpha$ must be
	separable.
\end{prp}
\begin{proof}
	Let $I$ denote the set of open intervals $(q,r)\subseteq\reals$ such that
	$q,r\in\rats$ and $(q,r)\cap\alpha\neq\emptyset$.  Clearly $I$ is countable
	so we may fix an enumeration $\family{(q_{i},r_{i})}{i<\omega}$ of $I$.
	Furthermore, for each $i<\omega$, there exists an
	$s_{i}\in(q_{i},r_{i})\cap\alpha$.

	Let $S=\setbuild{s_{i}}{i<\omega}$ then it follows by definition that $S$ is
	countable.  The aim is now to show that, additionally, $S$ is dense in
	$\alpha$.  Fix any $a,b\in\alpha$ and suppose that $a<b$.  Since $\rats$ is
	dense in $\lambda$, we may now choose $q,r\in\rats$ such that
	\begin{equation}
		a<q<r<b.
	\end{equation}

	Since $\alpha$ is dense in $\lambda$, it follows that
	$(q,r)\cap\alpha\neq\emptyset$. Thus, by definition of $I$, there exists a
	$k<\omega$ such that $(q,r)=(q_{k},r_{k})$.  Therefore, since $s_{k}\in(q,r)$,
	we may conclude that $a<s_{k}<b$, as required.
\end{proof}

\begin{prp}\label{prp:completesuslin}
	If there exists a Suslin line then there exists a complete Suslin line.
\end{prp}
\begin{proof}
	Let $\beta$ be a Suslin line and choose $\alpha=\comp(\beta)$.  Our aim is
	now to show that $\alpha$ is a Suslin line in its own right.  We may assume,
	without loss of generality, that $\beta$ has no endpoints.  Note that $\alpha$
	cannot be separable as Theorem \ref{thm:rchar} would imply that
	$\alpha\cong\lambda$ and thus, from Proposition \ref{prp:densesublambda},
	$\beta$ is separable.  As this contradicts the Suslinity of $\beta$, we may
	conclude that $\alpha$ is inseparable.


	Since a dense linear order is necessarily dense in its completion, any
	family $\family{U_{i}}{i\in I}$ of pairwise disjoint, nonempty open
	subintervals of $\alpha$ gives rise to a corresponding family
	$\family{\beta\cap U_{i}}{i\in I}$ of pairwise disjoint, nonempty open
	subintervals of $\beta$.  Consequently, the Suslinity of $\beta$ implies
	that any such family is at most countable, as required.
\end{proof}

\begin{cor}
	There exists a Suslin line iff there exists a complete Suslin line.
\end{cor}
\begin{proof}
	\forward Refer to Proposition \ref{prp:completesuslin}.

	\backward Immediate, by definition.
\end{proof}

\begin{dfn}[Suslin tree]
	A tree $\mathfrak{T}$ is called a \textbf{Suslin tree} whenever
	$\height(\mathfrak{T})=\omega_1$, all the antichains of $\mathfrak{T}$ are at
	most countable and, for every branch $\beta$ in $\mathfrak{T}$, it holds that
	$\length(\beta)<\omega_1$.
\end{dfn}

\begin{dfn}[Branching point]
	A \textbf{branching point} in a tree $\mathfrak{T}$ is a non-maximal node
	$x$ in $\mathfrak{T}$ such that there exist (necessarily distinct) branches
	$\beta$ and $\gamma$ in $\mathfrak{T}$ satisfying $\beta\cap\gamma=\down{x}$.
\end{dfn}

\begin{dfn}[Siblings]
	Distinct nodes $x$ and $y$ in a tree $\mathfrak{T}$ are \textbf{siblings} of
	one another whenever it holds that $\mathfrak{T}^{<x}=\mathfrak{T}^{<y}$.
\end{dfn}

As the name might suggest, a node $x$ in a tree $\mathfrak{T}$ is a branching
point iff it has at least two distinct immediate successors.  Note that distinct
nodes $z$ and $z^{\prime}$ of non-limit height are siblings iff there exists a
branching point $y$ of $\mathfrak{T}$ which has $z$ and $z^{\prime}$ among its
immediate successors.

Henceforth, if $\mathfrak{T}$ is any tree and $X\subseteq T$, we use
$\mathfrak{T}^{<X}$ to denote the partial order whose domain is
\begin{equation}
	T^{<X}\define\setbuild{y\in T}{y<x\text{ for all }x\in X},
\end{equation}
and whose order relation is that of $\mathfrak{T}$ restricted to $T^{<X}$.  In
the special case $X=\set{x_{0}}$, we simply write $\mathfrak{T}^{<x_{0}}$
in place of $\mathfrak{T}^{<X}$.  Similar notation is employed for the dual
relation $>$ as well as the non-strict counterparts of these relations.

\begin{lem}\label{lem:norm}
	If there exists a Suslin tree then there exists a normal Suslin tree.
\end{lem}
\begin{proof}
	Suppose there exists a Suslin tree $\mathfrak{T}$.  We now construct, in
    stages, a normal Suslin tree from $\mathfrak{T}$.  It already follows by
    definition that $\height(\mathfrak{T})=\omega_1$ and the levels of
    $\mathfrak{T}$ are at most countable, since each level is also an antichain.
    Hence, $\mathfrak{T}$ satisfies both \ref{dfn:n1} and \ref{dfn:n2} in the
    definition of normality.

	We now choose $\mathfrak{T}_0$ to be the tree such that
	$T_0=\setbuild{x\in T}{\card{\up{x}}\geq\aleph_1}$.  That is to say,
	$\mathfrak{T}_0$ is obtained from $\mathfrak{T}$ by discarding all nodes $x$
	such that $\up{x}$ is at most countable.  Regarding $\mathfrak{T}_{0}$'s
	status as a tree, it is sufficient to note that any subset of well-ordered
	set is itself well-ordered.

	Since $\mathfrak{T}_{0}$ is a subtree of $\mathfrak{T}$, it follows that
	\begin{equation}
		\height(\mathfrak{T}_{0})\leq\height(\mathfrak{T})=\omega_{1}.
	\end{equation}
     In order to conclude that $\height(\mathfrak{T}_0)=\omega_1$, it is
     sufficient to note that $\mathfrak{T}_0$ has property \ref{dfn:n4}, in the
     definition of normality, 
    

	Since each branch in $\mathfrak{T}_{0}$ is simply a truncation of a branch
	in $\mathfrak{T}$, the branches in $\mathfrak{T}_{0}$ are certainly at most
    countable.  Hence, $\mathfrak{T}_0$ is a Suslin tree and satifies properties
    \ref{dfn:n1}, \ref{dfn:n2} and \ref{dfn:n4}.

    We now set out to build a Suslin tree $\mathfrak{T}_1$, from
    $\mathfrak{T}_0$, such that it is the minimal extension of
    $\mathfrak{T}_{1}$ satisfying both of the following:
	\begin{enumerate}
        \item If $x\in\mathfrak{T}_{0}$ is either of limit height or is the root
            of $\mathfrak{T}_{0}$, and $X=\mathfrak{T}^{<x}$, then add an
            $a_{x}\in\mathfrak{T}_{1}$, distinct from $x$, such that $a_{x}>X$
            and $a_{x}<\mathfrak{T}_{0}^{>X}$.
		\item Whenever $x$ and $x^{\prime}$ are siblings in $\mathfrak{T}_{0}$,
		      each of limit height, then it follows that $a_{x}=a_{x^{\prime}}$.
	\end{enumerate}

	We are now required to show that antichains in $\mathfrak{T}_1$ are at most
	countable.  Consider an arbitrary antichain $X$ in $\mathfrak{T}_1$.  For each $x\in X$,
	choose $y_{x}$ to be of some successor of $x$.  Define $X^{\prime}$
	to be the set of all such $y_{x}$ then it follows by definition that
	$\card{X}=\card{X^{\prime}}$. Since $X^{\prime}$ is clearly an antichain in
	$\mathfrak{T}_{0}$, it now follows that $X$ is at most countable.  Hence,
	$\mathfrak{T}_{1}$ is a Suslin tree. Furthermore, by definition,
	$\mathfrak{T}_1$ satisfies properties \ref{dfn:n4} and \ref{dfn:n5} in the
	definition of normality.

	We now discard all nodes that aren't branching points by letting
	$\mathfrak{T}_2$ be the subtree of $\mathfrak{T}_{1}$ with domain
	$T_2=\setbuild{x\in T_1}{x\text{ is a branching point of }\mathfrak{T}_1}$.
    Note, for every $x\in T_1$, that $\mathfrak{T}_1$ has uncountably many
    branches $\beta$ such that $x\in\beta$.  If, by way of contradiction, we
    suppose that $\mathfrak{T}_2$ does not have property \ref{dfn:n4} then there
    exists an $a\in T_2$ such that $\mathfrak{T}_2^{\geq a}$ is at most
    countable.  Therefore, there is at most countably many branches in
    $\mathfrak{T}_1$ containing $a$ as a member, a contradiction.

    We now argue that $\mathfrak{T}_2$ also has property 5.  Fix $a,b\in T_2$ of
    limit height in $\mathfrak{T}_2$ such that
    $\mathfrak{T}_2^{<a}=\mathfrak{T}_2^{<b}$.  Let $A=T_2^{<a}$ and
    $B=T_2^{<b}$ then it follows that their respective suprema in
    $\mathfrak{T}_1$ are $a$ and $b$.  Therefore, neither $a$ nor $b$ can have
    an immediate predecessor in $\mathfrak{T}_1$ and thus must have limit height
    in $\mathfrak{T}_1$ as well.  Since every nonbranching point in
    $\mathfrak{T}_1^{<a}$ sits below some $a^\prime\in A$ and, similarly, every
    nonbranching point in $\mathfrak{T}_1^{<b}$ sits below some $b^\prime\in B$
    we may conclude that $\mathfrak{T}_1^{<a}=\mathfrak{T}_1^{<b}$ and thus
    $a=b$, as required.

    Choose $\mathfrak{T}_3$ to be the tree such that $T_3=\setbuild{x\in
    T_2}{h(x)\text{ is a limit ordinal}}$. As $\mathfrak{T}_2$ had no
    uncountable branches or antichains it immediately follows that neither will
    $\mathfrak{T}_3$.  Note also that, for each $\alpha<\omega_{1}$,
    $\mathfrak{T}_{3}$ must have a branch of length at least $\alpha\cdot\omega$
    and, therefore, we must have $h(\mathfrak{T}_{3})=\omega_{1}$, thereby
    making $\mathfrak{T}_3$ a Suslin tree.

	By construction, it follows that each node in $\mathfrak{T}_3$ has
	infinitely many immediate successors, thus satisfying property \ref{dfn:n3}.
	It satisfies properties \ref{dfn:n1} (for $\alpha=\omega_1$) and \ref{dfn:n2},
	by virtue of being a Suslin tree, and inherits properties \ref{dfn:n4} and
    \ref{dfn:n5}, via the construction, from $\mathfrak{T}_2$.  This concludes
    the proof as $\mathfrak{T}_3$ is therefore the desired normal Suslin tree.
\end{proof}

\begin{lem}\label{lem:ltree}
	If there exists a Suslin line then there exists a Suslin tree.
\end{lem}
\begin{proof}
    Suppose $\sigma=(S,<)$ is a Suslin line.  We now construct, by transfinite
    recursion, a tree from closed intervals in $\sigma$, each consisting of at
    least two elements.  First, we choose $I_0=S$ and fix any $I_1=[a_1,b_1]$
    such that $a_1,b_1\in S$ and $a_1<b_1$ have been defined.

    Suppose now that $\alpha$ is an ordinal satisfying $0<\alpha<\omega_1$ and
    assume, for each ordinal $\beta$ satisfying $0<\beta<\alpha$, that
    $I_\beta=[a_\beta,b_\beta]$ and $a_\beta<b_\beta$ for some
    $a_\beta,b_\beta\in S$.  Define
    $E=\setbuild{a_\beta}{\beta<\alpha}\cup\setbuild{b_\beta}{\beta<\alpha}$ and
    note that $E$, by definition, must be countable.  Therefore, since $\sigma$
    is not separable there exist $a_\alpha,b_\alpha\in S$ such that
    $a_\alpha<b_\alpha$ and $I_\alpha=[a_\alpha,b_\alpha]$ is disjoint from $E$.

    Now, defining $T=\setbuild{I_\gamma}{\gamma<\omega_1}$, it follows that $T$
    is uncountable and partially ordered by $\supsetneq$.  We now argue, for
    every $\alpha<\omega_1$, that the set $\down{I_\alpha}$ is well-ordered.
    This will then allow us to conclude that $\mathfrak{T}=(T,\supsetneq)$ is a
    tree.

    It follows from the construction of $\mathfrak{T}$ that if
    $\alpha,\beta<\omega_1$ and $\alpha<\beta$ then either $I_\alpha\supsetneq
    I_\beta$ or $I_\alpha\cap I_\beta=\emptyset$.  Choose any $J\subseteq
    \down{I_{\alpha}}$ then there exists a set $\Gamma$ of ordinals smaller than
    $\omega_{1}$ such that \begin{equation}
    J=\setbuild{I_{\gamma}}{\gamma\in\Gamma}. \end{equation} Since $I_{\alpha}$
    is contained in the intersection of any pair in $\down{I_{\alpha}}$, it
    follows that $\down{I_{\alpha}}$, and thus $J$, is linearly ordered.  Hence,
    if $\gamma_{0}$ is the least member of $\Gamma$ then it necessarily follows
    that $I_{\gamma_{0}}$ is the least element of $\down{I_{\alpha}}$.
    Therefore, since $J$ was arbitrary, it follows by definition that
    $\down{I_{\alpha}}$ is well-ordered.

    We are now required to show that $\mathfrak{T}$ is in fact a Suslin tree.
    That is, we are required to prove that $\height(\mathfrak{T})=\omega_1$ and
    $\mathfrak{T}$ has neither an uncountable antichain nor an uncountable
    branch.

    Recall from earlier, if $\alpha,\beta<\omega_1$ and $\alpha<\beta$ then
    either $I_\alpha\supsetneq I_\beta$ or $I_\alpha$ and $ I_\beta$ are
    disjoint.  From this observation, it follows that if $X\subseteq T$ is an
    antichain then $X$ is a pairwise disjoint set of closed intervals in $S$.
    Choose $X^\prime$ to now be the corresponding set of open intervals i.e.\
    $X^\prime=\setbuild{(a,b)}{[a,b]\in X}$.

    Since $\sigma$ is a Suslin line it now follows that $X^\prime$, and thus
    also $X$, is at most countable.  In order to show that $\mathfrak{T}$ has no
    uncountable branch suppose the contrary: there exists a branch
    $\beta=(B,\supsetneq)$ in $\mathfrak{T}$ such that
    $\length(\beta)=\delta\geq\omega_1$.  We may now assume
    $B=\setbuild{I_{\gamma_{i}}}{i<\delta}$ and $I_{\gamma_i}\supsetneq
    I_{\gamma_j}$ whenever $i<j$.

    If, for each $\gamma<\delta$, we choose $x_\gamma$ to be the left endpoint
    of the interval $I_\gamma$ then it follows that
    $\setbuild{(x_\gamma,x_{\gamma+1})}{\gamma<\delta}$ is an uncountable set of
    pairwise disjoint open intervals in $\sigma$ --- contradicting the fact that
    $\sigma$ is a Suslin line.

    All that remains is to show that $\height(\mathfrak{T})=\omega_1$.  Since
    $\mathfrak{T}$ has no uncountable branch, however, it follows that
    $\height(\mathfrak{T})\leq\omega_1$.  Note also, since levels are
    antichains, each level of $\mathfrak{T}$ must be countable.  Consequently,
    as $\card{\mathfrak{T}}=\aleph_1$, we cannot have
    $\height(\mathfrak{T})<\omega_1$ and thus we may conclude that
    $\height(\mathfrak{T})=\omega_1$, as required.
\end{proof}

\begin{lem}\label{lem:tline}
	If there exists a Suslin tree then there exists a Suslin line.
\end{lem}
\begin{proof}
	Suppose there exists a Suslin tree.  It follows from Lemma \ref{lem:norm}
	that there exists a normal Suslin tree $\mathfrak{T}$.  Since $\mathfrak{T}$ is
	normal, each node $x\in T$ has $\aleph_0$ many immediate successors.  We may
	thus order the successors of every node like $\eta$.  Note now that every
	branch $\beta$ in $\mathfrak{T}$ uniquely determines some transfinite sequence
	$\sigma\in S=\rats^{<\omega_1}=\bigcup_{\alpha<\omega_1}\rats^\alpha$.

	Note, by the maximality of branches, it is never the case that some
	$\sigma_0\in S$ is an initial subsequence of any $\sigma_1\in S$.  Let
	$(S,\prec)$ now denote the set of all such $\sigma$ ordered lexicographically:
	if $\sigma_0,\sigma_1\in S$, $\sigma_0\neq\sigma_1$ and
	$\length(\sigma_0)\leq\length(\sigma_1)$ then we define $\sigma_0\prec\sigma_1$
	whenever, for the least $\alpha<\length(\sigma_0)$ such that
	$\sigma_0(\alpha)\neq\sigma_1(\alpha)$, it holds that
	$\sigma_0(\alpha)<_{\eta}\sigma_1(\alpha)$.

	It should be clear from its definition that $S$ is linearly ordered by
	$\prec$ and lacks endpoints.  It remains to be shown that $(S,\prec)$ is a
	Suslin line.  We now show that $S$ is in fact densely ordered.  If
	$\sigma_0,\sigma_1\in S$ and $\sigma_0\prec\sigma_1$ then it follows from the
	density of $\eta$ that there exists a $\tau\in S$ such that
	$\sigma_0\prec\tau\prec\sigma_1$.

    From here onwards, we identify each $\sigma\in S$ with its corresponding
    branch in $T$ and treat them as such.  We are required to show that $\sigma$
    has the Suslin property:  suppose $\mathcal{I}$ is a pairwise disjoint
    collection of open intervals $(A,B)$ in $S$.  For each $x\in T$, define
    $I_x\subseteq S$ to be the convex set:
	\begin{equation}
		I_x=\setbuild{\sigma\in S}{x\in \sigma}.
	\end{equation}
	Note now that, for $x,y\in T$, $I_x\cap I_y=\emptyset$ iff $x$ and $y$ are
	incomparable.

	Choose any $(\sigma_0,\sigma_1)\in\mathcal{I}$ and let
	$\alpha<\min\set{\length(\sigma_0),\length(\sigma_1)}$ be the least ordinal such
	that $\sigma_0(\alpha)<_{\eta}\sigma_1(\alpha)$.  Now choose any
	$\tau\in(\sigma_0,\sigma_1)$ such that
	$\sigma_0(\gamma)=\tau(\gamma)=\sigma_1(\gamma)$, for each $\gamma<\alpha$, and
	$\sigma_0(\alpha)<_{\eta}\tau(\alpha)<_{\eta}\sigma_1(\alpha)$.  Define $x=\tau(\alpha)$ and
	note that it then follows that $I_x\subseteq(\sigma_0,\sigma_1)$.  Thus, since
	$\sigma_0$ and $\sigma_1$ were arbitrary, we may conclude that for any open
	interval $J\in\mathcal{I}$ there exists a $\tau_J\in S$ and an $x_J\in\tau_J$
	such that $I_{x_J}\subseteq J$.

	By definition of $\mathcal{I}$, it follows that the set
	$\setbuild{I_{x_J}}{J\in\mathcal{I}}$ is pairwise disjoint. Consequently
	$X=\setbuild{x_J}{J\in\mathcal{I}}$ is a pairwise incomparable set of elements
	of $\mathfrak{T}$, i.e.\ $X$ is an antichain of $\mathfrak{T}$, and is therefore
	at most countable.  This implies that $\mathcal{I}$ is also at most countable.

    All that remains is to show that $(S,\prec)$ is not separable.  By way of
    contradiction, assume $D\subseteq S$ is dense in $S$.  It immediately
    follows that $D$ is cofinal in $S$.  From property \ref{dfn:n4} in the
    definition of normality, it follows that there are arbitrarily long branches
    of length less than $\omega_1$ in $\mathfrak{T}$.  Hence, by definition of
    $\prec$, we may conclude that $D$ is uncountable.
\end{proof}

\begin{thm}
	There exists a Suslin line iff there exists a Suslin tree
\end{thm}
\begin{proof}
	This result is the combination of lemmas \ref{lem:ltree} and \ref{lem:tline}.
\end{proof}
