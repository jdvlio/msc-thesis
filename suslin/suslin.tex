\chapter{Suslin lines}

\section{Fundamental concepts}

\begin{dfn}
	If $P$ is a partially ordered set and $a\in P$ then we denote by $\down{a}$ the set
	\begin{equation}
		\down{a}=\setbuild{x\in P}{x\leq a}.
	\end{equation}
\end{dfn}

\begin{dfn}[Tree]
	A partial order $\mathfrak{T}=(T,<)$ is called a \textbf{tree} whenever it has a least element (referred to as a \textit{root}) and, for each $a\in T$, the set $\down{a}$ is well-ordered.
\end{dfn}

\begin{dfn}[Height]
	If $\mathfrak{T}=(T,<)$ is any tree and $a\in T$ then \textbf{height of }$\mathbf{a}$\textbf{ in }$\bm{\mathfrak{T}}$ is the order type of $\mathfrak{T}^{<a}$.
\end{dfn}

Note that, by definition of a tree, any $\mathfrak{T}^{<a}$ is well-ordered and is thus isomorphic to an ordinal.  Therefore, the height of any element (called a \textit{node}) of a tree is an ordinal.  One can now also define the related but distinct concept of height for trees themselves.

\begin{dfn}[Tree height]
	The height $\height(\mathfrak{T})$ of a tree $\mathfrak{T}$ is the least ordinal such that, for any ordinal $\alpha>\height(T)$, there exists no embedding $f\colon\alpha\hookrightarrow\mathfrak{T}$.
\end{dfn}

The relation between these notions of height is then given by the following proposition:

\begin{prp}
	If $\mathfrak{T}$ is any tree and $\alpha$ is any ordinal then $\height(\mathfrak{T})=\alpha$ iff it holds that
	\begin{equation}
		\alpha=\sup_{a\in T}(\height(a)+1).
	\end{equation}
\end{prp}

\begin{dfn}[Length]
	For any chain $\beta$ in a tree $\mathfrak{T}$, the ordinal $\length(\beta)$ in Proposition \ref{prp:length} is referred to as the \textbf{length} of the chain $\beta$.
\end{dfn}

\begin{prp}\label{prp:length}
	If $\beta$ is a chain in a tree $\mathfrak{T}$ then $\beta$ is isomorphic to some ordinal $\length(\beta)\leq\height(\mathfrak{T})$.
\end{prp}

\begin{dfn}[Branch]
	A \textbf{branch} $\beta$ in a tree $\mathfrak{T}$ is any maximal chain of nodes in $\mathfrak{T}$.
\end{dfn}

\begin{dfn}[Levels]
	If $\mathfrak{T}$ is a tree and $\alpha<\height(\mathfrak{T})$ is an ordinal then the set
	\begin{equation}
		\level{\alpha}(\mathfrak{T})=\setbuild{a\in T}{\height(a)=\alpha}
	\end{equation}
	is referred to as the $\bm{\alpha}$\textbf{-th level} of $\mathfrak{T}$.
\end{dfn}

\begin{dfn}[Normality]
	A \textbf{normal }$\bm{\alpha}$\textbf{-tree} is a tree $\mathfrak{T}$ such that
	\begin{enumerate}
		\item	$\height(\mathfrak{T})=\alpha$,
		\item	for each $\gamma<\alpha$ it holds that $\card{\level{\gamma}(\mathfrak{T})}\leq\aleph_0$,
		\item	if $x\in T$ is not maximal then $x$ has infinitely many immediate successors,
		\item	for each $x\in T$ and each ordinal $\beta<\alpha$, if $\beta>\height(x)$ then there exists some $y\in\level{\beta}(\mathfrak{T})$ such that $y>x$,
		\item	for every $x,y\in T$, if $\mathfrak{T}^{<x}=\mathfrak{T}^{<y}$ then $x=y$.
	\end{enumerate}
	As one might expect, we refer to a tree $\mathfrak{T}$ simply as being \textbf{normal} whenever it is $\height(\mathfrak{T})$-normal.
\end{dfn}


\section{On the existence of Suslin lines and Suslin trees}

\begin{dfn}[Suslin line]
	A complete linear order $\alpha\in\dense$ is said to be a \textbf{Suslin line} whenever $\alpha$ is not seperable and every pairwise disjoint set of open intervals in $\alpha$ is at most countable.
\end{dfn}

Restated succintly:  A Suslin line is a complete inseperable linear order without endpoints posessing the \textit{Suslin property}.

\begin{prp}
	If there exists a Suslin line then there exists a complete Suslin line.
\end{prp}

\begin{cor}
	There exists a Suslin line iff there exists a complete Suslin line.
\end{cor}

\begin{dfn}[Suslin tree]
	A tree $\mathfrak{T}$ is called a \textbf{Suslin tree} whenever $\height(\mathfrak{T})=\omega_1$, all the anti-chains of $\mathfrak{T}$ are at most countable and, for every branch $\beta$ in $\mathfrak{T}$, it holds that $\length(\beta)<\omega_1$.
\end{dfn}

\begin{lem}
	If there exists a Suslin line then there exists a Suslin tree.
\end{lem}
\begin{proof}
	Suppose $\sigma=(S,<)$ is a Suslin line.  We now construct, by transifinite recursion, a tree from closed intervals in $\sigma$, each consisting of at least two elements.  First, we choose $I_0=S$ and fix any $I_1=[a_1,b_1]$ such that $a_1,b_1\in S$ and $a_1<b_1$.  Suppose now that $\alpha$ is an ordinal satisfying $0<\alpha<\omega_1$ and assume, for each ordinal $\beta$ satisfying $0<\beta<\alpha$, that $I_\beta=[a_\beta,b_\beta]$ and $a_\beta<b_\beta$ for some $a_\beta,b_\beta\in S$.  Define $E=\setbuild{a_\beta}{\beta<\alpha}\cup\setbuild{b_\beta}{\beta<\alpha}$ and note that $E$, by definition, must be countable.  Therefore, since $\sigma$ is not seperable there exists an $a_\alpha,b_\alpha\in S$ such that $a_\alpha<b_\alpha$ and $I_\alpha=[a_\alpha,b_\alpha]$ is disjoint from $E$.  Now, defining $T=\setbuild{I_\alpha}{\alpha<\omega_1}$, it follows that $T$ is uncountable and partially ordered by $\supsetneq$.  Note, for every $\alpha<\omega_1$, that the set $\down{I_\alpha}$ is well-ordered.  We may thus conclude that $\mathfrak{T}=(T,\supsetneq)$ is a tree.

	We are now required to show that $\mathfrak{T}$ is in fact a Suslin tree.  That is we are required to prove that $\height(\mathfrak{T})=\omega_1$ and $\mathfrak{T}$ has neither an uncountable antichain nor an uncountable branch.  It follows from the construction of $\mathfrak{T}$ that if $\alpha,\beta<\omega_1$ and $\alpha<\beta$ then either $I_\alpha\supsetneq I_\beta$ or $I_\alpha\cap I_\beta=\emptyset$.  From this observation it follows that if $X\subseteq T$ is an antichain then $X$ is a pairwise disjoint set of closed intervals in $X$.  Choose $X^\prime$ to now be the corresponding set of open intervals i.e.\ $X^\prime=\setbuild{\Int(I)}{I\in X}$.  Since $\sigma$ is a Suslin line it now follows that $X^\prime$, and thus also $X$, is at most countable.  In order to show that $\mathfrak{T}$ has no uncountable branch suppose the contrary: there exists a branch $\beta=(B,\supsetneq)$ in $\mathfrak{T}$ such that $\length(\beta)=\delta\geq\omega_1$.  We may now assume $B=\setbuild{I_\gamma}{\gamma<\delta}$ and $I_{\gamma_0}\supsetneq I_{\gamma_1}$ whenever $\gamma_0<\gamma_1$.  If, for each $\gamma<\delta$, we choose $x_\gamma$ to be the left endpoint of the interval $I_\gamma$ then it follows, since $B$ pairwise disjoint, that $\setbuild{(a_\gamma,a_{\gamma+1})}{\gamma<\delta}$ is an uncountable set of pairwise disjoint open intervals in $\sigma$ --- contradicting the fact that $\sigma$ is a Suslin line.

	All that remains is to show that $\height(\mathfrak{T})=\omega_1$.  Since $\beta$ has no uncountable branch, however, it follows that $\height(\mathfrak{T})\leq\omega_1$.  Note also, since levels are antichains. each level of $\mathfrak{T}$ must be countable.  Therefore, as $\card{\mathfrak{T}}=\aleph_1$, we cannot have $\height(\mathfrak{T})<\omega_1$ and thus we may conclude that $\height(\mathfrak{T})=\omega_1$, as required.
\end{proof}

\begin{lem}
	If there exists a Suslin Tree then there exists a normal Suslin tree.
\end{lem}

\begin{lem}
	If there exists a Suslin tree then there exists a Suslin line.
\end{lem}

\begin{thm}
	There exists a Suslin line iff there exists a Suslin tree
\end{thm}

\begin{prp}
	Let $\mathfrak{T}$ be a tree in which every node has at least two immediate successors.  If $\mathfrak{T}$ has no uncountable antichains then for every branch $\beta$ in $\mathfrak{T}$ it necessarily holds that $\length(\beta)<\omega_1$.
\end{prp}

\begin{thm}
	There exists a Suslin line iff there exists a regular Suslin tree.
\end{thm}
