\chapter{The theorems of La\"uchli and Leonard}


\section{The result for scattered linear orders}


    \begin{dfn}[The class $\mathcal{M}_0$]
        The class $\mathcal{M}_0\subseteq\mathcal{L}$ is the smallest class of linear orders such that:
        \begin{enumerate}
            \item $\mathbf{0},\mathbf{1}\in\mathcal{M}_0$,
            \item if $\alpha,\beta\in\mathcal{M}_0$ then $\alpha+\beta\in\mathcal{M}_0$,
            \item if $\alpha\in\mathcal{M}_0$ then $\alpha\cdot\omega,\alpha\cdot\dual{\omega}\in\mathcal{M}_0$
        \end{enumerate}
    \end{dfn}

    \begin{rem}
        Take note of the following:
        \begin{enumerate}
            \item Every $\alpha\in\mathcal{M}_0$ is a scattered linear order,
            \item The class $\mathcal{M}_0$ is actually a (countable) set.  Furthermore, $\mathcal{M}_0$ is recursively enumerable.
        \end{enumerate}
    \end{rem}

    \begin{thm}[La\"uchli and Leonard]
        For every scattered linear order $\alpha$ and every $n\in\nats$ there exists a $\beta_n\in\mathcal{M}_0$ such that $\alpha\nequiv{n}\beta_n$.
    \end{thm}


\section{The result for arbitrary linear orders}
