\chapter{Dense linear orders}


\section{The shuffle operation}

\begin{prp}\label{prp:ratpart}
	Suppose $n\in\nats$ and $n\geq 2$ and let $P$ bet a set of $n-1$ primes $p\in\nats$.  Choose
	\begin{equation}
		D_p=\setbuild{q=\faktor{a}{p^k}}{a\in\ints\text{ and }k\in\ints\setminus\set{0}},\quad\text{for each }p\in P.
	\end{equation}
	If $P=\set{p_0,\dotsc,p_{n-2}}$ and $Q_i=\setbuild{D_p}{p\in P}$ when $i<n$ and $Q_n=\rats\setminus\bigcup_{i<n}Q_i$ then $\setbuild{Q_i}{0\leq i<n}$ is a partition of $\rats$ such that $Q_i$ is dense in $\eta$, for each $i=0,\dotsc,n-1$.
\end{prp}

\begin{dfn}[Canonical partition]
	The partition defined in proposition \ref{prp:ratpart} will be referred to as the \textbf{canonical partion} of the rationals into dense subsets.
\end{dfn}

\begin{cor}
	For every natural $n\geq 2$ there exists a partition $\mathcal{R}=\set{R_0,\dotsc,R_{n-1}}$ of $\reals$ such that:
	\begin{enumerate}
		\item $R_{n-1}=\irrats$,
		\item $R_k$ is a countable dense subset of $\lambda$, for each natural $k<n-1$.
	\end{enumerate}
\end{cor}

\begin{dfn}[The shuffle operation]
	Suppose $n\in\nats$ and let $F=\set{\alpha_0,\dotsc,\alpha_{n-1}}$ be an arbitrary set of $n$ linear orders.  Now suppose the following:
	\begin{enumerate}
		\item $\set{Q_0,\dotsc,Q_{n-1}}$ is the canonical partition of $\rats$,
		\item $h\colon\rats\to F$ is a colouring of $\eta$ such that $\inv{h}[\alpha_i]=Q_i$, for each natural $i<n$.
	\end{enumerate}
	We then define the \textbf{shuffle} $\sigma(F)$ of the set $F$ to be the linear order
	\begin{equation}
		\sigma(\set{\alpha_0,\dotsc,\alpha_{n-1}})=\sum_{q\in\eta}h(q),
	\end{equation}
	and we refer to the resulting (class) map $\sigma$ as the \textbf{shuffle operation} induced by the aforementioned partition.
\end{dfn}


\section{The L\"auchli and Leonard result for the class of linear orders}

\begin{dfn}[The class $\M$]
	The class $\M$ is the smallest class of linear orders such that the following holds:
	\begin{enumerate}
		\item	$\zero,\one\in\M$,
		\item	if $\alpha,\beta\in\M$ then $\alpha+\beta\in\M$,
		\item	if $\alpha\in\M$ then $\alpha\cdot\omega,\alpha\cdot\dual{\omega}\in\M$,
		\item	for every finite $F\subseteq\M$, it holds that $\sigma(F)\in\M$.
	\end{enumerate}
\end{dfn}

\begin{prp}\label{prp:Msumint}
	If $\alpha_i\in\M$ for each $i\in\ints$ then there exists a $\beta\in\M$ such that\
	\begin{equation}
		\sum_{i\in\zeta}\alpha_i\nequiv{n}\beta.
	\end{equation}
\end{prp}
\begin{proof}
	Since $\M$ is closed under finite sums as well as right multiplication by $\omega$ and $\dual{\omega}$, results analagous to propositions \ref{prp:M0sum} and \ref{prp:M0sumop} can be proven in the context of $\M$.  An argument similar to that which appears in the proof of proposition \ref{prp:M0sumint} then yields the desired result.
\end{proof}

\begin{thm}[L\"auchli and Leonard]
	For every countable linear order $\alpha$, and every $n\in\nats$, there exists some $\beta_n\in\M$ such that $\alpha\nequiv{n}\beta_n$.
\end{thm}
\begin{proof}
	Define a binary relation $R$ on $\alpha$ such that, for every $x,y\in\alpha$, $xRy$ iff $x\leq y$ and there exists a $\beta\in\M$ such that $[x,y]\nequiv{n}\beta$.  Since $R$ is clearly transitive (simply note $\M$ is closed under finite sums), it induces a congruence $\sim$ on $\alpha$.    We now claim the following:
	\begin{claim}
		For every $\gamma\in\faktor{\alpha}{\sim}$, there exists a $\beta\in\M$ such that $\gamma\nequiv{n}\beta$.
	\end{claim}
	\begin{proof}
		Fix any $\gamma\in\faktor{\alpha}{\sim}$ and note that $\M$ contains all finite ordinals so we may assume, without loss of generality, that $\gamma$ is infinite.  Consequently, at least one of $\omega$ and $\dual{\omega}$ is embeddable in $\gamma$.  There must then exist a family $\family{\gamma_i}{i\in\zeta}$ of (possibly empty) linear orders such that
		\begin{equation}
			\gamma\cong\sum_{i\in\zeta}\gamma_i.
		\end{equation}
		By definition of $\sim$ it now follows that there exists a $\gamma_i^\prime\in\M$, for each $i\in\ints$, such that $\gamma_i\nequiv{n}\gamma_i^\prime$.  Proposition \ref{prp:Msumint} then implies that there exists a $\gamma^\prime\in\M$ such that
		\begin{equation}
			\gamma^\prime\nequiv{n}\sum_{i\in\zeta}\gamma_i^\prime.
		\end{equation}
		From lemma \ref{lem:fvsum} we may conclude that
		\begin{equation}
			\gamma\nequiv{n}\sum_{i\in\zeta}\gamma_i^\prime\nequiv{n}\gamma^\prime,
		\end{equation}
		establishing the claim.
	\end{proof}

	Note that if $\card{\faktor{\alpha}{\sim}}=1$ then our task is complete so suppose, by way of contradiction, that $\card{\faktor{\alpha}{\sim}}>1$.
	\begin{claim}
		$\faktor{\alpha}{\sim}$ is dense.
	\end{claim}
	\begin{proof}
		With our goal being a contradiction, suppose that $\faktor{\alpha}{\sim}$ is not dense.  Therefore, there exists $a,b\in\alpha$ such that $a<b$, $a\not\sim b$ and for every $c\in(a,b)$ either $c\sim a$ or $c\sim b$ but not both.  Note that, since $a\not\sim b$, the interval $(a,b)$ cannot be finite.  Therefore, $(a,b)\cong\sum_{i\in\zeta}\gamma_i$ where $\gamma_i$ is $n$-equivalent to some member $\gamma_i^\prime$ of $\M$, for each $i\in\ints$.  Thus, by proposition \ref{prp:Msumint}, $(a,b)\nequiv{n}\gamma$ for some $\gamma\in\M$.  Since $\one+\gamma+\one\nequiv{n}[a,b]$ by lemma \ref{lem:fvsum} and $\one+\gamma+\one$ is a member of $\M$, it follows by definition that $a\sim b$ --- a contradiction.
	\end{proof}

	Now choose $K=\set{\chi_0,\dotsc,\chi_{m-1}}$, for some $m\in\posnats$, to be an $n$-spectrum for the class $\setbuild{\chi\in\M}{\chi\nequiv{n}\gamma,\text{ for some }\gamma\in\faktor{\alpha}{\sim}}$.
	\begin{claim}
		There exists a maximal $F\subseteq K$ and a nonempty interval $I=(A,B)$ of $\faktor{\alpha}{\sim}$ such that, for every $\chi\in F$, the set $D_\chi(I)=\setbuild{\gamma\in I}{\gamma\nequiv{n}\chi}$ is dense in $I$.
	\end{claim}
	\begin{proof}
		Argue by induction on $m$.  If $m=1$ then, since $\faktor{\alpha}{\sim}$ is dense we can choose any $A,B\in\faktor{\alpha}{\sim}$ such that $A<B$ and it will immediately follow that if $I=(A,B)$ then $D_{\chi_0}(I)$ is dense in $I$, since every member of $\faktor{\alpha}{\sim}$ is $n$-equivalent to $\chi_0$.

		Assume now that the claim holds for each $m<m^\prime\in\posnats$.  Let $m=m^\prime$ and suppose to the contrary that for every $F^\prime\subseteq K$ and every (nonempty) open interval $J\subseteq\faktor{\alpha}{\sim}$ there exists some $\chi_k\in F^\prime$ such that $0\leq k<m$ and $D_{\chi_k}(J)$ is not dense in $J$.  Note that, by assumption, there must exist a $\chi\in K$ such that $D_{\chi}(\faktor{\alpha}{\sim})$ is not dense in $\faktor{\alpha}{\sim}$.  Without loss of generality we may assume that $\chi=\chi_{m-1}$ and consequently there exists a (nonempty) open subset $J$ of $\faktor{\alpha}{\sim}$ such that $D_{\chi_{m-1}}(J)$ is not dense in $J$.  Note in particular that we can choose $J$ such that $\chi_{m-1}$ has no $n$-equivalent in $J$.

		By definition, there exists an $F^\prime\subseteq\set{\chi_0,\dotsc,\chi_{m-2}}$ which is an $n$-spectrum for the class $\setbuild{\chi^\prime\in\M}{\chi\nequiv{n}\gamma\text{ for some }\gamma\in J}$.  It then follows from the inductive hypothesis that there exists a maximal $F\subseteq F^\prime$ and some open interval $I=(A,B)\subseteq J$, where $A<B$, such that $D_{\chi}(I)$ is dense in $I$, for each $\chi\in F$.  Since $\chi_{m-1}\notin J\supseteq I$, it follows that $\chi_{m-1}$ is not dense in $I$ and, consequently, $F\subseteq K$ is maximal, as required.  This is the sought-after contradiction, thus establishing the claim.
	\end{proof}
	Note that $F$ is now an $n$-spectrum for the class
	\begin{equation}
		\setbuild{\chi\in\M}{\chi\nequiv{n}\gamma\text{ for some }\gamma\in I}.
	\end{equation}
	Consequently, since $D_{\chi}(I)$ is dense in $I$ for each $\chi\in F$, it follows that
	\begin{equation}
		\sigma(F)\nequiv{n}\sum I=\sum_{\gamma\in I}\gamma.
	\end{equation}
	Therefore, since $\sigma(F)\in\M$, it follows that $I$ has exactly one element.  This then contradicts the fact that $I$ is an open interval.  Hence, we may now conclude that $\faktor{\alpha}{\sim}$ has only one element, yielding the result.

\end{proof}

\begin{lem}
	The set $\setbuild{(\alpha,\sigma)}{\alpha\in\M\text{ and }\alpha\models\sigma}$ is recursively enumerable.  Assume, by way of contradiction, that for some natural $k<m^\prime$ it holds that $D_{\chi_k}$ is not dense in $I$.
\end{lem}

\begin{thm}
	The theory of linear orders is decidable.
\end{thm}


\section{Model-theoretic aspects of the class \text{$\dense$}}

\begin{dfn}[Prime model]
	Suppose $T$ is an $L$-theory and $\mathfrak{M}$ is a model of the theory $T$. It is then said to $\mathfrak{M}$ is a \textbf{prime model} of $T$ whenever, for every $\mathfrak{N}\in\Mod(T)$, there exists an elementary embedding $f\colon\mathfrak{M}\to\mathfrak{N}$.
\end{dfn}

\begin{thm}[Cantor's Theorem]
	If $\alpha\in\dense$ is countable then there exists an isomorphism $f\colon\eta\to\alpha$.
\end{thm}
\begin{proof}
	Let $\family{a_i}{i<\omega}$ and $\family{b_i}{i<\omega}$, respectively, be enumerations of $\eta$ and $\alpha$.  Define, for each $n\in\nats$, the set $A_n=\set{a_0,\dotsc,a_n}$.  Choose $A_0=\set{a_0}$, $B_0=\set{b_0}$ and let $h_0$ be the unique map $h\colon A_0\to B_0$.  Now suppose $h_{n-1}\colon A_{n-1}\to B_{n-1}$ has been defined for some $n\in\nats$ and choose any $b_n\in\domain{}\alpha\setminus B_{n-1}$.  Define $B_n=B_{n-1}\cup\set{b_n}$ and choose each $h_n\colon A_n\to B_n$ such that $h_n\restriction A_{n-1}=h_{n-1}$, where $A_n=\set{a_0,\dotsc,a_n}$ and the following holds:
	\begin{enumerate}
		\item
	\end{enumerate}
\end{proof}

\begin{dfn}[Quantifier elimination]
	An $L$-theory $T$ is said to \textbf{admit quantifier elimination} (or \textit{has} quantifier elimination) if for every $k\in\nats$ and every $\varphi\in L_k$ there exists a quantifier-free formula $\psi\in L_k$ such that $\varphi$ is $T$-equivalent (or equivalent modulo $T$) to $\psi$ i.e.\ $T\models\forall\bar{x}(\varphi(\bar{x})\leftrightarrow\psi(\bar{x}))$.
\end{dfn}

\begin{dfn}[Model-completeness]
	An $L$-theory $T$ is said to be \textbf{model-complete} if every embedding $f\colon\mathfrak{M}\to\mathfrak{N}$, between models $\mathfrak{M}$ and $\mathfrak{N}$ of $T$, is elementary.
\end{dfn}
