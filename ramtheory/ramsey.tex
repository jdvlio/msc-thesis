\bibliographystyle{amsalpha}

\chapter{Ramsey theory}

In the area of mathematics known as Ramsey Theory, one studies the relationship
between the cardinality of a specified mathematical structure and the occurence
(or total absence) of certain ``uniformities''.  The archetypical example
involves colourings of the edges of the (complete) graph $K_n$ on $n$ vertices:
how large should $n$ be to guarantee the existence of a particular monochromatic
cycle?

Problems of this flavour can also be formulated for infinite structures and form
the foundation of a body of set theory commonly referred to as combinatorial
set theory.  Many of the problems we will encounter will be of this nature.

We lay out the terminology for and briefly discuss Ramsey's theorem as it
appears in the context of $k$-colourings of $n$ element subsets of $\nats$.
This is followed by an analagous result, formulated by Shelah, which pertains to
$k$-colourings of subintervals of a given (infinite) linear order, irrespective
of the countability of the underlying set.

\section{Classical Ramsey Theorem}

A formalisation is required of the aforementioned notion of ``uniformity''.
This would be the essence of the following definition, formulated in terms of
partitions.  These partitions, of which there will be finitely many, are
intended to represent the ``colours'' of the elements: two elements belong to
the same member of the partition iff they share a colour.

\begin{dfn}[Homogeneous set for a partition]
   If $A$ is a nonempty set and, for some $k\in\posnats$, $\mathcal{C}$ is a
   partition of $\subsets{k}{A}$ then we will call $X\subseteq A$
   \textbf{homogeneous for the partition} $\mathcal{C}$ if there exists a $C\in
   \mathcal{C}$ such that $\subsets{k}{X}\subseteq C$.
\end{dfn}

Rephrased in terms of colours, a set $X$ is homogeneous for a partition whenever
$\subsets{k}{X}$ is a monochromatic subset of $\subsets{k}{\nats}$.  That is to
say all its elements share the same colour.

\begin{thm}[Ramsey's Theorem]
	Suppose $\mathcal{C}$ is a finite partition of $\subsets{k}{X}$ and
	$\card{X}=\aleph_0$, for some $k\in\posnats$, then there exists an infinite
	set $H\subseteq\nats$ such that $H$ is homogeneous for $\mathcal{C}$.
\end{thm}
\begin{proof}
	Note that it is enough to consider the case $X=\nats$ and proceed by
	induction on $k$. If $k=1$ then there must exist an infinite
	$C\in\mathcal{C}$ and thus $H=C$ is homogeneous for $\mathcal{C}$.

	Suppose the result holds for $k\in\posnats$ and let
	$\mathcal{C}=\set{C_0,\dotsc,C_{p-1}}$ be a partition of
	$\subsets{k+1}{\nats}$.  We proceed by recursively defining a sequence
	$a=\family{a_i}{i<\omega}$ of natural numbers and a sequence
	$\family{H_i}{i<\omega}$ of subsets of $\nats$ as follows.  For $i<p$,
	$b\in\nats$ and $S\subseteq\nats$, the set $C_i(S,b)$ is defined by
	\begin{equation}
		C_i(b,S)=\setbuild{A\in\subsets{k}{S\setminus\set{b}}}{A\cup\set{b}\in
		C_i}.
	\end{equation}

	Choose $a_0=0$ and $H_0=\nats\setminus\set{0}$.  Now assume $a_j$ and $H_j$
	have been defined, for some $j<\omega$, and that $H_j$ is infinite.  Define
	$a_{j+1}$ to be the least element of $H_j$ then
	\begin{equation}
		\mathcal{B}_{j+1}=\setbuild{C_i\big(a_{j+1},H_j\big)}{
		i<p}\setminus\set{\emptyset}
	\end{equation}
	is a partition of $\subsets{k}{H_j\setminus\set{a_{j+1}}}$.

	From the inductive hypothesis, there exists an infinite $H_{j+1}\subseteq
	H_j\setminus\set{a_{j+1}}$ such that $H_{j+1}$ is homogeneous for
	$\mathcal{B}_{j+1}$.  This completes the recursive definition so that all that
	remains is to show that the set
	\begin{equation}
		H=\setbuild{a_i}{0<i<\omega}
	\end{equation}
	is homogeneous for $\mathcal{C}$.

	Without loss of generality, we may assume it holds that
	\begin{equation}
		\subsets{k}{H_1}\subseteq C_0(a_1,H_0)
	\end{equation}
	and thus $\subsets{k}{H_1}\subseteq C_0$.  Consequently, it follows by
	definition that $\subsets{k}{H_j}\subseteq C_0$, for all $j>0$.

	By construction, for each $j>0$, all sets which are of the form
	$X\cup\set{a_j}$, for some $X\in\subsets{k+1}{H_{j+1}}$, must necessarily be
	members of $C_0$.  Now, suppose $X\in\subsets{k}{H}$ and let $b$ be the
	least element of $X$.  Define $X_0=X\setminus\set{b}$ and note that there
	exists some $j_0>0$ such that $b=a_{j_0}$.

	Since $a$ is clearly (strictly) increasing, the members of $X_0$ all occur
	in the sequence $a$ at some point after the index $j_0$. Therefore,
	$X_0\in\subsets{k}{H_{j+1}}$ and thus we may conclude that $X\in C_0$.  This
	then proves that $\subsets{k+1}{H}\subseteq C_0$, as required.
\end{proof}

In an appeal to the imagination, a colouring of a countably infinite set always
gives rise to an infinite monochromatic subset.  Thus, it is impossible to
arrange the colours in such a manner that all the monochromatic subsets are
finite.

It would now be prudent to formally define the term \textit{colouring}.  This is
done in terms of surjective maps and will serve in giving us a better (or at
least different) view of certain concepts already introduced.

\begin{dfn}[Colouring]\label{def:Col}
	If $S$ is a nonempty set and $\card{C}=k\in\posnats$ then a
	$\mathbf{k}$\textbf{-colouring} of $S$ is a surjective map $f\colon  S\to
	C$.  We call a surjection $f$ simply a \textbf{colouring} if it is a
	$k$-colouring for some $k$.
\end{dfn}

Note that if $\mathcal{C}$ is some partition of a nonempty set $S$ then there
exists a unique surjection $f\colon S\to \mathcal{C}$ such that $x\in f(x)$, for
each $x\in S$. Since every surjection $f\colon A\to B$ induces a partition of
its domain $A$, by taking $\mathcal{C}=\setbuild{f^{-1}[x]}{x\in B}$, this
justifies the following definition:

\begin{dfn}[Homogeneous set for a colouring]
	Suppose $A$ is a nonempty set and $f\colon A\to C$ is a colouring.  We call
	$X\subseteq A$ \textbf{homogeneous for} $\bm{f}$ whenever there exists a $c\in
	C$ such that $X\subseteq f^{-1}[c]$ or, equivalently, $f(x)=f(y)$ for every
	$x,y\in X$.
\end{dfn}

If $A$ is some linearly ordered set and $k\in\posnats$ we say that a tuple
$\bar{a}\in A^k$ is \textit{increasing} whenever its components form an
increasing sequence i.e.\ it holds that $a_0<\dotsb<a_{k-1}$.

Denote the set of increasing $k$-tuples of $A$ as $A^+(k)$ and let $f\colon
\subsets{k}{A}\to C$ be some arbitrary colouring of $A$.  Defining the map
$\tau_f\colon A^+(k)\to C$ such that
$\tau_f(a_0,\dotsc,a_{k-1})=f(\set{a_0,\dotsc,a_{k-1}})$, for
$(a_0,\dotsc,a_{k-1})\in A^+(k)$, there exists a unique bijection $h\colon
\subsets{k}{A}\to A^+(k)$ such that the diagram
\begin{center}
	\begin{tikzcd}
		\subsets{k}{A}\arrow[r,"h"]\arrow[rd,"f"']	& A^+(k)\arrow[d,"\tau_f"] \\
		& C
	\end{tikzcd}
\end{center}
commutes.  The unconvinced reader need only consider the function $h$ inverse
to:
\begin{equation}
	(a_0,\dotsc,a_{k-1})\mapsto\set{a_0,\dotsc,a_{k-1}}.
\end{equation}
Noting the surjectivity of $\tau_f$ guarantees the uniqueness of $h$,
relative to the commutativity of the diagram above.

\begin{conv}[Colourings]\label{rem:Col}
	In light of these observations, we will henceforth identify the maps $f$ and
	$\tau_f$.  This allows us to write $f(a_0,\dotsc,a_{k-1})$ as
	shorthand for $f(\set{a_0,\dotsc,a_{k-1}})$.

	When $k=2$ we take this a step further by also identifying $f$ with the map
	that sends subintervals $[a,b]$ of $A$, under the condition $a<b$, to the
	colour $\tau_f(a,b)$.
\end{conv}

\begin{rem}[Homogeneity]\label{rem:hom}
	Note that if $f$ is a colouring of $\subsets{k}{S}$, for some nonempty set
	$S$ and some $k\in\posnats$, then it follows by definition that $H\subseteq
	S$ is homogeneous for the partition $\setbuild{f^{-1}[c]}{c\in\range f}$ if
	and only if $\subsets{k}{H}$ is homogeneous for $f$.  This establishes the
	relationship between the two notions of homogeneity.
\end{rem}


\begin{dfn}[Cofinality]
	If $\alpha$ is a linear order then $\cf(\alpha)$, called the
	\textbf{cofinality} of $\alpha$, is the least ordinal $\beta$ with the
	following property: there exists an embedding $f\colon\beta\to\alpha$ such
	that $f[\beta]$ is not bounded above by a non-maximal element of $\alpha$.
\end{dfn}

In clarification of the following definition, we use the phrase
\textit{``transfinite sequence''} to refer to any function whose domain is some
ordinal.  Hence, finite sequences (tuples) may also be considered in this
regard.

\begin{dfn}[Cofinal sequence]
	A \textbf{cofinal sequence} in a linear order $\alpha$ is any (strictly)
	increasing transfinite sequence $(a_\gamma)_{\gamma<\beta}$, where
	$\beta\geq\cf(\alpha)$, such that $\setbuild{a_\gamma}{\gamma<\beta}$ is
	not bounded above by any non-maximal element of $\alpha$.
\end{dfn}

\begin{dfn}[Homogeneous sequence]
	If $\alpha$ is a linear order and $f$ is a colouring of
	$\subsets{2}{\domain\alpha}$ then a \textbf{homogeneous sequence for
	$\bm{f}$} is a cofinal sequence $\family{a_\gamma}{\gamma<\beta}$ in
	$\alpha$ such that
	\begin{equation}
		H=\setbuild{a_\gamma}{\gamma<\beta}
	\end{equation}
	is homogeneous for $f$.
\end{dfn}

\begin{cor}[Existence of homogeneous sequences]\label{cor:Cofinal}
	Suppose $A$ is an infinite linearly ordered set without a greatest element.
	If $A$ has cofinality $\omega$ and $f$ is a colouring of $\subsets{2}{A}$
	then there exists a homogeneous sequence $x=\family{a_i}{i<\omega}$ for $f$
	in $A$.
\end{cor}
\begin{proof}
	Choose $y=(b_i)_{i<\omega}$ to be cofinal in $A$.  Define
	\begin{equation}
		B=\setbuild{b_i}{i<\omega}
	\end{equation}
	and let $g=f\restriction B$.

	As superfluous colours may be discarded, thereby restricting the codomain of
	$g$, we may safely assume that $g$ is surjective and thus a colouring of
	$B$.  It follows from Remark \ref{rem:hom} and Ramsey's Theorem there exists
	an $H\subseteq B$ which is homogeneous for $g$ and, clearly then, also for
	$f$.

	Since $y$ is increasing, there exists a unique subsequence
	$x=(b_{k_i})_{i<\omega}$ of $y$ with image $H$.  By definition $x$ is the
	desired homogeneous sequence for the colouring $f$.
\end{proof}

We conclude this exposition by providing an illustrative example.  The reader is
encouraged to thoroughly internalise its contents.  Similar arguments will be
employed throughout the text and it is, consequently, more significant than
first appearances may suggest.

\begin{exm}[Cofinal sequences in $\lambda$]
	We aim to show that every colouring $f$ of $\reals$ is accompanied by a
	corresponding homogeneous sequence.  Note that $\cf(\lambda)=\omega$ since
	$\lambda$ has no greatest element and $\family{k}{k<\omega}$ is a cofinal
	sequence in $\lambda$.

	Now suppose $f$ is a colouring of $\subsets{2}{\reals}$ and let $x=(x_i)_{i<\omega}$
	be a cofinal sequence in $\lambda$.  Let $X=\setbuild{x_i}{i<\omega}$ and
	define $g=f\restriction \subsets{2}{X}$.

	Similar to the proof of Corollary \ref{cor:Cofinal}, we may assume $g$ is
	surjective and, hence, a colouring of the set $\subsets{2}{X}$.  Since $x$
	is clearly cofinal in $X$, it follows from Corollary \ref{cor:Cofinal} that
	there exists a subsequence $y$ of $x$ which is homogeneous for $g$.  As $y$
	is homogeneous for $g$ and $X$ is a cofinal subset of $\lambda$, it follows
	by definition of $g$ that $y$ is a homogeneous sequence for $f$.
\end{exm}


\section{Additive Ramsey Theorem}

A natural question that arises is whether or not Corollary \ref{cor:Cofinal}
could be extended to linear orders of \textit{uncountable} cofinality.
Unfortunately this is not the case, as illustrated by the following example.

\begin{exm}[Sierpi\'nski colouring]
	Suppose $\alpha=(\reals,\prec)$ is a well-order and assume
	$\cf(\alpha)\geq\omega_1$.  Note that the uncountability of $\reals$
	guarantees the existence of such an $\alpha$ e.g.\ by choosing $\alpha$
	isomorphic to the initial ordinal of cardinality $2^{\aleph_0}$.

	The \textit{Sierpi\'nski colouring} $s\colon\subsets{2}{\reals}\to\set{0,1}$
	is defined, for every $x,y\in\reals$, such that
	\begin{equation}
		s(x,y)=
		\begin{cases}
			1,  &\text{if }x\prec y\text{ and }x<y,\\
			1,  &\text{if }x\nprec y\text{ and }x\not< y,\\
			0,   &\text{otherwise.}
		\end{cases}
	\end{equation}
	That is to say that $s(x,y)=1$ iff $<$ and $\prec$ agree on the order of
	$x$ and $y$.

	Since $\lambda$ is clearly not a well-order it follows that $s$ is
	necessarily a surjection and, thus, a $2$-colouring of
	$\subsets{2}{\reals}$.  By way of contradiction, we assume there exists a
	homogeneous sequence $x=(x_{\gamma})_{\gamma<\beta}$ for $s$ in $\alpha$.

	By definition, $x$ must be either an increasing or (otherwise) decreasing
	sequence in $\lambda$.  Since $\lambda\cong\dual{\lambda}$, we may assume
	without loss of generality that $x$ is increasing.

	It now follows, by assumption, that $\beta\geq\cf(\alpha)\geq\omega_1$.
	However, this implies that there exists a (necessarily uncountable) set
	\begin{equation}
		\setbuild{(x_\gamma,x_{\gamma+1})}{\gamma<\omega_1}
	\end{equation}
	of pairwise disjoint open intervals of $\lambda$, thereby yielding the
	desired contradiction.
\end{exm}

Although our hopes may have been dashed by the late Sierpi\'nski, all hope is
not lost.  Our saving grace turns out to be the following definition:
\begin{dfn}[Additive colouring \cite{ShelahOrder}]
	An \textbf{additive colouring} of a linear order $\alpha$ is a $2$-colouring
	$f$ of $\domain{\alpha}$ such that the equations $f(x_0,y_0)=f(x_1,y_1)$ and
	$f(y_0,z_0)=f(y_1,z_1)$ imply $f(x_0,z_0)=f(x_1,z_1)$.
\end{dfn}

In the context of a $2$-colouring $f$ of the domain of a linear order $\alpha$,
one may interpret the ``joining together'' of subintervals $(a,b)$ and $(b,c)$
to form the interval $(a,c)$ as the process of mixing the colours $f(a,b)$ and
$f(b,c)$ to obtain the colour $f(a,c)$.

In this intuitive framework, one could think of $f$ as being additive precisely
when colours of intervals, under $f$, mix in a consistent fashion.  For example,
if a blue interval and a red interval can be mixed to create a purple interval
then this should be the case for \textit{any} choice of the component intervals,
given that the first one is blue and the second red.

\begin{rem}[Addition of colours]
	As the name might suggest, an \textit{additive} colouring $f$ of a linear
	order $\alpha$ gives rise to a (partial) binary operation $\oplus$ on the
	set of colours (i.e.\ the range of the function $f$).  That is to say one
	defines, in the obvious manner,
	\begin{equation}
		f(a,b)\oplus f(b,c)=f(a,c).
	\end{equation}
\end{rem}

\begin{thm}[Additive Ramsey Theorem \cite{ShelahOrder}]
	If $\delta$ is a limit ordinal, $\beta=\cf(\delta)$ and $f$ is an additive
	colouring of $\delta$ then there exists a homogeneous sequence
	$x=(\alpha_\gamma)_{\gamma<\beta}$ for $f$.
\end{thm}

\begin{proof}
	For every $\alpha,\alpha^\prime<\delta$, define $\alpha\sim\alpha^\prime$ whenever there exists a $\gamma_0<\delta$ such that $\alpha_0,\alpha_1<\gamma_0$ and $f(\alpha,\gamma_0)=f(\alpha^\prime,\gamma_0)$.  We now prove the following claim:
	\begin{claim}
		The binary relation $\sim$ is an equivalence relation
	\end{claim}
	\begin{proof}[Proof of claim.]
		Fix any $\alpha_0,\alpha_1,\alpha_2<\delta$.  Since $f(\alpha_0,\alpha_0+1)=f(\alpha_0,\alpha_0+1)$ it follows that $\sim$ is reflexive.

		If there exists a $\gamma_0>\alpha_0,\alpha_1$ such that $\gamma_0<\delta$ and $f(\alpha_0,\gamma_0)=f(\alpha_1,\gamma_0)$ then $f(\alpha_1,\gamma_0)=f(\alpha_0,\gamma_0)$ so that $\sim$ is symmetric.

		All that remains is to esablish transitivity.
		Suppse that $\gamma_0>\alpha_1,\alpha_0$ and $\gamma_1>\alpha_1,\alpha_2$
		such that $\gamma_0,\gamma_1<\delta$.  Furthermore, assume both the
		equations:
		\begin{align}
			f(\alpha_0,\gamma_0)&=f(\alpha_1,\gamma_0),\\
			f(\alpha_1,\gamma_1)&=f(\alpha_2,\gamma_1).
		\end{align}

		If $\gamma_0=\gamma_1$ then it follows immediately that
		$f(\alpha_0,\gamma_0)=f(\alpha_2,\gamma_0)$ and thus
		$\alpha_0\sim\alpha_2$, so suppose instead that $\gamma_0<\gamma_1$.

		Since, trivially, $f(\gamma_0,\gamma_1)=f(\gamma_0,\gamma_1)$, it
		follows by additivity of $f$ that
		$f(\alpha_0,\gamma_1)=f(\alpha_1,\gamma_1)$ and, therefore,
		$f(\alpha_0,\gamma_1)=f(\alpha_2,\gamma_1)$. so that
		$\alpha_0\sim\alpha_2$, as required.\noqed

		Since the case $\gamma_1<\gamma_0$ is similar, the relation $\sim$ is
		transitive, implying $\sim$ is an equivalence relation.
	\end{proof}

	Note that $\sim$ has at most $\card{\range f}<\aleph_0$ equivalence classes
	and there must exist an equivalence class $C$ under $\sim$ which is
	unbounded in $\delta$.  We will now define a cofinal sequence
	$x=(\alpha_{\gamma})_{\gamma<\beta}$, where $\beta=\cf(\delta)$, by means
	of transfinite recursion.

	Let $\alpha_0$ be the least element of $C$ and define, for each $c\in\range
	f$, the set $I_c=\setbuild{\alpha\in C}{\alpha_0<\alpha\text{ and
	}f(\alpha_0,\alpha)=c}$.  It then follows, by definition, that
	\begin{equation}
		C\setminus\set{\alpha_0}=\bigcup_{c\in\range f}I_c.
	\end{equation}

	Since $C$ is unbounded in $\delta$ and $\card{\range f}<\card{\cf(\delta)}$
	it follows that there exists a $d\in\range f$ such that $I_d$ is unbounded
	in $\delta$.  Note that there exists a cofinal sequence
	$(\delta_\gamma)_{\gamma<\beta}$ in $I_d$.

	Now, continuing the recursion, choose $\alpha_1$ to be the least member of
	$I_d$ and assume $\alpha_\gamma\in I_d$ has been defined for each
	$\gamma<\epsilon$ and some ordinal $\epsilon<\beta=\cf(\delta)$.  We
	suppose, whenever $\gamma<\xi<\epsilon$ and
	$\gamma^\prime<\xi^\prime<\epsilon$, that
	\begin{equation}
		f(\alpha_\gamma,\alpha_\xi)=f(\alpha_{\gamma^\prime},\alpha_{\xi^\prime}).
	\end{equation}
	For each $\gamma<\xi<\epsilon$ define $\sigma_{\gamma,\xi}$ to be the least
	member of $C$ such that
	\begin{equation}\label{eq:addram}
		f(\alpha_\gamma,\sigma_{\gamma,\xi})=f(\alpha_\xi,\sigma_{\gamma,\xi}).
	\end{equation}

	Choose $\alpha_\epsilon$ to be the first $\delta_\gamma$ such that
	$\gamma\geq\epsilon$ and
	$\delta_\gamma>\sup\setbuild{\sigma_{\nu,\xi}}{\nu<\xi<\epsilon}$.
	Consequently, for each $\gamma<\xi<\epsilon$, it follows from
	(\ref{eq:addram}) and the additivity of $f$ that
	\begin{equation}
		f(\alpha_\gamma,\alpha_\epsilon)=f(\alpha_\xi,\alpha_\epsilon).
	\end{equation}
	It now follows from the construction that the transfinite sequence
	$\family{\alpha_\gamma}{\gamma<\beta}$ is homogeneous for $f$.
\end{proof}

\begin{rem}
	If in the above proof we substitute for the colouring $f$ any surjection
	with the property that $\card{\range f}<\card{\cf(\delta)}$ then the proof
	remains valid.  Hence, it is not necessary for the proof that $\card{\range
	f}<\aleph_0$ when $\cf(\delta)\geq\omega_1$.
\end{rem}

One could also view additive colourings, through a categorical lense, as a
functor between appropriately chosen categories.  Leting $\cat{\alpha}$ denote
the category whose \textit{objects} are the elements of $\alpha$ and whose
\textit{arrows} are (closed) intervals of the form $[a,b]\colon a\to b$.  It is
readily verified that $\cat{\alpha}$ is in fact a legitimate category.

Given this perspective, an additive colouring of $\alpha$ is then nothing more
than a functor $\cat{\alpha}$ to a monoid $\cat{M}$, informally a ``group
without inverses'', considered as a category (specifically, a category with a
single object).

The arrows of $\cat{M}$, i.e.\ the monoid elements, then represent what we have
referred to as ``colours'' while $\circ$ corresponds simultaneously to the
aforementioned $\oplus$ operation and multiplication within the monoid itself.

\begin{cor}[Existence of homogeneous sequences]
	Suppose $\alpha$ is a linear order, without a greatest element, and $f$ is
	an additive colouring of $\alpha$. There must then exist a limit ordinal
	$\delta\geq\cf(\alpha)$ and a homogeneous sequence
	$(a_\gamma)_{\gamma<\delta}$ for $f$.
\end{cor}

\begin{proof}
	Let $(b_\gamma)_{\gamma<\delta^\prime}$ be some cofinal sequence in $\alpha$
	such that $\delta^\prime=\cf(\alpha)$.  Note that, since $\alpha$ has no
	greatest element, $\delta^\prime=\cf(\alpha)$ is a limit ordinal.

	Define $B=\setbuild{b_\gamma}{\gamma<\delta^\prime}$ and let
	$g=f\restriction{\subsets{2}{B}}$ then we may assume, without loss of
	generality, that $g$ is surjective and hence a colouring of $B$.

	Since $B$ has order type $\delta^\prime$, it follows from the additive
	Ramsey Theorem that there exists, for some $\delta\geq\cf(\alpha)$, a
	transfinite sequence $x=(a_\gamma)_{\gamma<\delta}$ which is homogeneous for
	$g$.  Since $B$ is not bounded above, $\delta$ must be a limit ordinal and,
	by definition of $g$, $x$ is a homogeneous sequence for $f$, as required.
\end{proof}


\bibliography{references}
