\bibliographystyle{amsalpha}

\chapter{Ramsey theory}

In the area of mathematics known as Ramsey Theory, one studies the relationship
between the cardinality of a specified mathematical structure and the occurence
(or total absence) of certain ``uniformities''.  The archetypical example
involves colourings of the edges of the (complete) graph $K_n$ on $n$ vertices:
how large should $n$ be to guarantee the existence of a particular monochromatic
cycle?

Problems of this flavour can also be formulated for infinite structures and form
the foundation for a body of set theory commonly referred to as combinatorial
set theory.  The infinite case is where our interest lies as we will be placing
a heavy emphasis on arguments which involve colourings of certain infinite
linear orders.

We lay out the terminology for and briefly discuss Ramsey's theorem as it
appears in the context of $k$-colourings of $n$ element subsets of $\nats$.
This is followed by an analagous result, formulated by Shelah, which pertains to
$k$-colourings of subintervals of a given (infinite) linear order, irrespective
of the countability of the underlying set.

\section{Classical Ramsey Theorem}


\begin{dfn}[Homogeneous set for a partition]
   If $A$ is a nonempty set and, for some $k\in\posnats$, $\mathcal{C}$ is a
   parition of $\subsets{k}{A}$ then we will call $X\subseteq A$
   \textbf{homogeneous for the partition} $\mathcal{C}$ if there exists a $C\in
   \mathcal{C}$ such that $\subsets{k}{X}\subseteq C$.
\end{dfn}

\begin{thm}[Ramsey's Theorem]
	Suppose $\mathcal{C}$ is a finite partition of $[\nats]^k$, for some
	$k\in\posnats$, then there exists an infinite set $X\subseteq\nats$ such
	that $X$ is homogeneous for $\mathcal{C}$.
\end{thm}
\begin{proof}
	We proceed by induction on $k$. If $k=1$ then there must exists an infinite
	$C\in\mathcal{C}$ and thus $X=C$ is homogeneous for $\mathcal{C}$.

	Suppose the result holds for $k\in\posnats$ and let
	$\mathcal{C}=\set{C_0,\dotsc,C_{m-1}}$ be a partition of
	$\subsets{k+1}{\nats}$.  If we let $s\colon\nats\to\posnats$
	be the successor map then $s$ is a bijection.

	Now define
	\begin{equation}
		B_i=\setbuild{s^{-1}[X]}{X\in\subsets{k}{\posnats}\text{ and
		}X\cup\set{0}\in C_i},
	\end{equation}
	for each $i=0,\dotsc,m-1$.  Since $\mathcal{C}$ is a partition of
	$\subsets{k+1}{\nats}$, $s(x)\neq 0$ for any $x\in\nats$ and $s^{-1}$ is
	bijective it follows that $\mathcal{B}=\set{B_0,\dotsc,B_{m-1}}$ is a
	partition of $\subsets{k}{\nats}$.

	By the inductive hypothesis there exists an infinite set $H\subseteq\nats$
	which is homogeneous for $\mathcal{B}$.  Without loss of generality we may
	assume $\subsets{k}{H}\subseteq B_0$.  Therefore, by definition of $B_0$,
	if we define $H^\prime=\setbuild{s[X]\cup\set{0}}{X\in\subsets{k}{H}}$ then
	$H^\prime\subseteq C_0$, as required.
\end{proof}

\begin{dfn}[Colouring]\label{def:Col}
	If $S$ is a nonempty set and $\card{C}=k\in\posnats$ then a
	$\mathbf{k}$\textbf{-colouring} of $S$ is a surjective map $f\colon  S\to
	C$.  We call a surjection $f$ simply a \textbf{colouring} if it is a
	$k$-colouring for some $k$.
\end{dfn}

Note that if $\mathcal{C}$ is some partition of a nonempty set $S$ then there
exists a unique surjection $f\colon S\to \mathcal{C}$ such that
$\mathcal{C}=\setbuild{f^{-1}[X]}{X\in \mathcal{C}}$. Since every surjection
$f\colon A\to B$ induces a partition of its domain $A$, by taking
$\mathcal{C}=\setbuild{f^{-1}[x]}{x\in B}$, this justifies the following
definition:

\begin{dfn}[Homogeneous set for a colouring]
	Suppose $A$ is a nonempty set and $f\colon A\to C$ is a colouring.  We call
	$X\subseteq A$ \textbf{homogeneous for} $f$ whenever there exists a $c\in
	C$ such that $X\subseteq f^{-1}[c]$ or, equivalently, $f(x)=f(y)$ for every
	$x,y\in X$.
\end{dfn}

If $A$ is some linearly ordered set and $k\in\posnats$ we say that a tuple
$\bar{a}\in A^k$ is \textit{increasing} whenever its components form an
increasing sequence i.e.\ it holds that $a_0<\dotsb<a_{k-1}$.

Denote the set of increasing $k$-tuples of $A$ as $A^+(k)$ and let $f\colon
\subsets{k}{A}\to C$ be some arbitrary $k$-colouring of $A$.  Defining the map
$\tau_f\colon A^+(k)\to C$ such that
$\tau_f(a_0,\dotsc,a_{k-1})=f(\set{a_0,\dotsc,a_{k-1}})$, for
$(a_0,\dotsc,a_{k-1})\in A^+(k)$, there exists a unique bijection $h\colon
\subsets{k}{A}\to A^+(k)$ such that the diagram
\begin{center}
	\begin{tikzcd}
		\subsets{k}{A}\arrow[r,"h"]\arrow[rd,"f"']	& A^+(k)\arrow[d,"\tau_f"] \\
		& C
	\end{tikzcd}
\end{center}
commutes.  The unconvinced reader need only consider the map
\begin{equation}
	h\colon\set{a_0,\dotsc,a_{k-1}}\mapsto(a_0,\dotsc,a_{k-1})
\end{equation}
and noting that the surjectivity of $\tau_f$ guarantees the uniqueness of $h$
relative to commutativity of the diagram above.

\begin{conv}[Colourings]\label{rem:Col}
	In light of these observations, we will henceforth identify the maps $f$ and
	$\tau_f$.  This allows us to write $f(a_0,\dotsc,a_{k-1})$ as
	shorthand for $f(\set{a_0,\dotsc,a_{k-1}})$.

	When $k=2$ we take this a step further by also identifying $f$ with the map
	that sends subintervals $(a,b)$ of $A$, under the condition $a<b$, to the
	colour $\tau_f(a,b)$.
\end{conv}

\begin{rem}[Homogeneity]\label{rem:hom}
	Note that if $f$ is a colouring of $\subsets{k}{S}$, for some nonempty set
	$S$ and some $k\in\posnats$, then it follows by definition that $H\subseteq
	S$ is homogeneous for the partition $\setbuild{f^{-1}[c]}{c\in\range f}$ if
	and only if $\subsets{k}{H}$ is homogeneous for $f$.  This establishes the
	relationship between the two notions of homogeneity.
\end{rem}


\begin{dfn}[Cofinality]
	If $\alpha$ is a linear order then $\cf(\alpha)$, called the
	\textbf{cofinality} of $\alpha$, is the least ordinal $\beta$ so that there
	exists an embedding $f\colon\beta\to\alpha$ such that $f[\beta]$ is not
	bounded above in $\alpha$.
\end{dfn}

In order to clarify the context of the following definition, it is important to
point out that we use the phrase \textit{``transfinite sequence''} to refer to
any function whose domain is some ordinal.  Hence, finite sequences are also
perfectly valid candidates for playing this role.

\begin{dfn}[Cofinal sequence]
	A \textbf{cofinal sequence} in a linear order $\alpha$ is any (strictly)
	increasing transfinite sequence $(a_\gamma)_{\gamma<\beta}$, where
	$\beta\geq\cf(\alpha)$, such that $\setbuild{a_\gamma}{\gamma<\beta}$ is an
	unbounded subset of $\alpha$
\end{dfn}

\begin{dfn}[Homogeneous sequence]
	If $\alpha$ is a linear order and $f$ is a colouring of
	$\subsets{2}{\domain\alpha}$ then a \textbf{homogeneous sequence for $f$}
	is a cofinal sequence $\family{a_\gamma}{\gamma<\beta}$ in $\alpha$ such
	that
	\begin{equation}
		H=\setbuild{\set{a_\gamma,a_\delta}}{\gamma<\delta<\beta}
	\end{equation}
	is homogeneous for $f$.
\end{dfn}

\begin{cor}[Existence: homogeneous sequences]\label{cor:Cofinal}
	Suppose $A$ is an infinite linearly ordered set without a greatest element.
	If $A$ has cofinality $\omega$ and $f$ is a colouring of $\subsets{2}{A}$
	then there exists a homogeneous sequence $x=\family{a_i}{i<\omega}$ for $f$
	in $A$.
\end{cor}

\begin{proof}
	Choose $y=(b_i)_{i<\omega}$ to be cofinal in $A$.  If we define
	\begin{equation}
		B=\setbuild{\set{b_i,b_j}}{i<j<\omega}
	\end{equation}
	then by Ramsey's Theorem, and Remark \ref{rem:hom}, there exists an
	$H\subseteq B$ which is homogeneous for $f$.  Since $y$ is increasing,
	there exists a unique subsequence $x=(b_{k_i})_{i<\omega}$ of $y$ with
	image $H$.  By definition $x$ is the desired homogeneous sequence for the
	colouring $f$.
\end{proof}

\begin{exm}[Cofinal sequences in $\lambda$]

	We aim to show that every $2$-colouring $f$ of $\reals$ is accompanied by a
	corresponding homogeneous sequence.  Note that $\cf(\lambda)=\omega$ since
	$\lambda$ has no greatest element and $\family{k}{k<\omega}$ is a cofinal
	sequence in $\lambda$.

	Now suppose $f$ is a $2$-colouring of $\reals$ and let $x=(x_i)_{i<\omega}$
	be a cofinal sequence in $\lambda$.  Let $X=\setbuild{x_i}{i<\omega}$ and
	define $g=f\restriction \subsets{2}{X}$ then we may assume, without loss of
	generality, that $g$ is surjective and thus a $2$-colouring of $X$.  Since
	$x$ is clearly cofinal in $X$ it follows from corollary (\ref{cor:Cofinal})
	that there exists a subsequence $y$ of $x$ which is homogeneous for $g$.

	As $y$ is homogeneous for $g$ and $X$ is a cofinal subset of $\lambda$, it
	follows by definition of $g$ that $y$ is a homogeneous sequence for $f$.

\end{exm}


\section{Additive Ramsey Theorem}

A natural question that arises is whether or not corollary (\ref{cor:Cofinal})
could be extended to linear orders of \textit{uncountable} cofinality.
Unfortunately this is not the case, as illustrated by the example that follows.

\begin{exm}[Sierpi\'nski colouring]
	Suppose $\alpha=(\reals,\prec)$ is a well-order.  Take note that
	$\alpha\geq\omega_1$ and thus the ordinal sum $\omega+\alpha$ is simply
	$\alpha$ itself.

	Therefore, since every countable ordinal is bijectively equivalent to the
	ordinal $\omega$, we may assume $\cf(\alpha)\geq\omega_1$.  Intuitively
	speaking, one can could have otherwise ``moved'' any countable tail to the
	beginning of the ordinal by using an appropriate bijection and applying the
	fact that $\omega+\alpha$ is isomorphic to $\alpha$.

	Recalling remark \ref{rem:Col}, the \textit{Sierpi\'nski colouring}
	$s\colon\subsets{2}{\reals}\to\set{0,1}$ is defined, for every
	$x,y\in\reals$, as
	\begin{equation}
		s(x,y)=
		\begin{cases}
			1,  &\text{if }x\prec y,\\
			0,   &\text{otherwise.}
		\end{cases}
	\end{equation}

	Since $\lambda$ is clearly not a well-order it follows that $s$ is a
	$2$-colouring of $\reals$.  By way of contradiction, we assume there
	exists a homogeneous sequence $x=(x_{\gamma})_{\gamma<\beta}$ for $s$ in
	$\alpha$.

	It now follows, by assumption, that $\beta\geq\cf(\alpha)\geq\omega_1$.
	However, this gives rise to a necessarily uncountable family
	\begin{equation}
		\setbuild{(x_\gamma,x_{\gamma+1})}{\gamma<\omega_1}
	\end{equation}
	of pairwise disjoint open intervals of $\lambda$, yielding the desired
	contradiction.
\end{exm}

Although our hopes may have been dashed by the late Sierpi\'nski, all hope is
not lost.  Our saving grace turns out to be the following definition:
\begin{dfn}[Additive colouring \cite{ShelahOrder}]
	An \textbf{additive colouring} of a linear order $\alpha$ is a $2$-colouring
	$f$ of $\domain{\alpha}$ such that the equations $f(x_0,y_0)=f(x_1,y_1)$ and
	$f(y_0,z_0)=f(y_1,z_1)$ imply $f(x_0,z_0)=f(x_1,z_1)$.
\end{dfn}

In the context of a $2$-colouring $f$ of the domain of a linear order $\alpha$,
one may interpret the ``joining together'' of subintervals $(a,b)$ and $(b,c)$
to form the interval $(a,c)$ as the process of mixing the colours $f(a,b)$ and
$f(b,c)$ to obtain the colour $f(a,c)$.

In this intuitive framework, one could think of $f$ as being additive precisely
when colours of intervals, under $f$, mix in a consistent fashion.  For example,
if a blue interval and a red interval can be mixed to create a purple interval
then this should be the case for \textit{any} choice of the component intervals,
given that the first one is blue and the second red.

\begin{rem}[Addition of colours]
	As the name might suggest, an \textit{additive} colouring $f$ of a linear
	order $\alpha$ gives rise to a (partial) binary operation $\oplus$ on the
	set of colours (i.e.\ the range of the function $f$).  That is to say one
	defines, in the obvious manner,
	\begin{equation}
		f(a,b)\oplus f(b,c)=f(a,c).
	\end{equation}
\end{rem}

\begin{thm}[Additive Ramsey Theorem \cite{ShelahOrder}]
	If $\delta$ is a limit ordinal, $\beta=\cf(\delta)$ and $f$ is an additive
	colouring of $\delta$ then there exists a homogeneous sequence
	$x=(\alpha_\gamma)_{\gamma<\beta}$ for $f$.
\end{thm}

\begin{proof}
	For every $\alpha,\alpha^\prime<\delta$, define $\alpha\sim\alpha^\prime$ whenever there exists a $\gamma_0<\delta$ such that $\alpha_0,\alpha_1<\gamma_0$ and $f(\alpha,\gamma_0)=f(\alpha^\prime,\gamma_0)$.  We now prove the following claim:
	\begin{claim}
		The binary relation $\sim$ is an equivalence relation
	\end{claim}
	\begin{proof}[Proof of claim.]
		Fix any $\alpha_0,\alpha_1,\alpha_2<\delta$.  Since $f(\alpha_0,\alpha_0+1)=f(\alpha_0,\alpha_0+1)$ it follows that $\sim$ is reflexive.

		If there exists a $\gamma_0>\alpha_0,\alpha_1$ such that $\gamma_0<\delta$ and $f(\alpha_0,\gamma_0)=f(\alpha_1,\gamma_0)$ then $f(\alpha_1,\gamma_0)=f(\alpha_0,\gamma_0)$ so that $\sim$ is symmetric.

		All that remains is to esablish transitivity.
		Suppse that $\gamma_0>\alpha_1,\alpha_0$ and $\gamma_1>\alpha_1,\alpha_2$
		such that $\gamma_0,\gamma_1<\delta$.  Furthermore, assume both the
		equations:
		\begin{align}
			f(\alpha_0,\gamma_0)&=f(\alpha_1,\gamma_0),\\
			f(\alpha_1,\gamma_1)&=f(\alpha_2,\gamma_1).
		\end{align}

		If $\gamma_0=\gamma_1$ then it follows immediately that
		$f(\alpha_0,\gamma_0)=f(\alpha_2,\gamma_0)$ and thus
		$\alpha_0\sim\alpha_2$, so suppose instead that $\gamma_0<\gamma_1$.

		Since, trivially, $f(\gamma_0,\gamma_1)=f(\gamma_0,\gamma_1)$, it
		follows by additivity of $f$ that
		$f(\alpha_0,\gamma_1)=f(\alpha_1,\gamma_1)$ and, therefore,
		$f(\alpha_0,\gamma_1)=f(\alpha_2,\gamma_1)$. so that
		$\alpha_0\sim\alpha_2$, as required.\noqed

		Since the case $\gamma_1<\gamma_0$ is similar, the relation $\sim$ is
		transitive, implying $\sim$ is an equivalence relation.
	\end{proof}

	Note that $\sim$ has at most $\card{\range f}<\aleph_0$ equivalence classes
	and there must exist an equivalence class $C$ under $\sim$ which is
	unbounded in $\delta$.  We will now define a cofinal sequence
	$x=(\alpha_{\gamma})_{\gamma<\beta}$, where $\beta=\cf(\delta)$, by means
	of transfinite recursion.

	Let $\alpha_0$ be the least element of $C$ and define, for each $c\in\range
	f$, the set $I_c=\setbuild{\alpha\in C}{\alpha_0<\alpha\text{ and
	}f(\alpha_0,\alpha)=c}$.  It then follows, by definition, that
	\begin{equation}
		C\setminus\set{\alpha_0}=\bigcup_{c\in\range f}I_c.
	\end{equation}

	Since $C$ is unbounded in $\delta$ and $\card{\range f}<\card{\cf(\delta)}$
	it follows that there exists a $d\in\range f$ such that $I_d$ is unbounded
	in $\delta$.  Note that there exists a cofinal sequence
	$(\delta_\gamma)_{\gamma<\beta}$ in $I_d$.

	Now, continuing with the recursion, assume $\alpha_\gamma\in C$ has been
	defined for each $\gamma<\epsilon$ and some ordinal
	$\epsilon<\beta=\cf(\delta)$.  We suppose, whenever $\gamma<\xi<\epsilon$
	and $\gamma^\prime<\xi^\prime<\epsilon$, that
	\begin{equation}
		f(\alpha_\gamma,\alpha_\xi)=f(\alpha_{\gamma^\prime},\alpha_{\xi^\prime}).
	\end{equation}

	Note for each $\gamma<\xi<\epsilon$, by definition of $\sim$, there exists a
	least $\sigma_{\gamma,\xi}\in C$ such that
	\begin{equation}
		f(\alpha_\gamma,\sigma_{\gamma,\xi})=f(\alpha_\xi,\sigma_{\gamma,\xi}).\label{eq:add}
	\end{equation}
	For each $\xi<\epsilon$, since $\xi<\beta$, the transfinite sequence is
	bounded above in $C$ and we may thus define
	\begin{equation}
			\sigma_\xi=\sup_{\gamma<\xi}\sigma_{\gamma,\xi}.
	\end{equation}

	Let $\alpha_\epsilon\in C$ such that
	\begin{equation}
		\alpha_\epsilon=\sup\setbuild{\sigma_{\xi}}{\xi<\epsilon}.
	\end{equation}
	We are now required to show, for each $\gamma<\xi<\epsilon$, it holds
	that
	\begin{equation}
		f(\alpha_\gamma,\alpha_\epsilon)=f(\alpha_\xi,\alpha_\epsilon).
	\end{equation}

	Fix any $\gamma<\xi<\epsilon$, then it follows from (\ref{eq:add}) and the
	additivity of $f$ that
	\begin{equation}
		f(\alpha_\gamma,\alpha_\epsilon)=f(\alpha_\xi,\alpha_\epsilon).
	\end{equation}
	As intended, it follows immediately from the definition of
	$\family{\alpha_\gamma}{\gamma<\beta}$ that it is a homogeneous sequence for
	$f$.
\end{proof}

\begin{rem}
	If in the above proof we substitute for the colouring $f$ any surjection
	with the property that $\card{\range f}<\card{\cf(\delta)}$ then the proof
	remains valid.  Hence, it is not necessary for the proof that $\card{\range
	f}<\aleph_0$ when $\cf(\delta)\geq\omega_1$.
\end{rem}


\begin{cor}[Existence of homogeneous sequences]
	Suppose $\alpha$ is a linear order, without a greatest element, and $f$ is
	an additive colouring of $\alpha$. There must then exist a limit ordinal
	$\delta\geq\cf(\alpha)$ and a homogeneous sequence
	$(a_\gamma)_{\gamma<\delta}$ for $f$.
\end{cor}

\begin{proof}
	Let $(b_\gamma)_{\gamma<\delta^\prime}$ be some cofinal sequence in $\alpha$
	such that $\delta^\prime=\cf(\alpha)$.  Note that, since $\alpha$ has no
	greatest element, $\delta^\prime=\cf(\alpha)$ is a limit ordinal.

	Define $B=\setbuild{b_\gamma}{\gamma<\delta^\prime}$ and let
	$g=f\restriction{\subsets{2}{B}}$ then we may assume, without loss of
	generality, that $g$ is surjective and hence a colouring of $B$.

	Since $B$ has order type $\delta^\prime$, it follows from the additive
	Ramsey Theorem that there exists, for some $\delta\geq\cf(\alpha)$, a
	transfinite sequence $x=(a_\gamma)_{\gamma<\delta}$ which is homogeneous for
	$g$.  Since $B$ is not bounded above, $\delta$ must be a limit ordinal and,
	by definition of $g$, $x$ is a homogeneous sequence for $f$, as required.
\end{proof}


\bibliography{references}
