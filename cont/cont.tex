\chapter{A La\"uchli and Leonard style result for continuous linear orders}


\section{Continuous linear orders}

\begin{dfn}[Continuous linear order]
	A \textbf{continuous linear order} is a complete linear order $\alpha$ such that $\alpha\in\dense$.  A \textit{coloured linear order} is said to be \textbf{continuous} whenever its monochromatic reduct is a continuous linear order.
\end{dfn}

\begin{dfn}[The class $\C$]\label{dfn:C}
	Let $\C$ be the smallest class of linear orders such that:
	\begin{enumerate}
		\item	$\lambda\in\C$\label{dfn:C1}
		\item	if $\alpha,\beta\in\C$ then $\alpha+\one+\beta\in\C$\label{dfn:C2}
		\item	if $\alpha\in\C$ then $(\alpha+\one)\cdot\omega,(\one+\alpha)\cdot\dual{\omega}\in\C$\label{dfn:C3}
		\item	if $\mathcal{F}\subseteq\C$ is finite and $h\colon\reals\to\mathcal{F}$ is a surjection such that, for each $\alpha\in\mathcal{F}$, the set $\inv{h}[\alpha]$ is dense in $\lambda$ then it follows that
			\begin{equation}
				\sum_{x\in\lambda}(\one+h(x)+\one)\in\C.
			\end{equation}\label{dfn:C4}
	\end{enumerate}
\end{dfn}

\begin{dfn}[The class $\C_k$]
	For each $k\in\nats$, let $\C_k$ denote the class of all $k$-coloured expansions of linear orders belonging to $\C$.
\end{dfn}

\begin{prp}\label{prp:cont}
	If $\alpha$ is a continuous linear order then there exists an embedding $h\colon\lambda\hookrightarrow\alpha$.  Furthermore, $h$ can be chosen such that $h[\reals]$ is an open interval in $\alpha$.
\end{prp}
\begin{proof}
	Since $\alpha$ is dense there exists an embedding $h_0\colon\eta\hookrightarrow\alpha$.  We now set out to extend $h_0$ to an embedding of $\lambda$ into $\alpha$.  Define the map $h\colon\reals\to\domain{}\alpha$ such that, for each $r\in\reals$,
	\begin{equation}
		h(r)=\sup\setbuild{h_0(q)}{q\in\rats\text{ and }q<r}.
	\end{equation}
	It clearly follows by definition that $h\restriction\rats=h_0$.  Note that, since $\eta$ is dense in $\lambda$ there exists a $q_0\in\rats$ such that $x<q_0<y$.  Again invoking the density of the rationals in the reals, there exists a $q_1\in(q_0,y)\cap\rats$.  Since $h_0$ is an embedding, it follows from the definition of $h$ that $h(x)\leq h(q_0)<h(q_1)\leq h(y)$ so that, in particular, it holds that $h(x)<h(y)$ and thus $h$ is an order-preserving map and, therefore, also an embedding $h\colon\lambda\hookrightarrow\alpha$.

	All that remains is to show that there now exists an embedding $h^\prime\colon\lambda\hookrightarrow\alpha$ such that $h^\prime[\reals]$ is a convex set in $\alpha$.  Consider now the open interval $I=(0,1)\subseteq\reals$ and its image $J=h[I]$ under $h$.  We now proceed to show that $J$ is convex.  Note that $h(0)<J<h(1)$ so we may therefore conclude that $J$ is bounded in $\alpha$.  Thus, since $\alpha$ is complete, we may define $a=\inf J$ and $b=\sup J$.  Now fix any $d\in(a,b)_\alpha$ and choose $\ell=\sup\setbuild{r\in\reals}{h(r)<d}$ and $u=\inf\setbuild{r\in\reals}{h(r)>d}$.  By definition, it must hold that $\ell\leq u$.  Suppose that $u<\ell$ then there exists some $r_0\in(\ell,u)$.  Since $h$ is an embedding, it follows that $h(\ell)<h(r_0)<h(u)$.  Now, if $h(r_0)<d$ then it follows by definition of $\ell$ that $r_0\leq\ell$, which is a contradiction.  Similarly, if $h(r_0)>d$ then it follows that $r_0\geq u$, also a contradiction.  Therefore, it follows that $h(r_0)=d$ from which we may conclude that, since $r_0\in(\ell,u)\subseteq I$ it follows that $d\in J$.  Note that if $J$ had a greatest element then its preimage under $h$ would be forced to be $1\in\reals$, contradicting the fact that $1\notin I=(0,1)$.  Similarly, since $0\notin I$, $J$ cannot have a least element either.  Therefore, since it holds that $(0,1)\cong\lambda$, $h$ is an embedding and $h\restriction(0,1)$ maps onto a convex subset of $\alpha$, the result follows.
\end{proof}

\begin{thm}\label{thm:Cll}
	For each $k,n\in\nats$, if $\alpha$ is a continuous $k$-coloured linear order then there exists a $\beta_n\in\C_k$ such that:
	\begin{equation}
		\alpha\nequiv{n}\beta_n
	\end{equation}
\end{thm}
\begin{proof}
	Fix any $k,n\in\nats$ and suppose $\alpha$ is a continuous $k$-coloured linear order.  Define a binary relation $R$ on $\domain{}\alpha$ such that, for every $a,b\in\reals$, $aRb$ iff  $a<b$ and there exists a $\beta\in\C_k$ such that $(a,b)\nequiv{n}\beta$.  It follows from clause \ref{dfn:C2} in definition \ref{dfn:C} that $R$ is transitive and, therefore, induces a congruence $\sim$ on $\alpha$.

	\begin{claim}
		The congruence $\sim$ is definable.
	\end{claim}
	\begin{proof}
		The defining formula $\varphi=\varphi(x,y)$ is of the same form as the formula in (\ref{eq:condef}), with the only differences being that $\varphi$ is a formulated in a language with a finite number of additional (unary) relation symbols and $\tau=\bigvee_{\delta\in\C_k}\cha{\delta}{n}$.
	\end{proof}

	\begin{claim}
		The linear order $\faktor{\alpha}{\sim}$ is dense.
	\end{claim}
	\begin{proof}
		Fix any $I,J\in\faktor{\alpha}{\sim}$ such that $I<J$ i.e.\ $x<y$ for each $x\in I$ and $y\in J$.  Since $\alpha$ is complete, there exists some $a,b\in\alpha$ such that $a=\sup I$ and $b=\inf J$.  Note, if $a=b$ then it follows from clause \ref{dfn:C2} in definition \ref{dfn:C} that $x\sim y$, for every $x\in I$ and $y\in J$, contradicting the definition of $I$ and $J$.  Hence, we may assume that $a<b$.  It now follows from the density of $\alpha$ that there exists a $c\in (a,b)$.  Consequently, it follows that $I<[c]<J$ --- as required.
	\end{proof}

	\begin{claim}
		The linear order $\faktor{\alpha}{\sim}$ is complete.
	\end{claim}
	\begin{proof}
		Suppose $X\subseteq\faktor{\alpha}{\sim}$ is bounded above.  Note that $\bigcup X\subseteq\alpha$ is now bounded above in $\alpha$.  It therefore follows from the completeness of $\alpha$ that there exists a $u\in\bigcup X$ such that
		\begin{equation}
			u=\sup\bigcup X.
		\end{equation}
		Define $I$ to now be the equivalence class $I=[u]$.  Suppose by way of contradiction that there exists a $J\in\faktor{\alpha}{\sim}$ such that $J\geq X$ and $J<I$.  Since $\faktor{\alpha}{\sim}$ is dense there must exist a $K\in\faktor{\alpha}{\sim}$ such that $J<K<I$.  Therefore, there exists an $x_0\in K$ such that $x<x_0<u$ for each $x\in J$.  This implies that $x_0$ is an upper bound of $\bigcup X$ less than $u$, a contradiction.  Therefore, $I$ is in fact the supremum of $X$.
	\end{proof}

	\begin{claim}\label{clm:contInt}
		For every $I\in\faktor{\alpha}{\sim}$ there exists a $\beta\in\C_k$ such that $\Int I\nequiv{n}\beta$.
	\end{claim}
	\begin{proof}
		Note that, by the Downwards L\"owenheim-Skolem Theorem, we may assume $I$ is at most countable.  Consider first the case where $I$ is bounded below but not above in $\alpha$ and choose a cofinal sequence $a=\family{a_i}{i<\omega}$ in $\alpha$ such that $a_0=\inf I$.  If $h$ denotes some colouring of $I$ such that, for $x,y\in I$, $h(x,y)\in\C_k$ and $(x,y)\nequiv{n}h(x,y)$ then it follows from Ramsey's Theorem that there exists a homogeneous subsequence $a^\prime=\family{a^\prime_i}{i<\omega}$ of $a$ for $h$.  It then follows by definition of $a^\prime$ that
		\begin{equation}
			\sum_{i<\omega}(a^\prime_i,a^\prime_{i+1}]\nequiv{n}\left(h(a^\prime_0,a^\prime_1)+\one\right)\cdot\omega.
		\end{equation}
		We now choose $\beta\in\C_k$ such that
		\begin{equation}
			\beta=\begin{cases}
				\left(h(a^\prime_0,a^\prime_1)+\one\right)\cdot\omega,&\text{if }a_0=a^\prime_0,\\
				h(a_0,a^\prime_0)+\one+\left(h(a^\prime_0,a^\prime_1)+\one\right)\cdot\omega,& \text{otherwise}.\\
			\end{cases}
		\end{equation}
		It follows by definition then that $\beta\in\C_k$ and
		\begin{equation}
			\Int I\nequiv{n}\beta,
		\end{equation}
		as required.  The case where $I$ is bounded above but not below in $\alpha$ is obtained dually i.e.\ by considering a cofinal sequence $b=\family{b_i}{i<\omega}$ in $\dual{I}\subseteq\dual{\alpha}$ and repeating the above argument.  In order to establish the claim, the remaining cases are obtained by similar means: considering the relevant cofinal, or coinitial, sequences and taking the appropriate sums.
	\end{proof}

	We now wish to show that $\card{\faktor{\alpha}{\sim}}=1$ so it follows from the previous claim that $\alpha\nequiv{n}\beta_n$, for some $\beta_n\in\C_k$.  To this end, we suppose to the contrary that $\card{\faktor{\alpha}{\sim}}>1$ and note that proposition \ref{prp:cont} implies the existence of an interval $I=(a,b)\subseteq\alpha$ and an embedding $g\colon\lambda\hookrightarrow\alpha$ such that $g[\reals]=I$.  Let $f$ be the surjective homomorphism $f\colon\alpha\to\faktor{\alpha}{\sim}$ induced by the congruence $\sim$ and define $S=\set{\delta_0,\dotsc,\delta_{m-1}}$ to be an $n$-spectrum for the class of $\chi\in\C_k$ such that $\chi\nequiv{n}X$ for some $X\in f[I]$.

	\begin{claim}
		There exists an open interval $J=(a^\prime,b^\prime)\subseteq I$ and a nonempty $S^\prime\subseteq S$ such that, for each $\sigma\in S$, the set
		\begin{equation}
			D_\delta=\setbuild{\faktor{d}{\sim}}{d\in J\text{ and }\faktor{d}{\sim}\nequiv{n}\delta}
		\end{equation}
		is dense in $f[J]$.
	\end{claim}
	\begin{proof}
		Proceed by induction on $m$ and note that, by definition of $I$, every nonempty open interval in $I$ has order type $\lambda$, implying that $I$ contains no singletons.  If $m=1$ then the result follows immediately from the definition of $S$ as then every member of $I$ is $n$-equivalent to the same (unique) linear order in $S$.  Assume the result holds whenever $m<m^\prime$, for some arbitrary $m^\prime\geq 1$.  Suppose to the contrary that $\delta\in S$ and $D_\delta$ is not dense in any open interval $(a^\prime,b^\prime)\subseteq I$.  By definition, there must then exist an open interval $J\subsetneq I$ such that no $X\in J$ satisfies $X\nequiv{n}\delta$.  If we now let $S^\prime=\setbuild{\delta^\prime\in S}{X\nequiv{n}\delta^\prime,\text{ for some }X\in J}$ then it follows by definition that $\delta\notin S^\prime$ and thus $\card{S^\prime}<\card{S}$.  Applying the induction hypothesis now to $J$ and $S^\prime$ yields the desired contradiction, thus establishing the claim.
	\end{proof}

	Note, since $I$ is bounded in $\alpha$, that each $X\in f[I]$ is also bounded in $\alpha$ and thus (by completeness of $\alpha$ and claim \ref{clm:contInt}) $X$ has a least and greatest element.  If we now choose $h$ to be the map $h\colon\reals\to S$ such that, for each $x\in\reals$, $h(x)$ is the interior of the (unique) $n$-equivalent of $fg(x)$ belonging to $S$ then it follows from lemma \ref{lem:fvsum} that
	\begin{equation}
		I\nequiv{n}\sum_{x\in\lambda}(\one+h(x)+\one)\in\C_k.\label{eq:rsum}
	\end{equation}
	Therefore, by definition of $I$, it follows that $a\sim b$ and thus $a,b\in I$, which is the desired contradiction.  Hence $\faktor{\alpha}{\sim}$ has only one element and the result follows from claim 4.
\end{proof}

It is inconvenient, however, that the class $\C$ as it stands, due to the final clause in its definition, is not countable let alone recursively enumerable.  This is due to the uncountability of $\reals$ itself as it guarantees the uncountability of at least one of it equivalence classes under any equivalence relation.  Should $A$ be any such uncountable equivalence class and $B$ any equivalence class distinct from it, uncountably many new partitions can be generated by substituting, for some $r\in A$, the set $A\setminus\set{r}$ for the equivalence class $A$ and the set $B\cup\set{r}$ for the equivalence class $B$ in the chosen partition of $\reals$.

\begin{prp}\label{prp:dsum}
	Suppose, for some $k\in\nats$, $\mathcal{S}$ is a set of $k$-coloured linear orders and fix any $\alpha\in\dense$.  If $h,h^\prime\colon\alpha\to\mathcal{S}$ are surjections such that, for every $\delta\in\mathcal{S}$, $\inv{h}[\delta]$ and $\inv{h^\prime}[\delta]$ are respectively dense in $\alpha$ then it follows, for each $n\in\nats$, that
	\begin{equation}
		\sum_{x\in\alpha}h(x)\nequiv{n}\sum_{x\in\alpha}h^\prime(x).
	\end{equation}
\end{prp}
\begin{proof}
	The case $n=0$ is immediate from the fact that the language of coloured linear orders has no constant symbols so we may assume $n>0$.  Define the linear orders $\beta_0$ and $\beta_1$ such that
	\begin{equation}
		\beta_0=\sum_{x\in\alpha}h(x)\quad\text{and}\quad\beta_1=\sum_{x\in\alpha}h^\prime(x).
	\end{equation}
	We now argue by describing the winning strategy of $\Right$ in the game $\EF_n(\beta_0,\beta_1)$.  On the first round of the game if $\Left$ plays $a_0\in\beta_0$ then $\Right$ responds with an element from a summand of $\beta_1$ which, leveraging the definition of $h^\prime$, corresponds to the summand of $\beta_0$ from which $a_0$ arose.  Should $\Left$ have played some $b_0\in\beta_1$ in stead a similar cuntermove would suffice.

	Consider now a position $(\bar{a},\bar{b})$ of length $\ell$ such that $0<\ell<n$.  Since $\alpha$ is dense, it follows from the definitions of $h$ and $h^\prime$ that (irrespective of $\Left$'s $\ell$-th move) $\Right$ always has available a response from an appropriate summand, of either $\beta_0$ or $\beta_1$, in the opposing $k$-coloured linear order.  If $\Left$ plays $d\in\beta_0$ and $d>\bar{a}$ then $\Right$ responds, from a corresponding summand, with an element $e\in\beta_1$ such that $d>\bar{b}$.  One argues dually when $d<\bar{a}$.  Otherwise, it must hold that $d\in(a_i,a_j)$, for some $i,j<\ell$, such that $a_{i^\prime}\notin(a_i,a_j)$ when $i^\prime<\ell$.  In this latter case it follows again from the definition of $h^\prime$ that $\Right$ has an ``appropriate'' response $e\in\beta_1$ from a summand corresponding to the one from which $d$ arose: meaning $e$ can be chosen such that $e\in(b_i,b_j)$ and $d_{i^\prime}\notin(d_i,d_j)$ when $i^\prime<\ell$.
\end{proof}

\begin{dfn}[Canonical partitions of $\reals$]
	Suppose $n\in\posnats$ then we refer to a partition $\mathcal{R}=\set{R_0,\dotsc,R_{n-1}}$ of $\reals$ as the \textbf{canonical partition of $\reals$ into $\bm{n}$ dense subsets} whenever:
	\begin{enumerate}
		\item	$n=1$ implies $\mathcal{R}=\set{\reals}$,
		\item	if $n>1$ then, for some $k\in\set{0,\dotsc,n-1}$, it holds that $R_k=\irrats$ and $\mathcal{R}\setminus\set{R_k}$ is the canonical partition of $\rats$ into $n-1$ dense subsets.
	\end{enumerate}
\end{dfn}

\begin{dfn}[The class $\Cast$]
	Let $\Cast$ be the smallest class of linear orders such that:
	\begin{enumerate}
		\item	$\lambda\in\Cast$\label{dfn:C1}
		\item	if $\alpha,\beta\in\Cast$ then $\alpha+\one+\beta\in\Cast$\label{dfn:C2}
		\item	if $\alpha\in\Cast$ then $(\alpha+\one)\cdot\omega,(\one+\alpha)\cdot\dual{\omega}\in\Cast$\label{dfn:C3}
		\item	if $\mathcal{F}\subseteq\Cast$ is nonempty and finite while $h\colon\reals\to\mathcal{F}$ is a surjection which induces the canonical partition of $\reals$ into $\card{\mathcal{F}}$ subsets then
			\begin{equation}
				\sum_{x\in\lambda}(\one+h(x)+\one)\in\Cast.
			\end{equation}\label{dfn:C4}
	\end{enumerate}
\end{dfn}

\begin{dfn}[The class $\Cast_k$]
	For each $k\in\nats$, let $\Cast_k$ denote the class of all $k$-coloured expansions of linear orders belonging to $\Cast$.
\end{dfn}

\begin{thm}
	For each $k,n\in\nats$, if $\alpha$ is a continuous $k$-coloured linear order then there exists a $\beta_n\in\Cast_k$ such that:
	\begin{equation}
		\alpha\nequiv{n}\beta_n
	\end{equation}
\end{thm}
\begin{proof}
	Near-identical to the proof of Theorem \ref{thm:Cll}.  The only variation is that one invokes Proposition \ref{prp:dsum} to choose the surjection $h$, occuring in (\ref{eq:rsum}), that induces the required canonical partition of $\reals$ into dense subsets.
\end{proof}


\section{Decidability: the coloured case}

\begin{dfn}[Definable completeness]
	A coloured linear order $\alpha$ is said to be \textbf{definably complete} whenever every nonempty definable subset of $\alpha$, which is bounded above, has a supremum.
\end{dfn}

Note that, for any subset $A$ of a definably complete $\alpha$, if there exists a tuple $\bar{a}$ of $\alpha$ such that $A$ is defined by the formula $\varphi(x,\bar{a})$ and $A$ is bounded bounded \textit{below} then the set of lower bounds of $A$ is defined by the formula
\begin{equation}
	\psi(x,\bar{a})=\forall y\big(\varphi(y,\bar{a})\rightarrow x<y\big).
\end{equation}

Additionally, the set defined by $\psi$ will necessarily be bounded above and thus have a supremum.  It can then be shown that the aforementioned supremum is the infimum of $A$.  Therefore, if a coloured linear order $\alpha$ is definably complete then every definable subset of $\alpha$, which is bounded below, has an infimum.  A similar argument shows that the converse also holds.

\begin{lem}\label{lem:compsum}
	Suppose $\beta$ is a coloured linear order and, for each $i\in\beta$, $\alpha_i$ is a complete coloured linear order then the following holds:
	\begin{enumerate}
		\item	If $\beta$ is complete and, for each $i\in\beta$, $\alpha_i$ has both a least and a greatest element then $\sum_{i\in\beta}\alpha_i$ is complete.
		\item	If $\beta$ is well-ordered and, for each $i\in\beta$, $\alpha_i$ has a least element then $\sum_{i\in\beta}\alpha_i$ is complete.
	\end{enumerate}
\end{lem}
\begin{proof}
	\begin{enumerate}[nosep]
		\item	Suppose $B$ is a nonempty subset of $\sum_{i\in\beta}\alpha_i$ which is bounded above.  By definition, there must then exist some $u\in\beta$ such that, for each $i\geq u$, $B$ is disjoint from the set $A_i=\setbuild{(a,i)}{a\in\alpha_i}$.  Since $\beta$ is complete we may choose $u$ to be the least element of $\beta$ with this property.  If $B\cap A_u$ is unbounded in $A_u$ then the supremum of $B$ is the greatest element of $A_u$.  Otherwise, the supremum of $B$ is $\sup A_u$.
		\item	Similar to 1 except one leverages the well-ordering property of $\beta$ to obtain the necessary $u\in\beta$.\qedhere
	\end{enumerate}
\end{proof}

\begin{prp}\label{prp:defcomp}
	If $\alpha$ is a definably complete coloured linear order without endpoints and $n\in\nats$ then there exists a complete coloured linear order $\beta_n$ such that $\alpha\nequiv{n}\beta_n$.
\end{prp}
\begin{proof}
	Let $n\in\nats$ be fixed but arbitrary and define a binary relation $r$ on $\domain{}\alpha$ such that $aRb$ iff $(a,b)$ is $n$-equivalent to a complete coloured linear order.  It follows from Lemma \ref{lem:IndCong} that $R$ induces a congruence $\sim$ on $\alpha$.

	Take note that, as in the proof of Theorem \ref{thm:Cll}, the congruence $\sim$ is definable.  Therefore, each equivalence class is also definable and we may prove the following:

	\begin{claim}
		If $I\in\faktor{\alpha}{\sim}$ then $I$ is $n$-equivalent to a complete coloured linear order.  Furthermore, if $I$ is bounded above (below) in $\alpha$ then $I$ has a greatest (least) element.
	\end{claim}
	\begin{proof}
		We first consider the case where $I$ is bounded above in $\alpha$.  Choose $u=\sup I$ and let $\family{a_\xi}{\xi<\beta}$ be a cofinal sequence in $I^{<u}$, for some ordinal $\beta\geq\omega$.

		By definition of $I$ it now follows, for each $\xi<\beta$, that the interval $[a_\xi,a_{\xi+1})$ is $n$-equivalent to some complete linear order $\delta_\xi$.  Define the linear order
		\begin{equation}
			\delta=\sum_{\xi<\beta}\delta_\xi
		\end{equation}
		so that it follows from Lemma \ref{lem:compsum} that $\delta$ is a complete coloured linear order.  Lemma \ref{lem:fvsum} then tells us that $\delta\nequiv{n}I\restriction(a_0,u)$ and thus, by definition, we have $a_0\sim u$.  Consequently, $u\in I$ and therefore $u$ is the greatest element of $I$.

		Were it the case that $I$ is unbounded in $\alpha$ then one, again, chooses $\family{a_\xi}{\xi<\beta}$ cofinal in $\alpha$ and a similar argument as before suffices in showing that $I^{>a_0}$ has a complete $n$-equivalent $\delta$.

		Similarly, one may obtain a $b_0\in I$ such that $b_0<a_0$ and $I^{<b_0}$ has a complete $n$-equivalent $\delta^\prime$.  Note, by definition, that $I\restriction (b_0,a_0)$ has a complete $n$-equivalent $\epsilon$.  Therefore, we may conclude that there exists a coloured expansion of the sum
		\begin{equation}
			\delta^\prime+\one+\epsilon+\one+\delta
		\end{equation}
		which is a complete $n$-equivalent of $I$, as required.
	\end{proof}

	\begin{claim}
		The coloured linear order $\faktor{\alpha}{\sim}$ is dense.
	\end{claim}
	\begin{proof}
		Choose any $I,J\in\faktor{\alpha}{\sim}$ such that $I<J$ and, aiming for a contradiction, suppopose there exists no $K\in\faktor{\alpha}{\sim}$ such that $I<K<J$.

		Note that, respectively, $I$ is bounded above and $J$ is bounded below in $\alpha$.  Since $I$ and $J$ are both definable, and $\alpha$ is definably complete, it now follows that there exists $a,b\in\alpha$ such that $a=\sup I$ and $b=\inf J$.

		From our assumption it now follows that $(a,b)=\emptyset$ so that $\alpha\restriction(a,b)$ is (vacuously) complete.  Therefore, since the previous claim yields $a\in I$ and $b\in J$, we may conclude that $a\sim b$ and thus $I=J$ --- the desired contradiction.
	\end{proof}

	Observe that if $\card{\faktor{\alpha}{\sim}}=1$ then the result follows immediately.  By way of contradiction, suppose now that $\card{\faktor{\alpha}{\sim}}>1$.  Let $S=\set{\alpha_0,\dotsc,\alpha_{k-1}}$ be an $n$-spectrum for the class of complete coloured linear orders. The contradiction desired is contained in the following claim:
	\begin{claim}
		There exists an $S^\prime\subseteq S$ and an open interval $D\subseteq\faktor{\alpha}{\sim}$ such that, for each $\delta\in S^\prime$, the set $\setbuild{I\in D}{I\nequiv{n}\delta}$ is dense in $D$.
	\end{claim}
	\begin{proof}
		Choose some open interval $D_0\subseteq\faktor{\alpha}{\sim}$ and define
		\begin{equation}
			S_0=\setbuild{\delta\in S}{\text{there exists an }I\in D_0\text{ such that }\delta\nequiv{n}I}.
		\end{equation}

		Recursively, whenever $0<k<\card{S_0}-1$, we obtain $S_{k+1}$ from $S_k$ by removing any order type $\delta_k\in S_k$ whose $n$-equivalents in the open interval $D_k\subseteq D_{k-1}$ are not dense in $D_k$.  Additionally, we choose $D_{k+1}\subseteq D_k$ to be some open interval, not having $\delta_k$ as a member.  Otherwise, if no such $\delta_k$ exists, we declare $S_{k+1}=S_k$ and $D_{k+1}=D_k$.

		Since $\faktor{\alpha}{\sim}$ is dense, if $m=\card{S_0}-1$ then it follows that $\card{S_m}>0$.  By definition, choosing $S^\prime=S_m$ and $D=D_m$ will establish the claim.
	\end{proof}

	We may assume that $D=(I,J)$ for some $I,J\in\faktor{\alpha}{\sim}$.  Now, let $a,b\in\bigcup D$ satisfy $a<b$.  Our aim is to show that the interval $(a,b)$ of $\alpha$ has some $n$-equivalent in $S$.

	Let $h\colon\reals\to S^\prime$ be the map that induces the canonical partition of $\reals$ into $\card{S^\prime}$ sets.  As in claim 1, note that $I^{>a}$ and $J^{<b}$ will have respective $n$-equivalents $\nu_a,\nu_b\in S$.

	Define a coloured linear order $\chi_{a,b}$ such that
	\begin{equation}
		\chi_{a,b}=\nu_a+\sum_{x\in\lambda}h(x)+\nu_b.
	\end{equation}
	As $\one+\lambda+\one$ is a complete order type, it follows from Lemma \ref{lem:fvsum} that $\chi_{a,b}$ is a complete coloured linear order.  Since the $n$-equivalents in $D$ of any member of $S^\prime$ form a dense subset of $D$, $\Right$ has a winning strategy in the $n$-game between the coloured linear orders $(a,b)$ and $\chi_{a,b}$.

	By definition, we now have $a\sim b$.  Since $a$ and $b$ were chosen arbitrarily, it follows by definition of $\faktor{\alpha}{\sim}$ that $\bigcup D$ is itself an equivalence class.  Thus $D$ has only one element, contradicting that $D$ is an open interval and thereby concluding the proof.
\end{proof}

For the purposes of the proof of the following proposition, given a fixed $k\in\nats$, let $\Sigma_k$ be the theory consisting of the following sentences which, together, express that its models are (nontrivially) dense coloured linear orders without endpoints and are definably complete:
\begin{itemize}
	\item	$\exists x(x=x)$
	\item	$\axmlin\wedge\axmden$,
	\item	$\forall x\exists y(x<y)\wedge\forall x\exists y(y<x)$,
	\item	for each formula $\varphi(x)$ the sentence given by:
		\begin{multline}
			\exists x\varphi(x)\wedge\exists y\forall x\big(\varphi(x)\rightarrow x<y\big)\rightarrow\\
			\exists z\Big(\forall x\big(\varphi (x)\rightarrow x<z\big)\wedge\forall y\big(\forall x(\varphi(x)\rightarrow x<y)\rightarrow z\leq y\big)\Big).
		\end{multline}
\end{itemize}

\begin{prp}
	For each $k\in\nats$, the theory $T$ of complete $k$-coloured linear orders is recursively enumerable.
\end{prp}
\begin{proof}
	Fix any $k\in\nats$ and note that it is enough to show that every model $\alpha$ of $\Sigma_k$ is also a model of $T$.  In line with this we suppose $\alpha\models\Sigma_k$ is fixed but arbitrary.

	Choose any $\sigma\in T$ and let $n=\qrank(\sigma)$.  It follows now from Proposition \ref{prp:defcomp} that there exists a complete coloured linear order $\beta_n\nequiv{n}\alpha$.  By definition, we must have $\beta_n\models T$ and therefore $\beta\models\sigma$.  In conclusion, since $\sigma$ has quantifier rank at most $n$ and $\beta_n\nequiv{n}\alpha$, it holds that $\alpha\models\sigma$.
\end{proof}

\begin{dfn}[Regularity]
	We call a $k$-coloured linear order $\alpha$, with colours say $r_0,\dotsc,r_{k-1}$, \textbf{regular} whenever \textit{a subset of} its colours form a partition of $\alpha$ and no member of $\alpha$ posesses two distinct colours.  Equivalently, $\alpha$ is regular whenever it holds that:
	\begin{equation}
		\alpha\models\left(\forall x\bigvee_{0\leq i<n}r_i(x)\right)\wedge\left(\neg\exists x\bigvee_{i<j<k}(r_i(x)\wedge r_j(x))\right).
	\end{equation}
\end{dfn}

\begin{dfn}[Normality]
	We call a $k$-coloured linear order $\alpha$, with colours say $r_0,\dotsc,r_{k-1}$, \textbf{normal} whenever \textit{all its colours} (together) form a partition of $\alpha$.  Equivalently, $\alpha$ is normal whenever it is regular and satisfies:
	\begin{equation}
		\alpha\models\bigwedge_{0\leq i<k}\exists xr_i(x).
	\end{equation}
\end{dfn}

\begin{dfn}[$k$-Pattern]
	A $k$-coloured linear order $\alpha$ is called a $\bm{k}$\textbf{-pattern} whenever $\alpha$ is both regular and finite.  Should $k\in\nats$ be clear from the context we will refer to $\alpha$ as simply being a \textit{pattern}.
\end{dfn}

\begin{prp}
	If $n\in\nats$ and $\alpha$ is a normal $k$-coloured linear order of order type $\omega$, with colours $r_0,\dotsc,r_{k-1}$, such that
	\begin{equation}
		i<k\quad\implies\quad\alpha\models\forall x\exists y\big(x<y\wedge r_i(y)\big),
	\end{equation}
	then there exists a $k$-coloured linear order $\beta_n$ such that $\alpha\nequiv{n}\beta_n$ and, for each $i<n$, $r_i^{\beta_n}$ is a recursively enumerable set of natural numbers.
\end{prp}
\begin{proof}
	The case $n=0$ is trivial so fix any natural $n>0$.  Choose any $\beta_n$ such that the initial segment of $\beta_n$ of length $2^n$ is coloured identically to the corresponding initial segment of $\alpha$, each of the colours $r_0^{\beta_n},\dotsc,r_{k-1}^{\beta_n}$ are recursively enumerable subsets of $\beta_n$ and there exists arbitrary large elements in $\beta_n$ of any given colour.

	We proceed by means of an $n$-game between $\alpha$ and $\beta_n$.  The winning strategy for the player $\Right$ is described as follows.  Should $\Left$ choose some $a_0\in\alpha$ such that $r_i^\alpha(a_0)$, for some $i<k$, and there are at most $2^n-1$ elements (strictly) below $a_0$ in $\alpha$ then $\Right$ responds with the corresponding element (per the definition of $\beta_n$) in $\beta_n$.

	On the other hand, if the elements (strictly) below $a_0$ in $\alpha$ number at least $2^n$ then $\Right$ chooses the least $b_0\in\beta$ such that $r^{\beta_n}_i(b_0)$ and every pattern, of length at most $n-1$, embeddable in $\alpha^{<a_0}$ can also be embedded in $\beta_n^{<b_0}$.  A similar approach is followed if, in stead, $\Left$ has some $b_0\in\beta_n$ as his first move.

	Suppose now that $0<m\leq n$ and it is round $m$, of the $n$-round game, and $\Left$ has just played some $a_{k-1}\in\alpha$.  Note that there is no harm in assuming $\Left$ never repeats a move and, for each $i<m-1$, that $a_i<a_{i+1}$.

	Additionally, we need only consider the case where the following holds: (1) the first $m-1$ rounds of the current game constitute a winning position for $\Right$ in the game $\EF_{m-1}(\alpha,\beta_n)$; (2) every pattern, of length at most $n-(m-1)$, which embeds in $\alpha^{<a_0}$ also embeds in $\beta_n^{<b_0}$ and (3) for each $i<m-1$ and every pattern, of length at most $n-(m-1)$, which can be embedded in $(a_i,a_{i+1})$ also embeds in $(b_i,b_{i+1})$.  Otherwise, $\Right$ would be free to pick any element from $\beta_n$ as his move.

	It now follows by assumption, for each $i<m-1$ and every $j<k$, that $b_i<b_{i+1}$ and
	\begin{equation}
		\alpha\models r_j(a_i)\quad\iff\quad\beta_n\models r_j(b_i).
	\end{equation}

	Should the length of $\alpha^{<a_{k-1}}$ be at most $2^{n-(m-1)}-1$ then $\Right$ simply responds with the corresponding element of $\beta_n$, according to the definition of $\beta_n$ by noting that $2^{n-(m-1)}\leq 2^n-1$.

	Now suppose the length of $\alpha^{<a_{k-1}}$ is at least $2^{n-(m-1)}$. In the case where $a_{k-1}>a_{k-2}$, $\Right$ responds with the least $b_{k-1}\in\beta_n$ such that every pattern, of length at most $2^{n-(m-1)}-1$, that embeds in $(a_{k-2},a_{k-1})$ also embeds in $(b_{k-2},b_{k-1})$.

	The only remaining case is if there exists some natural $\ell<m-1$ such that $a_k\in(a_\ell,a_{\ell+1})$.  The proper response is then for $\Right$ to play the least $b_k\in\beta_n$ such that every pattern embeddable in $(a_\ell,a_{k-1})$ also embeds in $(b_\ell,b_k)$.
\end{proof}

\begin{prp}
	Suppose $\alpha$ is a normal $k$-coloured linear order of order type $\omega$ such that each of the colours, say $r_0,\dotsc,r_{k-1}$, is cofinal in $\alpha$ i.e.\ for each $i<k$:
	\begin{equation}
		\alpha\models\forall x\exists y\big(x<y\wedge r_i(y)\big).
	\end{equation}
	It then necessarily follows that $\Th(\alpha)$ is a recursively enumerable theory.
\end{prp}

\begin{lem}
	For each $k\in\nats$, the set
	\begin{equation}
		R_k=\setbuild{(\alpha,\sigma)}{\alpha\in\Cast_k\text{ is normal and }\alpha\models\sigma}
	\end{equation}
	is recursively enumerable.
\end{lem}
\begin{proof}
	We proceed by induction on $k\in\nats$.  Since the order types in $\Cast$ are all dense and without endpoints, the result is trivial for the case $k=0$: each member of $\Cast$ is elementarily equivalent to order type of the rationals $\eta$ and $\Th(\eta)$ is recursively enumerable as it is both finitely axiomatisable and complete.

	Assume now that $R_k$ is recursively enumerable for some $k\in\nats$.  In what follows, we choose $\family{\sigma_i}{i<\omega}$ to be a (mechanically generated) enumeration of all instances of the first-order scheme which expresses the completeness property:
	\begin{equation}
		\forall\bar{z}\big(\exists x\varphi(x,\bar{z})\wedge\exists y\forall x(\varphi(x,\bar{z})\rightarrow x\leq y)\big).
	\end{equation}
	Additionally, if $\chi_\varphi$ denotes the sentence expressing that ``the binary relation defined by $\varphi(x,y,\bar{z})$ is a congruence``: choose $\family{\tau_i}{i\in\nats}$ be a (mechanically generated) enumeration of all instances of the first-order scheme expressing quasi-seperability:
	\begin{equation}
		\forall\bar{z}\big(\chi_\varphi\rightarrow\forall x_0\forall x_1\exists y_0(x_0<y_0<x_1\wedge\forall y_1(\varphi(y_0,y_1,\bar{z})\rightarrow y_0=y_1))\big)
	\end{equation}

	For each $\alpha\in\Cast$, we proceed to define theories $S_\alpha\subseteq T_\alpha$ such that $S_\alpha$ is finite and $T_\alpha$ is a complete recursively enumerable theory.

	Define $S_\lambda$ to consist of only the sentence which expresess that $\lambda$ is a dense linear order without endpoints:
	\begin{equation}
		\axmlin\wedge\axmden\wedge\forall x\exists y\exists z(y<x<z).
	\end{equation}
	Now choose $T_\lambda=S_\lambda$
\end{proof}

\begin{dfn}[$k$-Complete theory]
	A first-order theory $T$, for some $k\in\nats$, is called $\bm{k}$\textbf{-complete} whenever it holds, for every sentence $\sigma$ of quantifier rank at most $k$, that $T\models\sigma$ or $T\models\neg\sigma$.
\end{dfn}

\begin{dfn}[$k$-Decidability]
	A first-order theory $T$, for some $k\in\nats$, is called $\bm{k}$\textbf{-decidable} whenever it holds, for every sentence $\sigma$ of quantifier rank at most $k$, that there exists a machine that can decide the statement ``$T\models\sigma$''.
\end{dfn}
