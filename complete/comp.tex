\bibstyle{amsalpha}

\chapter{Complete linear orders and the reals}

\section{Continuous linear orders and separability}

\begin{dfn}[Completeness]
	A linear order $\alpha$ is said to be \textbf{complete} if every (nonempty)
	subset of $\domain\alpha$, bounded above in $\alpha$, has a least upper
	bound.
\end{dfn}

Equivalently, $\alpha$ is complete whenever all of its nonempty subsets, bounded
below, have a greatest lower bound.

\begin{dfn}[Continuous linear order]
	If $\alpha\in\dense$ then we call $\alpha$ \textbf{continuous} precisely when it is complete.
\end{dfn}

\begin{dfn}[Separable]
	A linear order $\alpha$ is called \textbf{separable} whenever
	$\alpha\notin\set{\zero,\one}$ and there exists a countable subset of
	$\domain{}\alpha$ which is dense in $\alpha$.
\end{dfn}

A notable property of all separable linear orders is that they are necessarily
also dense.  Suppose, for instance, $\alpha$ is a separable linear order then
choose $D\subseteq\alpha$ to be countable and dense in $\alpha$.  If
$a<a^{\prime}$ are arbitrary elements of $\alpha$ then, by definition, there
must exist some $d\in D$ such that $a<d<a^{\prime}$.  Hence, the separable
linear orders form a subclass of the class $\Dense$ of all (nontrivially) dense
linear orders.

\begin{prp}[Separability]\label{prp:sep}
	If $\alpha$ is a separable linear order then there exists an embedding
	$h\colon\alpha\to\lambda$.
\end{prp}
\begin{proof}
	Suppose $\alpha$ is separable.  By definition, there exists a countable
	$\beta\subseteq\alpha$ which is dense in $\alpha$.  From Cantor's Theorem, there
	exists an isomorphism $h_{0}\colon\beta\to\eta$. Define a map
	$h\colon\alpha\to\lambda$ such that, for $a\in\alpha$,
	\begin{equation}
		h(a)=\sup\setbuild{h_{0}(b)}{b\in\beta\text{ and }b<a}.
	\end{equation}
	It is now sufficient to prove that $h$ is an embedding.

	We first establish that $h$ is strictly increasing on $\alpha$, that is to
	say:
	\begin{equation}
		a<a^{\prime}\implies h(a)<h(a^{\prime}),
	\end{equation}
	for all $a,a^{\prime}\in\alpha$.  Fix any $a<a^{\prime}$ in $\alpha$. Since
	$\beta$ is dense in $\alpha$, there must exist a $b\in\beta$ such that
	\begin{equation}
		a<b<a^{\prime}.
	\end{equation}
	Consequently, by definition of $h$, we must have
	\begin{equation}
		h(a)<h(a^{\prime}).
	\end{equation}
	It follows that $h$ is a (strictly) increasing map and, thus, $h$ is
	an embedding of $\alpha$ into $\lambda$.
\end{proof}

\begin{thm}[Characterising the reals]\label{thm:rchar}
	Every separable continuous linear order is isomorphic to $\lambda$.
\end{thm}
\begin{proof}
	Suppose $\alpha$ is a separable continuous linear order.  From Proposition
	\ref{prp:sep}, there exists an embedding $h\colon\alpha\to\lambda$.  Since
	$\alpha$ is separable there exists some denumerable $\beta\subseteq\alpha$
	which is dense in $\alpha$.

	Since $h$ is an embedding and $\beta$ denumerable, dense and without
	endpoints, it follows from Cantor's theorem that we must have
	\begin{equation}
		\beta\cong\eta\cong h[\beta].
	\end{equation}
	Now define $\alpha^\prime=\comp(h[\beta])$ so that we must have
	$\alpha\cong\comp(\beta)$, since $\alpha$ is complete and $\beta$ is dense
	in $\alpha$, hence $\alpha\cong\alpha^{\prime}$.  Moreover,
	$\alpha^{\prime}\cong\comp(\eta)=\lambda$ so $\alpha\cong\lambda$.
\end{proof}


\section{The Suslin property and the first order theory of the coloured reals}

Note that in this section we will refer to \textit{coloured linear orders}
(linear orders with finitely many unary relations defined on them) simply as
linear orders.  This simplifies the language used but it should be noted that
all the following results also apply in the monochromatic case i.e.\ in the
absence of any colours.

\begin{dfn}[Quasi-separable]\label{dfn:quasiseparable}
	A linear order $\alpha$ is called \textbf{quasi-separable} iff every
	densely ordered condensation $\beta\not\cong\one$ of $\alpha$ has a subset
	which is dense in $\beta$ and consists of only singleton subsets of $\alpha$.
\end{dfn}

\begin{dfn}[Definably quasi-separable]
	A linear order $\alpha$ is called \textbf{definably quasi-separable} iff
	each densely ordered condensation $\beta\not\cong\one$ induced by a definable
	congruence of $\alpha$ contains a set of singletons that is dense in $\alpha$.
\end{dfn}

\begin{dfn}[Suslin property]
	A linear order $\alpha$ is said to possess the \textbf{Suslin property} iff
	every set of \textit{pairwise disjoint open intervals} in $\alpha$ is at most
	countable.
\end{dfn}

Clearly $\lambda$ has the Suslin property, since any open interval in $\lambda$
necessarily contains a rational number.  A question that immediately arises is
whether or not Suslinity could be a stand-in for separability in characterising
the real number line.  Unfortunately, this problem cannot be decided within
ordinary ZFC and additional set-theoretic assumptions are required in order to
give a definite yes or no answer.  The reader may refer to any of the standard
texts on set theory, should they decide to embark on this quest.

The following lemma illustrates that quasi-separable linear orders are in
abundance and that $\lambda$ is among them:

\begin{prp}
	If $\alpha$ is a complete linear order with the Suslin property then
	$\alpha$ must be quasi-separable.
\end{prp}
\begin{proof}
	Let $\alpha$ be any continuous linear order that has the Suslin property and
	choose $\beta\not\cong\one$ to be any dense condensation of $\alpha$, say
	$\beta=\faktor{\alpha}{\sim}$.  By way of contradiction, assume there exists no
	$D\subseteq\beta$ which is dense in $\beta$ and consists of only singleton
	subsets of $\alpha$.

	Given this setting, there must exist elements $a<b$ in $\alpha$ such that
	$\faktor{a}{\sim}<\faktor{b}{\sim}$ and no singleton subset of $\alpha$ is a
	member of the open interval
	\begin{equation}
		I_{a,b}=\left(\faktor{a}{\sim},\faktor{b}{\sim}\right).
	\end{equation}

	Since $I_{a,b}$ has no singleton members, each of its elements necessarily
	contains an open interval as a subset.  Therefore, by Suslinity, it follows
	that $I_{a,b}$ must be countable.  Since $\beta$ (and thus $I_{a,b}$) is dense,
	it follows by Cantor's Theorem that $I_{a,b}$ has order type $\eta$.

	Since $\eta$ is not complete, we may choose $G\subseteq I_{a,b}$ to be
	bounded above in $I_{{a,b}}$ but without a supremum in $I_{a,b}$.  We
	now argue that $\bigcup G$ has no supremum, thereby contradicting the
	completeness of $\alpha$.

	Suppose to the contrary that $\bigcup G$ has a supremum
	$u\in \bigcup I_{a,b}$.  By definition, it follows that $\faktor{u}{\sim}$ is an
	upper bound of $G$ in $I_{a,b}$.  Now choose any $u^{\prime}\in\bigcup I_{a,b}$
	such that $\faktor{u^{\prime}}{\sim}$ is an upper bound of $G$.  Consequently,
	by definition, $u^{\prime}$ is an upper bound of $\bigcup G$ and thus
	$u^{\prime}\geq u$.  By definition, we may conclude that
	\begin{equation}
		\faktor{u^{\prime}}{\sim}\geq\faktor{u}{\sim}
	\end{equation}
	so it follows that $\faktor{u}{\sim}$ is the supremum of $G$.  This
	contradicts the definition of $G$ and the result now follows.
\end{proof}

\begin{prp}\label{prp:qdense}
	For each natural $n\geq 2$, a partition $Q_0,\dotsc,Q_{n-1}$ of $\rats$
	exists such that, for each $i=0,\dotsc,n-2$, $Q_i$ is dense in $\eta$.
\end{prp}
\begin{proof}
	Let $p_{0},\dotsc,p_{n-1}$ denote the first $n-1$ prime numbers.  Now, for each
	$i<n-1$, define the set
	\begin{equation}
		Q_{i}=\setbuild{\frac{m}{p_{i}^{k}}}{k,m\in\ints\setminus\set{0}\text{
				and  }\gcd(m,p_{i})=1},
	\end{equation}
	and let $Q_{n-1}=\rats\setminus\bigcup_{i<n-1}Q_{i}$.

	We now claim that each such $Q_{i}$, for $i<n$, is dense in $\eta$.  Fix any
	distinct $q,q^{\prime}\in\eta$ and first consider the case $i<n-1$.  Note
	that there must exist a least $k\in\posnats$ such that
	\begin{equation}
		\frac{1}{p_{i}^{k}}<\abs{q^{\prime} - q}.
	\end{equation}

	Since the distance between two elements of $\eta$ is preserved under the map
	$x\mapsto -x$, we may assume without loss of generality that
	$0\leq q<q^{\prime}$.  There now exists a least $m\in\nats$ such that
	\begin{equation}
		q<\frac{m}{p_{i}^{k}}.
	\end{equation}
	By definition of $m$, it follows that
	\begin{equation}
		\frac{m-1}{p_{i}^{k}}\leq q.
	\end{equation}
	Therefore, by definition of $k$, we may conclude that
	\begin{equation}
		q<\frac{m}{p_{i}^{k}}<q^{\prime}.
	\end{equation}
	Reducing the fraction above to its simplest form then yields the desired
	result.

	Let $p$ be the least prime number such that $p>p_{n-1}$.  Repeating the
	argument above, and substituting $p$ for $p_{i}$, will show that $Q_{n-1}$
	is also dense in $\eta$.
\end{proof}

\begin{prp}\label{prp:csums}
	If $\delta$ is a complete linear order and $\family{\alpha_i}{i\in\delta}$
	is a family of complete linear orders with endpoints then the sum
	\begin{equation}
		\alpha=\sum_{i\in\delta}\alpha_i,
	\end{equation}
	is also a complete linear order.
\end{prp}
\begin{proof}
	Suppose $B\subseteq\alpha$ is bounded above in $\alpha$ and identify each
	$\alpha_{i}$ with its corresponding image under the canonical embedding of
	$\alpha_{i}$ into $\alpha$. Choose $d\in\delta$ to be the supremum of
	all $i\in\delta$ such that $B\cap\alpha_{i}\neq\emptyset$.

	If $B\cap\alpha_{d}=\emptyset$ then let $u$ be the least element of
	$\alpha_{d}$, otherwise, choose $u$ to be the supremum of $B\cap\alpha_{d}$.
	Since $\alpha_{d}$ has a greatest element, it follows that $u\in\alpha_{d}$
	in either case.  Consequently, by definition, $u$ is an upper bound of $B$.

	All that remains is to demonstrate the minimality of $u$.  Suppose then that
	$u^{\prime}$ is any other upper bound of $B$.  If $u^{\prime}\in\alpha_{e}$,
	for some $e>d$, then we immediately have $u^{\prime}\geq u$ so assume the
	contrary.  By definition of $d$, we cannot have $u^{\prime}\in\alpha_{c}$
	for any $c<d$ so we may conclude that $u^{\prime}\in\alpha_{d}$.  It then
	follows immediately, by definition of $u$, that $u^{\prime}\geq u$, as required.
\end{proof}

\begin{prp}\label{prp:countdensumsep}
	Suppose $\gamma$ is a countable dense linear order and
	$\family{\alpha_{i}}{i\in\gamma}$ is a family of separable linear orders.
	The sum
	\begin{equation}
		\alpha=\sum_{i\in\gamma}\alpha_{i}
	\end{equation}
	must then also be a separable linear order.
\end{prp}
\begin{proof}
	Identify each $\alpha_{i}$, for $i\in\gamma$, with its image under the
	canonical embedding into $\alpha$.  By definition, for any $i\in\gamma$,
	there exists a countable $S_{i}\subseteq\alpha_{i}$ which is dense in
	$\alpha_{i}$.

	Define $S=\bigcup_{i\in\gamma}S_{i}$, then it follows by definition that
	\begin{equation}
		\card{S}= \aleph_0\cdot\aleph_0=\aleph_0.
	\end{equation}
	We are now required to show that $S$ is dense in $\alpha$ so fix any $a<b$
	in $\alpha$.  If there exists a $j\in\gamma$ such that $a,b\in\alpha_{j}$
	then there must exist a $c\in S_{j}\subseteq S$ such that $a<c<b$.

	Otherwise, there exists $j<k$ in $\gamma$ such that $a\in\alpha_{j}$ and
	$b\in\alpha_{k}$.  Since $\gamma$ is dense, there must exist an $m\in\gamma$
	such that $j<m<k$.  If we now choose any $c\in S_{m}\subseteq S$ then it
	follows that $a<c<b$, as required.
\end{proof}

\begin{dfn}[Definable completeness]
	A linear order $\alpha$ is \textbf{definably complete} whenever each of its
	definable subsets, bounded above, has a supremum.
\end{dfn}

Similar to the definition of definable quasi-separability, $\alpha$ is definably
complete in the sense just defined iff each of its definable subsets, bounded
below, has an infimum.  To this end, suppose $\bar{a}$ is a finite tuple in
$\alpha$ and $\varphi(x,\bar{a})$ defines a subset $B_{\varphi}$ of $\alpha$,
bounded below.

Define the formula
\begin{equation}
	\psi(x,\bar{a})\define\neg\exists y(\varphi(y,\bar{a})\wedge y<x).
\end{equation}
It follows immediately that $\psi=\psi(x,\bar{a})$ defines a subset $A_{\psi}$
of $\alpha$, bounded above by every element of $B_{\varphi}$.  Furthermore, we
have
\begin{equation}
	\sup A_{\psi}=\inf B_{\varphi}.
\end{equation}

The reverse direction of this implication follows similarly.  Hence, as
anticipated, the two alternate formulations of definable completeness are
equivalent.

\begin{thm}
	Suppose $\alpha\in\dense$ is countable, definably complete and definably
	quasi-separable.  For each $n\in\nats$, $\alpha$ has an $n$-equivalent of order
	type $\lambda$.
\end{thm}
\begin{proof}
	As in the statement, let $\alpha\in\dense$ be countable, definably complete
	and definably quasi-separable.  Suppose also that there are $k$ colours defined
	on $\alpha$, for some fixed $k\in\nats$.  Now choose some $n\in\nats$ and define
	a (binary) relation $R$ on $\alpha$ such that, for every $a,b\in\alpha$:  $aRb$
	iff $a\leq b$ and $(a,b)\nequiv{n}\lambda$ whenever $a\neq b$.  Since sums
	preserve $n$-equivalence, we may conclude that $R$ is transitive and therefore
	induces a congruence $\sim$ on $\alpha$.  Note that it follows from the
	definition of $\sim$ that each $\beta\in\faktor{\alpha}{\sim}$ satisfies exactly
	one of: $\beta\nequiv{n}\one$, $\beta\nequiv{n}\one+\lambda$,
	$\beta\nequiv{n}\lambda+\one$, $\beta\nequiv{n}\one+\lambda+\one$ or
	$\beta\nequiv{n}\lambda$.

	\begin{claim}
		The congruence $\sim$ is a definable (binary) relation and the
		condensation $\faktor{\alpha}{\sim}$ is dense.
	\end{claim}
	Let $\tau$ be the (finite) disjunction of all characteristic sentences, of
	quantifier-rank at most $n$, of $k$-colourings of $\lambda$.  Define
	$\varphi(x,y)$ to be the formula given by
	\begin{equation}\label{eq:condef}
		x=y\vee(x<y\wedge\tau^{(x,y)})\vee(y<x\wedge\tau^{(y,x)}).
	\end{equation}
	By definition of $\varphi$, as well as that of aforementioned characteristic
	sentences, it follows that $\varphi$ is a defining formula for the
	congruence relation $\sim$.

	It now remains to be shown that $\faktor{\alpha}{\sim}$ is dense.  We
	suppose, to the contrary, that there exists $I,J\in\faktor{\alpha}{\sim}$ such
	that the interval $(I,J)$ is empty.  Therefore, there cannot exist a
	$c\in\alpha$ such that $I<c<J$.

	If we now let $a=\sup I$ and $b=\inf J$ then it must hold that
	$a,b\in I\cup J$ and $a\leq b$.  Since $\alpha$ is dense we cannot
	simultaneously have $a\in I$ and $b\in J$.  Therefore, either $a,b\in I$ or
	$a,b\in J$.

	As the cases are similar, we may suppose that $a,b\in I$.  Note that, since
	$a$ is the supremum of $I$, there cannot exist a $c\in I$ such that $a<c<b$.
	Hence, $(a,b)$ must be empty and thus $a=b$.  Also, since $b\notin J$, it
	follows that $J$ cannot have order type $\one$.  From the countability of
	$\alpha$, it now follows that there exists a strictly decreasing sequence
	$(b_{i})_{i<\omega}\subseteq J$ such that
	\begin{equation}
		\inf_{i<\omega}b_{i}=b.
	\end{equation}

	For each $i<j<\omega$, by definition of $\sim$ the order type of the
	interval $[b_{j},b_{i})$ is $n$-equivalent to $\one+\lambda$.  From Lemma
	\ref{lem:fvsum}, it now follows that the order type of $(b,b_{0})$ is
	$n$-equivalent to $(\one+\lambda)\cdot\dual{\omega}\cong\lambda$.  Thus, by
	definition, it holds that $b\sim b_{0}$.  Since this contradicts that
	$b\notin J$, we have successfully established the above claim.  Furthermore,
	a similar argument may be used to show that each $I\in\faktor{\alpha}{\sim}$
	must contain their respective suprema and infima whenever they exist.

	Note that if $\card{\faktor{\alpha}{\sim}}=1$ then, since $\alpha$ is
	countable, it follows from Lemma \ref{lem:fvsum} that $\alpha$ is $n$-equivalent
	to a $k$-colouring of order type
	\begin{equation}
		(\one+\lambda)\cdot\dual{\omega}+\one+(\lambda+\one)\cdot\omega\cong\lambda,
	\end{equation}
	as required.  Suppose then, by way of contradiction, that
	$\card{\faktor{\alpha}{\sim}}>1$.
	\begin{claim}
		There exists a (proper) open interval $I$ of $\faktor{\alpha}{\sim}$ and
		a finite set $\Sigma$ of coloured linear orders, each of which either has order
		type $\one$ or $\one+\lambda+\one$, such that the following holds:
		\begin{enumerate}[nosep]
			\item for each $\beta\in I$ there exists a $\sigma_\beta\in\Sigma$
			      such that $\beta\nequiv{n}\sigma_\beta$,

			\item if $\sigma\in\Sigma$ then the set $\setbuild{\beta\in I}{\beta\nequiv{n}\sigma}$ is dense in $I$.
		\end{enumerate}
	\end{claim}

	Define $C$ to be the set (up to logical equivalence) of $n$-characteristic
	sentences $\cha{\beta}{n}$ for $\beta\in\faktor{\alpha}{\sim}$. It follows that
	$C$ is a nonempty finite set of sentences. Fix some $n$-spectrum $\Sigma_{C}$
	for the class $\faktor{\alpha}{\sim}$.  As noted previously, since equivalence
	classes under $\sim$ must contain any existing suprema and infima, each
	$\beta\in\Sigma_C$ is of order type either $\one$ or $\one+\lambda+\one$.

	We then proceed by means of induction on $\card{C}$. If $\card{C}=1$ then
	every $\beta\in\faktor{\alpha}{\sim}$ has order type $\one$, since $\alpha$ is
	definably quasi-separable and thus $\faktor{\alpha}{\sim}$ must contain a dense
	set of singletons.  If we now choose any fixed
	$\beta_0\in\faktor{\alpha}{\sim}$, as well as any (proper) open subinterval $I$
	of $\faktor{\alpha}{\sim}$ such that $\beta_0\in I$, then $\Sigma=\set{\beta_0}$
	and $I$ satisfy properties 1 and 2 above.

	Assume that the claim holds whenever $\card{C}<m$, for some fixed
	$m\in\posnats$.  Suppose the claim fails when $\card{C}=m$ then, by definition,
	there exists a $\tau_0\in C$ and an open interval
	$I^\prime\subsetneq\faktor{\alpha}{\sim}$ such that
	$\beta\not\models\tau_0$, for any $\beta\in I^\prime$.  Since
	$\card{C\setminus\set{\tau_0}}<m$, it follows from the inductive hypothesis that
	there exists an open interval $I\subsetneq I^\prime$ and a
	$\Sigma\subseteq\Sigma_C$ satisfying properties 1 and 2.  This then
	establishes the desired claim.

	\smallskip	The goal is now to show that $\card{I}=1$, contradicting that
	$I$ is an open interval.  Once established, this contradiction will imply
	that the assumption $\card{\faktor{\alpha}{\sim}}>1$ was erroneous and,
	therefore, that $\faktor{\alpha}{\sim}$ is a singleton partition of
	$\alpha$.

	Since $\alpha$ is definably quasi-separable, there exists an
	$S\subseteq\faktor{\alpha}{\sim}$ dense in $\faktor{\alpha}{\sim}$ such that $S$
	is a set of singletons.  Therefore, $S\cap I$ is nonempty so we may choose some
	$\beta_{1}\in S\cap I$. Note, by definition of $I$, that the
	set of $B\in I$ satisfying $B\nequiv{n}\beta_{1}$ is necessarily dense in $I$.

	We now proceed to construct a coloured linear order $\delta$ of order type
	$\lambda$ which is $n$-equivalent to $\bigcup I$.  Let
	$\Sigma=\set{\sigma_0,\dotsc,\sigma_{\ell-1}}$ be an $n$-spectrum for
	$I$.  Since the labelling is irrelevant, we may assume
	without loss of generality that $\sigma_{0}\nequiv{n}\beta_{1}$.

	From Proposition \ref{prp:ratpart}, we may choose a surjection
	$h\colon\reals\to\Sigma$ such that $h(x)=\sigma_0\nequiv{n}\beta_1$, for each
	$x\in\irrats$, and $h^{-1}[\sigma]$ is dense in $\lambda$ for any
	$\sigma\in\Sigma\setminus\set{\beta_1}$. We now define the (coloured) linear
	order
	\begin{equation}
		\delta=\sum_{x\in\lambda}h(x).
	\end{equation}
	Since each $\sigma\in\Sigma$ has order type $\one$ or $\one+\lambda+\one$,
	it follows from Proposition \ref{prp:csums} that $\delta$ is a complete
	linear order.  Whenever $x\in\irrats$, it follows (by definition) that
	$h(x)$ has order type $\one$.  Hence, the sum
	\begin{equation}
		\gamma=\sum_{x\in\eta}h(x)
	\end{equation}
	embeds densely into $\delta$.

	Since each summand of $\gamma$ is separable and dense, using Proposition
	\ref{prp:countdensumsep}, it follows that $\delta$ is both separable and
	complete.  Consequently, from Theorem \ref{thm:rchar}, we may conclude that
	$\delta$ has order type $\lambda$.

	A winning strategy for $\Right$ in the $n$-round game
	$\EF_{n}(\delta,\bigcup I)$ may now be devised by leveraging the fact that each
	respective member of the $n$-spectrum $\Sigma$ is dense in $I$.  Hence, it
	may be concluded that $\delta\nequiv{n}\bigcup I$. Therefore, since the
	equivalence classes of $\sim$ must be convex, we may conclude that
	$\card{I}=1$.  This is the desired contradiction.
\end{proof}
