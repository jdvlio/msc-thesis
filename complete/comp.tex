\bibstyle{amsalpha}

\chapter{Complete linear orders and the reals}

	\section{Characteristics of complete linear orders}

	\begin{dfn}[Separable]
		A linear order $\alpha$ is \textbf{seperable} whenever there exists a coutable subset of $\domain{}\alpha$ which is dense in $\alpha$.
	\end{dfn}

	\begin{dfn}[Completeness]
		A linear order $\alpha$ is \textbf{complete} if every (nonempty) subset of $\domain\alpha$ which is bounded above in $\alpha$, has a least upper bound (supremum).
	\end{dfn}

	\begin{dfn}[Continuous linear order]
		If $\alpha\in\dense$ then we call $\alpha$ \textbf{continuous} precisely when it is complete.
	\end{dfn}

	\begin{prp}[Seperability]\label{prp:sep}
		A continuous linear order $\alpha$ is seperable iff there exists an embedding $h\colon\alpha\to\lambda$
	\end{prp}

	\begin{thm}[Characterising the reals]
		uppose $\alpha$ is a continuous linear order.  If $\alpha$ is also seperable then, ncessarily, we must have $\alpha\cong\lambda$.  Therefore a linear order is continuous and seperable iff it is isomorphic to the reals.
	\end{thm}
	\begin{proof}
		Suppose $\alpha$ is a seperable continuous linear order.  From proposition (\ref{prp:sep}) there exists an embedding $h\colon\alpha\to\lambda$.  Since $\alpha$ is seperable there exists some $\beta\subseteq\alpha$ which is dense in $\alpha$.  Note that we must have $\beta\cong\eta\cong h[\beta]$, since $\beta$ is countable and dense.  Now define $\alpha^\prime=\comp(h[\beta])$ so that we must have $\alpha\cong\alpha^\prime$, since $\alpha$ is complete and $\beta$ is dense in $\alpha$.

		Lastly, we will claim that $\alpha^\prime$ is convex.  Suppose to the contrary there exists $a,b\in\alpha^\prime$, with $a<b$ and a $c\in\reals$ such that $a<c<b$ but $c\notin\alpha^\prime$.  Define $c_0=\sup\setbuild{x\in\alpha^\prime}{x<c}$. Since, by definition, we have $c_0\leq c$.  If $c_0=c$ then the completeness of $\alpha^\prime$ guarantees $c=c_0\in\alpha^\prime$.  Assume now instead that $c_0<c$ then, since $h[\beta]$ must be dense in $\alpha^\prime$, there exists some $d\in\beta$ such that $c_0<d<c$.  However, since $h(d)\in\beta^\prime\subseteq\alpha^\prime$, this contradicts the definition of $c_0$ which requires that $h(d)\leq c_0$.  This then concludes the proof.
	\end{proof}

	\begin{thm}\label{thm:rchar}
		Every seperable continuous linear order is isomorphic to $\lambda$.
	\end{thm}


	\section{The  suslin property and the first order theory of the coloured reals}

	Note that in this section we will refer to \textit{coloured linear orders} (linear orders with finitely many relations defined on them) merely as linear orders.  Though all of it applies to "ordinary" monochromatic linear orders as well.  This is done to simplify the wording of some of the results.  The reader should, in particular,,take note how this setting differs from the monochromatic case.  For monochromatic linear orders some the results that follow, in fact, become either trivial or otherwise uninteresting.

	\begin{dfn}[Quasi-separable]
		A continuous linear order $\alpha$ is said to be $\textbf{quasi-separable}$ iff every condensation $\beta\in\dense$ of $\alpha$ contains, as a subset, a dense set of singletons of elements in $\alpha$.
	\end{dfn}

	\begin{dfn}[Suslin property]
		A continuous linear order $\alpha$ is said to posess the \textbf{Suslin property} iff every set of \textit{disjoint open intervals} in $\alpha$ is at most countable.
	\end{dfn}

	Clearly $\lambda$ has the suslin property as any set of open intervals of reals must contain a rational number, due to $\eta$ being dense in $\lambda$.  The more interesting question, posed by Suslin, was whether this property characterises the reals.

	The following lemma illustrates that quasi-separable linear orders are in abundance and among these is order type of the reals.

	\begin{prp}
		If $\alpha$ is a continuous linear order with the suslin property then $\alpha$ must be quasi-separable.
	\end{prp}

	\begin{dfn}[Definably quasi-separable]
		A linear oder $\alpha\in\dense$ is called \textbf{definably quasi-separable} iff each condensation induced by a definable congruence onf $\alpha$ has a dense set of singletons (of elements in $\alpha$).
	\end{dfn}
	In the following definition we call a linear order \textit{definably complete} whenever each of its definable subsets bounded above have a supremum.  Clearly, all complete linear orders are definably complete.

	\begin{prp}\label{prp:qdense}
		For every $n\in\posnats$, there exists a partition $Q_0,\dotsc,Q_{n-1}$ of $\rats$ such that, for each $i=0,\dotsc,n-1$, $Q_i$ is dense in $\eta$.
	\end{prp}

	\begin{prp}\label{prp:csums}
		If $\delta$ is a complete linear order and $\family{\alpha_i}{i\in\delta}$ is a family of complete linear orders then the sum
		\begin{equation}
			\sum_{i\in\delta}\alpha_i,
		\end{equation}
		is also a complete linear order.
	\end{prp}

	\begin{thm}
		Suppose $\alpha\in\dense$ is definably complete and definably quasi-separable. If $\card{\alpha}=\aleph_0$ then, for each $n\in\nats$, $\alpha$ has an $n$-equivalent of order type $\lambda$.
	\end{thm}
	\begin{proof}

		As in the statement, let $\alpha$ be countable, definably complete and definably quasi-separable.  Suppose also that there are $k$ colours defined on $\alpha$, for some fixed $k\in\nats$.  Now choose some $n\in\nats$ and define a (binary) relation $R$ on $\alpha$ such that, for every $a,b\in\alpha$:  $aRb$ iff $a\leq b$ and $(a,b)\nequiv{n}\lambda$ whenever $a\neq b$.  Since sums preserve $n$-equivalence, we may conclude that $R$ is transitive and therefore induces a congruence $\sim$ on $\alpha$.  Note that it follows from definition of $\sim$ that each $\beta\in\faktor{\alpha}{\sim}$ satisfies exactly one of: $\beta\nequiv{n}\one$, $\beta\nequiv{n}\one+\lambda$, $\beta\nequiv{n}\lambda+\one$, $\beta\nequiv{n}1+\lambda+1$ or $\beta\nequiv{n}\lambda$.

		\begin{claim}
			The congruence $\sim$ is a definable (binary) relation and the condensation $\faktor{\alpha}{\sim}$ is dense.
		\end{claim}
		If $\tau$ is the (finite) disjunction of characteristic sentences of $k$-colourings of $\lambda$, let $\varphi(x,y)$ be the formula given by
		\begin{equation}\label{eq:condef}
			x=y\vee(x<y\wedge\tau^{(x,y)})\vee(y<x\wedge\tau^{(y,x)}).
		\end{equation}
		Note, by definition of $\varphi$ as well as that of characteristic sentences, that $\varphi$ is a defining formula for the binary relation $\sim$.  It remains to be shown that $\faktor{\alpha}{\sim}$ is dense.  We suppose to the contrary that there exists $I,J\in\faktor{\alpha}{\sim}$ such that the interval $(I,J)$ is empty.  Therefore, there cannot exists an $c\in\alpha$ such that $I<c<J$ and thus there must exist some $a\in I$ and some $b\in J$ such that $(a,b)=\emptyset$, contradicting the density of $\alpha$.  We have thus established the aforementioned claim.

		Note that if $\card{\faktor{\alpha}{\sim}}=1$ then, since $\alpha$ is countable, it follows from lemma \ref{lem:fvsum} that $\alpha$ is $n$-equivalent to a $k$-colouring the order type
		\begin{equation}
			\sum_{k\in\dual{\omega}}(\one+\lambda)+\one+\sum_{k\in\omega}(\lambda+\one)\cong\lambda,
		\end{equation}
		as required.  Suppose then, by way of contradiction, that $\card{\faktor{\alpha}{\sim}}>1$.
		\begin{claim}
			There exists a (proper) open interval $D$ of $\faktor{\alpha}{\sim}$ and a finite set $\Sigma$ of coloured linear orders, each of which either has order type $\one$ or $\one+\lambda+\one$, such that the following holds:
			\begin{enumerate}[nosep]
				\item for each $\beta\in D$ there exists a $\sigma_\beta\in\Sigma$  such that $\beta\nequiv{n}\sigma_\beta$,

				\item if $\sigma\in\Sigma$ then the set $\setbuild{\beta\in D}{\beta\nequiv{n}\sigma}$ is dense in $D$.
			\end{enumerate}
		\end{claim}

		Define $C=\setbuild{\cha{\beta}{n}}{\beta\in\faktor{\alpha}{\sim}\text{ and }\beta\text{ is neither cofinal nor coinitial in }\alpha}$, then it immediately follows that $C$ is a finite set of sentences and, since $\card{\faktor{\alpha}{\sim}}>1$, $C$ is nonempty.  Now fix some set $\Sigma_C$ of coloured linear orders such that each of its members is at most countable and models precisely one sentence in $C$.  Note that, from its definition, each $\beta\in\Sigma_C$ either has order type $\one$ or, otherwise, order type $\one+\lambda+\one$.

		We will now argue, by means of a contradiction, that for some suitable $\Sigma\subseteq\Sigma_C$ there will exist an interval $D$ of $\faktor{\alpha}{\sim}$ satisfying properties 1 and 2 in the claim above.  Suppose that no such $D$ and $\Sigma\subseteq\Sigma_C$ exists.  We note that, since $\sim$ is definable and $\alpha$ is definably quasi-seperable, that $\card{C}\geq 1$ and then argue by induction on $\card{C}$.  If $\card{C}=1$ then every $\beta\in\faktor{\alpha}{\sim}$ has order type $\one$, since $\alpha$ is definably quasi-seperable and thus $\faktor{\alpha}{\sim}$ must contain a dense set of singletons.  If we now choose any fixed $\beta_0\in\faktor{\alpha}{\sim}$ as well as $D$ any (proper) open subinterval such that $\beta_0\in D$ then $\Sigma=\set{\beta_0}$ and $D$ satisfy properties 1 and 2 above.

		Assume now the claim holds whenever $\card{C}<m$, for some fixed $m\in\posnats$.  Suppose the claim fails when $\card{C}=m$ then, by definition, there exists a $\tau_0\in C$ and an open interval $D^\prime\subsetneq\faktor{\alpha}{\sim}$ such that $\beta\not\models\tau_0$ for any $\beta\in D^\prime$.  Since $\card{C\setminus\set{\tau_0}}<m$, it follows from the inductive hypothesis that there exists an open interval $D\subsetneq D^\prime$ and a $\Sigma\subseteq\Sigma_C$ satisfying properties 1 and 2.   Thus claim 2 holds.

		\smallskip	In order to attain the desired contradiction we will show that $\card{D}=1$, contradicting that $D$ is an open interval.  Since $\alpha$ is definably quasi-separable, there must exist a $\beta_1\in\faktor{\alpha}{\sim}$ which has order type $\one$ and such that $\setbuild{\beta\in\faktor{\alpha}{\sim}}{\beta\nequiv{n}\beta_1}$ is dense in the condensation $\faktor{\alpha}{\sim}$ and thus also dense in $D$.  We will now proceed to construct a coloured linear order $\delta$ of order type $\lambda$ which is $n$-equivalent to $\bigcup D$.  Suppose that $\Sigma=\set{\sigma_0,\dotsc,\sigma_{\ell-1}}$ then we may assume $\sigma_0\nequiv{n}\beta_1$, without loss of generality (as we are free to relabel the elements of $\Sigma$).  Define the map $h\colon\reals\to\Sigma$ such that $h(x)=\sigma_0\nequiv{n}\beta_1$, for each $x\in\irrats$, and $h^{-1}[\sigma]$ is dense in $\lambda$ for each $\sigma\in\Sigma\setminus\set{\beta_1}$.  The latter is possible due to proposition \ref{prp:qdense}.  We now define the (coloured) linear order
		\begin{equation}
			\delta=\sum_{x\in\lambda}h(x),
		\end{equation}
		where the elements of $\delta$ are coloured precisely as in the summands.  Note that, since $\sigma\in\set{\one,\one+\lambda+\one}$ for each $\sigma\in\Sigma$, it follows from proposition \ref{prp:csums} that $\delta$ is a complete linear order.  Additionally, since $h(x)$ has order type $\one$ whenever $x\in\irrats$ it follows that there exists a dense embedding of
		\begin{equation}
			\gamma=\sum_{x\in\eta}h(x)
		\end{equation}
		into $\delta$.  However, since each summand of $\gamma$ is seperable and $\eta$ is countable it follows immediately that $\delta$ is also seperable.  Consequently, from theorem \ref{thm:rchar}, we may conclude that $\delta$ has order type $\lambda$.  By definition of $\delta$, one can conclude via an $n$-game that we must have $\delta\nequiv{n}\bigcup D$.  Therefore, since the equivalence classes of $\sim$ must be convex we may conclude that $\card{D}=1$, which is the required contradiction.
	\end{proof}
